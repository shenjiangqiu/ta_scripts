\begin{document}
\extrawidth{0.5in} \extrafootheight{-0in} \pagestyle{headandfoot}
\headrule \header{\textbf{CS2311 - Spring 2021}}{\textbf{HW
 9 \ifprintanswers - Solutions \fi}}{\textbf{Due: ***. **/**/21}} \footrule \footer{}{Page \thepage\
of \numpages}{}
\ifprintanswers
\noindent \textbf{Instructions:} All assignments are due \underline{by
\textbf{midnight} on the due date} specified.  Assignments should be typed or
scanned and submitted as a PDF in Canvas.

\medskip
\noindent Every student or student group must write up their own solutions in
their own manner.

\medskip
\noindent You should \underline{complete all problems}, but \underline{only a
subset will be graded} (which will be graded is not known to you ahead of
time).
\else
\noindent \textbf{Instructions:} All assignments are due \underline{by \textbf{midnight} on the due date} specified.

\medskip
\noindent Every student or student group  must write up their own solutions in
their own manner.

\medskip
\noindent Please present your solutions in a clean, understandable
manner.  Use the provided files that give mathematical notation in Word, Open Office, Google Docs, and \LaTeX.

\medskip
\noindent Assignments should be typed or scanned and submitted as a PDF.

\medskip
\noindent You should \underline{complete all problems}, but \underline{only a
subset will be graded} (which will be graded is not known to you ahead of
time).
\fi

\begin{questions}



\section*{Sequences} 

\gquestion{12}{6}{e} Rosen Ch 2.4 \# 16(c,e), p. 178.
    \ifprintanswers
        \vspace{-10pt}
   \fi
\begin{solution}
    % \begin{itemize}[itemsep=0pt,parsep=0pt,topsep=0pt,partopsep=0pt]
    %     \item[a)]
        % \small \vspace{-0.2in}
        % \begin{align*}
        %     (a) \quad \quad a_n &= -a_{n-1} = -1\cdot a_{n-1} \\
        %       &= -(-a_{n-2}) = (-1)^2a_{n-2} \\
        %       &= -(-(-a_{n-3})) = (-1)^3a_{n-3} \\
        %       &= \cdots \\
        %       &= (-1)^n\cdot a_{n-n} = (-1)^n\cdot a_{0} = 5\cdot(-1)^n
        % \end{align*}
        % \item[(c)] 
        \vspace{-0.1in} 
        \begin{align*}
            (c) \quad \quad a_n &= a_{n-1} - n \\
            &= a_{n-2} -(n-1) - n = a_{n-1} - (n + (n-1)) \\
            &= a_{n-3} -(n-2) - (n + (n-1)) = a_{n-2} - (n + (n-1) + (n-2)) \\
            &= \cdots \\
            &= a_{n-n} -(n + (n-1) + (n-2) + \ldots + (n- (n-1)) \\
            &= a_0 - (n + (n-1) + (n-2) + \ldots + 1) \\
            &= 4 - (n + (n-1) + (n-2) + \ldots + 1) = 4 -\frac{n(n+1)}{2}
        \end{align*}
    % \end{itemize}
        \vspace{-0.1in}
        \begin{align*}
            (e) \quad \quad a_n &= (n+1)a_{n-1} \\
              &= (n+1)na_{n-2} \\
              &= (n+1)n(n-1)a_{n-3} \\
              &= \cdots \\
              &= (n+1)n(n-1)\cdots(n-(n-2))a_{n-n} \\
              &= (n+1)n(n-1)\cdots2a_{0} \\
              &= 2(n+1)!
        \end{align*}
\end{solution}


\ugquestion{6} A mortage loan is paid off in periodic (monthly) installments, while interest is also charged each period.  A mortage with an annual interest rate of $r$ has a monthly interest rate of $i = \frac{r}{12}$.  A mortage of $M$ dollars at monthly interest rate $i$ has payments of $P$ dollars.  At the end of the month, interest is added to the previous balance, and then the payment $P$ is subtracted from the result.  Let $m_n$ be the balance due after $n$ months, where $m_0 = M$, $m_1 = M(1+i) - P, \ldots$
\begin{enumerate}[label=(\alph*),itemsep=0pt,parsep=0pt,
    topsep=0pt,partopsep=0pt]
    \item Let $M = 10,000$, $i = 0.03$ and $P = 105.13$, determine $m_2$ and $m_3$.
    % \item If $m_{20} = 79,495.98$, $i=0.02$ and $R = 822.89$, determine $m_{21}$.
    \item Find a recursive formula for the mortage balance, $m_n$
\end{enumerate}
    \ifprintanswers
        \vspace{-10pt}
   \fi
\begin{solution}
    \begin{enumerate}[label=(\alph*),itemsep=0pt,parsep=0pt,
    topsep=0pt,partopsep=0pt]
        \item $m_0 = 10,000$, $m_1 = 10194.87$, $m_2 = 10395.59$, $m_3 =  10602.32$
        \item $m_n = m_{n-1}(1+i) - P$, $m_o = M$
    \end{enumerate}
\end{solution}


\section*{Summations}


\ugquestion{2} Write out in sigma notation the sum of the first 40 terms of the series $3 + 6 + 9 + 12 + \ldots$.
    \ifprintanswers
        \vspace{-10pt}
   \fi
\begin{solution}
    The sum can take many form depending on index of summation and limits.  Here are two samples

\begin{center}
\begin{tabular}{ccc}
   $\ds \sum_{i=1}^{40} 3\cdot i$ \hspace{0.25in} & $\ds \sum_{j=1}^{40} 3\cdot j$ \hspace{0.25in} &
    $\ds \sum_{k=0}^{39} 3\cdot (k+1)$
\end{tabular}
\end{center}
\end{solution}


\gquestion{12}{8}{(a,c-e)} What are the values of the sums:\\
\begin{tabular}{lll}
    (a) $\ds \sum_{j\in S} (2j - 1), \text{ where } S = \{1, 2, 4, 6\}$
    & (b) $\ds \sum_{i=1}^{8} 4$  \hspace*{1in} 
    & (c) $\ds \sum_{k=0}^{4} (-3)^k$ \\
    & & \\
    (d) $\ds \sum_{j=0}^{7} (2^{j+1} - 2^{j-1})$
    & (e) $\ds \sum_{i=1}^{4} \sum_{j=1}^{3} (2i + 3j)$
    & (f) $\ds \sum_{j=0}^{3} \sum_{k=1}^{3} j*2^{k}$
\end{tabular}
    \ifprintanswers
        \vspace{-10pt}
   \fi
\begin{solution}
    \begin{enumerate}[label=(\alph*),itemsep=2pt,parsep=0pt,
    topsep=0pt,partopsep=0pt]
        \item $(2\cdot1 - 1)+(2\cdot2 - 1)+(2\cdot4 -1)+(2\cdot6-1) = 1+3+7+11 = 22$ 
        \item $\ds \sum_{i=1}^8 4 = 8\cdot4 = 32$
        \item $(-3^0) + (-3^1) + (-3^2) + (-3^3) + (-3^4) = 1 -3 + 9 -27 + 81 = 61 $
        \item $\ds \sum_{j=0}^7 2^{j+1} - \sum_{j=0}^7 2^{j-1} = 510 - 127.5 = 382.5$
        \item $\ds \sum_{i=1}^4 \left( \sum_{j=1}^3 2i + \sum_{j=1}^3 3j \right) = 132$
        \item $\ds \sum_{j=0}^3 \sum_{k=1}^3 j*2^k = 84$
    \end{enumerate}
\end{solution}



\uplevel{\ifprintanswers 
\else 
A \textit{closed form} of a summation is an equation
in which no summation symbol appears.  The classic example is
$\ds \sum_{i=1}^n i = \frac{n(n+1)}{2}$.  The fraction on
the right is the \textit{closed form} of the summation. 

\medskip
\fi
}
% \ifprintanswers
\small
\begin{tabular}{llll}
  \multicolumn{3}{l}{\textit{Arithmetic Properties of Summations}} &  \\
  \hline   
  Fact 1: & $\ds \sum_{i=1}^n c  = nc$     & Fact 2: & $\ds \sum_{i=j}^n c  = (n-j+1)c$ \\
    & when $c$ is a constant & & when $c$ is a constant \\ 
  Fact 3: & $\ds \sum_{i=j}^n (f(i) \pm g(i)) = \sum_{i=j}^n f(i) \pm \sum_{i=j}^n g(i)$ \hspace{0.4in} & Fact 4: & $\ds \sum_{i=j}^n cf(i) = c \sum_{i=j}^n f(i)$ \\
    & & & when $c$ is a constant \\
  \hline 
\end{tabular}

\medskip
\begin{tabular}{llllll}
  \multicolumn{3}{l}{\textit{Closed-form Summation Formulae}} \\
  \hline 
  Name & Sum & Closed Form & Name & Sum & Closed Form \\
  Table 2.1 & $\ds \sum_{k=0}^n ar^k \;(r \neq 0)$ 
    & $\ds \frac{ar^{n+1} - a}{r-1}, \;\text{if}\; r \neq 1$ \hspace{0.25in}
    & Table 2.2 & $\ds \sum_{k=1}^n k $
    & $\ds \frac{n(n+1)}{2}$  \\
  Table 2.3 & $\ds \sum_{k=1}^n k^2 $ 
    & $\ds \frac{n(n+1)(2n+1)}{6}$  
    & Table 2.4 & $\ds \sum_{k=1}^n k^3 $
    & $\ds \frac{n^2(n+1)^2}{4} $  \\
  Table 2.5 & $\displaystyle \sum_{k=0}^{\infty} x^k, \;|x| < 1$ 
    & $\displaystyle \frac{1}{1-x}$ 
    & Table 2.6 & $\displaystyle \sum_{k=1}^{\infty} kx^{k-1}, \;|x| < 1$ & $\displaystyle \frac{1}{(1-x)^2} $\\
  \hline
\end{tabular}


% \else
% \small
% \begin{tabular}{lll}
%     \multicolumn{2}{l}{Facts:} & Example: \\
%   Fact 1: & $\ds \sum_{i=1}^n c  = nc$     & $\ds \sum_{i=1}^n 7 = 7n$ \\
%     & when $c$ is a constant \\
%   Fact 2: & $\ds \sum_{i=j}^n c  = (n-j+1)c$  & $\ds \sum_{i=0}^n = 7(n+1) $ \\
%     & when $c$ is a constant \\
%   Fact 3: & $\ds \sum_{i=j}^n (f(i) \pm g(i)) = \sum_{i=j}^n f(i) \pm \sum_{i=j}^n g(i)$ \hspace{0.2in} &  $\ds \sum_{i=j}^n (2n - n^2) = \sum_{i=j}^n 2n - \sum_{i=j}^n n^2 $ \\
%     & \\
%   Fact 4: & $\ds \sum_{i=j}^n cf(i) = c \sum_{i=j}^n f(i)$  & $\ds \sum_{i=j}^n (3 \times 2^i) = 3 \sum_{i=j}^n 2^i $ \\
%    & when $c$ is a constant \\
% \end{tabular}


% \begin{tabular}{llll}
%      Name & Summation & & Closed Form \\
%      \\
%      Table 2.1 & $\displaystyle \sum_{k=0}^n ar^k \;(r \neq 0)$ 
%       & \hspace{0.3in} & $\displaystyle \frac{ar^{n+1} - a}{r-1}, \;\text{if}\; r \neq 1$ \\
%      \\
%      % $\displaystyle \sum_{k=0}^n ar^k \;(r = 1)$ & & $\displaystyle (n+1)a, \;\text{if}\; r=1$ \\
%      % \\
%      Table 2.2 & $\displaystyle \sum_{k=1}^n k $ & 
%       & $\displaystyle \frac{n(n+1)}{2}$ \\
%      \\
%      Table 2.3 & $\displaystyle \sum_{k=1}^n k^2 $ & 
%       & $\displaystyle \frac{n(n+1)(2n+1)}{6}$ \\
%      Table 2.4 & $\displaystyle \sum_{k=1}^n k^3 $ & 
%       & $\displaystyle \frac{n^2(n+1)^2}{4} $ \\
%      \\
%      Table 2.5 & $\displaystyle \sum_{k=0}^{\infty} x^k, \;|x| < 1$ & 
%       & $\displaystyle \frac{1}{1-x}$ \\
%      \\
%      Table 2.6 & $\displaystyle \sum_{k=1}^{\infty} kx^{k-1}, \;|x| < 1$ &
%       & $\displaystyle \frac{1}{(1-x)^2} $
%     \end{tabular}

% \fi


\gquestion{8}{8}{all} Find a closed form for the summation $\displaystyle \sum_{i=0}^n (2 \cdot 3^i - 3 \cdot 2^i)$.   Show how you find the closed form solution; justify each step using the four facts of the arithmetic properties of summations and Table 2 above (p. 166 in the book).
    \ifprintanswers
        \vspace{-10pt}
   \fi
\begin{solution}
\begin{align*}
  \sum_{i=0}^n (2 &\cdot 3^i - 3 \cdot 2^i) \\
     &= \sum_{i=0}^n 2 \cdot 3^i - \sum_{i=0}^n 3 \cdot 2^i \tag{Fact 3} \\
     &= 2 \sum_{i=0}^n 3^i - 3 \sum_{i=0}^n 2^i \tag{Fact 4} \\
     &= 2 \left( \frac{3^{n+1} - 1}{2} \right) 
        - 3 \left( \frac{2^{n+1} - 1}{1} \right) \tag{Table 2.1} \\
     &= 3^{n+1} - 1 - 3(2^{n+1} - 1) \tag{algebra} \\
     &= 3^{n+1} - 3\cdot 2^{n+1} +2 \tag{algebra}
\end{align*}
\end{solution}



\ugquestion{8} Find a closed form for the summation $\displaystyle \sum_{i=4}^n 7\cdot 5^i$.  Show how you find the closed form solution; justify each step using the four facts of the arithmetic properties of summations and Table 2 above (p. 166 in the book).
    \ifprintanswers
        \vspace{-10pt}
   \fi
\begin{solution}
  \begin{align*}
      \sum_{i=4}^n 7\cdot 5^i  &= \sum_{j=0}^{n-4} 7\cdot 5^{j+4} \tag{change of index} \\
       &= \sum_{j=0}^{n-4} 7\cdot 5^{j}5^{4} \tag{algebra} \\
       &= 7\cdot 5^4 \sum_{j=0}^{n-4} 5^j \tag{Fact 4} \\
       &= 7\cdot 5^4 \left( \frac{5^{n-3} - 1}{4}  \right) \tag{Table 2.1} \\
       &= \frac{7\cdot 5^4(5^{n-3} - 1)}{4} \tag{algebra}
  \end{align*}

  or, 
  \begin{align*}
      \sum_{i=4}^n 7\cdot 5^i  &= \sum_{j=0}^{n-4} 7\cdot 5^{j+4} \tag{change of index} \\
       &= \sum_{j=0}^{n-4} 7\cdot 5^{j}5^{4} \tag{algebra} \\
       &= \frac{7\cdot 5^4 \cdot 5^{n-3} - 7\cdot 5^4}{4} \tag{Table 2.1} \\
       &= \frac{7\cdot 5^4(5^{n-3} - 1)}{4} \tag{algebra}
  \end{align*}

  Alternatively, 
  \begin{align*}
    \sum_{i=4}^n 7\cdot5^i  &= \sum_{i=0}^n 7\cdot5^i - \sum_{i=0}^3 7\cdot5^i \tag{prop. of sum }\\
      &= \frac{7\cdot5^{n+1} - 7}{4} - \frac{7\cdot5^4 - 7}{4} \tag{Table 2.1} \\
      &= \frac{7\cdot5^{n+1} - 7\cdot5^4}{4} \tag{algebra} \\
      &= \frac{7\cdot5^4\cdot5^{n-3} - 7\cdot5^4}{4} \tag{algebra} \\
      &= \frac{7\cdot5^4 (5^{n-3} - 1)}{4}  
  \end{align*}
\end{solution}



\section*{Induction}

\uplevel{Follow the template for inductive proofs given on p. 329 of the book.}

\gquestion{10}{10}{all} Let $P(n)$ be the statement $\sum_{j=1}^n 2j = n + n^2$ for $n \geq 1$.
\begin{enumerate}[label=(\alph*),itemsep=0pt,parsep=0pt,
    topsep=0pt,partopsep=0pt]
  \item (1 pt) What is the statement $P(1)$? 
  \item (1 pt) Show that P(1) is true, completing the basis step of the proof.
  \item (1.5 pts) What is the inductive hypothesis? 
  \item (1.5 pts) What do you need to prove in the inductive step? 
  \item (4 pts) Complete the inductive step.
  \item (1 pt) Explain why these steps show that this formula is true whenever $n$ is a positive integer. 
\end{enumerate}
    \ifprintanswers
        \vspace{-10pt}
   \fi
\begin{solution}
\begin{enumerate}[label=(\alph*),itemsep=0pt,parsep=0pt,
    topsep=0pt,partopsep=0pt]
  \item (1 pt) What is the statement $P(1)$? \\
    $P(1)$ is the statement $\sum_{j=1}^1 2j = 1 + 1^2 $.
  \item (1 pt) Show that P(1) is true, completing the basis step of the proof. \\
    $\sum_{j=1}^1 2j = 2\cdot 1 = 2 = 1 + 1^2 $.
  \item (1.5 pts) What is the inductive hypothesis? \\
    The inductive hypothesis is to assume $P(k)$ is true for an arbitrary, fixed integer $k \geq 1$, that is
    \[ \sum_{j=1}^k 2j = k + k^2 \]
  \item (1.5 pts) What do you need to prove in the inductive step? \\
    For the inductive step, show for each $k \geq 1$ that $P(k)$ implies $P(k+1)$. \\
    That is, show $P(k+1)$:
    \[ \sum_{j=1}^{k+1} 2j = (k+1) + (k+1)^2  = k^2 + 3k + 2\]
  \item (4 pts) Complete the inductive step.
    Start with $P(k+1)$ 
      \begin{align*}
        \sum_{j=1}^{k+1} 2j &= \sum_{j=1}^k 2j + 2(k+1) \\
        &= k + k^2 +  2(k+1) \tag{IH} \\
        &= k^2 + k + 2k + 2 \\
        &= k^2 + 3k + 2 
      \end{align*}
      This shows $P(k+1)$ is true, assuming $P(k)$ is true, completing the inductive step.
  \item (1 pt) Explain why these steps show that this formula is true whenever $n$ is a positive integer. \\
    The basis step and inductive step are completed.  Therefore by principle of mathematical induction, the statement, $P(n)$, is true for every positive integer $n$.
\end{enumerate}
\end{solution}
  % \textit{Proof:}
  % Let $P(n)$ be the statement $\sum_{j=1}^n 2j = n + n^2$ for $n \geq 1$.

  % \smallskip
  % \smallskip
  % \begin{tabular}{lp{4in}}
  %   \textit{Basis Step:} & Show $P(1)$ is true, $\sum_{j=1}^1 2j = 2\cdot 1 = 2 =  1 + 1^2 = n + n^2 $ \\
  %    & \\
  %  \textit{Inductive Step:} &  \\
  % \end{tabular}

  % Assume $P(k)$ is true for an arbitrary, fixed integer $k \geq 1$, that is,
  % \begin{align*}
  %   \sum_{j=1}^k 2j = k + k^2 \tag{IH}
  % \end{align*}

  % Show $P(k+1)$ is true, that is, 
  % \[ \sum_{j=1}^{k+1} 2j = (k+1) + (k+1)^2 = k+1 + k^2 + 2k + 1 = k^2 + 3k + 2 = (k+1)(k+2) \]

  % Start with $P(k+1)$ 
  % \begin{align*}
  %   \sum_{j=1}^{k+1} 2j = \sum_{j=1}^k 2j + 2(k+1) \\
  %   &= k + k^2 +  2(k+1) \tag{IH} \\
  %   &= k^2 + k + 2k + 2 \\
  %   &= k^2 + 3k + 2 = (k+2)(k+1) \\
  %   &= k+1 + k^2 + 2k + 1 = (k+1) + (k+1)^2 \tag{alternatively} 
  % \end{align*}
  % This shows $P(k+1)$ is true, assuming $P(k)$ is true, completing the inductive step.

  % Therefore by mathematical induction, $P(n)$ is true for $n \geq 1$.
% \end{solution}


\ugquestion{8} Prove using mathematical induction that
\[ 1 + 5 + 5^2 + 5^3 + \cdots + 5^n = \frac{5^{n+1} - 1}{4} \text{ for all } n\geq 0. \]
    \ifprintanswers
        \vspace{-10pt}
   \fi
\begin{solution}
    Let $P(n)$ be $1 + 5 + 5^2 + 5^3 + \cdots + 5^n = \frac{5^{n+1} - 1}{4}$.

    \smallskip
    Show, for all $n\geq 0, P(n)$.

    \smallskip
    \textit{Basis Step:}\\ Show $n=0$, $1 = \frac{5^{0+1} - 1}{4} = \frac{5-1}{4} = 1.$ \\
    Therefore, $P(0)$ is true.

    \smallskip
    \textit{Inductive Step:} \\
    Assume $P(k)$ is true, for some arbitrary, fixed integer $k \geq 0$,
       \[ 1 + 5 + 5^2 + 5^3 + \cdots + 5^k = \frac{5^{k+1} - 1}{4} \]
    Show $P(k+1)$ is true,
      \[ 1 + 5 + 5^2 + 5^3 + \cdots + 5^{k+1} = \frac{5^{k+2} - 1}{4} \]
    Begin with $P(k)$ and add the next term, $5^{k+1}$ to both sides.
    \begin{align*}
        1 + 5 + 5^2 + 5^3 + \cdots + 5^k &= \frac{5^{k+1} - 1}{4} \\
        1 + 5 + 5^2 + 5^3 + \cdots + 5^k  + 5^{k+1} &= \frac{5^{k+1} - 1}{4} + 5^{k+1} \\
          &= \frac{5^{k+1} - 1 + 4\cdot 5^{k+1}}{4} \\
          &= \frac{5\cdot 5^{k+1} - 1}{4} = \frac{5^{k+2} - 1}{4} \\
    \end{align*}
    This is the form of $P(k+1)$, thus completing the inductive step.

    Therefore, by mathematical induction $P(n)$ is true for all $n \geq 0$.
\end{solution}



\gquestion{8}{8}{all} Rosen Ch 5.1 \# 32, p. 351
    \ifprintanswers
        \vspace{-10pt}
   \fi
\begin{solution}
Let $P(n)$ be the proposition that $3 \;|\; n^3 + 2n$ for positive integers $n$.

\smallskip
\begin{tabular}{ll}
  \textit{Basis Step:} & Show $P(1)$ is true, $3 \;|\; n^3 + 2n$ or $3 \;|\; 1 + 2$ or $3 \;|\; 3$ which is true. \\
   & \\
 \textit{Inductive Step:} &  \\
\end{tabular}

Assume $P(k)$ is true for an arbitrary, fixed
 integer $k \geq 1$, that is,
\begin{align*}
  3 \;|\; k^3 + 2k \tag{IH} 
\end{align*}

Show $P(k+1)$ is true, that is,
\[ 3 \;|\; (k+1)^3 + 2(k+1) \]

Start with the expression used in $P(k+1)$, we want to show this is divisible by 3.
\begin{align*}
  (k+1)^3 + 2(k+1) &=  k^3 + 3k^2 + 3k + 1 + 2k + 2 \\
   &= (k^3 + 2k) + (3k^2 + 3k + 3) \\
\end{align*}
Each of the terms in parenthesis are divisible by 3.  The first term $k^3 + 2k$ is divisible by 3 using the Inductive Hypothesis, the second term has a 3 pulled from the expression, $3\cdot(k^2 + k + 1)$ so it is also divisible by 3. 

This shows $P(k+1)$ is true when $P(k)$ is true, completing the
inductive step.

\smallskip
Hence, the basis step and inductive step are completed, by mathematical induction $P(n)$ is true for all $n$ such
that $n\geq 1$.
\end{solution}


\section*{Bonus Questions} 


\bonusquestion[4] Find a closed form for the summation $\displaystyle \sum_{i=1}^n \sum_{j=1}^n (6i^2 - 2j)$.  Show how you find the closed form solution; justify each step using the four facts of the arithmetic properties of summations and Table 2, p. 166 in the book.
    \ifprintanswers
        \vspace{-10pt}
   \fi
\begin{solution}
\begin{align*}
        \sum_{i=1}^n \sum_{j=1}^n (6i^2 - 2j) 
        &= \sum_{i=1}^n ( \sum_{j=1}^n (6i^2 - 2j) ) \tag{implied parentheses} \\
        &= \sum_{i=1}^n (\sum_{j=1}^n 6i^2 - \sum_{j=1}^n 2j) \tag{Fact 3} \\
        &= \sum_{i=1}^n (6i^2 \sum_{j=1}^n 1 - 2 \sum_{j=1}^n j) \tag{Fact 4, twice} \\
        &= \sum_{i=1}^n (6i^2n - 2 \sum_{j=1}^n j) \tag{Fact 1} \\
        &= \sum_{i=1}^n (6i^2n - 2 \frac{n(n+1)}{2}) \tag{Table 2} \\
        &= \sum_{i=1}^n 6i^2n - \sum_{i=1}^n  n(n+1) \tag{Fact 3} \\
        &= 6n \sum_{i=1}^n i^2 - n(n+1) \sum_{i=1}^n 1 \tag{Fact 4, twice} \\
        &= 6n (\frac{n(n+1)(2n+1)}{6}) - n(n+1) \sum_{i=1}^n 1 \tag{Table 2} \\
        &= n^2(n+1)(2n+1) - n^2(n+1) \tag{algebra} \\
        &= n^2(n+1)(2n + 1  - 1) \\
        &= 2n^3(n+1) 
    \end{align*}
\end{solution}



% Ferland, p. 191, Exercise #17.
\bonusquestion[4] Let $\{s_n\}$ be the sequence defined as, 
\[ s_1 = 4 \quad \text{and} \quad s_n = 3s_{n-1} - 2, \forall n \geq 2. \]
Show $\forall n \geq 1, s_n = 3^n + 1.$
    \ifprintanswers
        \vspace{-10pt}
   \fi
\begin{solution}
  \textit{Proof:}
  Let $P(n)$ be that the $n$th term of the sequence is found as $s_n = 3^n + 1$ for $n \geq 1$

  \smallskip
  \begin{tabular}{lp{4in}}
    \textit{Basis Step:} & Show $P(1)$ is true, $s_1 = 3^1 + 1 = 4$ \\
     & \\
   \textit{Inductive Step:} &  \\
  \end{tabular}

  Assume $P(k)$ is ture for an arbitrary fixed integer $k \geq 1$, that is, 
  \begin{align*}
    s_k &= 3^k + 1 \tag{IH} \\
  \end{align*}

  Show $P(k+1)$ is true, that is,
  \[ s_{k+1} = 3^{k+1} + 1 \]

  Start with $P(k+1)$ and the definition of the sequence, 
  \begin{align*}
    s_{k+1} &= 3s_{k} - 2 \\
    &= 3(3^k + 1) - 2 \tag{IH}\\
    &= 3^{k+1} + 3 - 2 = 3^{k+1} + 1
  \end{align*}
  This shows $P(k+1)$ is true, assuming $P(k)$, completing the inductive step.

  \smallskip
  Therefore, we have shown by mathematical induction, $P(n)$ is true for all $n \geq 1$.
\end{solution}


\end{questions}
\end{document}
