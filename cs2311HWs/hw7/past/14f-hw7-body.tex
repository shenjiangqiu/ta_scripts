\begin{document}
\extrawidth{0.5in} \extrafootheight{-0in} \pagestyle{headandfoot}
\headrule \header{\textbf{cs2311 - Fall 2014}}{\textbf{HW
 7 \ifprintanswers - Solutions \fi}}{\textbf{Due: Mon. 11/3/14}} \footrule \footer{}{Page \thepage\
of \numpages}{}

\ifprintanswers
% \noindent You should \underline{complete all problems}, but \underline{only a subset will be graded} (which will be graded is not known to you ahead of time). 
\else
\noindent \textbf{Instructions:} All assignments are due \underline{by \textbf{4pm} on the due date} specified.  There will be a box in the CS department office (Rekhi 221) where assignments may be turned in.  Solutions will be posted on-line at the following lecture.  Every student
must write up their own solutions in their own manner.

\medskip
\noindent Please present your solutions in a clean, understandable
manner; pages should be stapled before being turned in, no ragged edges of
paper.

% \medskip
% \noindent Be clear with you penmanship to distinguish set brackets and parentheses.
%   For example, $\{1, 2\}$ is a set where $(1,2)$ is an ordered pair.  Also,
% differentiate the empty set $\emptyset$ and the number zero 0.

\medskip
\noindent You should \underline{complete all problems}, but \underline{only a subset will be graded} (which will be graded is not known to you ahead of time). 
\fi

\begin{questions}

\gquestion{16}{8}{(b)} Rosen Ch 1.6 \#14 (a,b), p. 79 \\
Let $C(x)$ be ``$x$ is in this class'', $R(x)$ be ``$x$ owns a red convertible'', and $T(x)$ be ``$x$ has gotten a ticket''. \\
Let $M(x)$ be ``$x$ is one of the five roommates'', $D(x)$ be ``$x$ has taken a course in discrete math'', and $A(x)$ be ``$x$ can take a course in algorithms''.
    \ifprintanswers
        \vspace{-15pt}
    \fi
  \begin{solution}
    \begin{itemize}[itemsep=0pt,parsep=0pt,topsep=0pt,partopsep=0pt]
    \item[(a):] The premises are:
    \begin{itemize}[itemsep=0pt,parsep=0pt,topsep=0pt,partopsep=0pt]
        \item[1.] $C(Linda)$
        \item[2.] $R(Linda)$
        \item[3.] $\forall x\; (R(x) \rightarrow T(x))$
    \end{itemize}
    To conclude: $\exists x\; (C(x) \wedge T(x))$

    \smallskip
    \begin{tabular}{lll}
        Step        & \hspace{0.2in} & Reason \\
        1. $\forall x\; (R(x) \rightarrow T(x))$    &   & Hypothesis \\
        2. $R(Linda) \rightarrow T(Linda)$          &   & Universal Instantiation \\
        3. $R(Linda)$                               &   & Hypothesis \\
        4. $T(Linda)$                               &   & Modus ponens using (2) and (3) \\
        5. $C(Linda)$                               &   & Hypothesis \\
        6. $C(Linda) \wedge T(Linda)$               &   & Conjunction with (4) and (5) \\
        7. $\exists x\; (C(x) \wedge T(x))$         &   & Existential generalizaion \\
    \end{tabular}

    \item[(b):] The premises are:
    \begin{itemize}[itemsep=0pt,parsep=0pt,topsep=0pt,partopsep=0pt]
        \item[1.] $\forall x\;(M(x) \rightarrow D(x))$
        \item[2.] $\forall x\;(D(x) \rightarrow A(x))$
    \end{itemize}
    The conclusion is: $\forall x\;(M(x) \rightarrow A(x))$.  Let $y$ be an arbitrary person.

    \smallskip
    \begin{tabular}{lll}
        Step        & \hspace{0.2in} & Reason \\
        1. $\forall x\;(M(x) \rightarrow D(x))$     &   & Hypothesis \\
        2. $M(y) \rightarrow D(y)$                  & & Univ. Inst. with (1) \\
        3. $\forall x\; (D(x) \rightarrow A(x))$    &   & Hypothesis \\
        4. $D(u) \rightarrow A(y)$                  &   & Univ. Inst. with (3) \\
        5. $R(y) \rightarrow A(y)$                  &   & Hyp. syl. with (2) and (4) \\
        6. $\forall x\;(R(x) \rightarrow A(x))$     &   & Univ. Generalization with (5) \\
    \end{tabular}
    \end{itemize}
  \end{solution}



\gquestion{8}{6}{(a-c)} For each argument determine whether it is valid or not and explain why (in a sentence).
    \begin{itemize}[itemsep=0pt,parsep=0pt,topsep=0pt,partopsep=0pt]
    \item[(a)] From  ``all healthy people eat an apple a day" and  ``Samantha eats an apple a day" conclude ``Samantha is a healthy person."  
    \item[(b)] ``Some math majors left the campus for the weekend" and ``All seniors left the campus for the weekend" implies the conclusion ``Some seniors are math majors."
    \item[(c)] ``Everyone who left campus for the weekend is a senior" and ``All math majors left campus for the weekend"  implies the conclusion ``All math majors are seniors." 
    \item[(d)] ``No juniors left campus for the weekend" and ``Some math majors are not juniors" implies the conclusion ``Some math majors left campus for the weekend."
    \end{itemize}
   \ifprintanswers
        \vspace{-10pt}
    \fi
\begin{solution}
    \begin{itemize}
        \item[(a)] Not Valid argument.  Fallacy of affirming the conclusion
    \item[(b)] Not Valid argument.  The two premises do not imply the conclusion.
        \item[(c)] Valid argument. The argument uses Univ. instantiation (x2), hyp. syllogism, followed by Univ. generalization.
        \item[(d)] Not Valid argument.  The two premises do not imply the conclusion.
    \end{itemize}
\end{solution}




\ugquestion{4} Rosen Ch 1.6 \# 16(a,b), p. 79
    \ifprintanswers
        \vspace{-15pt}
    \fi
\begin{solution}
    \begin{itemize}[itemsep=0pt,parsep=0pt,topsep=0pt,partopsep=0pt]
        \item[(a)] Correct, using universal instantiation and modus tollens.
        \item[(b)] Incorrect, first apply universal instantiation then fallacy of denying the hypothesis.
        \item[(c)] Incorrect, first apply universal instantiation then fallacy of denying the hypothesis.
%        \item[(d)] Correct, using universal instantiation and modus ponens.
    \end{itemize}
\end{solution}




\ugquestion{4} Rosen Ch 1.6 \# 18, p. 79
    \ifprintanswers
        \vspace{-15pt}
    \fi
\begin{solution}
    It is true that there is some $s$ in the domain s.t. $S(s,Max)$; however, there is no guarantee that Max is the $s$.  This first step of the proof is invalid.
\end{solution}



\gquestion{6}{6}{(all)} Rosen Ch 1.6 \# 24, p. 79
    \ifprintanswers
        \vspace{-15pt}
    \fi
\begin{solution}
    The incorrect steps are 3 and 5.  The simplification rule can not be applied to disjunctions.  Also, step 7 the conjunction rule should ``AND" terms together.
\end{solution}



\gquestion{12}{12}{(all)} Consider the statements of Rosen Example 27, Ch 1.4, p. 51.  Show from the premises a valid argument that leads to the conclusion.
    \ifprintanswers
        \vspace{-10pt}
    \fi
\begin{solution}
    For the problem, the domain description and translation into logical expression has already been performed.

    \begin{tabular}{lll}
            Step                    & \hspace{0.15in} & Reason \\
            1. $\forall x\; (P(x) \ra S(x))$            & & Given \\
            2. $\neg \exists x\; (Q(x) \wedge R(x))$    & & Given \\
            3. $\forall x\; (\neg R(x) \ra \neg S(x))$  & & Given \\
            4. $\forall x\; \neg (Q(x) \wedge R(x))$    & & DeMorgan's with Quantifiers with (2) \\
            5. $P(a) \ra S(a)$                          & & Univ. instantiation with (1) \\
            6. $\neg R(a) \ra \neg S(a)$                & & Univ. instantiation with (3) \\
            7. $\neg (Q(a) \wedge R(a))$                & & Univ. instantiation with (4) \\
            8. $\neg Q(a) \vee \neg R(a)$               & & DeMorgan's with (7) \\
            9. $\neg R(a) \vee \neg Q(a)$               & & Commutative with (8) \\
            10. $\neg \neg R(a) \ra \neg Q(a)$          & & Table 7, rule 3 with (9) \\
            11. $R(a) \ra \neg Q(a)$                    & & Double negation with (10) \\
            12. $S(a) \ra R(a)$                         & & Table 7, rule 2 with (6) \\
            13. $P(a) \ra R(a)$                         & & Hyp. syllogism with (5) and (12) \\
            14. $P(a) \ra \neg Q(a)$                    & & Hyp. syllogism with (13) and (11) \\
            15. $\forall x\; (P(x) \ra \neg Q(x))$      & & Univ. generalization with (14)
    \end{tabular}
\end{solution}



% Proofs
\ifprintanswers
\else
\uplevel{
  For the next two problems, you will use the definition of
  $n$ being \textit{divisible} by $m$, specifically:

  \smallskip
  \textbf{Definition:} For integers $n$ and $m$ and $m\neq0$, $n$ is
  \underline{divisible} by $m$ if and only if there is a integer $k$ such that $n=km$.  The notation $m | n$ denotes, ``$m$ divides $n$".}

\smallskip
\uplevel{
  For example, 56 is divisible by 8; there is a natural number $k$
  (specifically, 7) such that $56 = 8k$. Also, thinking of the
  bi-implication in the other direction, you can state because $39 = 3
  \cdot 13$, 39 is divisible by 3 (or 39 is divisible by 13).}

  \uplevel{An example of a proof using this new definition is to show,
  \begin{center}
   "For all integers $n$, if $n$ is divisible by 6, then $n^2$
   is divisible by 9."
  \end{center}
  Follow the examples in class,
  \begin{quote}
  Assume $n$ is divisible by 6.  Then, by definition of divisible
  there is some integer $k$ s.t. $n=6k$. Compute $n^2$:
  \[ n^2 = (6k)^2 = 36k^2 = 9(4k^2). \]
  where $n^2$ is divisible by 9 by definition. Therefore, for all
  integers, if $n$ is divisible by 6, then $n^2$ is divisible
  by 9.
  \end{quote}}
\fi



\gquestion{6}{6}{(all)} Prove: For all natural numbers $m$ and $n$, if $m$ is divisible by 5 and $n$ is divisible by 4, then $m\cdot n$ is divisible by 10.
    \ifprintanswers
        \vspace{-10pt}
    \fi
  \begin{solution} \textbf{Proof:} Assume $m$ is divisible by 5 and also assume $n$ is divisible by 4.  By definition of divisibility, then there exists a integer $k_m$ and a integer $k_n$ such that $m = 5k_m$ and $n=4k_n$.  Then, $m\cdot n$ is:
    \[ m\cdot n = 5k_m \cdot 4k_n = 20k_mk_n = 10(2k_mk_n). \]
  From this expression, you can see $m\cdot n$ is divisible by 10.  Therefore, if $m$ is divisible by 5 and $n$ is divisible by 4, then $m \cdot n$ is divisible by 10.
  \end{solution}



\ugquestion{6} Prove: For all integers $a$, $b$, and $c$, if $b$ is divisible by $a$ and $c$ is divisible by $a$, then $2b - 3c$ is divisible by $a$.
    \ifprintanswers
        \vspace{-10pt}
    \fi
  \begin{solution} \textbf{Proof:} Assume $a$, $b$, and $c$ are integers and $a | b$ and $a | c$.  By definition of divisibility, there exists integers $k_b$ and $k_c$ such that $b = ak_b$ and $c = ak_c$.  By substitution,
    \[ 2b - 3c = 2(ak_b) - 3(ak_c) = a(2k_b - 3k_c), \]
  where the expression $2k_b - 3k_c$ is also an integer.  Hence, $2b - 3c$ is also divisible by $a$.  Therefore, if $b$ is divisible by $a$ and $c$ is divisible by $a$, then $2b-3c$ is divisible by $a$.
  \end{solution}


\gquestion{12}{6}{(b)} Give a (a) proof by contradiction and (b) proof by contraposition that if $n$ is a natural
number and $3n+3$ is odd, then $n$ is even.
    \ifprintanswers
        \vspace{-10pt}
    \fi
\begin{solution}
    \textbf{Proof by contradiction:} Assume $3n+3$ is odd and $n$ is
    odd.  Then $n=2k + 1$ for some integer $k$.  Compute
    $3n+3$:
    \[ 3n + 3 = 3(2k+1) + 3 = 6k + 6 = 2(3k + 3)\]
    showing $3n+3$ is even. This contradicts the assumption that $3n
    + 3$ is odd. Consequently, if $n$ is a natural number and $3n + 3$ is odd, then $n$ is even.

    \textbf{Proof by contraposition} Assume $n$ is odd. Then by definition of odd, there exists an integer $k$ such that $n = 2k+1$.  Compute $3n+3$:
    \[ 3n+3 = 3(2k+1) + 3 = 6k + 3 + 3 = 6k + 6 = 2(3k + 3) = 2k' \]
    where the expression $2(3k+3)$ is an even integer.  Therefore, by contraposition if $3n+3$ is odd, then $n$ is even.
\end{solution}


\gquestion{12}{12}{(all)} Prove for all natural numbers $n$ and $m$, $nm$ is odd if
and only if $n$ and $m$ are both odd.
    \ifprintanswers
        \vspace{-10pt}
    \fi
\begin{solution}
     This proof is for a theorem using ``if and only if" therefore,
     it must be proved in both directions.  \\
    \textbf{Prove ``if p, then q" by contradiction:} Assume $nm$ is
    odd and $n$ and $m$ are not both odd.  Then at least one of $n$
    and $m$ are even.  Suppose $n$ is the one that is even. Then
    $n=2k$ for some natural number $k$.  Then $nm = 2km$ and so $nm$
    is even.  This contradicts the assumption that $nm$ is odd.
    Similarly, if we assume instead that $m$ is the one that is even,
    we reach the same contradiction.   If both $m$ and $n$ are both
    even we reach the same result.  Thus, it must be that $n$ and
    $m$ are both odd.

    \smallskip
    \textbf{Prove ``if q then p" directly:} Assume $n$ and $m$ are
    both odd.  Then, there are natural numbers $k$ and $j$ such
    that $n = 2k+1$ and $m=2j+1$.  Then,
    \[ nm = (2k+1)(2j+1) = 4kj + 2(k+j) +1 = 2(2kj +k + j) +1 \]
    So, $nm$ is odd by definition. Therefore, if $n$ and $m$ are
    odd, $nm$ is odd.
\end{solution}


\ugquestion{6} Use a proof by cases to show that $min(a, min(b,c)) =
min(min(a,b),c)$ whenever $a$, $b$, and $c$ are real numbers.
% SEE Jean Mayo hw5
    \ifprintanswers
        \vspace{-10pt}
    \fi
\begin{solution}
    There are 6 cases for the values of $a$,$b$, $c$. \\
    \textbf{case 1:} $a \leq b \leq c$, then \\
        $min(a, min(b,c)) = min(a,c) = a$ \\
        $min(min(a,b),c)) = min(a,c) = a$ \\
    \textbf{case 2:} $a \leq c \leq b$, then \\
        $min(a, min(b,c)) = min(a,c,) = a$ \\
        $min(min(a,b),c) = min(a,c) = a$ \\
    \textbf{case 3:} $b \leq a \leq c$, then \\
        $min(a, min(b,c)) = min(a,b) = b$\\
        $min(min(a,b),c) = min(b,c) = b$ \\
    \textbf{case 4:} $b \leq c \leq a$, then \\
        $min(a, min(b,c)) = min(a,b) = b$ \\
        $min(min(a,b),c) = min(b,c) = b$ \\
    \textbf{case 5:} $c \leq a \leq b$, then \\
        $min(a, min(b,c)) = min(a,c) = c$\\
        $min(min(a,b),c) = min(a,c) = c$ \\
    \textbf{case 6:} $c \leq b \leq a$, then \\
        $min(a, min(b,c)) = min(a,c) = c$ \\
        $min(min(a,b),c) = min(b,c) = c$ \\
    We have shown it to be true, in all cases.
\end{solution}



\ugquestion{2} Prove that there is a positive integer that equals the
sum of the positive integers not exceeding it.
    \ifprintanswers
        \vspace{-10pt}
    \fi
\begin{solution}
    This is an existence proof.  3 is an example of such a positive
    integer, 3 = 1 + 2.
\end{solution}


\gquestion{2}{2}{(all)} Prove or disprove: If $a$ and $b$ are rational
numbers, then $a^b$ is also rational.
    \ifprintanswers
        \vspace{-10pt}
    \fi
\begin{solution}
    \textbf{Disprove:} Let $a=2$ and $b=\frac{1}{2}$, which are both rational numbers.  Then $a^b =
    \sqrt{2}$ which is irrational.
\end{solution}


\gquestion{2}{2}{(all)} Prove or disprove: The sum of four consecutive integers is divisible by 4.
    \ifprintanswers
        \vspace{-10pt}
    \fi
\begin{solution}
    \textbf{Disprove:} Let the consecutive integers be 1, 2, 3, 4; the sum is 10 which is not divisible by 4.
\end{solution}


\bonusquestion[4] Prove the following proposition, ``For any integer $n \geq 2$, $n^2-3$ is never divisible by 4."
    \ifprintanswers
        \vspace{-10pt}
    \fi
\begin{solution}
    Consider two cases:
    \begin{itemize}
        \item[Case 1]: If $n$ is even, then by definition $n^2$is also even and $n^2-3$ will be odd.  Therefore, $n^2-3$ is not divisible by 4.
        \item[Case 2]: If $n$ is odd, there are 4 cases to consider. $n$ can be written as $4k$, $4k+1$, $4k+2$, and $4k+3$ for an integer $k$.  The forms of $4k$ and $4k+2$ are even, not matching the condition $n$ is odd, and will not be considered further.
            \begin{itemize}
                \item[Case a]:
                \begin{align*}
                    n &= 4k + 1 \\
                    n^2 - 3 &= (4k + 1)^2 - 3 \\
                        &= 16k^2 + 8k + 1 - 3 \\
                        &= 16k^2 +8k - 2
                \end{align*}
                The number $n^2-3$ is not divisible by 4.
                \item[Case b]:
                \begin{align*}
                    n &= 4k+3 \\
                    n^2 - 3 &= (4k + 3)^2 - 3 \\
                     &= 16k^2 + 24k + 9 - 3 \\
                     &= 16k^2 + 24k + 6
                \end{align*}
                The number $n^2-3$ is not divisible by 4.
                Therefore, if $n$ is odd, $n^2-3$ is not divisible by 4.
            \end{itemize}
        \end{itemize}
        Therefore, we have shown in all cases that for any integer $n \geq 2$, $n^2 - 3$ is not divisible by 4.
\end{solution}

\end{questions}
\end{document}