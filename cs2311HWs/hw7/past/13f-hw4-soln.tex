\documentclass[11pt,addpoints]{exam}
% can include option [answers] to print out solutions, or command \printanswers
%  can turn addpoint on and off with commands, \addpoints and \noaddpoints

\usepackage{amsthm}
\usepackage{amssymb}
\usepackage{amsmath}
\usepackage{epsfig,graphicx}
\usepackage{color}
\usepackage{enumitem}
\usepackage[top=0.75in,bottom=0.75in,left=0.75in,right=0.75in]{geometry}

\setlength{\itemsep}{0pt} \setlength{\topsep}{0pt}
\newcommand{\ra}{\rightarrow}
\newcommand{\lra}{\leftrightarrow}
\newcommand{\xor}{\oplus}
\newcommand{\es}{\emptyset}
\newcommand{\s}{\subseteq}
\newcommand{\pss}{\subset}
\newcommand{\N}{\mathbb{N}}
\newcommand{\Z}{\mathbb{Z}}
\newcommand{\Zp}{\mathbb{Z}^+}
\newcommand{\Zn}{\mathbb{Z}^-}
\newcommand{\Q}{\mathbb{Q}}
\newcommand{\R}{\mathbb{R}}



\begin{document}

\extrawidth{0.5in} \extrafootheight{-0in} \pagestyle{headandfoot}
\headrule \header{\textbf{cs2311 - Fall 2013}}{\textbf{HW
 4 - \numpoints\; points \ifprintanswers Solutions \fi}}{\textbf{Due: Wed.
10/9/13}} \footrule \footer{}{Page \thepage\
of \numpages}{}

\noindent \textbf{Instructions:} All assignments are due \underline{by 5pm on the due date} specified.  There will be a box in the CS department office (Rekhi 221) where assignments may be turned in.  Solutions will be handed out (or posted on-line) shortly thereafter.  Every student
must write up their own solutions in their own manner.

\smallskip
\noindent Please present your solutions in a clean, understandable
manner; pages should be stapled before class, no ragged edges of
paper.

\hrulefill

\begin{questions}
\printanswers


\question[4] Determine which of the statements are propositions? What are the truth values of those that are propositions?
    % \begin{enumerate}[label=(\alph*),itemsep=0pt,parsep=0pt,topsep=0pt,partopsep=0pt]
    %     \item 7 is even.
    %     \item Who is Heisenberg? 
    %     \item $x + 2 = 3$
    %     \item 4 + 4 = 8.
    % \end{enumerate}
    \begin{tabular}{ll}
        (a) 7 is even.  \hspace{1in}    & (b) Who is Heisenberg? \\
        (c) $x + 2 = 3.$                & 4 + 4 = 8.
    \end{tabular}
    \ifprintanswers
        \vspace{-5pt}
    \fi
    \begin{solution}
        \begin{enumerate}[label=(\alph*),itemsep=0pt,parsep=0pt,
        topsep=0pt,partopsep=0pt]
        	\item 7 is even. \hspace{1.6in} proposition, $F$
        	\item Who is Heisenberg?. \hspace{0.80in} not a proposition, question
        	\item $x + 2 = 3$ \hspace{1.6in} not a proposition, depends on $x$
        	\item 4 + 4 = 8. \hspace{1.5in} proposition, $T$
        \end{enumerate}
    \end{solution}


\question[8] Rosen Ch 1.1, \#8 (a,c,d,g), p. 13 
	\ifprintanswers
        \vspace{-15pt}
    \fi
    \begin{solution}
    	\begin{itemize}[itemsep=0pt,parsep=0pt,
    	topsep=0pt,partopsep=0pt]
    		\item[(a)] I did not buy a lottery ticket this week.
    		\item[(c)] If I bought a lottery ticket this week, then I won the million dollar jackpot on Friday. 
    		\item[(d)] I bought a lottery ticket this week and I won the million dollar jackopot on Friday.
    		\item[(g)] I did not buy a lottery ticket this week, and I did not win the million dollar jackpot on Friday. 
    	\end{itemize}
    \end{solution}


\question[6] Rosen Ch 1.1 \#14 (b,c,e), p. 13-14
    \ifprintanswers
        \vspace{-15pt}
    \fi
    \begin{solution}
    	(b) $p \wedge q \wedge r$ \hfill (c) $r \ra p$ \hfill (e) $(p \wedge q) \ra r$ \hfill
    \end{solution}
    


\question[12] Rosen Ch 1.1, \#22 (a,d) \#24 (g,h), p. 13-14.
    \ifprintanswers
        \vspace{-15pt}
    \fi
    \begin{solution}
    \begin{itemize}[itemsep=0pt,parsep=0pt,
    topsep=0pt,partopsep=0pt]
        \item[22(a)] If you get promoted, then you wash the boss's car.
        \item[22(d)] If Willy cheats, then he gets caught.
        \item[24(g)] If you log on to the server, then you have a valid password.
        \item[24(h)] If you do not begin your climb too late, then you will reach the summit.
    \end{itemize}
    \end{solution}



\question[12] State the converse, contrapositive, and inverse of each of the following statements.
	\begin{enumerate}[label=(\alph*),itemsep=0pt,parsep=0pt,topsep=0pt,partopsep=0pt]
        \item If it is summer, then I take a vacation.
        \item To pass the course it is necessary that you get a high grade on the final exam.
    \end{enumerate}
    \ifprintanswers
        \vspace{-15pt}
    \fi
    \begin{solution}
    \begin{enumerate}[label=(\alph*),itemsep=0pt,parsep=0pt,topsep=0pt,partopsep=0pt]
        \item Converse: If I take a vaction, then it is summer. \\
        	Contrapositive: If I do not take a vacation, then it is not summer. \\
        	Inverse: If it is not summer, then I do not take a vacation. 
        \item If you pass the course, then you get a high grade on the final exam. \\
        	Converse: If you get a high grade on the final exam, then you pass the course. \\
        	Contrapositve: If you do not get a high grade on the final exam, then you do not pass the course. \\
        	Inverse: If you do not pass the course, then you do not get a high grade on the final exam.
    \end{enumerate}
    \end{solution}


\question[32]\label{tt} Construct a truth table for each compound proposition.
    \ifprintanswers
        \vspace{-10pt}
    \fi
    \footnotesize
    \begin{center}
    \begin{tabular}{ll}
       (a) (2 pt) $p \ra \neg p$  
       & (b) (2 pt) $\neg p \lra p$ \\
       (c) (6 pt) $(p \vee q) \wedge \neg p$ 
       & (d) (6 pt) $(\neg q \ra p) \xor (p \wedge \neg q)$ \\
       (e) (6 pt) $(p \vee q) \wedge (\neg p \vee (q \ra p))$ 
       & (f) (8 pt) $(p \lra \neg r) \ra (q \vee \neg (p \vee \neg r))$
    \end{tabular}
    \end{center}
    \normalsize
    \ifprintanswers
        \vspace{-15pt}
    \fi
    \begin{solution}
    \scriptsize
        % $p \ra \neg p$
	   \begin{tabular}{c|c||c}
            \multicolumn{3}{l}{ (a) } \\
            $p$ & $\neg p$ & $p \ra \neg p$ \\
         \hline
            T & F & F \\
            F & T & T \\
         \end{tabular} \hspace{0.5in}
         % $\neg p \lra p$ 
         \begin{tabular}{c|c||c}
            \multicolumn{3}{l}{ (b) } \\
            $p$ & $\neg p$ & $\neg p \lra p$ \\
         \hline
            T & F & F \\
            F & T & F \\
         \end{tabular} \hspace{0.5in}
        % $(p \vee q) \wedge \neg p$
        \begin{tabular}{c|c|c||c|c}
            \multicolumn{5}{l}{ (c)  $(p \vee q) \wedge \neg p$} \\
            $p$ & $q$ & $\neg p$ & $p \vee q$ & $(p \vee q) \wedge \neg p$ \\
         \hline
            T & T & F & T & F  \\
            T & F & F & T & F  \\
            F & T & T & T & T  \\
            F & F & T & F & F  \\
        \end{tabular} 

        \smallskip
        % $(\neg q \ra p) \xor (p \wedge \neg q)$
        \begin{tabular}{c|c|c||c|c|c} 
            \multicolumn{6}{l}{(d) $(\neg q \ra p) \xor (p \wedge \neg q)$ } \\
            $p$ & $q$ & $\neg q$ & $\neg q \ra p$ & $p \wedge \neg q$ & (d)  \\
         \hline
            T & T & F & T & F  & T   \\
            T & F & T & T & T  & F  \\
            F & T & F & T & F  & T  \\
            F & F & T & F & F  & F  \\
        \end{tabular} \hspace{0.5in}
        % $(p \vee q) \wedge (\neg p \vee (q \ra p))$ 
        \begin{tabular}{c|c|c||c|c|c|c} 
            \multicolumn{7}{l}{(e) $(p \vee q) \wedge (\neg p \vee (q \ra p))$} \\
            $p$ & $q$ & $\neg p$ & $p \vee q$ & $q \ra p$ & $(\neg p \vee (q \ra p))$ & (e) \\
         \hline
            T & T & F & T & T  & T  & T \\
            T & F & F & T & T  & T  & T \\
            F & T & T & T & F  & T  & T \\
            F & F & T & F & T  & T  & F \\
        \end{tabular}

        \smallskip
        % $(p \lra \neg r) \ra (q \vee \neg (p \vee \neg r))$
        \begin{tabular}{c|c|c|c||c|c|c|c|c}
            \multicolumn{9}{l}{(f) $(p \lra \neg r) \ra (q \vee \neg (p \vee \neg r))$ } \\
            $p$ & $q$ & $r$ & $\neg r$ 
             & $p \vee \neg r$ & $\neg (p \vee \neg r)$ 
             & $(q \vee \neg (p \vee \neg r))$ & $p \lra \neg r$ & (f) \\
         \hline
            T & T & T & F  & T & F & T  & F & T \\
            T & T & F & T  & T & F & T  & T & T \\
            T & F & T & F  & T & F & F  & F & T \\
            T & F & F & T  & T & F & F  & T & F \\
          \hline
            F & T & T & F  & F & T & T  & T & T \\
            F & T & F & T  & T & F & T  & F & T \\
            F & F & T & F  & F & T & T  & T & T \\
            F & F & F & T  & T & F & F  & F & T
         \end{tabular} 
    \end{solution}


\question[4] List those expressions from problem \ref{tt} that are tautologies, contradictions, and contingencies?
    \ifprintanswers
        \vspace{-15pt}
    \fi
    \begin{solution}
       Tautology:  none;
       Contradiction: (b);
       Contingency:  (a), (c), (d), (e), (f)
    \end{solution}

% XXX Check solutions
\question[10] There are 16 possible truth tables for propositions of two variables $p$ and $q$.
 All sixteen possibilities are given in the table below (numbered 15, 14, \ldots, 0).
 For example, the proposition $p \vee q$ is 14. What are the numbers of the truth
 tables for propositions:
     \ifprintanswers
        \vspace{-10pt}
    \fi
\begin{center}
 \begin{tabular}{ll}
    (a) $\neg p \vee q$ \quad 
    &  (b) $q \vee (\neg p \wedge p)$ \\
    (c) $p \ra (q \ra \neg p) $ \quad \quad
    & (d) $(\neg p \ra q) \vee (\neg q \wedge p)$ \\
    (e) $(p \oplus \neg q) \wedge (\neg p)$ \\
 \end{tabular}

 \footnotesize
 \begin{tabular}{cc|cccc|cccc|cccc|cccc}
    $p$ & $q$ & 15 & 14 & 13 & 12 & 11 & 10 & 9 & 8 & 7 & 6 & 5 & 4 & 3 & 2 & 1 & 0 \\
    \hline
    T & T & T & T & T & T & T & T & T & T & F & F & F & F & F & F & F & F \\
    T & F & T & T & T & T & F & F & F & F & T & T & T & T & F & F & F & F \\
    F & T & T & T & F & F & T & T & F & F & T & T & F & F & T & T & F & F \\
    F & F & T & F & T & F & T & F & T & F & T & F & T & F & T & F & T & F \\
 \end{tabular}
 \end{center}
 \normalsize
     \ifprintanswers
        \vspace{-15pt}
    \fi
    \begin{solution} (a) 11, (b) 10 (c) 7, (d) 14, (e) 1
    \end{solution}


% Ch 1.3, \# 10c, p. 35
\question[8] Prove the following statement is a tautology without using truth tables (use the logical equivalences from Table 6-8 of the book).  Justify each step with the law used.  Model the solutions in the style of Examples 6-8, pp. 29-30 of the book.
\[ [ p \wedge (\neg p \vee q) ] \ra q \]
    \ifprintanswers
        \vspace{-30pt}
    \fi
\begin{solution}  There are many possible solutions; two are given.
\small
    \begin{align*}
        [p \wedge (\neg p \vee q) ] \ra q \\
        & \equiv \neg [ p \wedge (\neg p \vee q) ] \vee q \tag{Table 7, rule 1} \\
        & \equiv \neg p \vee \neg (\neg p \vee q) \vee q \tag{DeMorgan's law} \\
        & \equiv \neg p \vee q \vee \neg (\neg p \vee q)   \tag{Commutative} \\
        & \equiv (\neg p \vee q) \vee \neg (\neg p \vee q) \tag{algebra} \\
        & \equiv \mathbf{T} \tag{Negation} 
    \end{align*}

    \begin{align*}
        [ p \wedge (\neg p \vee q) ] \ra q \\
        & \equiv [ (p \wedge \neg p) \vee (p \wedge q) ] \ra q \tag{Distributive} \\
        & \equiv [ \mathbf{F} \vee (p \wedge q) ] \ra q \tag{Negation} \\
        & \equiv [ (p \wedge q) \vee \mathbf{F} ] \ra q \tag{Commutative} \\
        & \equiv (p \wedge q) \ra q \tag{Identity} \\
        & \equiv \neg (p \wedge q) \vee q \tag{Table 7, rule 1} \\
        & \equiv (\neg p \vee \neg q) \vee q \tag{DeMorgans} \\
        & \equiv \neg p \vee (\neg q \vee q) \tag{Associative} \\
        & \equiv \neg p \vee (q \vee \neg q) \tag{Commutative} \\
        & \equiv \neg p \vee \mathbf{T} \tag{Negation} \\
        & \equiv \mathbf{T} \tag{Domination} 
    \end{align*}
\end{solution}

% \question[10] Prove the following logical equivalence using the
% equivalence laws from Table 6-8 (model the solutions in the style of
% book Examples 6-8, pp. 29-30).  Justify each step with laws.
% %\begin{enumerate}[label=(\alph*),itemsep=0pt,parsep=0pt,topsep=0pt,partopsep=0pt]
% %    \item $(p \wedge \neg(q \vee \neg p)) \equiv p \wedge \neg q$
% %    \item $\neg [ r \vee (q \wedge (\neg r \ra \neg p))] \equiv \neg r \wedge (p \vee \neg q) $
% %\end{enumerate}
% $$\neg [ r \vee (q \wedge (\neg r \ra \neg p))] \equiv \neg r \wedge (p \vee \neg q) $$
%     \begin{solution}

% %    (a)
% %    \begin{align*}
% %        p \ra (q \vee r) &\equiv (p \wedge \neg q) \ra r \\
% %            & \equiv \neg p \vee (q \vee r) \tag{Table 7, rule 1} \\
% %            & \equiv (\neg p \vee q) \vee r) \tag{Associative} \\
% %            & \equiv \neg (\neg p \vee q) \ra r \tag{Table 7, rule 1} \\
% %            & \equiv (\neg \neg p \vee \neg q) \ra r \tag{DeMorgan's} \\
% %            & \equiv (p \vee \neg q) \ra r \tag{Double Negation}
% %    \end{align*}
% %
% %    (b)
%     \begin{align*}
%       \neg [ r \vee (q & \wedge (\neg r \ra \neg p))] \equiv \neg r \wedge (p \vee \neg q) \\
%         & \equiv \neg [r \vee (q \wedge (p \ra r))] \tag{Table 7, rule 2} \\
%         & \equiv \neg [r \vee (q \wedge (\neg p \vee r))] \tag{Table 7, rule 1} \\
%         & \equiv \neg [r \vee (q \wedge \neg p) \vee (q \wedge r)] \tag{Distributive} \\
%         & \equiv \neg [r \vee (q \wedge r) \vee (q \wedge \neg p)] \tag{Commutative} \\
%         & \equiv \neg [r \vee (q \wedge \neg p)] \tag{Absorption} \\
%         & \equiv \neg [r \vee (\neg p \wedge q)] \tag{Commutative} \\
%         & \equiv \neg r \wedge \neg (\neg p \wedge q) \tag{DeMorgan's} \\
%         & \equiv \neg r \wedge (\neg \neg p \vee \neg q) \tag{DeMorgan's} \\
%         & \equiv \neg r \wedge (p \vee \neg q) \tag{Double Negation}
%     \end{align*}
%     \end{solution}

\end{questions}

\end{document}