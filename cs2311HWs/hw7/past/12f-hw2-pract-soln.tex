\documentclass[12pt,addpoints]{exam}
% can include option [answers] to print out solutions, or command \printanswers
%  can turn addpoint on and off with commands, \addpoints and \noaddpoints

\usepackage{amsthm}
\usepackage{amssymb}
\usepackage{amsmath}
\usepackage{color}
\usepackage{enumitem}
\usepackage{multicol}
\usepackage[top=0.75in,bottom=0.75in,left=0.9in,right=0.9in]{geometry}

\setlength{\itemsep}{0pt} \setlength{\topsep}{0pt}
\newcommand{\ra}{\rightarrow}
\newcommand{\lra}{\leftrightarrow}
\newcommand{\xor}{\oplus}

\begin{document}
\extrawidth{0.5in} \extrafootheight{-0in} \pagestyle{headandfoot}
\headrule 
\header{\textbf{cs2311 - Fall 2012}}{\textbf{HW
2 - Practice Problems}}{\textbf{Solutions}} \footrule 
\footer{}{Page \thepage\ of \numpages}{}

\noindent \textbf{Instructions:} The following practice problems are
not due and are not graded.  The solutions will be provided to allow
for extra practice.


\begin{questions}
\printanswers

\question Rosen Ch 1.4 \#8 (b,c,d) p. 53
    \ifprintanswers
        \vspace{-12pt}
    \fi
\begin{solution}
    \begin{itemize}[itemsep=0pt,parsep=0pt,topsep=0pt,partopsep=0pt]
%        \item[(a)] ``For every animal, if it is a rabbit then then that animal hops."  \\Or, ``Every rabbit hops."
        \item[(b)] ``For every animal, it is a rabbit and it hops." \\Or, ``Every animal is a rabbit and hops."
        \item[(c)] ``There exists an animal such that if it is a rabbit then it hops."
        \item[(d)] ``There exists an animal that is a rabbit and it hops." \\Or, ``Some rabbits hop." \\Or, ``Some hopping animals are rabbits."
    \end{itemize}
\end{solution}


\question Rosen Ch 1.4 \#10(a,c,d) p.53
    \ifprintanswers
        \vspace{-12pt}
    \fi
\begin{solution}
%    \begin{itemize}[itemsep=0pt,parsep=0pt,topsep=0pt,partopsep=0pt]
%        \item[(a)] $\exists x\; (C(x) \wedge D(x) \wedge F(x))$
%%        \item[(b)] $\forall x\; (C(x) \vee D(x) \vee F(x))$
%        \item[(c)] $\exists x\; (C(x) \wedge F(x) \wedge \neg D(x))$
%        \item[(d)] $\neg \exists x\; (C(x) \wedge D(x) \wedge F(x))$
%%        \item[(e)] $(\exists x\; C(x)) \wedge (\exists y\; D(y)) \wedge (\exists z\; F(z))$ \\Or, can use the same variable $(\exists x\; C(x)) \wedge (\exists x\; D(x)) \wedge (\exists x\; F(x))$
%    \end{itemize}
    \begin{tabular}{ll}
    	(a) $\exists x\; (C(x) \wedge D(x) \wedge F(x))$ 
    	 & (c) $\exists x\; (C(x) \wedge F(x) \wedge \neg D(x))$ \\
    	(d) $\neg \exists x\; (C(x) \wedge D(x) \wedge F(x))$
	\end{tabular}
\end{solution}


\question Rosen Ch 1.4 \#12(a,c,f) p. 53
    \ifprintanswers
        \vspace{-12pt}
    \fi
\begin{solution}
%    \begin{itemize}[itemsep=0pt,parsep=0pt,topsep=0pt,partopsep=0pt]
%        \item[(a)] T, since $0 + 1 > 2\cdot 0$.
%%        \item[(b)] T, since $-1 + 1 > 2\cdot -1$.
%        \item[(c)] F, since $1 + 1 \ngtr 2\cdot 1$.
%%        \item[(d)] T, we know there is at least one $x$ that makes $Q(x)$ true.
%%        \item[(e)] F, we know there is at least one $x$ that makes $Q(x)$ false.
%        \item[(f)] T, we know there is at least one $x$ that makes $Q(x)$ false.
%%        \item[(g)] F, we know there is at least one $x$ that makes $Q(x)$ true.
%    \end{itemize}
    \begin{tabular}{ll}
    	(a) T, since $0 + 1 > 2\cdot 0$. 
    	 & (c) F, since $1 + 1 \ngtr 2\cdot 1$. \\
    	\multicolumn{2}{l}{(f) T, we know there is at least one $x$ that makes $Q(x)$ false.}
    \end{tabular}
\end{solution}


\question Rosen Ch 1.4 \#18(c,d,f) p. 53
    \ifprintanswers
        \vspace{-12pt}
    \fi
\begin{solution}
    \begin{itemize}[itemsep=0pt,parsep=0pt,topsep=0pt,partopsep=0pt]
%        \item[(a)] $P(-2) \vee P(-1) \vee P(0) \vee P(1) \vee P(2)$
%        \item[(b)] $P(-2) \wedge P(-1) \wedge P(0) \wedge P(1) \wedge P(2)$
        \item[(c)] $\neg P(-2) \vee \neg P(-1) \vee \neg P(0) \vee \neg P(1) \vee \neg P(2)$
        \item[(d)] $\neg P(-2) \wedge \neg P(-1) \wedge \neg P(0) \wedge \neg P(1) \wedge \neg P(2)$
%        \item[(e)] $\neg (P(-2) \vee P(-1) \vee P(0) \vee P(1) \vee P(2))$
        \item[(f)] $\neg (P(-2) \wedge P(-1) \wedge P(0) \wedge P(1) \wedge P(2))$
    \end{itemize}
\end{solution}


\question Rosen Ch 1.4 \# 24(a,d,e), p. 54 \\
Use the following predicates:
\begin{itemize}[itemsep=0pt,parsep=0pt,topsep=0pt,partopsep=0pt]
    \item $C(x)$ be ``$x$ is in your class"
    \item $P(x)$ be ``$x$ has a cell phone"
    \item $M(x)$ be ``$x$ has seen a foreign movie"
    \item $S(x)$ be ``$x$ can swim"
\end{itemize}
    \ifprintanswers
        \vspace{-12pt}
    \fi
\begin{solution}
    \begin{itemize}[itemsep=0pt,parsep=0pt,topsep=0pt,partopsep=0pt]
        \item[(a)] Everyone in your class has a cell phone.
        $$ \forall x\; P(x) \hspace{2in} \forall x\; (C(x) \ra P(x))$$
%        \item[(b)] Somebody in your class has seen a foreign movie.
%        $$ \exists x\; M(x) \hspace{2in} \exists x\; (C(x) \wedge F(x))$$
%        \item[(c)] There is a person in your class who cannot swim.
%        $$ \exists x\; \neg S(x) \hspace{2in} \exists x\; (C(x) \wedge \neg S(x))$$
        \item[(d)] All students in your class can solve quadratic equations
        $$ \forall x\; Q(x) \hspace{2in} \forall x\; (C(x) \ra Q(x))$$
        \item[(e)] Some student in your class does not want to be rich.
        $$ \exists x\; \neg R(x) \hspace{2in} \exists x\; (C(x) \wedge \neg R(x))$$
    \end{itemize}
\end{solution}


\question Rosen Ch 1.4 \#28(c,d,e), p. 54 \\
Use the following predicates:
\begin{itemize}[itemsep=0pt,parsep=0pt,topsep=0pt,partopsep=0pt]
    \item $C(x)$ be ``$x$ is in the correct place"
    \item $T(x)$ be ``$x$ is a tool"
    \item $E(x)$ be ``$x$ is in excellent condition"
\end{itemize}
    \ifprintanswers
        \vspace{-12pt}
    \fi
\begin{solution}
    \begin{itemize}[itemsep=0pt,parsep=0pt,topsep=0pt,partopsep=0pt]
%        \item[(a)] $\exists x\; \neg C(x) $
%        \item[(b)] $\forall x\; (T(x) \ra (C(x) \wedge E(x))) $
        \item[(c)] $\forall x\; (C(x) \wedge E(x)) $
        \item[(d)] $\neg \exists x\; (C(x) \wedge E(x)) \equiv \forall x\; \neg (C(x) \wedge E(x)) \equiv \forall x\; (\neg C(x) \vee \neg E(x))$
        \item[(e)] $\exists x\; (T(x) \wedge \neg C(x) \wedge E(x))$
    \end{itemize}
\end{solution}


\question Rosen Ch 1.5 \#4(b,d,e), p. 64
    \ifprintanswers
        \vspace{-12pt}
    \fi
\begin{solution}
    \begin{itemize}[itemsep=0pt,parsep=0pt,topsep=0pt,partopsep=0pt]
%        \item[(a)] There is a student in your class who has taken some computer science class at your school.
       \item[(b)] There is a student in your class who has taken every computer science class at your school.
%        \item[(c)] Every student in your class has taken a computer science class at your school.
        \item[(d)] There is a computer science class at your school that every student in your class has taken.
        \item[(e)] Every computer science class at your school has been taken by a student in your class.
%        \item[(f)] Every student in your class has taken every computer science class at your school.
    \end{itemize}
\end{solution}


\question Rosen Ch 1.5 \#10(c,f), p. 65
    \ifprintanswers
        \vspace{-12pt}
    \fi
\begin{solution}
    \begin{itemize}[itemsep=0pt,parsep=0pt,topsep=0pt,partopsep=0pt]
%        \item[(a)] $\forall x\; F(x,Fred)$
%        \item[(b)] $\forall y\; F(Evelyn, y)$
        \item[(c)] $\forall x\; \exists y\; F(x,y)$
%        \item[(d)] $\neg \exists x\; \forall y\; F(x,y)$
%        \item[(e)] $ \forall y\; \exists x\; F(x,y)$
        \item[(f)] $\neg \exists x\; (F(x, Fred) \wedge F(x,Jerry))$
%        \item[(g)] $\exists x\; \exists y; (F(Nancy, x) \wedge F(Nancy, y) \wedge x \neq y \wedge \forall z\; (F(Nancy,z) \ra (x = z \vee y = z)))$
%        \item[(h)] $\exists y\; (\forall x\; F(x,y) \wedge \forall z\; (\forall x\; F(x,z) \ra z = y))$
%        \item[(i)] $\neg \exists x F(x,x)$
%        \item[(j)] $\exists x\; \exists y\; (x \neq y \wedge F(x,y) \wedge \forall z\; ((F(x,z) \wedge z \neq x) \ra z = y))$
    \end{itemize}
\end{solution}


\question Rosen Ch 1.5 \#20(a,d), p. 67 \\
Statements about whether a number $x$ is positive and negative can be made as $(x > 0)$ and $(x < 0)$ respectively and the absolute value of $x$ is $|x|$.  Use this notation throughout this problem.
    \ifprintanswers
        \vspace{-12pt}
    \fi
\begin{solution}
    \begin{itemize}[itemsep=0pt,parsep=0pt,topsep=0pt,partopsep=0pt]
        \item[(a)] $\forall x\; \forall y\; (((x < 0) \wedge (y < 0)) \ra (xy > 0))$
        \item[(d)] $\forall x\; \forall y\; (|x + y| \leq |x| + |y|)$
    \end{itemize}
\end{solution}


\question Rosen Ch 1.5 \#25(a,b), p. 67
    \ifprintanswers
        \vspace{-10pt}
    \fi
\begin{solution}
    \begin{itemize}[itemsep=0pt,parsep=0pt,topsep=0pt,partopsep=0pt]
        \item[(a)] There exists a real number such that it can be added to any real number $y$ such that it equals $y$.
        \item[(b)] For all real numbers, if $x$ is positive and $y$ is negative then the difference of $x$ and $y$ is positive.
    \end{itemize}
\end{solution}


\question Rosen Ch 1.5 \#26(d,f,g), p. 67
    \ifprintanswers
        \vspace{-10pt}
    \fi
\begin{solution}
    \begin{itemize}[itemsep=0pt,parsep=0pt,topsep=0pt,partopsep=0pt]
%        \item[(a)] F, since $1 + 1 \neq 1 - 1$.
%        \item[(b)] T, since $2 + 0 = 2 - 0$.
%        \item[(c)] F, since many values of $y$ for which $1 + y \neq 1-y$.
        \item[(d)] F, since no solution to $x + 2 = x - 2$.
%        \item[(e)] True, let $x = y = 0$.
        \item[(f)] True, let $y=0$ for each $x$.
        \item[(g)] True, let $y = 0$.
%        \item[(h)] False, same as $d$.
%        \item[(i)] False.
    \end{itemize}
\end{solution}


\question Rosen Ch 1.5, \#30(a,c), p. 67
    \ifprintanswers
        \vspace{-10pt}
    \fi
\begin{solution}
    \begin{itemize}[itemsep=0pt,parsep=0pt,topsep=0pt,partopsep=0pt]
        \item[(a)]
        \vspace{-25pt}
        \begin{align*}
          \neg \exists y\; \exists x\; P(x,y) & \equiv \forall y\; \neg \exists x\; P(x,y) \\
          & \equiv \forall y\; \forall x\; \neg P(x,y)
        \end{align*}
%        \item[(b)]
%        \begin{align*}
%          \neg \forall x\; \exists y\; P(x,y) & \equiv \exists x\; \neg \exists y\; P(x,y) \\
%          & \equiv \exists x\; \forall y\; \neg P(x,y)
%        \end{align*}
        \item[(c)]
        \vspace{-25pt}
        \begin{align*}
          \neg \exists y\; (Q(y) \wedge \forall x\; \neg R(x,y)) & \equiv \forall y\; \neg(Q(y) \wedge \forall x\; \neg R(x,y)) \\
          & \equiv \forall y\; (\neg Q(y) \vee \neg \forall x\; \neg R(x,y)) \\
          & \equiv \forall y\; (\neg Q(y) \vee \exists x\; \neg \neg R(x,y))\\
          & \equiv \forall y\; (\neg Q(y) \vee \exists x\; R(x,y))
        \end{align*}
%        \item[(d)]
%        \begin{align*}
%          \neg \exists y\; (\exists x\; R(x,y) \vee \forall x\; S(x,y)) & \equiv \forall y\; \neg (\exists x\; R(x,y) \vee \forall x\; S(x,y)) \\
%          & \equiv \forall y\; (\neg \exists x\; R(x,y) \wedge \neg \forall x\; S(x,y)) \\
%          & \equiv \forall y\; (\forall x\; \neg R(x,y) \wedge \exists x\; \neg S(x,y))
%        \end{align*}
%        \item[(e)] \small
%        \vspace{-15pt}
%        \begin{align*}
%          \neg \exists y\; (\forall x\; \exists z\; T(x,y,z) \vee \exists x\; \forall z\; U(x,y,z)) & \equiv \forall y\; \neg (\forall x; \exists z\; T(x,y,z) \vee \exists x\; \forall z\; U(x,y,z)) \\
%          & \equiv \forall y\; (\neg \forall x; \exists z\; T(x,y,z) \wedge \neg \exists x\; \forall z\; U(x,y,z)) \\
%          & \equiv \forall y\; (\exists x\; \neg \exists z\; T(x,y,z) \wedge \forall x\; \neg \forall z\; U(x,y,z)) \\
%          & \equiv \forall y\; (\exists x\; \forall z\; \neg T(x,y,z) \wedge \forall x\; \exists z\; \neg U(x,y,z))
%        \end{align*}
        \end{itemize}
\end{solution}


\question Rosen Ch 1.6, \#10(e,f), p.79 
    \ifprintanswers
        \vspace{-10pt}
    \fi
\begin{solution}
    \begin{itemize}[itemsep=0pt,parsep=0pt,topsep=0pt,partopsep=0pt]
%        \item[(a)] Use modus tollens to conclude that ``I am not sore."  With this, you can use modus tollens again to conclude that ``I did not play hockey."
%        \item[(b)] There is nothing to conclude here.
        \item[(e)]  By Universal instantiation and modus ponens, ``Tofu does not taste good." The fourth hypothesis follows from the first three by modus tollens.  No conclusions can be drawn about cheeseburgers.
        \item[(f)] The conclusion ``I am hallucinating," follows from the first two hypotheses using disjunctive syllogism.  Then, ``I see elephants running down the road," follows by modus ponens.
    \end{itemize}
\end{solution}
    

\question Rosen Ch 1.6 \#6, p. 78.
    \ifprintanswers
        \vspace{-10pt}
    \fi
\begin{solution}
Let $r$ be the proposition ``It rains", let $f$ be ``It is foggy",
    let $s$ be ``The sailing race will be held", let $l$ be ``The life
    saving demonstration will go on", and let $t$ be ``The trophy will
    be awarded".

    Then, the premises are: $(\neg r \vee \neg f) \rightarrow (s \wedge l)$, $s \rightarrow t$, $\neg t$. \\
    The conclusion we want is: $r$.

    \begin{tabular}{lll}
        Step    & \hspace{0.2in} & Reason \\
        1. $\neg t$                 &       & Hypothesis \\
        2. $s \rightarrow t$        &       & Hypothesis \\
        3. $\neg s$                 &       & Modus tollens with (1) and (2) \\
        4. $(\neg r \vee \neg f) \rightarrow (s \wedge l)$  &   & Hypothesis \\
        5. $(\neg(s \wedge l)) \rightarrow \neg(\neg r \vee \neg f)$    & & Contrapositive \\
        6. $(\neg s \vee \neg l) \rightarrow (\neg \neg r \wedge \neg \neg f)$ & & De Morgans law with (5) (x2) \\
        7. $(\neg s \vee \neg l) \rightarrow (r \wedge f)$  & & Double negation with (6) (x2) \\
        8. $\neg s \vee \neg l$     &       & Addition with (3) \\
        9. $r \wedge f$             &       & Modus ponens with (7) and (8) \\
        10. $r$                     &       & Simplification with (9)
    \end{tabular}
\end{solution}


\question Rosen Ch 1.6 \#14 (a,b), p. 79
    \ifprintanswers
        \vspace{-10pt}
    \fi
\begin{solution}
    \begin{itemize}[itemsep=0pt,parsep=0pt,topsep=0pt,partopsep=0pt]
    \item[(a):] Let $c(x)$ be ``$x$ is in this class", $r(x)$ be ``$x$ owns a red convertible", $t(x)$ be ``$x$ has gotten a speeding ticket."  The premises are:
    \begin{itemize}[itemsep=0pt,parsep=0pt,topsep=0pt,partopsep=0pt]
        \item[1.] $c(Linda)$
        \item[2.] $r(Linda)$
        \item[3.] $\forall x\; (r(x) \rightarrow t(x))$
    \end{itemize}
    To conclude: $\exists x\; (c(x) \wedge t(x))$

    \smallskip
    \begin{tabular}{lll}
        Step        & \hspace{0.2in} & Reason \\
        1. $\forall x\; (r(x) \rightarrow t(x))$    &   & Hypothesis \\
        2. $r(Linda) \rightarrow t(Linda)$          &   & Universal Instantiation \\
        3. $r(Linda)$                               &   & Hypothesis \\
        4. $t(Linda)$                               &   & Modus ponens using (2) and (3) \\
        5. $c(Linda)$                               &   & Hypothesis \\
        6. $c(Linda) \wedge t(Linda)$               &   & Conjunction with (4) and (5) \\
        7. $\exists x\; (c(x) \wedge t(x))$         &   & Existential generalizaion \\
    \end{tabular}

    \item[(b):] Let $r(x)$ be ``$x$ is one of the five roommates", $d(x)$ be ``$x$ has taken a course in discrete math", and $a(x)$ be ``$x$ can take a course in algorithms." The premises are:
    \begin{itemize}[itemsep=0pt,parsep=0pt,topsep=0pt,partopsep=0pt]
        \item[1.] $\forall x\;(r(x) \rightarrow d(x))$
        \item[2.] $\forall x\;(d(x) \rightarrow a(x))$
    \end{itemize}
    The conclusion is: $\forall x\;(r(x) \rightarrow a(x))$.  Let $y$ be an arbitrary person.

    \smallskip
    \begin{tabular}{lll}
        Step        & \hspace{0.2in} & Reason \\
        1. $\forall x\;(r(x) \rightarrow d(x))$     &   & Hypothesis \\
        2. $r(y) \rightarrow d(y)$                  & & Univ. Inst. with (1) \\
        3. $\forall x\; (d(x) \rightarrow a(x))$    &   & Hypothesis \\
        4. $d(u) \rightarrow a(y)$                  &   & Univ. Inst. with (3) \\
        5. $r(y) \rightarrow a(y)$                  &   & Hyp. syl. with (2) and (4) \\
        6. $\forall x\;(r(x) \rightarrow a(x))$     &   & Univ. Generalization with (5) \\
    \end{tabular}
    \end{itemize}
\end{solution}


\question Rosen Ch 1.6 \# 16(a,b,c), p. 79
    \ifprintanswers
        \vspace{-10pt}
    \fi
\begin{solution}
    \begin{itemize}[itemsep=0pt,parsep=0pt,topsep=0pt,partopsep=0pt]
        \item[(a)] Correct, using universal instantiation and modus tollens.
        \item[(b)] Incorrect, after applying universal instantiation it uses the fallacy of denying the hypothesis.
        \item[(c)] Incorrect, after applying universal instantiation it uses the fallacy of denying the hypothesis.
%        \item[(d)] Correct, using universal instantiation and modus ponens.
    \end{itemize}
\end{solution}


\question Rosen Ch 1.6 \# 18, p. 79
    \ifprintanswers
        \vspace{-10pt}
    \fi
\begin{solution}
    It is true that there is some $s$ in the domain s.t. $S(s,Max)$; however, there is no guarantee that Max is the $s$.  This first step of the proof is invalid.
\end{solution}


\question Rosen Ch 1.6 \# 24, p. 79
    \ifprintanswers
        \vspace{-10pt}
    \fi
\begin{solution}
    The incorrect steps are 3 and 5.  The simplification rule can not be applied to disjunctions.  Also, step 7 the conjunction rule should ``AND" terms together.
\end{solution}


\question Rosen Ch 1.6 \# 26, p. 80
    \ifprintanswers
        \vspace{-10pt}
    \fi
\begin{solution}
    This is to show that the conditional statement is true for all elements in the domain, that is $\forall x\; P(x) \ra R(x)$.  Show if $P(x)$ is true for a particular $x$ then $R(x)$ is also true.  For an item in the domain $x$, universal modus ponens with the first hypothesis gives $Q(x)$.  Then, apply universal modus ponens with the second hypothesis to get $R(x)$.
\end{solution}

    
\end{questions}
\end{document}