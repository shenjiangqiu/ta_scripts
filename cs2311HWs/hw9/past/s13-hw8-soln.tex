\documentclass[10pt,addpoints]{exam}
% can include option [answers] to print out solutions, or command \printanswers
%  can turn addpoint on and off with commands, \addpoints and \noaddpoints

\usepackage{amsthm}
\usepackage{amssymb}
\usepackage{amsmath}
\usepackage{epsfig,graphicx}
\usepackage{color}
\usepackage{enumitem}
\usepackage[top=0.75in,bottom=0.75in,left=0.9in,right=0.9in]{geometry}

\setlength{\itemsep}{0pt} \setlength{\topsep}{0pt}
\newcommand{\ra}{\rightarrow}
\newcommand{\lra}{\leftrightarrow}
\newcommand{\xor}{\oplus}
\newcommand{\es}{\emptyset}
\newcommand{\s}{\subseteq}
\newcommand{\pss}{\subset}
\newcommand{\N}{\mathbf{N}}
\newcommand{\Z}{\mathbf{Z}}
\newcommand{\Zp}{\mathbb{Z}^+}
\newcommand{\Zn}{\mathbb{Z}^-}
\newcommand{\Q}{\mathbb{Q}}
\newcommand{\R}{\mathbb{R}}
\newcommand{\ds}{\displaystyle}

\begin{document}
\extrawidth{0.5in} \extrafootheight{-0in} \pagestyle{headandfoot}
\headrule \header{\textbf{cs2311 - Fall 2013}}{\textbf{HW
 8 - \numpoints$\;$ points \ifprintanswers - Solutions \fi}}{\textbf{Due: Fri. 11/8/13}} \footrule \footer{}{Page \thepage\
of \numpages}{}

\ifprintanswers
\else
\noindent \textbf{Instructions:} All assignments are due \underline{by \textbf{5pm} on the due date} specified.  There will be a box in the CS department office (Rekhi 221) where assignments may be turned in.  Solutions will be handed out (or posted on-line) shortly thereafter.  Every student
must write up their own solutions in their own manner.

\smallskip
\noindent Please present your solutions in a clean, understandable
manner; pages should be stapled before class, no ragged edges of
paper.
\fi

\begin{questions}
\printanswers


\question[10] Prove that $\displaystyle \sum_{j=1}^n 2j = n + n^2$ for the positive integers $n$.
    \ifprintanswers
        \vspace{-10pt}
   \fi
\begin{solution}
	\textit{Proof:}
	Let $P(n)$ be the statement $\sum_{j=1}^n 2j = n + n^2$ for $n \geq 1$.

	\smallskip
	\smallskip
	\begin{tabular}{lp{4in}}
	  \textit{Basis Step:} & Show $P(1)$ is true, $\sum_{j=1}^1 2j = 2\cdot 1 = 2 =  1 + 1^2 = n + n^2 $ \\
	   & \\
	 \textit{Inductive Step:} &  \\
	\end{tabular}

	Assume $P(k)$ is true for an arbitrary, fixed integer $k \geq 1$, that is,
	\begin{align*}
		\sum_{j=1}^k 2j = k + k^2 \tag{IH}
	\end{align*}

	Show $P(k+1)$ is true, that is, 
	\[ \sum_{j=1}^{k+1} 2j = (k+1) + (k+1)^2 = k+1 + k^2 + 2k + 1 = k^2 + 3k + 2 = (k+1)(k+2) \]

	Start with $P(k+1)$ 
	\begin{align*}
		\sum_{j=1}^{k+1} 2j = \sum_{j=1}^k 2j + 2(k+1) \\
		&= k + k^2 +  2(k+1) \tag{IH} \\
		&= k^2 + k + 2k + 2 \\
		&= k^2 + 3k + 2 = (k+2)(k+1) \\
		&= k+1 + k^2 + 2k + 1 = (k+1) + (k+1)^2 \tag{alternatively} 
	\end{align*}
	This shows $P(k+1)$ is true, assuming $P(k)$ is true, completing the inductive step.

	Therefore by mathematical induction, $P(n)$ is true for $n \geq 1$.
\end{solution}


% cs2311 - summer 2012 Assignment 4, Chuck
\question[12] The Fibonacci numbers are defined as $F(0) = 0$, $F(1) = 1$, and $F(n) = F(n-1) + F(n-2)$ for $n \geq 2$.   Use induction to prove the following for natural numbers $n$:
\[ F(0) + F(1) + \cdots + F(n) = F(n+2) - 1.  \]
    \ifprintanswers
        \vspace{-10pt}
   \fi
\begin{solution}
	\textit{Proof:}
	Let the $P(n)$ be the statement $F(0) + F(1) + \cdots + F(n) = F(n+2) -1$.

	\smallskip
	\begin{tabular}{lp{4in}}
	  \textit{Basis Step:} & Show $P(0)$ is true, $F(0) = 0 = 1 - 1 = F(0+2) - 1$ \\
	   & \\
	 \textit{Inductive Step:} &  \\
	\end{tabular}

	Assume $P(k)$ is true for an arbitrary, fixed integer $k \geq 0$, that is,
	\begin{align*}
		F(0) + F(1) + \cdots + F(k) &= F(k+2) - 1  \tag{IH} \\
		\sum_{i=0}^k F(i) &= F(k+2) - 1
	\end{align*}

	Show $P(k+1)$ is true, that is
	\begin{align*}
	 	F(0) + F(1) + \cdots + F(k) + F(k+1) &= F(k+1+2) - 1 = F(k+3) - 1 \\
	 	\sum_{i=0}^{k+1} F(k) &= F(k+3) - 1
	\end{align*}

	Start with $P(k+1)$:
	\begin{align*}
		F(0) + F(1) + \cdots + F(k) + F(k+1) &= F(k+3) - 1  \\
		F(k+2) - 1 + F(k+1) &= \tag{IH} \\
		F(k+2) + F(k+1) - 1 &= \\
		F(k+3) - 1
	\end{align*}
	This shows $P(k+1)$ is true, assuming $P(k)$ is true, completing the inductive step. 

	\smallskip
	Therefore, by mathematical induction, $P(n)$ is true for all $n \geq 0$.

	\smallskip
	Note, the inductive step could also be shown by, starting with $P(k)$ and adding $F(k+1)$ to both sides:
	\begin{align*}
		F(0) + F(1) + \cdots + F(k) &= F(k+2) - 1 \tag{IH} \\
		F(0) + F(1) + \cdots + F(k) + F(k+1) &= F(k+2) - 1 + F(k+1) \\
		 &= F(k+2) + F(k+1) - 1 \\
		 &= F(k+3) -1 
	\end{align*}
\end{solution}


% Rosen Ch 5.3, \# 12, p. 358.
\question[10]  Also, with the Fibonacci numbers, use induction to prove for the positive integers $n$: 
\[ F(1)^2 + F(2)^2 + \cdots + F(n)^2 = F(n)F(n+1). \]
    \ifprintanswers
        \vspace{-10pt}
   \fi
\begin{solution}
	\textit{Proof:}
	Let the $P(n)$ be the statement $F(1)^2 + F(2)^2 + \cdots + F(n)^2 = F(n)F(n+1)$.

	\smallskip
	\begin{tabular}{lp{4in}}
	  \textit{Basis Step:} & Show $P(1)$ is true, $F(1) = 1 = 1\cdot1 = F(1)F(1+1)$ \\
	   & \\
	 \textit{Inductive Step:} &  \\
	\end{tabular}

	Assume $P(k)$ is true for an arbitrary, fixed integer $k \geq 1$, that is,
	\begin{align*}
		F(1)^2 + F(2)^2 + \cdots + F(k)^2  = F(k)F(k+1)  \tag{IH} \\
		\sum_{i=1}^k F(i)^2 = F(k)F(k+1)
	\end{align*}

	Show $P(k+1)$ is true, that is
	\begin{align*}
	 	F(1)^2 + F(2)^2 + \cdots + F(k)^2 + F(k+1)^2 = F(k+1)F(k+2) \\
	 	\sum_{i=1}^{k+1} F(k)^2 = F(k+1)F(k+2) 
	\end{align*}

	Start with $P(k+1)$:
	\begin{align*}
		F(1)^2 + F(2)^2 + \cdots + F(k)^2 + F(k+1)^2 &= F(k+1)F(k+2) \\
		F(k)F(k+1) + F(k+1)^2 &= \tag{IH} \\
		F(k+1)\left[ F(k) + F(k+1) \right] &= \\
		F(k+1)F(k+2) 
	\end{align*}
	This shows $P(k+1)$ is true, assuming $P(k)$ is true, completing the inductive step. 

	Therefore, by mathematical induction, $P(n)$ is true for all $n \geq 1$.
\end{solution}



% Ferland, p. 188, Ex. 4.19
\question[10] Show for all integers $n \geq 4$, $n^2 \geq 3n + 4$. 
    \ifprintanswers
        \vspace{-10pt}
   \fi
\begin{solution}
	\textit{Proof:}
	Let $P(n)$ be $n^2 \geq 3n + 4$ for all integers $n \geq 4$.

	\smallskip
	\begin{tabular}{lp{4in}}
	  \textit{Basis Step:} & Show $P(4)$ is true, $4^2 = 16 \geq 16 = 3\cdot 4 + 4$ \\
	   & \\
	 \textit{Inductive Step:} &  \\
	\end{tabular}

	Assume $P(k)$ is true for an arbitrary, fixed integer $k \geq 4$, that is, 
	\begin{align*}
		k^2 \geq 3k + 4 \tag{IH} 
	\end{align*}

	Show $P(k+1)$ is true, that is, 
	\[ (k+1)^2 \geq 3(k+1) + 4 = 3k + 7 \]

	Start with \textit{lhs} of $P(k+1)$
	\begin{align*}
		(k+1)^2 &= k^2 + 2k + 1 \\
		 &\geq (3k + 4) + 2k + 1 \tag{IH} \\
		 &= 3k + (2k + 5) \\
		 &\geq 3k + 7 = 3(k+1) + 4
	\end{align*}
	This shows $P(k+1)$ is true, assuming $P(k)$ is true, completing the inductive step. 

	Therefore, by mathematical induction, $n^2 \geq 3n + 4$ for all $n \geq 4$.
\end{solution}



% Ferland, p. 191, Exercise #17.
\question[12] Let $\{s_n\}$ be the sequence defined as, 
\[ s_1 = 4 \quad \text{and} \quad s_n = 3s_{n-1} - 2, \forall n \geq 2. \]
Show $\forall n \geq 1, s_n = 3^n + 1.$
    \ifprintanswers
        \vspace{-10pt}
   \fi
\begin{solution}
	\textit{Proof:}
	Let $P(n)$ be that the $n$th term of the sequence is found as $s_n = 3^n + 1$ for $n \geq 1$

	\smallskip
	\begin{tabular}{lp{4in}}
	  \textit{Basis Step:} & Show $P(1)$ is true, $s_1 = 3^1 + 1 = 4$ \\
	   & \\
	 \textit{Inductive Step:} &  \\
	\end{tabular}

	Assume $P(k)$ is ture for an arbitrary fixed integer $k \geq 1$, that is, 
	\begin{align*}
		s_k &= 3^k + 1 \tag{IH} \\
	\end{align*}

	Show $P(k+1)$ is true, that is,
	\[ s_{k+1} = 3^{k+1} + 1 \]

	Start with $P(k+1)$ and the definition of the sequence, 
	\begin{align*}
		s_{k+1} &= 3s_{k} - 2 \\
		&= 3(3^k + 1) - 2 \tag{IH}\\
		&= 3^{k+1} + 3 - 2 = 3^{k+1} + 1
	\end{align*}
	This shows $P(k+1)$ is true, assuming $P(k)$, completing the inductive step.

	\smallskip
	Therefore, we have shown by mathematical induction, $P(n)$ is true for all $n \geq 1$.
\end{solution}


% Ferland p. 206, Example 4.28
\question[14] Consider the game of football (that is, the American game of football).  Let's assume teams can either score via field goal (3 points) or touchdowns (7 points, assume all point after touchdowns are made).   Safeties are ignored for this problem.  

Show that it is then possible (assuming no time constraints) for a team to score any number of points from 12 on up. 
    \ifprintanswers
        \vspace{-10pt}
   \fi
\begin{solution}
	\textit{Proof:}
	Let $P(n)$ be that it is possible for a team to score $n$ points, for $n \geq 12$.
	
	\smallskip
	\begin{tabular}{lp{4in}}
	  \textit{Basis Step:} 	& Show $P(12)$ is true, 4 field goals \\
	  						& Show $P(13)$ is true, 1 touchdown and 2 field goals, and \\
	  						& Show $P(14)$ is true, 2 touchdowns \\
	   & \\
	 \textit{Inductive Step:} &  \\
	\end{tabular}

	Assume $P(j)$ is true where $12 \leq j \leq k$ and an arbitrary, fixed integer $k \geq 14$, that is, a team can score $j$ points. 

	Show $P(k+1)$ is true, that is, a team can score $k+1$ points through field goals and touchdowns.

	We know a team can score $P(k-2)$ and $k -2 \geq 12$ from the inductive hypothesis. A team can score one additional field goal (3 points) to reach $k+1$ points, competing the inductive step. 

	Therefore, by mathematical induction, $P(n)$ for integers $n \geq 12$. 
\end{solution}



% Ferland, p. 207, Exercise #1.
\question[14] Let $\{s_n\}$ be the sequence defined as, 
\[ s_0 = 0, \quad s_1 = 1,  \quad \text{and} \quad s_n = 3s_{n-1} - 2s_{n-2}, \forall n \geq 2. \]
Show $\forall n \geq 0, s_n = 2^n - 1 .$
    \ifprintanswers
        \vspace{-10pt}
   \fi
\begin{solution}
	\textit{Proof:}
	Let $P(n)$ be that the $n$th term of the sequence is determine as $s_n = 2^n - 1$ for $n \geq 0$.

	\smallskip
	\begin{tabular}{lp{4in}}
	  \textit{Basis Step:} 	& Show $P(0)$ is true, $2^0 - 1 = 0 = s_0$ \\
	  						& Show $P(1)$ is true, $2^1 - 1 = 1 = s_1$ \\
	   & \\
	 \textit{Inductive Step:} &  \\
	\end{tabular}

	Assume $P(j)$ is true $0 \leq j \leq k$ with $k \geq 1$, that is, 
	\begin{align*}
		s_j = 2^j - 1 \quad \forall 0 \geq j \geq k \tag{IH}
	\end{align*}

	Show that $P(k+1)$ is true, that is,
	\[ s_{k+1} = 2^{k+1} - 1 \] 

	Consider the definition of the sequence, 
	\begin{align*}
		s_{k+1} &= 3s_{k} - 2s_{k-1} \\
		&= 3(2^k - 1) - 2(2^{k-1} - 1) \\
		&= 3\cdot 2^k - 3 - 2^k + 2 \\
		&= 2\cdot 2^k - 1 \\
		&= 2^{k+1} - 1
	\end{align*}
	When the inductive hypothesis is true, then $P(k+1)$ is true, completing the inductive step. 

	Then, by strong mathmatical induction, we conclude $P(n)$ for $n \geq 0$.
\end{solution}



\question[4] Rosen Ch 5.3 \# 6(a,b), p. 357. 
    \ifprintanswers
        \vspace{-10pt}
   \fi
\begin{solution}
	\begin{itemize}
		\item[(a)] Valid, $f(n) = (-1)^n$.  Informally, it is true with $n=0$, since $(-1)^0 = 1$. Assume it is true for $n=k$, then $f(k+1) = -f(k+1-1) = -f(k) - (-1)^k$ by the inductive hypothesiss, $f(k+1) = (-1)^{k+1}$.
		\item[(b)] Valid, $f(n) = (1 - (n \mod 3) \mod 2) 2^{\lfloor \frac{n}{3} \rfloor + \frac{n \mod 3}{2}}$.  Alternatively, break into cases: when $n \equiv 0 (\mod 3)$, $f(n) = 2^{\frac{n}{3}}$, when $n \equiv 1 (\mod 3)$, $f(n) = 0$, and when $n \equiv 2 (\mod 3)$, $f(n) = 2^{\frac{n+1}{3}}$.  Informally, prove the base cases: $f(0) = 1 = 2^{\frac{0}{3}}$, $f(1) = 0$, and $f(2) = 2 = 2^{\frac{2+1}{3}}$.  Assume the inductive hypothesis, then :
		\begin{itemize}
			\item[] when $n \equiv 0 (\mod 3)$, $f(n) = 2f(n-3) = 2\cdot 2^{\frac{n-3}{3}} = 2\cdot 2^{\frac{n}{3}} \cdot 2^{-1}$,
			\item[] when $n \equiv 1 (\mod 3)$, $f(n) = 2f(n-3) = 2\cdot 0 = 0$, 
			\item[] when $n \equiv 2 (\mod 3)$, $f(n) = 2f(n-3) = 2\cdot 2^{\frac{n-3+1}{3}} = 2\cdot 2^{\frac{n+1}{3}} \cdot 2^{-1} = 2^{\frac{n+1}{3}}$
		\end{itemize}
		as desired.
	\end{itemize}
\end{solution}



\question[6] Rosen Ch 5.3 \# 24(a,b), p. 358.
    \ifprintanswers
        \vspace{-10pt}
   \fi
\begin{solution}
	\begin{itemize}
		\item[(a)] Odd integers are obtained from other odds, by adding 2. Thus, $O$, the set of odd integers, can be defined as:  $1 \in O$;  and if $n \in O$, then $n+2  \in O$.
		\item[(b)] Powers of 3 are obtained from other powers of 3 by multiplying by 3.  The set $S$ of powers of three is:  $3 \in S$; and if $n \in S$, then $3n \in S$.
	\end{itemize}
\end{solution}


\question[6] Rosen Ch 5.3 \# 26a, p. 358. (See book examples and \#
27 to help with this problem)
\begin{solution}
Let $S$ be the subset of the set of ordered pairs of integers defined recursively by \\
\textit{Basis Step}: $(0,0) \in S$. \\
\textit{Recursive Step}: If $(a,b) \in S$, then $(a+2,b+3)\in S$ and $(a+3,b+2) \in S$. \\
\begin{enumerate}
    \item List the elements of $S$ produced by the first five applications of the recursive definition.
\end{enumerate}
\begin{quote}
    From the basis step $(0,0) \in S$.  Each application of the recursive definition will be given:

    \begin{tabular}{rllllll}
    1 & (2,3), & (3,2) \\
    2 & (4,6), & (5,5), & (6,4) \\
    3 & (6,9), & (7,8), & (8,7), & (9,6) \\
    4 & (8,12), & (9,11), & (10,10), & (11,9), & (12,8) \\
    5 & (10,15), & (11,14), & (12,13), & (13,12), & (14,11), & (15,10) \\
    \end{tabular}
\end{quote}
\end{solution}


\end{questions}
\end{document}