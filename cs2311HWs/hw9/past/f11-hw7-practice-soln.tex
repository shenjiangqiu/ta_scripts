\documentclass[11pt]{exam}

\usepackage[top=0.75in,bottom=0.75in,left=1in,right=1in]{geometry}
%\usepackage{fancyhdr}
%\usepackage{mdwlist}
\usepackage{epsfig,graphicx}
\usepackage{amsmath}
\usepackage{enumerate}

%\usepackage[normalmargins,normalsections,normalindent,normalleading]{savetrees}

\newcommand{\N}{\mathbf{N}}

\begin{document}
\extrawidth{0.5in}%
%\extrafootheight{-0.25in}%
\pagestyle{headandfoot}%
 \headrule
 \header{\textbf{cs2311 - Fall 2011}}{\textbf{HW 7} Practice Problems Solutions}{}%
  \footrule \footer{}{Page \thepage\ of \numpages}{}

\addpoints

\noindent \textbf{Instructions:} All assignments are due at the
beginning of class on the due date specified.  Solutions will be
handed out (or posted on-line) shortly thereafter.  Every student
must write up their own solutions in their own manner.

\begin{questions}
\printanswers

\question A \textit{closed form} of a summation is an equation
in which no summation symbol appears.  The classic example is
$\displaystyle \sum_{i=1}^n i = \frac{n(n+1)}{2}$.  The fraction on
the right is the \textit{closed form} of the summation.  For this
problem you will determine closed forms for summations. % The four facts below will be useful:
%
%\begin{tabular}{lll}
%    \multicolumn{2}{l}{Fact} & Example: \\
%  Fact 0: & $\displaystyle \sum_{i=1}^n c  = nc$     & $\displaystyle \sum_{i=1}^n 7 = 7n$ \\
%    & when $c$ is a constant \\
%  Fact 0': & $\displaystyle \sum_{i=j}^n c  = (n-j+1)c$  & $\displaystyle \sum_{i=0}^n = 7(n+1) $ \\
%    & when $c$ is a constant \\
%  Fact 1: & $\displaystyle \sum_{i=j}^n (f(i) \pm g(i)) = \sum_{i=j}^n f(i) \pm \sum_{i=j}^n g(i)$ \hspace{0.2in} &  $\displaystyle \sum_{i=j}^n (2n - n^2) = \sum_{i=j}^n 2n - \sum_{i=j}^n n^2 $ \\
%    & \\
%  Fact 2: & $\displaystyle \sum_{i=j}^n cf(i) = c \sum_{i=j}^n f(i)$  & $\displaystyle \sum_{i=j}^n (3 \times 2^i) = 3 \sum_{i=j}^n 2^i $ \\
%   & when $c$ is a constant \\
%\end{tabular}

Find a closed form for the summation $\displaystyle \sum_{i=0}^n (2 \cdot 3^i - 3 \cdot 2^i)$.  Show how you find the closed form solution; justify each step.

\begin{solution}
\begin{tabular}{rll}
  $\displaystyle \sum_{i=0}^n (2 \cdot 3^i - 3 \cdot 2^i)$ & $\displaystyle =  \sum_{i=0}^n 2 \cdot 3^i - \sum_{i=0}^n 3 \cdot 2^i$ & Fact 1 \\
   & $\displaystyle = 2 \sum_{i=0}^n 3^i - 3 \sum_{i=0}^n 2^i$ & Fact 2 (twice) \\
   & $\displaystyle = 2 \frac{3^{n+1} -1}{2} - 3(2^{n+1} - 1) $ & Table 2 (twice) \\
   & $\displaystyle = 3^{n+1} - 1 - 3 \cdot 2^{n+1} + 3 $ &  algebra \\
   & $\displaystyle = 3^{n+1} - 3 \cdot 2^{n+1} + 2 $ & algebra \\
\end{tabular}
\end{solution}


\question Give a proof by induction that for every natural
number $n \geq 0$,
\[ \sum_{i=0}^n i^2 = \frac{n(n+1)(2n+1)}{6}.\]

\begin{solution}
 Let $P(n)$ be $\displaystyle \sum_{i=0}^n i^2 = \frac{n(n+1)(2n+1)}{6}$ for every natural number $n\geq 0$. \\

    \textit{Basis Step}: For $n=0$, check $ \displaystyle \sum_{i=0}^0 i^2 = 0$ and $\frac{0(0+1)(2\cdot 0 +1)}{6} = 0$. \\
    \textit{Inductive Step}: Assume that for some $k \geq 0$,
    \[ \sum_{i=0}^k i^2 = \frac{k(k+1)(2k+1)}{6}. \]
    Show that
    \[ \sum_{i=0}^{k+1} i^2 = \frac{k+1(k+2)(2k+3)}{6}. \]

    \begin{align*}
        \sum_{i=0}^{k} i^2 + (k+1)^2 &= \sum_{i=1}^k i^2 + (k+1)^2   & \text{add term} \\
        \sum_{i=0}^{k+1} i^2 &= \frac{k(k+1)(2k+1)}{6} + (k+1)^2               & \text{by def. of sum, ind. hyp} \\
          &= \frac{k(k+1)(2k+1) +6(k+1)^2}{6}  \\
          &= \frac{(k+1)(k(2k+1) +6(k+1))}{6} \\
          &= \frac{(k+1)(2k^2 + k + 6k + 6}{6} \\
          &= \frac{(k+1)(k+2)(2k +3)}{6}. \\
    \end{align*}
    This completes the inductive step. By mathematical induction, $P(n)$ is true, for every natural number $n \geq 0$.
\end{solution}


\question Rosen, Ch 5.1 \# 6, p. 329. \\ Prove that $1\cdot 1! +
2\cdot 2! + \cdots + n\cdot n! = (n+1)! - 1$ whenever $n$ is a
positive integer. In other words, give a proof by induction for
every natural number $n > 0$,
\[ \sum_{i=1}^n i\cdot i! = (n+1)! - 1.\]

\begin{solution}
    Let $P(n)$ be $\displaystyle \sum_{i=1}^n i\cdot i! = (n+1)! - 1$
    for every natural number $n >0$. \\
    \textit{Basis Step}: For $n=1$, verify
    $\sum_{i=1}^1 i \cdot i! = 1 \cdot 1! = 1$ and
    $(1+1)!-1 = 1$, so $P(1)$ is true. \\
    \smallskip
    \textit{Inductive Step}: Assume that for some $k \geq 1$,
    \[ \sum_{i=1}^k i\cdot i! = (k+1)! - 1. \]
    Show that
    \[ \sum_{i=1}^{k+1} i\cdot i! = (k+2)! - 1. \]
    \begin{align*}
        \sum_{i=1}^{k} i\cdot i! + (k+1)(k+1)! &= \sum_{i=1}^k i\cdot i! + (k+1)(k+1)!  & \text{add term to both sides} \\
        \sum_{i=1}^{k+1} i\cdot i!  &= (k+1)!-1 + (k+1)(k+1)! & \text{by def. of sum and ind. hyp.} \\
          &= (n+2)(n+1)! - 1  \\
          &= (n+2)! - 1 \\
    \end{align*}
    This completes the inductive step.  By mathematical induction, $P(n)$ is true for every natural number $n>0$.
\end{solution}


\question Rosen Ch 5.1 \# 10, p. 330.
\begin{solution}
(a) Find a closed formula for $\displaystyle \sum_{i=1}^n \frac{1}{i(i+1)}$ by examining the value of this summation for small values of $n$ (Look at $n=1,2,3,4$ to determine this pattern).\\
\begin{quote}
    For $n=1,2$, and $3$, the summation $\displaystyle \sum_{i=1}^n \frac{1}{i(i+1)}$ is $\frac{1}{2}$, $\frac{2}{3}$, $\frac{3}{4}$, respectively.  Therefore, the closed form of this summation is
    \[ \sum_{i=1}^n \frac{1}{i(i+1)} = \frac{n}{n+1}. \]
\end{quote}

(b) Use induction to prove that the formula you found is correct.\\
\begin{quote}
Let $P(n)$ be for all $n \geq 1$, $\displaystyle \sum_{i=1}^n \frac{1}{i(i+1)} =  \frac{n}{n+1}.$ \\
\textit{Basis Step}: For $n=1$, compute $\sum_{i=1}^1 \frac{1}{1(1+1)} = \frac{1}{2}$ and $\frac{n}{n+1} = \frac{1}{1+1} = \frac{1}{2}$.  \\
\textit{Inductive Step}: Assume for some $k \geq 1$
 \[ \sum_{i=1}^k \frac{1}{i(i+1)} = \frac{k}{k+1}. \]
 Show that
 \[ \sum_{i=1}^{k+1} \frac{1}{i(i+1)} = \frac{k+1}{k+2}. \]
 \begin{align*}
    \sum_{i=1}^{k+1} \frac{1}{i(i+1)} &= \sum_{i=1}^k \frac{1}{i(i+1)}  + \frac{1}{(k+1)(k+2)}  & \text{by def. of}\; \sum \\
      &= \frac{k}{k+1} + \frac{1}{(k+1)(k+2)}  & \text{by ind. hyp.} \\
      &= \frac{k(k+2) + 1}{(k+1)(k+2)} \\
      &= \frac{k^2 + 2k + 1}{(k+1)(k+2)} \\
      &= \frac{(k+1)^2}{(k+1)(k+2)} \\
      &= \frac{k+1}{k+2} \\
 \end{align*}
 This completes the inductive step.  By mathematical induction, $P(n)$ is true, for every pos. integer $n$.
\end{quote}
\end{solution}

\question Give a proof by induction that for every natural number $n
\geq 4$, $2^n \geq n^2$.

\begin{solution}
    Let $P(n)$ be $2^n \geq n^2$ for $n \geq 4$. \\
    \textit{Basis Step}: For $n=4$, $2^4 = 16 \geq 4^2 = 16$. So, $P(4)$ is true. \\
    \textit{Inductive Step}: Assume for some $k \geq 4$
    \[ 2^k \geq k^2 \]
    Show that
    \[ 2^{k+1} \geq (k+1)^2 \]
    \begin{align*}
        2^{k+1} &= 2 \cdot 2^{k}  & \text{by def. of exponents} \\
          &\geq 2 \cdot k^2     & \text{by ind. hyp} \\
          &\geq k^2 + k^2  \\
          &\geq k^2 + 2k + 1 & \text{because}\; k^2 \geq 2k +1 \;\text{when}\; n \geq 4 \\
          &= (k+1)^2 \\
    \end{align*}
    This completes the inductive step.  By mathematical induction, $P(n)$ is true, for $n \geq 4$.
\end{solution}

% 4(a,c)
\question Rosen Ch 5.3 \# 4b, p. 357.
\begin{solution}
Find $f(2)$, $f(3)$, $f(4)$, and $f(5)$ if $f$ is defined
recursively by $f(0) = f(1) = 1$ and for $n=1,2,\ldots$
\begin{itemize}
%    \item[(a)] $f(n+1) = f(n) - f(n-1)$
%    \begin{quote}
%    \begin{align*}
%        f(2) &= f(1) - f(0) = 1 - 1 = 0 \\
%        f(3) &= f(2) - f(1) = 0 - 1 = -1 \\
%        f(4) &= f(3) - f(2) = -1 - 0 = -1 \\
%        f(5) &= f(4) - f(3) = -1 - -1 = 0 \\
%    \end{align*}
%    \end{quote}
    \item[(b)] $f(n+1) = f(n)f(n-1)$
    \begin{quote}
    \begin{align*}
        f(2) &= f(1)f(0) = 1 \\
        f(3) &= f(2)f(1) = 1 \\
        f(4) &= f(3)f(2) = 1 \\
        f(5) &= f(4)f(3) = 1 \\
    \end{align*}
    \end{quote}
%    \item[(c)] $f(n+1) = f(n)^2 + f(n-1)^3$.
%    \begin{quote}
%    \begin{align*}
%        f(2) &= f(1)^2 + f(0)^3 = 1^2 + 1^3 = 2 \\
%        f(3) &= f(2)^2 + f(1)^3 = 2^2 + 1^3 = 5 \\
%        f(4) &= f(3)^2 + f(2)^3 = 5^2 + 2^3 = 33 \\
%        f(5) &= f(4)^2 + f(3)^2 = 33^2 + 5^3 = 1214 \\
%    \end{align*}
%    \end{quote}
\end{itemize}
\end{solution}

% 8a
\question Rosen Ch 5.3 \# 8c p. 308.
\begin{solution}
Give a recursive definition of the sequence $\{a_n\}$, $n=1,2,3,
\ldots$ if
\begin{itemize}
%    \item[(a)] $a_n = 4n-2$
%    \begin{quote}
%        The sequence contains the following number $a_1 = 2$, $a_2 = 6$, $a_3 = 10$, $a_4 = 14$, ...
%        A recursive definition (there are many such definitions) is
%        \[  a_1 = 2, \;\; a_{n+1} = a_n + 4 \;\; \forall n \geq 2 \]
%    \end{quote}
    \item[(c)] $a_n = n(n+1)$.
    \begin{quote}
        This sequences is as follows: $a_1 = 2$, $a_2 = 6$, $a_3 = 12$, $a_4 = 20$, \ldots
        A recursive definition is
        \[ a_1 = 2, \;\; a_{n} = a_{n-1} + 2n \]
    \end{quote}
\end{itemize}
\end{solution}


\question Rosen Ch 5.3 \# 49, p. 359.
\begin{solution}
 Show that A(m,2) = 4 whenever
$m \geq 1$.
\begin{quote}
    Let $P(n)$ be $A(n,2) = 4$ for $n \geq 1$. \\
    \textit{Basis Step}: $A(1,2) = A(0,A(1,1)) = A(0,2) = 2*2 = 4$, therefore $P(1)$ is true completing the basis step. \\
    \textit{Inductive Step}: Assume that $P(k)$ is true, that is $A(k,2) = 4$.  Then, $A(k+1,2) = A(k,A(k+1,1)) = A(k,2) = 4$.  This completes the inductive step.  Therefore, by mathematical induction, $A(n,2) = 4$ for $n \geq 1$.
\end{quote}
\end{solution}

\end{questions}
\end{document}
