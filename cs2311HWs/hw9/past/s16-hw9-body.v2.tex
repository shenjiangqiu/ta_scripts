% Homework Specific Commands


\begin{document}
\extrawidth{0.5in} \extrafootheight{-0in} \pagestyle{headandfoot}
\headrule \header{\textbf{cs2311 - Fall 2016}}{\textbf{HW
 9 \ifprintanswers - Solutions \fi}}{\textbf{Due: Mon. 04/04/16}} \footrule \footer{}{Page \thepage\
of \numpages}{}

\ifprintanswers
% \noindent You should \underline{complete all problems}, but \underline{only a subset will be graded} (which will be graded is not known to you ahead of time). 
\else
\noindent \textbf{Instructions:} All assignments are due \underline{by \textbf{4pm} on the due date} specified.  There will be a box in the CS department office (Rekhi 221) where assignments may be turned in.  Solutions will be posted on-line at the following lecture.  Every student
must write up their own solutions in their own manner.

\medskip
\noindent Please present your solutions in a clean, understandable
manner; pages should be stapled before being turned in, no ragged edges of
paper.

% \medskip
% \noindent Be clear with you penmanship to distinguish set brackets and parentheses.
%   For example, $\{1, 2\}$ is a set where $(1,2)$ is an ordered pair.  Also,
% differentiate the empty set $\emptyset$ and the number zero 0.

\medskip
\noindent You should \underline{complete all problems}, but \underline{only a subset will be graded} (which will be graded is not known to you ahead of time). 

\medskip
\noindent \textbf{Include the following information at the top, left corner of your submission:} (not including this information will result in -3pt)\\
\textbf{Last Name, First Name (as listed in Canvas)} \\
\textbf{Section (R01, R02)}
\fi

\begin{questions}




\uplevel{{\bf Summations}}


% % in HW8 
% \gquestion{8}{4}{c,e} What are the values of the sums:\\
% % \begin{tabular}{lll}
% %     (a) $\ds \sum_{j\in S} (2j - 1), \text{ where } S = \{1, 2, 4, 6\}$
% %     & (b) $\ds \sum_{i=1}^{8} 4$  \hspace*{1in} 
% %     & (c) $\ds \sum_{k=0}^{4} (-3)^k$ \\
% %     & & \\
% %     (d) $\ds \sum_{j=0}^{7} (2^{j+1} - 2^{j-1})$
% %     & (e) $\ds \sum_{i=1}^{4} \sum_{j=1}^{3} (2i + 3j)$
% %     & (f) $\ds \sum_{j=0}^{3} \sum_{k=1}^{3} j*2^{k}$
% % \end{tabular}
% \begin{tabular}{lll}
%     (a) $\ds \sum_{j\in S} (2j - 1), \text{ where } S = \{1, 2, 4, 6\}$
%     & (b) $\ds \sum_{k=0}^{4} (-3)^k$ \\
%     & & \\
%     (c) $\ds \sum_{i=1}^{4} \sum_{j=1}^{3} (2i + 3j)$
%     & (d) $\ds \sum_{j=0}^{3} \sum_{k=1}^{3} j*2^{k}$
% \end{tabular}
%     \ifprintanswers
%         \vspace{-10pt}
%    \fi
% \begin{solution}
%     \begin{enumerate}[label=(\alph*),itemsep=3pt,parsep=0pt,
%     topsep=0pt,partopsep=0pt]
%         \item $\ds (2\cdot1 - 1)+(2\cdot2 - 1)+(2\cdot4 -1)+(2\cdot6-1) = 1+3+7+11 = 22$ 
%         % \item $\sum_{i=1}^8 4 = 8\cdot4 = 32$
%         \item $\ds (-3^0) + (-3^1) + (-3^2) + (-3^3) + (-3^4) = 1 -3 + 9 -27 + 81 = 61 $
%         % \item $\sum_{j=0}^7 2^{j+1} - \sum_{j=0}^7 2^{j-1} = 510 - 127.5 = 382.5$
%         \item $\ds \sum_{i=1}^4 \left( \sum_{j=1}^3 2i + \sum_{j=1}^3 3j \right) = 132$
%         \item $\ds \sum_{j=0}^3 \sum_{k=1}^3 j*2^k = 34$
%     \end{enumerate}
% \end{solution}

% \gquestion{8}{8}{all} \ifprintanswers
\ugquestion{8} \ifprintanswers 
\else 
A \textit{closed form} of a summation is an equation
in which no summation symbol appears.  The classic example is
$\displaystyle \sum_{i=1}^n i = \frac{n(n+1)}{2}$.  The fraction on
the right is the \textit{closed form} of the summation. 
\fi  Determine closed forms for summation, $\displaystyle \sum_{i=1}^n \sum_{j=1}^n (3i+ 2j)$.  Make sure to justify and separate each step using the four facts of the arithmetic properties of summations (indicate which fact is used) and Table 2, p. 166 in the book.

\ifprintanswers
\small
\begin{tabular}{llll}
    \multicolumn{2}{l}{Fact} & Example: \\
  Fact 1: & $\displaystyle \sum_{i=1}^n c  = nc$     & Fact 2: & $\displaystyle \sum_{i=j}^n c  = (n-j+1)c$ \\
    & when $c$ is a constant & & when $c$ is a constant \\ 
  Fact 3: & $\displaystyle \sum_{i=j}^n (f(i) \pm g(i)) = \sum_{i=j}^n f(i) \pm \sum_{i=j}^n g(i)$ \hspace{0.2in} & Fact 4: & $\displaystyle \sum_{i=j}^n cf(i) = c \sum_{i=j}^n f(i)$ \\
    & & & when $c$ is a constant \\
\end{tabular}
\else
\small
\begin{tabular}{lll}
    \multicolumn{2}{l}{Fact} & Example: \\
  Fact 1: & $\displaystyle \sum_{i=1}^n c  = nc$     & $\displaystyle \sum_{i=1}^n 7 = 7n$ \\
    & when $c$ is a constant \\
  Fact 2: & $\displaystyle \sum_{i=j}^n c  = (n-j+1)c$  & $\displaystyle \sum_{i=0}^n = 7(n+1) $ \\
    & when $c$ is a constant \\
  Fact 3: & $\displaystyle \sum_{i=j}^n (f(i) \pm g(i)) = \sum_{i=j}^n f(i) \pm \sum_{i=j}^n g(i)$ \hspace{0.2in} &  $\displaystyle \sum_{i=j}^n (2n - n^2) = \sum_{i=j}^n 2n - \sum_{i=j}^n n^2 $ \\
    & \\
  Fact 4: & $\displaystyle \sum_{i=j}^n cf(i) = c \sum_{i=j}^n f(i)$  & $\displaystyle \sum_{i=j}^n (3 \times 2^i) = 3 \sum_{i=j}^n 2^i $ \\
   & when $c$ is a constant \\
\end{tabular}
\fi
    \ifprintanswers
        \vspace{-10pt}
   \fi
\begin{solution}
\begin{tabular}{rll}
  $\displaystyle \sum_{i=1}^n \sum_{j=1}^n (3i+ 2j)$ & $\displaystyle = \sum_{i=1}^n \left[ \sum_{j=1}^n (3i + 2j) \right] $ & add implied parentheses \\
   & $\displaystyle = \sum_{i=1}^n \left[ \sum_{j=1}^n 3i + \sum_{j=1}^n 2j \right]$ & Fact 3 \\
   & $\displaystyle = \sum_{i=1}^n \left[ 3i \sum_{j=1}^n 1 + 2 \sum_{j=1}^n j \right]$ & Fact 4 and Fact 4 \\
   & $\displaystyle = \sum_{i=1}^n \left[ 3i\cdot n + 2 \sum_{j=1}^n j \right]$ & Fact 1  \\
   & $\displaystyle = \sum_{i=1}^n \left[ 3i\cdot n + 2 \cdot \frac{n(n+1)}{2} \right]$ & Table 2 \\
   & $\displaystyle = \sum_{i=1}^n 3i\cdot n + \sum_{i=1}^n n(n+1)$ & Fact 3 \\
   & $\displaystyle = 3n \sum_{i=1}^n i + n(n+1) \sum_{i=1}^n 1$ & Fact 4 (twice) \\
   & $\displaystyle = 3n \cdot \frac{n(n+1)}{2} + n(n+1) \sum_{i=1}^n 1$ & Table 2 \\
   & $\displaystyle = 3n \cdot \frac{n(n+1)}{2} + n(n+1)\cdot n$ & Fact 1 \\
   & $\displaystyle = \frac{5n^2(n+1)}{2} $ & algebra \\
\end{tabular}
\end{solution}


% \ugquestion{8} Find a closed form for the summation $\displaystyle \sum_{i=1}^n \sum_{j=1}^n (6i^2 - 2j)$.  Show how you find the closed form solution; justify each step using the four facts of the arithmetic properties of summations (listed above) and Table 2, p. 166 in the book.
%     \ifprintanswers
%         \vspace{-10pt}
%    \fi
% \begin{solution}
% \begin{align*}
%         \sum_{i=1}^n \sum_{j=1}^n (6i^2 - 2j) 
%         &= \sum_{i=1}^n ( \sum_{j=1}^n (6i^2 - 2j) ) \tag{implied parentheses} \\
%         &= \sum_{i=1}^n (\sum_{j=1}^n 6i^2 - \sum_{j=1}^n 2j) \tag{Fact 3} \\
%         &= \sum_{i=1}^n (6i^2 \sum_{j=1}^n 1 - 2 \sum_{j=1}^n j) \tag{Fact 4, twice} \\
%         &= \sum_{i=1}^n (6i^2n - 2 \sum_{j=1}^n j) \tag{Fact 1} \\
%         &= \sum_{i=1}^n (6i^2n - 2 \frac{n(n+1)}{2}) \tag{Table 2} \\
%         &= \sum_{i=1}^n 6i^2n - \sum_{i=1}^n  n(n+1) \tag{Fact 3} \\
%         &= 6n \sum_{i=1}^n i^2 - n(n+1) \sum_{i=1}^n 1 \tag{Fact 4, twice} \\
%         &= 6n (\frac{n(n+1)(2n+1)}{6}) - n(n+1) \sum_{i=1}^n 1 \tag{Table 2} \\
%         &= n^2(n+1)(2n+1) - n^2(n+1) \tag{algebra} \\
%         &= n^2(n+1)(2n + 1  - 1) \\
%         &= 2n^3(n+1) 
%     \end{align*}
% \end{solution}




\gquestion{8}{8}{all} Find a closed form for the summation $\displaystyle \sum_{i=5}^n 3^i$.  Show how you find the closed form solution; justify each step using the four facts of the arithmetic properties of summations (listed above) and Table 2, p. 166 in the book.
    \ifprintanswers
        \vspace{-10pt}
   \fi
\begin{solution}
    \begin{align*}
        \sum_{i=5}^n 3^i  & = \sum_{j=0}^{n-5} 3^{j+5} \tag{change of index} \\
        & = \sum_{j=0}^{n-5} 3^{j}3^{5} \tag{algebra} \\
        & = 3^5 \sum_{j=0}^{n-5} 3^j    \tag{Fact 4} \\
        & = 3^5 \left[ \frac{3^{n-5+1} - 1}{3 - 1} \right] \tag{Table } \\
        & = \frac{ 3^5(3^{n-4} - 1) }{ 2 } \tag{algebra}
    \end{align*}

    Alternatively, 
    \begin{align*}
        \sum_{i=5}^n 3^i &= \sum_{i=0}^n 3^i - \sum_{i=0}^4 3^i \tag{change index} \\
        &= \frac{3^{n+1} - 1}{3 - 1} - \frac{3^{4+1} - 1}{3 -1} \tag{Table 2}\\
        &= \frac{3^{n+1} - 3^{5}}{2} = \frac{3^5(3^{n-4} - 1)}{2} \tag{algebra} \\
    \end{align*}
\end{solution}



\ugquestion{4} Rosen Ch 2.4 \#40, p. 169
    \ifprintanswers
        \vspace{-10pt}
   \fi
\begin{solution}
 We can find $\ds \sum_{k=1}^{200} k^3 = \frac{200^2(201)^2}{4}$.
 Also, $\ds \sum_{k=1}^{98} k^3 = \frac{98^2(99)^2}{4}$.

 Therefore,
$$ \sum_{k=99}^{200} = \sum_{k=1}^{200} k^3 - \sum_{k=1}^{98} k^3
   = \frac{200^2(201)^2}{4} - \frac{98^2(99)^2}{4} 
   = 380,477,799 $$
\end{solution}



\uplevel{{\bf Induction and Recursion}}



\gquestion{8}{8}{all} Rosen, Ch 5.1 \# 4, p. 329
    \ifprintanswers
        \vspace{-10pt}
   \fi
\begin{solution}
Let $P(n)$ be the statement that $1^3 + 2^3 + \cdots + n^3 = \left( \frac{n(n+1)}{2} \right)^2$ for the positive integer $n$.
\begin{parts}
    \part (1 point) What is the statement P(1)? \\
        $P(1)$ is the statement $ 1^3 = [ \frac{1 \cdot (1+1) }{2} ]^2 $.
    \part (1 point) Show that P(1) is true, completing the basis step of the proof.
        \[ \left[ \frac{1 \cdot (1+1) }{2} \right]^2 = 1^2 = 1 = 1^3 \]
    \part (2 points) What is the inductive hypothesis? \\
        The inductive hypothesis is to assume $P(k)$ is true for an arbitrary, fixed integer $k \geq 1$, that is
        \[ 1^3 + 2^3 + \cdots + k^3 = \left( \frac{k(k+1)}{2} \right)^2 \]
    \part (2 point) What do you need to prove in the inductive step? \\
        For the inductive step, show for each $k \geq 1$ that $P(k)$ implies $P(k+1)$. \\
        That is, show $P(k+1)$:
        \[ 1^3 + 2^3 + \cdots + k^3 + (k+1)^3 = \left( \frac{(k+1)(k+2)}{2} \right)^2 \]
    \part (3 points) Complete the inductive step. \\
        Since, $P(k)$ holds,
        \begin{align*}
            1^3 + 2^3 + \cdots + k^3 &= \left( \frac{k(k+1)}{2} \right)^2 \tag{Ind. Hyp.} \\
            1^3 + 2^3 + \cdots + k^3 + (k+1)^3 &=  \left( \frac{k(k+1)}{2} \right)^2 + (k+1)^3   \\
              &= \left(\frac{k^2(k+1)^2}{2^2} \right) + (k+1)^3 \\
              &= (k+1)^2 \left( \frac{k^2}{4} + (k + 1) \right) \\
              &= (k+1)^2 \left( \frac{k^2 + 4k + 4}{4} \right) \\
              &= \frac{(k+1)^2(k+2)^2}{2^2} =  \left( \frac{(k+1)(k+2)}{2} \right)^2\\
        \end{align*}
        That is, $1^3 + 2^3 + \cdots +(k+1)^3 = \left( \frac{(k+1)(k+2)}{2} \right)^2 $, the $P(k+1)$ statement.  This completes the inductive step.
    \part (1 point) Explain why these steps show that this formula is true whenever $n$ is a positive integer. \\
        The basis step and inductive step are completed.  Therefore by principle of mathematical induction, the statement is true for every positive integer $n$.
\end{parts}
\end{solution}




% Ferland, p. 189
\ugquestion{8} Use induction to prove the following statement:
    \[\forall n \geq 0, \;\; 4 \,|\, (5^{n} - 1)\]
That is, for the natural numbers, 4 divides $5^n - 1$.
    \ifprintanswers
        \vspace{-10pt}
   \fi
\begin{solution}
Let $P(n)$ be the statement ``4 divides $5^n - 1$" for the natural numbers $n$. 

\textit{Basis Step:} Show $P(0)$ is true. 
    4 does divide $5^n - 1$ where $n=0$;  $4 \,|\, 5^0 - 1 = 1 - 1 = 0$ because there exists an integer $k$ such that $4k = 0$.

\textit{Inductive Step:} \\
\textit{IH} Assume $P(n)$ holds for some $k \geq 0$, that is, there exists an integer $a$ such that $4a = 5^k - 1$. 

Show that $P(k+1)$ holds, that is, there exists an integer $b$ such that $4b = 5^{k+1} - 1$. 

\begin{align*}
    5^{k+1} - 1 &= 5\cdot 5^{k} - 1 \\
     &= (4 + 1)5^{k} - 1 \\
     &= 4\cdot 5^{k} + 5^{k} - 1 \\
     &= 4\cdot 5^{k} + 4a \tag{Ind. Hyp.} \\
     &= 4(5^{k} + a)
     = 4b
\end{align*}
This shows that assuming $P(k)$, $P(k+1)$ holds. \\
Therefore, by completing the basis and inductive step and principle of mathematical induction, the statement $P(n)$ is true for every natural number.
\end{solution}




% cs2311 - summer 2012 Assignment 4, Chuck
\bonusquestion[6] The Fibonacci numbers are defined as $F(0) = 0$, $F(1) = 1$, and $F(n) = F(n-1) + F(n-2)$ for $n \geq 2$.   Use induction to prove the following for natural numbers $n$:
\[ F(0) + F(1) + \cdots + F(n) = F(n+2) - 1.  \]
    \ifprintanswers
        \vspace{-10pt}
   \fi
\begin{solution}
  \textit{Proof:}
  Let the $P(n)$ be the statement $F(0) + F(1) + \cdots + F(n) = F(n+2) -1$.

  \smallskip
  \begin{tabular}{lp{4in}}
    \textit{Basis Step:} & Show $P(0)$ is true, $F(0) = 0 = 1 - 1 = F(0+2) - 1$ \\
     & \\
   \textit{Inductive Step:} &  \\
  \end{tabular}

  Assume $P(k)$ is true for an arbitrary, fixed integer $k \geq 0$, that is,
  \begin{align*}
    F(0) + F(1) + \cdots + F(k) &= F(k+2) - 1  \tag{IH} \\
    \sum_{i=0}^k F(i) &= F(k+2) - 1
  \end{align*}

  Show $P(k+1)$ is true, that is
  \begin{align*}
    F(0) + F(1) + \cdots + F(k) + F(k+1) &= F(k+1+2) - 1 = F(k+3) - 1 \\
    \sum_{i=0}^{k+1} F(k) &= F(k+3) - 1
  \end{align*}

  Start with $P(k+1)$:
  \begin{align*}
    F(0) + F(1) + \cdots + F(k) + F(k+1) &= F(k+3) - 1  \\
    F(k+2) - 1 + F(k+1) &= \tag{IH} \\
    F(k+2) + F(k+1) - 1 &= \\
    F(k+3) - 1
  \end{align*}
  This shows $P(k+1)$ is true, assuming $P(k)$ is true, completing the inductive step. 

  \smallskip
  Therefore, by mathematical induction, $P(n)$ is true for all $n \geq 0$.

  \smallskip
  Note, the inductive step could also be shown by, starting with $P(k)$ and adding $F(k+1)$ to both sides:
  \begin{align*}
    F(0) + F(1) + \cdots + F(k) &= F(k+2) - 1 \tag{IH} \\
    F(0) + F(1) + \cdots + F(k) + F(k+1) &= F(k+2) - 1 + F(k+1) \\
     &= F(k+2) + F(k+1) - 1 \\
     &= F(k+3) -1 
  \end{align*}
\end{solution}



% Rosen Ch 5.3, \# 12, p. 358.
\ugquestion{8}  Also, with the Fibonacci numbers, use induction to prove for the positive integers $n$: 
\[ F(1)^2 + F(2)^2 + \cdots + F(n)^2 = F(n)F(n+1). \]
    \ifprintanswers
        \vspace{-10pt}
   \fi
\begin{solution}
  \textit{Proof:}
  Let the $P(n)$ be the statement $F(1)^2 + F(2)^2 + \cdots + F(n)^2 = F(n)F(n+1)$.

  \smallskip
  \begin{tabular}{lp{4in}}
    \textit{Basis Step:} & Show $P(1)$ is true, $F(1) = 1 = 1\cdot1 = F(1)F(1+1)$ \\
     & \\
   \textit{Inductive Step:} &  \\
  \end{tabular}

  Assume $P(k)$ is true for an arbitrary, fixed integer $k \geq 1$, that is,
  \begin{align*}
    F(1)^2 + F(2)^2 + \cdots + F(k)^2  = F(k)F(k+1)  \tag{IH} \\
    \sum_{i=1}^k F(i)^2 = F(k)F(k+1)
  \end{align*}

  Show $P(k+1)$ is true, that is
  \begin{align*}
    F(1)^2 + F(2)^2 + \cdots + F(k)^2 + F(k+1)^2 = F(k+1)F(k+2) \\
    \sum_{i=1}^{k+1} F(k)^2 = F(k+1)F(k+2) 
  \end{align*}

  Start with $P(k+1)$:
  \begin{align*}
    F(1)^2 + F(2)^2 + \cdots + F(k)^2 + F(k+1)^2 &= F(k+1)F(k+2) \\
    F(k)F(k+1) + F(k+1)^2 &= \tag{IH} \\
    F(k+1)\left[ F(k) + F(k+1) \right] &= \\
    F(k+1)F(k+2) 
  \end{align*}
  This shows $P(k+1)$ is true, assuming $P(k)$ is true, completing the inductive step. 

  Therefore, by mathematical induction, $P(n)$ is true for all $n \geq 1$.
\end{solution}



% Ferland, p. 188, Ex. 4.19
\gquestion{8}{8}{all} Show for all integers $n \geq 4$, $n^2 \geq 3n + 4$. 
    \ifprintanswers
        \vspace{-10pt}
   \fi
\begin{solution}
  \textit{Proof:}
  Let $P(n)$ be $n^2 \geq 3n + 4$ for all integers $n \geq 4$.

  \smallskip
  \begin{tabular}{lp{4in}}
    \textit{Basis Step:} & Show $P(4)$ is true, $4^2 = 16 \geq 16 = 3\cdot 4 + 4$ \\
     & \\
   \textit{Inductive Step:} &  \\
  \end{tabular}

  Assume $P(k)$ is true for an arbitrary, fixed integer $k \geq 4$, that is, 
  \begin{align*}
    k^2 \geq 3k + 4 \tag{IH} 
  \end{align*}

  Show $P(k+1)$ is true, that is, 
  \[ (k+1)^2 \geq 3(k+1) + 4 = 3k + 7 \]

  Start with \textit{lhs} of $P(k+1)$
  \begin{align*}
    (k+1)^2 &= k^2 + 2k + 1 \\
     &\geq (3k + 4) + 2k + 1 \tag{IH} \\
     &= 3k + (2k + 5) \\
     &\geq 3k + 7 \tag{2k +5 $\geq 7$ or k $\geq$ 1, for all k $\geq$ 4} \\
     &= 3(k+1) + 4 \\
  \end{align*}
  This shows $P(k+1)$ is true, assuming $P(k)$ is true, completing the inductive step. 

  Therefore, by mathematical induction, $n^2 \geq 3n + 4$ for all $n \geq 4$.
\end{solution}


% Ferland p. 206, Example 4.28
\gquestion{10}{10}{all} Consider the game of football (that is, the American game of football).  Let's assume teams can either score via field goal (3 points) or touchdowns (7 points, assume all point after touchdowns are made).   Safeties are ignored for this problem.  

Show that it is then possible (assuming no time constraints) for a team to score any number of points from 12 on up. 
    \ifprintanswers
        \vspace{-10pt}
   \fi
\begin{solution}
    \textit{Proof:}
    Let $P(n)$ be that it is possible for a team to score $n$ points, for $n \geq 12$.
    
    \smallskip
    \begin{tabular}{lp{4in}}
      \textit{Basis Step:}  & Show $P(12)$ is true, 4 field goals \\
                            & Show $P(13)$ is true, 1 touchdown and 2 field goals, and \\
                            & Show $P(14)$ is true, 2 touchdowns \\
       & \\
     \textit{Inductive Step:} &  \\
    \end{tabular}

    Assume $P(j)$ is true where $12 \leq j \leq k$ and an arbitrary, fixed integer $k \geq 14$, that is, a team can score $j$ points. 

    Show $P(k+1)$ is true, that is, a team can score $k+1$ points through field goals and touchdowns.

    We know a team can score $P(k-2)$ and $k -2 \geq 12$ from the inductive hypothesis. A team can score one additional field goal (3 points) to reach $k+1$ points, competing the inductive step. 

    Therefore, by mathematical induction, $P(n)$ for integers $n \geq 12$. 
\end{solution}



% % Ferland p. 206, Example 4.28
% \gquestion{10}{10}{all} Consider the game of rugby.  Let's assume teams can either score via trys, 5 pts, or drop goals, 3 pts.   Ignore conversions and penalty goals for this problem.  

% Show that it is then possible (assuming no time constraints) for a team to score any number of points from 16 on up. 
%     \ifprintanswers
%         \vspace{-10pt}
%    \fi
% \begin{solution}
%   \textit{Proof:}
%   Let $P(n)$ be that it is possible for a team to score $n$ points, for $n \geq 16$.
  
%   \smallskip
%   \begin{tabular}{lp{4in}}
%     \textit{Basis Step:}  & Show $P(16)$ is true, 2 trys and 2 drop goals \\
%                 & Show $P(17)$ is true, 1 try and 4 drop goals, and \\
%                 & Show $P(18)$ is true, 3 trys and 1 drop goals \\
%      & \\
%    \textit{Inductive Step:} &  \\
%   \end{tabular}

%   Assume $P(j)$ is true where $16 \leq j \leq k$ and an arbitrary, fixed integer $k \geq 18$, that is, a team can score $j$ points. 

%   Show $P(k+1)$ is true, that is, a team can score $k+1$ points through trys and drop goals.

%   We know a team can score $P(k-2)$ and $k -2 \geq 16$ from the inductive hypothesis. A team can score one additional drop goal (3 points) to reach $k+1$ points, competing the inductive step. 

%   Therefore, by mathematical induction, $P(n)$ for integers $n \geq 12$. 
% \end{solution}




% Ferland, p. 207, Exercise #1.
\ugquestion{8} Let $\{s_n\}$ be the sequence defined as, 
\[ s_0 = 0, \quad s_1 = 1,  \quad \text{and} \quad s_n = 3s_{n-1} - 2s_{n-2}, \forall n \geq 2. \]
Show $\forall n \geq 0, s_n = 2^n - 1 .$
    \ifprintanswers
        \vspace{-10pt}
   \fi
\begin{solution}
  \textit{Proof:}
  Let $P(n)$ be that the $n$th term of the sequence is determine as $s_n = 2^n - 1$ for $n \geq 0$.

  \smallskip
  \begin{tabular}{lp{4in}}
    \textit{Basis Step:}  & Show $P(0)$ is true, $2^0 - 1 = 0 = s_0$ \\
                & Show $P(1)$ is true, $2^1 - 1 = 1 = s_1$ \\
     & \\
   \textit{Inductive Step:} &  \\
  \end{tabular}

  Assume $P(j)$ is true $0 \leq j \leq k$ with $k \geq 1$, that is, 
  \begin{align*}
    s_j = 2^j - 1 \quad \forall 0 \geq j \geq k \tag{IH}
  \end{align*}

  Show that $P(k+1)$ is true, that is,
  \[ s_{k+1} = 2^{k+1} - 1 \] 

  Consider the definition of the sequence, 
  \begin{align*}
    s_{k+1} &= 3s_{k} - 2s_{k-1} \\
    &= 3(2^k - 1) - 2(2^{k-1} - 1) \\
    &= 3\cdot 2^k - 3 - 2^k + 2 \\
    &= 2\cdot 2^k - 1 \\
    &= 2^{k+1} - 1
  \end{align*}
  When the inductive hypothesis is true, then $P(k+1)$ is true, completing the inductive step. 

  Then, by strong mathmatical induction, we conclude $P(n)$ for $n \geq 0$.
\end{solution}



\gquestion{6}{6}{} Rosen Ch 5.3 \# 26a, p. 358. (See book examples and \#
27 to help with this problem)
\begin{solution}
Let $S$ be the subset of the set of ordered pairs of integers defined recursively by \\
\textit{Basis Step}: $(0,0) \in S$. \\
\textit{Recursive Step}: If $(a,b) \in S$, then $(a+2,b+3)\in S$ and $(a+3,b+2) \in S$. \\
\begin{enumerate}
    \item List the elements of $S$ produced by the first five applications of the recursive definition.
\end{enumerate}
\begin{quote}
    From the basis step $(0,0) \in S$.  Each application of the recursive definition will be given:

    \begin{tabular}{rllllll}
    1 & (2,3), & (3,2) \\
    2 & (4,6), & (5,5), & (6,4) \\
    3 & (6,9), & (7,8), & (8,7), & (9,6) \\
    4 & (8,12), & (9,11), & (10,10), & (11,9), & (12,8) \\
    5 & (10,15), & (11,14), & (12,13), & (13,12), & (14,11), & (15,10) \\
    \end{tabular}
\end{quote}
\end{solution}

\end{questions}
\end{document}