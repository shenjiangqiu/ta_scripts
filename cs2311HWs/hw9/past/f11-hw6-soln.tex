\documentclass[11pt]{exam}

\usepackage[top=0.75in,bottom=0.75in,left=1in,right=1in]{geometry}
%\usepackage{fancyhdr}
%\usepackage{mdwlist}
\usepackage{epsfig,graphicx}
\usepackage{amsmath}
\usepackage{enumerate}
\usepackage{amssymb}

%\usepackage[normalmargins,normalsections,normalindent,normalleading]{savetrees}

%\pagestyle{fancy} \lhead{\textbf{cs2311} - Fall 2010}
%\chead{\textbf{HW 4}} \rhead{\textbf{Due: Fri. 10/1/10}}
%%\rfoot{\thepage}

\newcommand{\es}{\emptyset}
\newcommand{\ra}{\rightarrow}
\newcommand{\N}{\mathbb{N}}
\newcommand{\Z}{\mathbb{Z}}
\newcommand{\Zp}{\mathbb{Z}^+}
\newcommand{\Zn}{\mathbb{Z}^-}
\newcommand{\Q}{\mathbb{Q}}
\newcommand{\R}{\mathbb{R}}

\begin{document}
\extrawidth{0.5in}%
%\extrafootheight{-0.5in}%
\pagestyle{headandfoot}%
 \headrule
 \header{\textbf{cs2311 - Fall 2011}}{\textbf{HW 6 - Solutions} (\numpoints$\;$ points)}{\textbf{Due: Mon. 10/17/11}}%
  \footrule \footer{}{Page \thepage\ of \numpages}{}

\addpoints

\noindent \textbf{Instructions:} All assignments are due at the
beginning of class on the due date specified.  Solutions will be
handed out (or posted on-line) shortly thereafter.  Every student
must write up their own solutions in their own manner.

\begin{questions}
\printanswers

\question[4] Rosen, Ch 2.3 \# 4 (a,c), p. 152.
\begin{solution}
\begin{itemize}
  \setlength{\itemsep}{1pt}
  \setlength{\parskip}{0pt}
  \setlength{\parsep}{0pt}
    \item[(a)] domain: set of natural numbers; range: $\{0, 1, 2, 3, 4, 5, 6, 7, 8, 9 \}$.
    \item[(b)] domain: set of integers; range: $\mathbb{Z}^+$.
    \item[(c)] domain: all possible bit strings; range: $\mathbf{N}$.
\end{itemize}
\end{solution}

\question[4] Rosen, Ch 2.3, \# 8(a,b,c,d), p. 153.
\begin{solution}
Find the values.

\renewcommand{\arraystretch}{1.3}
\begin{tabular}{ll}
    a) $\lfloor 1.1 \rfloor = 1$ \hspace*{0.5in} & b) $\lceil 1.1 \rceil = 2$\\
    c) $\lfloor -0.1 \rfloor = -1$ & d) $\lceil -0.1 \rceil = 0$ \\
%    e) $\lceil 2.99 \rceil = 3$ &     f) $\lceil-2.99 \rceil = -2$\\
%    \multicolumn{1}{l}{g) $\lfloor \frac{1}{2} + \lceil \frac{1}{2} \rceil \rfloor = \lfloor \frac{1}{2} + 1 \rfloor = \lfloor \frac{3}{2} \rfloor = 1$ } \\
%    \multicolumn{1}{l}{h) $\lceil \lfloor \frac{1}{2} \rfloor + \lceil \frac{1}{2} \rceil + \frac{1}{2} \rceil = \lceil 0 + 1 + \frac{1}{2} \rceil = \lceil \frac{3}{2} \rceil = 2$} \\
\end{tabular}
\end{solution}


\question[4] Rosen Ch 2.3 \# 12(a,b), p.153.  %(determine if function is one-to-one)
\begin{solution}
Determine whether each of these functions from $\mathbf{Z}$ to $\mathbf{Z}$ is \textbf{one-to-one}.

\begin{tabular}{l}
 a) $f(n) = n-1$, \textbf{one-to-one} \\
 b) $f(n) = n^2 + 1$, \textbf{not one-to-one}, $f(n) = f(-n)$\\
% c) $f(n) = n^3$, \textbf{one-to-one} \\
% d) $f(n) = \lceil \frac{n}{2} \rceil$, \textbf{not one-to-one}, $f(1) = f(2) = 1$ \\
\end{tabular}
\end{solution}



\question[6] Rosen Ch 2.3 \# 14(a,c,e), p. 153.  %(determine if function is onto)
\begin{solution}
Determine whether $f: \mathbf{Z} \times \mathbf{Z} \rightarrow \mathbf{Z}$ is \textbf{onto}

\begin{tabular}{p{5in}}
  a) $f(m,n) = 2m - n$, \textbf{onto}, $f(0,-n) = n$, for $\forall\;n \in \mathbf{Z}$ \\
%  b) $f(m,n) = m^2 - n^2$, \textbf{not onto}, there does not exist a pair of integers $m$ and $n$ such that $m^2 - n^2$ = 2. \\
  c) $f(m,n) = m + n + 1$, \textbf{onto}, $f(n-1,0) = n$, for $\forall\;n \in \mathbf{Z}$ \\
%  d) $f(m,n) = |m| - |n|$, \textbf{onto}, for $\forall\; n \in \mathbf{Z}$, if $n\geq 0$, $f(n,0) = n$; if $n < 0$, $f(0,n) = n$. \\
  e) $f(m,n) = m^2 - 4.$, \textbf{not onto}, for the same reason as part (b) \\
\end{tabular}
\end{solution}



\question[4] Rosen Ch. 2.3 \# 22(a,d) p. 153.  %(determine if function is bijection)
\begin{solution}
Determine whether each of these functions is a bijection from $\mathbf{R}$ to $\mathbf{R}$.

\begin{tabular}{l}
 a) $ f(x) = -3x + 4$, \textbf{Yes} \\
% b) $f(x) = -3x^2 + 7$, \textbf{No}, neither one-to-one nor onto. \\
 d) $f(x) = x^5 + 1$ \textbf{Yes} \\
\end{tabular}
\end{solution}



\question[6] Rosen Ch 2.3 \# 30(a,b) p. 154.
\begin{solution}
Find $f(S)$ for the given functions.

\begin{tabular}{l}
  a) $f(S) = \{ 1, 1, 1, 1, 1 \} = \{ 1 \}$ \\
  b) $f(S) = \{ -1, 1, 5, 9, 15 \}$ \\
%  c) $f(S) = \{ 0, 0, 1, 1, 2 \} = \{0, 1, 2\}$ \\
\end{tabular}
\end{solution}


\question[4] Rosen Ch 2.3 \# 32(a,b) p. 154.
\begin{solution}
Let $f(x) = 2x$ where the domain is the set of real numbers. What is:

\begin{tabular}{l}
  a) $f(\Z) =$ set of positive integers \\
  b) $f(\N) =$ set of even positive integers \\
%  c) $f(\R) =$ set of real numbers \\
\end{tabular}
\end{solution}

\question[6] Determine if the following describe a function with the given domain and codomain.
\begin{parts}
    \part $f: \N \ra \N$ where $f(n)$ is equal to sum of the digits in $n$.
    \part $g: \N \ra \N$ where $f(n) = 7-n$.
    \part $h: \N \ra \Z$ where $f(n) = 7-n$.
\end{parts}
\begin{solution}
    \begin{parts}
    \part This is a function, each input value, there is one number that is the sum of the digits of $n$.
    \part This is not a function with codomain $\N$ because $f(8) = 7-8 = -1 \not \in \N$.
    \part This is a function.
    \end{parts}
\end{solution}

\uplevel{In the following problems recall that $\N = \{0, 1, 2, \ldots \}$, we will define a set $\N_i$ for each $i \in \N$ so the set $\N_i = \{ 0, 1, 2, \ldots, i-1 \}$.  For example, $\N_3 = \{ 0, 1, 2 \}$ and $\N_8 = \{0, 1, 2, 3, 4, 5, 6, 7 \}$. }

\question Let $f$ be the function $f: \N_5 \rightarrow \N_5$ defined by the formula $f(n) = (3n) \mod 5$.  For example, $f(4) = (3 \times 4) \mod 5 = 12 \mod 5 = 2$.

\begin{parts}
    \part[2] What is the domain of $f$?
    \part[2] What is the codomain of $f$?
    \part[2] What is the range of $f$?
    \part[2] Is $f$ one-to-one? If not, say why.
    \part[2] is $f$ onto? If not, say why.
    \part[8] \label{ref-q} Find a simple formula for each of the following function compositions: \\
    $f \circ f$, $f^3$, $f^4$, and $f^5$, \\
    where, the notation : $f \circ f$ is the same as $f^2$; $f \circ f \circ f$ is the same as $f^3$, etc.\\
    \textit{Hint:} Try creating a table for values of $n$ and the values of each function compositions, to      understand the relationships (and help answering the next questions.  Simple formulas to consider are of the    form $g(n) = (\alpha n) \mod 5$ for the lowest values of $\alpha$ possible, i.e., $\alpha = 1, 2, \ldots$.
    \part[2] Which of the functions from part (\ref{ref-q}) is the same as the identity function (the identity function     is the function $I$ such that $I(n) = n$ for all $n$)?
    \part[2] Which of the functions from part (\ref{ref-q}) is the same as $f^{-1}$?
    \part[2] Which one of the functions from part (\ref{ref-q}) is the same as $f$?
\end{parts}


\begin{solution}
\begin{parts}
    \part The domain of $f$ is $\N_5 = \{0, 1, 2, 3, 4\}$
    \part The codomain of $f$ is $\N_5 = \{0, 1, 2, 3, 4\}$
    \part The range of $f$ is $\N_5 = \{0, 1, 2, 3, 4\}$
    \part $f$ is one-to-one.
    \part $f$ is onto.
    \part The table below shows the relationship between the various function compositions \\
    \begin{tabular}{c|c|c|c|c|c}
        $n$ & $f$ & $f \circ f$ & $f^3$ & $f^4$ & $f^5$ \\
        \hline
        0   & 0 & 0 & 0 & 0 & 0\\
        1   & 3 & 4 & 2 & 1 & 3\\
        2   & 1 & 3 & 4 & 2 & 1\\
        3   & 4 & 2 & 1 & 3 & 4\\
        4   & 2 & 1 & 3 & 4 & 2\\
        \hline \hline
        n   & $(3n) \mod 5$ & $(4n) \mod 5$ & $(2n) \mod 5$ & $n$ & $(3n) \mod 5$ \\
        \hline \hline
    \end{tabular}

    We can identify the simple function using the following table \\
    \begin{tabular}{c|c|c|c|c}
         & $f(n) = n$ &  &   \\
        $n$ & $(1n) \mod 5$ & $(2n) \mod 5$ & $(3n) \mod 5$ & $(4n) \mod 5$   \\
        \hline
       0   & 0 & 0 & 0 & 0 \\
        1   & 1 & 2 & 3 & 4 \\
        2   & 2 & 4 & 1 & 3 \\
        3   & 3 & 1 & 4 & 2 \\
        4   & 4 & 3 & 2 & 1 \\
    \end{tabular}
    \part $f^4(n) = n$ for all $n \in \N_5$, so it is the identity function.
    \part $f^3(n) = f^{-1}(n)$
    \part $f^5(n) = f(n)$
\end{parts}
\end{solution}


\uplevel{The following problems will ask questions about \textit{keys} and \textit{locks}.  Assume Eric has a $a$ keys and $b$ locks, but he does not know which key opens a lock.}

\question[6] Assume that each key can open one and only one lock. \\
Let $a=4$ and $b=6$; assume all the $a$ keys are labeled $1, 2, 3, \ldots$ and the $b$ locks are labeled $I,II,III, \ldots$. Consider the ``open" function $f : Keys \rightarrow Locks$.
\begin{parts}
    \part With $a=4$ and $b=6$, can all locks be opened?
    \part With $a=4$ and $b=6$, is the ``open" function one-to-one? onto? Explain why or why not.
    \part What needs to be true about $a$ and $b$ so that every lock will be opened?
\end{parts}

\begin{solution}
    \begin{parts}
        \part No
        \part The function is one-to-one because each key only opens one lock; the function is not onto, because not every lock can be opened.
        \part $b$ needs to be greater than or equal to $a$ for every lock to be opened.
    \end{parts}
\end{solution}

\question[6] Assume that there are three keys that open the same lock and the remaining keys open one and only one lock.
\begin{parts}
    \part With $a=9$ and $b=7$, is Eric able to open all of the locks?
    \part With $a=9$ and $b=7$, is the ``open" function one-to-one? onto? Explain why or why not.
    \part What is the minimum number of keys to open $b$ locks?
\end{parts}

\begin{solution}
    \begin{parts}
        \part Yes
        \part The function is not one-to-one because three keys open the same lock; the function is onto, because every lock can be opened.
        \part $b+2$ keys
    \end{parts}
\end{solution}

\question[3] Assume Eric has a single key that can open exactly two locks;  each of the other keys can open one and only one lock.
    \begin{parts}
        \part Is the ``open" a function?
        \part If it is a function, is it one-to-one? onto?
    \end{parts}

\begin{solution}
    \begin{parts}
        \part No
        \part not applicable
    \end{parts}
\end{solution}

%\question[10] Computer memory is a large, 1-dimensional array.  However, in computer programs you can create a 2-dimensional array such as \texttt{a[3,4]}.  This array would actually be laid out as a sequence of 12 items in actual memory.  Using \textit{column-major order} (this is the ordering used in \textsc{Matlab, FORTRAN,} and \textsc{R} however, \textsc{C} uses \textit{row-major order}), the elements are laid out by columns in sequence starting with the first column of the array.  The complete 2-d array can be viewed as a 1-d array in the following order:
%
%\texttt{a[0,0], a[1,0], a[2,0], a[0,1], a[1,1], a[2,1], a[0,2], a[1,2], a[2,2], a[0,3], a[1,3], a[2,3]}
%
%The $0$th element is \texttt{a[0,0]}.  The 5th element is \texttt{a[2,1]}.
%
%From the machine perspective the array is accessed as if it is a 1-d array"
%
%\texttt{A[0], A[1], A[2], A[3], A[4], A[5], A[6], A[7], A[8], A[9], A[10], A[11], A[12]}
%
%Therefore, the computer must be able to translate from 1-d indexing to 2-d indexing and vice versa.  That is, convert a 2-d subscript pair (ex. \texttt{[2,1]}) to the 1-d subscript, \texttt{[5]}, and back.  To translate 2-d subscripts to 1-d, the function $f$ should take pairs as inputs and return an integer, s.t. in the example $f(2,1) = 5$.  The correct function for the example in general is $f(i,j) = 3j + i$, where 3 is the number of rows.  In general, for an array of size \texttt{a[n,m]}, the formula is $f(i,j) = nj + i$.
%
%There need to be two functions to translate from 1-d to 2-d subscripts.  The first gives the row number of the elements, lets call is $g$, and the other gives the column number, $h$.  In the example given, $g(5) = 2$ and $h(5) = 1$.  Create a simple general formula for these functions with any array \texttt{a[n,m]}.
%    \begin{parts}
%    \part Give a simple formula for $g$.
%    \part Give a simple formula for $h$.
%    \end{parts}
%
%\begin{solution}
%    \begin{parts}
%        \part $g(i) = i \;\;\text{div}\; n$
%        \part $h(i) = i \mod n$
%    \end{parts}
%\end{solution}

\uplevel{\textbf{Sequences and Summations}}

\question[4]  Rosen, Ch 2.4, \# 4(a,d), p. 167.
\begin{solution}
What are the term $a_0, a_1, a_2$, and $a_3$ of the sequence $\{a_n\}$, where $a_n$ equals \\

\begin{tabular}{l}
  (a) $(-2)^n$, $a_0 = 1, a_1 = -2, a_2 = 4, a_3 = -8$ \\
%  (b) 3, $a_0, a_1, a_2, a_3 = 3$  \\
%  (c) $7 + 4^n$, $a_0 = 8, a_1 = 11, a_2 = 23, a_3 = 71$ \\
  (d) $2^n + (-2)^n$, $a_0 = 2, a_1 = 0, a_2 = 8, a_3 = 0$ \\
\end{tabular}
\end{solution}


\question[6] Rosen Ch 2.4 \# 10(a,b), p. 168.
\begin{solution}
Find the terms or recurrence relations.
\begin{parts}
    \part $a_0 = -1$, $a_1 = -2a_0 = 2$, $a_2 = -2a_1 = -4$, $a_3 = -2a_2 = 8$, $a_4 = -2a_3 = -16$, $a_5 = -2a_4 = 32$
    \part $a_0 = 2$, $a_1 = -1$, $a_2 = a_1 - a_0 = -3$, $a_3 = a_2 - a_1 = -2$, $a_4 = a_3 - a_2 = 1$, $a_5 = a_4 - a_3 = -3$
%    \part $a_0 = 1$, $a_1 = 3a_0^2 = 3$, $a_2 = 3a_1^2 = 27 = 3^3$, $a_3 = 3a_2^2 = 2187 = 3^7$, $a_4 = 3a_3^2 = 14348907 = 3^{15}$, $a_5 = 3a_4^2 = 6.17 \times 10^{14} = 3^{31}$
\end{parts}
\end{solution}

\question[6] Rosen Ch 2.4 \# 16(a,c), p. 168.
\begin{solution}
Find the solution to the recurrence relations using an iterative approach.
\begin{itemize}
  \setlength{\itemsep}{1pt}
  \setlength{\parskip}{0pt}
  \setlength{\parsep}{0pt}
    \item[a)]
    \vskip -0.1in \begin{align*}
        a_n &= -a_{n-1} = -1\cdot a_{n-1} \\
          &= -(-a_{n-2}) = (-1)^2a_{n-2} \\
          &= -(-(-a_{n-3})) = (-1)^3a_{n-3} \\
          &= \cdots \\
          &= (-1)^n\cdot a_{n-n} = (-1)^n\cdot a_{0} = 5\cdot(-1)^n
    \end{align*}
%    \item[b)]
%    \begin{align*}
%        a_n &= a_{n-1} + 3 \\
%         &= a_{n-2} + 3 + 3 = a_{n-2} + 2\cdot 3 \\
%         &= a_{n-3} + 3 + 2\cdot 3 = a_{n-3} + 3\cdot 3 \\
%         &= \cdots \\
%         &= a_{n-n} + 3 + (n-1)\cdot 3 = a_{0} + n\cdot 3 = 1 + 3\cdot n
%    \end{align*}
    \item[c)]
    \vskip -0.1in \begin{align*}
        a_n &= a_{n-1} - n \\
         &= a_{n-2} -(n-1) - n = a_{n-1} - (n + (n-1)) \\
         &= a_{n-3} -(n-2) - (n + (n-1)) = a_{n-2} - (n + (n-1) + (n-2)) \\
         &= \cdots \\
         &= a_{n-n} -(n - (n-1) + \ldots + (n-1) + (n-1) + n) \\
         &= -\frac{n(n+1)}{2} + 4
    \end{align*}
\end{itemize}
\end{solution}

\question[15] Rosen Ch 2.4, \# 26(a,c,e,f,h), p. 169
\begin{solution}
For each of these lists of integers, provide a simple formula or rule that generates the terms of an integer sequence that begins with the given list.  Assuming that your formula or rule is correct, determine the next three terms of the sequence. \\
%\begin{parts}
%    \part $3, 6, 11, 18, 27, 38, 51, 66, 83, 102, \ldots $
%    \part $7, 11, 15, 19, 23, 27, 31, 35, 39, 43, \ldots $
%    \part $1, 10, 11, 100, 101, 110, 111, 1000, 1001, 1010, 1011, \ldots $
%    \part $1, 2, 2, 2, 3, 3, 3, 3, 3, 5, 5, 5, 5, 5, 5, 5, \ldots $
%    \part $0, 2, 8, 26, 80, 242, 728, 2186, 6560, 19682, \ldots $
%    \part $1, 3, 15, 105, 945, 10395, 135135, 2027025, 34459425, \ldots $
%    \part $1, 0, 0, 1, 1, 1, 0, 0, 0, 0, 1, 1, 1, 1, 1, \ldots $
%    \part $2, 4, 16, 256, 65536, 4294967296, \ldots $
%\end{parts}

\begin{parts}
    \part Look at the difference between terms $3, 5, 7, 11, 13, ...$.  The $n$th term is obtained by adding $2n-1$ to the previous term.  One could also observe that starting with $n=1$ then, $a_n = n^2 + 2$.  \\
    Using this formula the next three terms will be: \\
    3, 6, 11, 18, 27, 38, 51, 66, 83, 102, \textbf{123, 146, 171,} \ldots
    \part This is an arithmetic progression with an initial term of 7 and common difference of 4.  The $n$th term is $a_n = 7 + 4(n-1) = 4n + 3$ starting from $n=1$.  The next three terms will be: \\
    7, 11, 15, 19, 23, 27, 31, 35, 39, 43, \textbf{47, 51, 55,} \ldots
    \part Here the pattern is the sequence of increasing integers if counting in binary starting from 1. The $n$th term is then $a_n = binary(a_n)$ where binary(x) is the number $x$ expressed in base 2.  The next three terms will be: \\
    1, 10, 11, 100, 101, 110, 111, 1000, 1001, 1010, 1011, \textbf{1100, 1101, 1110,} \ldots
    \part This sequence has two patterns to recognize what number is listed, and how many times it appears.  The number of times a number appears is increasing odd, 1, 3, 5, 7, \ldots.  The number itself is the sum of the two previous numbers 3 = 1+2, 5 =2+3.   The next three terms will be:
    1, 2, 2, 2, 3, 3, 3, 3, 3, 5, 5, 5, 5, 5, 5, 5, \textbf{8, 8, 8,} \ldots
    \part  This sequence is close to the expression $3^n$.  In fact, the $n$th term is $a_n = 3^n - 1$, where $n=0, 1, 2, \ldots$.  The next three terms will be: \\
    0, 2, 8, 26, 80, 242, 728, 2186, 6560, 19682, \textbf{56048, 177146, 531440,}\ldots
    \part In this sequence, the ratio of a term divided by the previous term starts with 3 then increases by two as the terms increase.  That is, $\frac{3}{1} = 3$, $\frac{15}{3} = 5$, $\frac{105}{15} = 7$, $\frac{945}{105} = 9$, \ldots.  The general form is $a_n = a_{n-1} \cdot (2n -1)$.  The next three terms will be:
    1, 3, 15, 105, 945, 10395, 135135, 2027025, 34459425, \textbf{585810225, 11130394275, 233738279775,} \ldots

    \part The pattern is 1 appears once, the next number 0 appears twice, the next number 1 appears three times, then 0 four times, etc.  The next three terms will be: \\
    1, 0, 0, 1, 1, 1, 0, 0, 0, 0, 1, 1, 1, 1, 1, \textbf{0, 0, 0,} \ldots
    \part Every term is the square of the previous term. The next three terms are:
    2, 4, 16, 256, 65536, 4294967296, \textbf{ 18446744073709551616,}\\ \textbf{340282366920938463463374607431768211456,}\\ \footnotesize\textbf{115792089237316195423570985008687907853269984665640564039457584007913129639936}, \normalsize\ldots

\end{parts}
\end{solution}


\question[4] Rosen Ch. 2.4 \# 30(a,b) p. 169
\begin{solution}
What are the values of these sums, where $S = \{1, 3, 5, 7\}$ ?

\begin{tabular}{l}
  (a) $\displaystyle \sum_{j \in S} j = 16$ \\
  (b) $\displaystyle \sum_{j \in S} j^2 = 84$ \\
%  (c) $\displaystyle \sum_{j \in S} \frac{1}{j} = \frac{176}{105}$ \\
%  (d) $\displaystyle \sum_{j \in S} 1 = 4$ \\
\end{tabular}
\end{solution}

\question[4] Rosen Ch 2.4 \# 34(a,b), p. 169
\begin{solution}
Compute each of the double sums.

\begin{tabular}{l}
    (a) $\displaystyle \sum_{i=1}^2 \sum_{j=1}^2 (i-j) = 0$ \\
    (b) $\displaystyle \sum_{i=0}^3 \sum_{j=0}^2 (3i + 2j) = 78$ \\
\end{tabular}
\end{solution}


\question[4] Rosen Ch 2.4 \# 40, p. 169
\begin{solution}
Find $\sum_{k=99}^{200} k^3$. (Use Table 2, p. 157)

\begin{eqnarray*}
    \sum_{k=99}^{200} k^3 &=& \sum_{k=1}^{200} k^3 - \sum_{k=1}^{98} k^3 \\
      &=& \frac{200^2 \cdot 201^2}{4} - \frac{98^2 \cdot 99^2}{4} \\
      &=& 404010000 - 23532301 = 380477799
\end{eqnarray*}
\end{solution}



\end{questions}
\end{document}
