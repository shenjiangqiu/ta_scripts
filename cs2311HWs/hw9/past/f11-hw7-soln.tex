\documentclass[11pt]{exam}

\usepackage[top=0.75in,bottom=0.75in,left=1in,right=1in]{geometry}
%\usepackage{fancyhdr}
%\usepackage{mdwlist}
\usepackage{epsfig,graphicx}
\usepackage{amsmath,amssymb}
\usepackage{enumerate}


%\usepackage[normalmargins,normalsections,normalindent,normalleading]{savetrees}

\newcommand{\N}{\mathbf{N}}

\begin{document}
\extrawidth{0.5in}%
%\extrafootheight{-0.25in}%
\pagestyle{headandfoot}%
 \headrule
 \header{\textbf{cs2311 - Fall 2011}}{\textbf{HW 7 Solutions} (\numpoints$\;$ points)}{\textbf{Due: Fri. 10/21/11}}%
  \footrule \footer{}{Page \thepage\ of \numpages}{}

\addpoints

\noindent \textbf{Instructions:} All assignments are due at the
beginning of class on the due date specified.  Solutions will be
handed out (or posted on-line) shortly thereafter.  Every student
must write up their own solutions in their own manner.

\begin{questions}
\printanswers

\question[16] A \textit{closed form} of a summation is an equation
in which no summation symbol appears.  The classic example is
$\displaystyle \sum_{i=1}^n i = \frac{n(n+1)}{2}$.  The fraction on
the right is the \textit{closed form} of the summation.  For this
problem you will determine closed forms for summations.  %The four facts below will be useful:
%
%\begin{tabular}{lll}
%    \multicolumn{2}{l}{Fact} & Example: \\
%  Fact 0: & $\displaystyle \sum_{i=1}^n c  = nc$     & $\displaystyle \sum_{i=1}^n 7 = 7n$ \\
%    & when $c$ is a constant \\
%  Fact 0': & $\displaystyle \sum_{i=j}^n c  = (n-j+1)c$  & $\displaystyle \sum_{i=0}^n = 7(n+1) $ \\
%    & when $c$ is a constant \\
%  Fact 1: & $\displaystyle \sum_{i=j}^n (f(i) \pm g(i)) = \sum_{i=j}^n f(i) \pm \sum_{i=j}^n g(i)$ \hspace{0.2in} &  $\displaystyle \sum_{i=j}^n (2n - n^2) = \sum_{i=j}^n 2n - \sum_{i=j}^n n^2 $ \\
%    & \\
%  Fact 2: & $\displaystyle \sum_{i=j}^n cf(i) = c \sum_{i=j}^n f(i)$  & $\displaystyle \sum_{i=j}^n (3 \times 2^i) = 3 \sum_{i=j}^n 2^i $ \\
%   & when $c$ is a constant \\
%\end{tabular}

Find a closed form for the summation $\displaystyle \sum_{i=1}^n \sum_{j=1}^n (3i+ 2j)$.  Show how you find the closed form solution; justify each step.


\begin{solution}
\begin{tabular}{rll}
  $\displaystyle \sum_{i=1}^n \sum_{j=1}^n (3i+ 2j)$ & $\displaystyle = \sum_{i=1}^n \left[ \sum_{j=1}^n (3i + 2j) \right] $ & add implied parentheses \\
   & $\displaystyle = \sum_{i=1}^n \left[ \sum_{j=1}^n 3i + \sum_{j=1}^n 2j \right]$ & Fact 1 \\
   & $\displaystyle = \sum_{i=1}^n \left[ 3i \sum_{j=1}^n 1 + 2 \sum_{j=1}^n j \right]$ & Fact 2 and Fact 2 \\
   & $\displaystyle = \sum_{i=1}^n \left[ 3in + 2 \cdot \frac{n(n+1)}{2} \right]$ & Fact 0 and Table 2 \\
   & $\displaystyle = \sum_{i=1}^n 3in + \sum_{i=1}^n n(n+1)$ & Fact 1 \\
   & $\displaystyle = 3n \sum_{i=1}^n i + n(n+1) \sum_{i=1}^n 1$ & Fact 2 (twice) \\
   & $\displaystyle = 3n \cdot \frac{n(n+1)}{2} + n(n+1)\cdot n$ & Table 2 and Fact 0 \\
   & $\displaystyle = \frac{5n^2(n+1)}{2} $ & algebra \\
\end{tabular}
\end{solution}



\question[10] Rosen, Ch 5.1 \# 4, p. 329.
\begin{solution}
Let $P(n)$ be the statement that $1^3 + 2^3 + \cdots + n^3 = \left( \frac{n(n+1)}{2} \right)^2$ for the positive integer $n$.
\begin{parts}
    \part (1 point) What is the statement P(1)? \\
        $P(1)$ is the statement $ 1^3 = [ \frac{1 \cdot (1+1) }{2} ]^2 $.
    \part (1 point) Show that P(1) is true, completing the basis step of the proof.
        \[ [ \frac{1 \cdot (1+1) }{2} ]^2 = 1^2 = 1 = 1^3 \]
    \part (2 points) What is the inductive hypothesis? \\
        The inductive hypothesis is to assume $P(k)$ is true, that is
        \[ 1^3 + 2^3 + \cdots + k^3 = \left( \frac{k(k+1)}{2} \right)^2 \]
    \part (2 point) What do you need to prove in the inductive step? \\
        For the inductive step, show for each $k \geq 1$ that $P(k)$ implies $P(k+1)$. \\
        That is, show $P(k+1)$:
        \[ 1^3 + 2^3 + \cdots + k^3 + (k+1)^3 = \left( \frac{(k+1)(k+2)}{2} \right)^2 \]
    \part (3 points) Complete the inductive step. \\
        Since, $P(k)$ holds,
        \begin{align*}
            1^3 + 2^3 + \cdots + k^3 + (k+1)^3 &=  \left( \frac{k(k+1)}{2} \right)^2 + (k+1)^3  \tag{Ind. Hyp.} \\
              &= \left(\frac{k^2(k+1)^2}{2^2} \right) + (k+1)^3 \\
              &= (k+1)^2 \left( \frac{k^2}{4} + (k + 1) \right) \\
              &= (k+1)^2 \left( \frac{k^2 + 4k + 4}{4} \right) \\
              &= \frac{(k+1)^2(k+2)^2}{2^2} \\
        \end{align*}
        That is, $1^3 + 2^3 + \cdots +(k+1)^3 = \left( \frac{(k+1)(k+2)}{2} \right)^2 $, the $P(k+1)$ statement.  This completes the inductive step.
    \part (1 point) Explain why these steps show that this formula is true whenever $n$ is a positive integer. \\
        The basis step and inductive step are completed.  Therefore by principle of mathematical induction, the statement is true for every positive integer $n$.
\end{parts}
\end{solution}


\question[10] Rosen Ch 5.1 \# 12, p. 330.
\begin{solution}
Prove that $\displaystyle \sum_{j=0}^{n} \left( \frac{-1}{2} \right)^j = \frac{2^{n+1} + (-1)^n}{3\cdot 2^n}$, whenever $n \geq 0$.

\medskip
Let $P(n)$ be $\displaystyle \sum_{j=0}^{n} \left( \frac{-1}{2} \right)^j = \frac{2^{n+1} + (-1)^n}{3\cdot 2^n}$ for $n \geq 0$.

\textit{Basis Step:} Show $P(0)$ is true,
\[ \left( \frac{-1}{2} \right)^0 = 1 = \frac{2^{0+1} + (-1)^0}{3\cdot 2^0} = \frac{2+1}{3} = 1 \]
this shows $P(0)$ is true. \\

\textit{Inductive Step:} Assume for some $k \geq 0$, $P(k)$ is true:
\[ \displaystyle \sum_{j=0}^{k} \left( \frac{-1}{2} \right)^j = \frac{2^{k+1} + (-1)^k}{3\cdot 2^k}\;\; \text{for}\; n \geq 0 \]
Show that $P(k+1)$ is true:
\[ \displaystyle \sum_{j=0}^{k+1} \left( \frac{-1}{2} \right)^j = \frac{2^{k+2} + (-1)^{k+1}}{3\cdot 2^{k+1}} \]
\begin{align*}
    \sum_{j=0}^{k+1} \left( \frac{-1}{2} \right)^j
    & = \sum_{j=0}^{k} \left( \frac{-1}{2} \right)^j + \left( \frac{-1}{2} \right)^{k+1} \tag{def. of summ.} \\
    & = \frac{2^{k+1} + (-1)^k}{3\cdot 2^k} + \frac{-1^{k+1}}{2^{k+1}} \tag{Ind. Hyp.}\\
    & = \frac{2}{2}\cdot\frac{2^{k+1} + (-1)^k}{3\cdot 2^k}
        + \frac{3}{3}\cdot\frac{(-1)^{k+1}}{2^{k+1}} \tag{algebra} \\
    & = \frac{2^{k+2} + 2\cdot(-1)^k}{3\cdot 2^{k+1}}
        + \frac{3\cdot (-1)^{k+1}}{3\cdot 2^{k+1}} \\
    & = \frac{2^{k+2} + 2\cdot(-1)^k + 3\cdot (-1)\cdot (-1)^k}{3\cdot 2^{k+1}} \\
    & = \frac{2^{k+2} + (-1)\cdot (-1)^k}{3\cdot 2^{k+1}} \\
    & = \frac{2^{k+2} + (-1)^{k+1}}{3\cdot 2^{k+1}}
\end{align*}
We use $P(k)$ to show $P(k+1$ is true, completing the inductive step.

The basis and inductive step are completed.  Therefore, by mathematical induction, $P(n)$ is true, for $n \geq 0$.
\end{solution}

\question[10] Rosen Ch 5.1 \# 20, p. 330.
\begin{solution}
Prove that $3^n < n!$ if $n$ is an integer greater than 6.

\smallskip
    Let $P(n)$ be $3^n < n!$ for $n > 6$.

    \textit{Basis Step}: Look at $P(7)$, $3^7 = 2187$ and $7! = 5040$, so $P(7)$ is true $2187 < 5040$. \\

    \textit{Inductive Step}:  Assume for some $k > 6$, $P(k)$ is true.
    \[ 3^k < k! \]
    Show that $P(k+1)$ is true.
    \[ 3^{k+1} < (k+1)! \]
    $3^{k+1} = 3 \cdot 3^{k} < (k+1) \cdot 3^k < (k+1)k! = (k+1)!$
    This completes the inductive step.  By mathematical induction, $P(n)$ is true, for integers greater than 6.
\end{solution}


\question[10] Rosen Ch 5.2 \# 4, p. 341-342.
\begin{solution}
\begin{parts}
    \part Show $P(18)$, $P(19)$, $P(20)$, $P(21)$ are true.

        \begin{tabular}{l}
           $7 + 7 + 4 = 18$ \\
           $7 + 4 + 4 + 4 = 19$\\
           $4 + 4 + 4 + 4 + 4 = 20$\\
           $7 + 7 + 7 = 21$
        \end{tabular}
    \part What is the inductive hypothesis? \\
        Using just 4-cent and 7-cent stamps we can form $j$ cents of postage, $P(j)$ for all $j$ with $ 18 \leq j \leq k$, assuming $k \geq 21$.
    \part What do you need to prove in the inductive step? \\
        Show, assuming the inductive hypothesis, that we can form $k+1$ cents using only 4-cent and 7-cent stamps, $P(k+1)$.
    \part Complete the inductive step. \\
        From the hypothesis, we can assume $P(k-3)$ is true, that is we can form $k-3$ cents of postage.  To form $k+1$ cents, add another 4-cent stamp.
    \part Explain why this statement is true for $n \geq 18$. \\
        We have completed both the basis step and inductive step, therefore, by strong induction, the statement is true for all $n \geq 18$.
\end{parts}
\end{solution}




\question[8] Rosen Ch 5.3 \# 4a,c, p. 357.
\begin{solution}
Find $f(2)$, $f(3)$, $f(4)$, and $f(5)$ if $f$ is defined
recursively by $f(0) = f(1) = 1$ and for $n=1,2,\ldots$
\begin{itemize}
    \item[(a)] $f(n+1) = f(n) - f(n-1)$
    \begin{align*}
        f(2) &= f(1) - f(0) = 1 - 1 = 0 \\
        f(3) &= f(2) - f(1) = 0 - 1 = -1 \\
        f(4) &= f(3) - f(2) = -1 - 0 = -1 \\
        f(5) &= f(4) - f(3) = -1 - -1 = 0 \\
    \end{align*}
%    \item $f(n+1) = f(n)f(n-1)$
%    \begin{quote}
%    \begin{align*}
%        f(2) &= f(1)f(0) = 1 \\
%        f(3) &= f(2)f(1) = 1 \\
%        f(4) &= f(3)f(2) = 1 \\
%        f(5) &= f(4)f(3) = 1 \\
%    \end{align*}
%    \end{quote}
    \item[(c)] $f(n+1) = f(n)^2 + f(n-1)^3$.
    \begin{align*}
        f(2) &= f(1)^2 + f(0)^3 = 1^2 + 1^3 = 2 \\
        f(3) &= f(2)^2 + f(1)^3 = 2^2 + 1^3 = 5 \\
        f(4) &= f(3)^2 + f(2)^3 = 5^2 + 2^3 = 33 \\
        f(5) &= f(4)^2 + f(3)^2 = 33^2 + 5^3 = 1214 \\
    \end{align*}
\end{itemize}
\end{solution}


\question[4] Rosen Ch 5.3 \# 8a p. 358.
\begin{solution}
Give a recursive definition of the sequence $\{a_n\}$, $n=1,2,3,
\ldots$ if
\begin{itemize}
    \item[(a)] $a_n = 4n-2$
    \begin{quote}
        The sequence contains the following number $a_1 = 2$, $a_2 = 6$, $a_3 = 10$, $a_4 = 14$, ...
        A recursive definition (there are many such definitions) is
        \[  a_1 = 2, \;\; a_{n+1} = a_n + 4 \;\; \forall n \geq 2 \]
    \end{quote}
%    \item[(c)] $a_n = n(n+1)$.
%    \begin{quote}
%        This sequences is as follows: $a_1 = 2$, $a_2 = 6$, $a_3 = 12$, $a_4 = 20$, \ldots
%        A recursive definition is
%        \[ a_1 = 2, \;\; a_{n} = a_{n-1} + 2n \]
%    \end{quote}
\end{itemize}
\end{solution}


\question[6] Rosen Ch 5.3 \# 24(a,b), p. 358.
\begin{solution}
    \begin{parts}
        \part initial condition: $1 \in S$; recursive rule:  if $n \in S$, then $n+1 \in S$
        \part initial condition: $3 \in S$; recursive rule: if $n \in S$, then $3n \in S$.
    \end{parts}
\end{solution}


\question[6] Rosen Ch 5.3 \# 26a, p. 358. (See book examples and \#
27 to help with this problem)
\begin{solution}
Let $S$ be the subset of the set of ordered pairs of integers defined recursively by \\
\textit{Basis Step}: $(0,0) \in S$. \\
\textit{Recursive Step}: If $(a,b) \in S$, then $(a+2,b+3)\in S$ and $(a+3,b+2) \in S$. \\
\begin{enumerate}
    \item List the elements of $S$ produced by the first five applications of the recursive definition.
\end{enumerate}
\begin{quote}
    From the basis step $(0,0) \in S$.  Each application of the recursive definition will be given:

    \begin{tabular}{rllllll}
    1 & (2,3), & (3,2) \\
    2 & (4,6), & (5,5), & (6,4) \\
    3 & (6,9), & (7,8), & (8,7), & (9,6) \\
    4 & (8,12), & (9,11), & (10,10), & (11,9), & (12,8) \\
    5 & (10,15), & (11,14), & (12,13), & (13,12), & (14,11), & (15,10) \\
    \end{tabular}
\end{quote}
\end{solution}




\end{questions}
\end{document}
