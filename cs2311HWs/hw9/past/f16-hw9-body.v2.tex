\begin{document}
\extrawidth{0.5in} \extrafootheight{-0in} \pagestyle{headandfoot}
\headrule \header{\textbf{cs2311 - Fall 2016}}{\textbf{HW
 9 \ifprintanswers - Solutions \fi}}{\textbf{Due: Mon. 11/7/16}} \footrule \footer{}{Page \thepage\
of \numpages}{}

\ifprintanswers
\noindent \textbf{Instructions:} All assignments are due \underline{by \textbf{midnight} on the due date} specified.  Assignments should be typed and submitted as a PDF.  Every student must write up their own solutions in their own manner.

\medskip
\noindent You should \underline{complete all problems}, but \underline{only a subset will be graded} (which will be graded is not known to you ahead of time). 
\else
\noindent \textbf{Instructions:} All assignments are due \underline{by \textbf{midnight} on the due date} specified.  Every student must write up their own solutions in their own manner.

\noindent Please present your solutions in a clean, understandable
manner.  Use the provided files that give mathematical notation in Word, Open Office, Google Docs, and \LaTeX. 

\noindent Assignments should be typed and submitted as a PDF.   

\noindent You should \underline{complete all problems}, but \underline{only a subset will be graded} (which will be graded is not known to you ahead of time). 
\fi

\begin{questions}


\section*{Proofs}




\ugquestion{13} For an integer $a$, then $a^2 + 4a + 5$ is odd if and only if  $a$ is even. 
    \ifprintanswers
        \vspace{-10pt}
    \fi
\begin{solution}
     This proof is for a theorem using ``if and only if" therefore,
     it must be proved in both directions.  \\
    \textbf{Proof ``if p, then q" by contradiction:} Assume $a^2 + 4a + 5$ is
    odd and $a$ is odd.  Then, by definition of odd, there exists an integer $k$ such that $a = 2k+1$.  Consider the expression, 
    \[ a^2 + 4a + 5 = (2k+1)^2 + 4(2k+1) + 5 = 4k^2 + 4k + 1 + 8k + 4 + 5 = 4k^2 + 12k + 10 = 2(2k^2 + 6k + 5) = 2k', \]
    which is of the form of an even number.  This contradicts the assumption that $a^2 + 4a + 5$ is odd.  Therefore, if $a^2 + 4a + 5$ is odd, then $a$ is even. 


    \textit{Alternative method}\textbf{ Proof by contrapositon:}  Assume $a$ is odd.  Then, by by definition of odd, there exists an integer $k$ such that $a = 2k+1$.  Consider the expression, 
    \[ a^2 + 4a + 5 = (2k+1)^2 + 4(2k+1) + 5 = 4k^2 + 4k + 1 + 8k + 4 + 5 = 4k^2 + 12k + 10 = 2(2k^2 + 6k + 5) = 2k', \]
    which is of the form of an even number (the negation of the hypothesis).  Therefore, using proof by contraposition, if $a^2 + 4a + 5$ is odd, then $a$ is even.  

    \smallskip
    \textbf{Prove ``if q then p" directly:} Assume $a$ is even.  By definition of even, $a = 2k$ for some integer $k$.  The expression $a^2 + 4a + 5$ becomes 
    \[ a^2 + 4a + 5 = (2k)^2 + 4(2k) + 5 = 4k^2 + 8k + 5 = 2(2k^2 + 4k + 2) + 1 = 2k' + 1, \]
    which is of the form of an odd integer.  Therefore, if $a$ is even, then $a^2 + 4a + 5$ is odd. 

    \smallskip
    Combining both parts, we have shown for an integer $a$, $a^2 + 4a + 5$ is odd if and only if  $a$ is even. 
\end{solution}


\vspace{-20pt}
\section*{Sequences}


\gquestion{10}{4}{d-e} What are the first four terms of each sequence: 

\begin{minipage}{0.5\textwidth}
\begin{enumerate}[label=(\alph*),itemsep=0pt,parsep=0pt,topsep=0pt]
  \item $a_n = 5 + 4n \quad \forall n \geq 0$ \hspace{0.25in} 
  \item $b_n = 6 - 2\cdot3^{n-1} \quad \forall n \geq 0$ 
  \item $c_0 = 3, c_n = 2\cdot c_{n-1} + 3 \quad \forall n \geq 1$ 
  \item $d_1 = -1, d_n = 3\cdot d_{n-1} + n \quad \forall n \geq 2$ 
  \item $e_0 = 1, e_1 = 1$, $e_n = ne_{n-1} + n^2e_{n-2} + 1\;\forall n \geq 2$
\end{enumerate}
\end{minipage}
% 
\begin{minipage}{0.5\textwidth}
\begin{solution}
\begin{enumerate}[label=(\alph*),itemsep=0pt,parsep=0pt,topsep=0pt]
  \item $a_0 = 5$, $a_1 = 9$, $a_2 = 13$, $a_3 = 17$
  \item $b_0 = 4$, $b_1 = 0$, $b_2 = -12$, $b_3 = -48$
  \item $c_0 = 3$, $c_1 = 9$, $c_2 = 21$, $c_3 = 45$
  \item $d_1 = -1$, $d_2 = -1$, $d_3 = 0$, $d_4 = 4$ 
  \item $e_0 = 1$, $e_1 = 1$, $e_2 = 6$, $e_3 = 27$
\end{enumerate}
\end{solution}
\end{minipage}





\gquestion{10}{6}{a-c} Find a closed formula for each sequence; assume the sequence starts with $n=0, 1, 2, \ldots$. 

\begin{minipage}{0.5\textwidth}
\begin{enumerate}[label=(\alph*),itemsep=0pt,parsep=0pt,topsep=0pt]
  \item $7, 14, 28, 56, 112, 224, \ldots$
  \item $9, 4, -1, -6, -11, -16, \ldots$
  \item $3, 4, 7, 12, 19, 28, 39 \ldots$
  \item $2, 6, 18, 54, 162, 486, 1458\ldots$
  \item $1, 0, 1, -4, 9, -16, 25, -36 \ldots$
\end{enumerate}
\end{minipage}
% 
\begin{minipage}{0.5\textwidth}
\begin{solution}
\begin{enumerate}[label=(\alph*),itemsep=0pt,parsep=0pt,topsep=0pt]
  \item $a_n = 7\cdot 2^n$
  \item $a_n = 9 - 5n$
  \item $a_n = n^2 + 3$
  \item $a_n = 2\cdot 3^n$ 
  \item $a_n = (n-1)^2 (-1)^n$
\end{enumerate}
\end{solution}
\end{minipage}



\gquestion{8}{4}{a-b} Find a recursive formula for each sequence; assume the sequence with $n=0, 1, 2, \ldots$.

\begin{minipage}{0.5\textwidth}
\begin{enumerate}[label=(\alph*),itemsep=0pt,parsep=0pt,topsep=0pt]
  \item $7, 3, -1, -5, -9, -13, \ldots$
  \item $4, 12, 36, 108, 324, 972, \ldots$
  \item $2, 4, 7, 11, 16, 22, \ldots$
  \item $2, 5, 13, 42, 171, 858, 5151, \ldots$
\end{enumerate}
\end{minipage}
% 
\begin{minipage}{0.5\textwidth}
\begin{solution}
\begin{enumerate}[label=(\alph*),itemsep=0pt,parsep=0pt,topsep=0pt]
  \item $a_0 = 7$, $a_n = a_{n-1} -4$
  \item $a_0 = 6$, $a_n = 3\cdot a_{n-1}$
  \item $a_0 = 2$, $a_n = a_{n-1} + n + 1$
  \item $a_0 = 2$, $a_n = n\cdot a_{n-1} + 3$
\end{enumerate}
\end{solution}
\end{minipage}



\gquestion{6}{3}{b} Determine whether each answer is a solution to the recurrence relation, 
$$ a_n = a_{n-1} + 2a_{n-2} + 2n - 9 $$
    \begin{enumerate}[label=(\alph*),topsep=0pt,itemsep=0pt,parsep=0pt]
        \item $a_n = 0$
        \item $a_n = -n +2$
    \end{enumerate}
    \ifprintanswers
        \vspace{-10pt}
    \fi
\begin{solution}
    % \begin{enumerate}[label=(\alph*),topsep=0pt,itemsep=0pt,parsep=0pt]
    %     \item Not a solution.
    %     \begin{align*}
    %         a_n &= a_{n-1} + 2a_{n-2} + 2n - 9 \\
    %         &= (0) + 2\cdot0 + 2n - 9 \\
    %         &= 2n - 9 \\
    %         0 &\neq 2n - 9
    %     \end{align*}
    %     \item Yes, $a_n = -n + 2$ is a solution.
    %     \begin{align*}
    %         a_n &= a_{n-1} + 2a_{n-2} + 2n - 9 \\
    %         &= (-(n-1) + 2) + 2(-(n-2) + 2) + 2n -9 \\
    %         &= -n + 1 + 2 -2n + 4 + 4 + 2n -9 \\
    %         &= -n +2 \\
    %     \end{align*}
    % \end{enumerate}
    \begin{minipage}{0.5\textwidth}
      (a) Not a solution. 
        \begin{align*}
            a_n &= a_{n-1} + 2a_{n-2} + 2n - 9 \\
            &= (0) + 2\cdot0 + 2n - 9 \\
            &= 2n - 9 \\
            0 &\neq 2n - 9
        \end{align*}
    \end{minipage}
    % 
    \begin{minipage}{0.5\textwidth}
      (b) Yes, $a_n = -n + 2$ is a solution.
        \begin{align*}
            a_n &= a_{n-1} + 2a_{n-2} + 2n - 9 \\
            &= (-(n-1) + 2) + 2(-(n-2) + 2) + 2n -9 \\
            &= -n + 1 + 2 -2n + 4 + 4 + 2n -9 \\
            &= -n +2 \\
        \end{align*}
    \end{minipage}
\end{solution}



\gquestion{18}{6}{c} Rosen Ch 2.4 \# 16(a,c,e), p. 168.
    \ifprintanswers
        \vspace{-10pt}
   \fi
\begin{solution}
    \begin{itemize}[itemsep=0pt,parsep=0pt,topsep=0pt,partopsep=0pt]
        \item[]
        \small \vspace{-0.2in}
        \begin{align*}
            (a) \quad \quad a_n &= -a_{n-1} = -1\cdot a_{n-1} \\
              &= -(-a_{n-2}) = (-1)^2a_{n-2} \\
              &= -(-(-a_{n-3})) = (-1)^3a_{n-3} \\
              &= \cdots \\
              &= (-1)^n\cdot a_{n-n} = (-1)^n\cdot a_{0} = 5\cdot(-1)^n
        \end{align*}

        an alternative answer is 
        $ a_n = \begin{cases}
                  5 & \text{if n is even} \\
                  -5 & \text{if n is odd} 
                \end{cases} $
        \item[]
        \vspace{-0.1in} 
        \begin{align*}
            (c) \quad \quad a_n &= a_{n-1} - n \\
            &= a_{n-2} -(n-1) - n = a_{n-1} - (n + (n-1)) \\
            &= a_{n-3} -(n-2) - (n + (n-1)) = a_{n-2} - (n + (n-1) + (n-2)) \\
            &= \cdots \\
            &= a_{n-n} -(n + (n-1) + (n-2) + \ldots + (n- (n-1)) \\
            &= a_0 - (n + (n-1) + (n-2) + \ldots + 1) \\
            &= 4 - (n + (n-1) + (n-2) + \ldots + 1) = 4 -\frac{n(n+1)}{2}
        \end{align*}
    % \end{itemize}
        \vspace{-0.1in}
        \item[]
        \begin{align*}
            (e) \quad \quad a_n &= (n+1)a_{n-1} \\
              &= (n+1)na_{n-2} \\
              &= (n+1)n(n-1)a_{n-3} \\
              &= \cdots \\
              &= (n+1)n(n-1)\cdots(n-(n-2))a_{n-n} \\
              &= (n+1)n(n-1)\cdots2a_{0} \\
              &= 2(n+1)!
        \end{align*}
    \end{itemize}
\end{solution}



\ugquestion{6} A mortage loan is paid off in periodic (monthly) installments, while interest is also charged each period.  A mortage with an annual interest rate of $r$ has a monthly interest rate of $i = \frac{r}{12}$.  A mortage of $M$ dollars at monthly interest rate $i$ has payments of $P$ dollars.  At the end of the month, interest is added to the previous balance, and then the payment $P$ is subtracted from the result.  Let $m_n$ be the balance due after $n$ months, where $m_0 = M$, $m_1 = M(1+i) - P, \ldots$
\begin{enumerate}[label=(\alph*),itemsep=0pt,parsep=0pt,
    topsep=0pt,partopsep=0pt]
    \item Let $M = 10,000$, $i = 0.03$ and $P = 105.13$, determine $m_2$ and $m_3$.
    % \item If $m_{20} = 79,495.98$, $i=0.02$ and $R = 822.89$, determine $m_{21}$.
    \item Find a recursive formula for the mortage balance, $m_n$
\end{enumerate}
    \ifprintanswers
        \vspace{-10pt}
   \fi
\begin{solution}
    \begin{enumerate}[label=(\alph*),itemsep=0pt,parsep=0pt,
    topsep=0pt,partopsep=0pt]
        \item $m_0 = 10,000$, $m_1 = 10194.87$, $m_2 = 10395.59$, $m_3 =  10602.32$
        \item $m_n = m_{n-1}(1+i) - P$, $m_o = M$
    \end{enumerate}
\end{solution}


\section*{Summations}


\gquestion{18}{9}{c-e} What are the values of the sums:\\
\textit{Hint: think about using some of the properties and closed-forms of summations to make the computations easier.} \\
\begin{tabular}{lll}
    (a) $\ds \sum_{j\in S} (2j - 1), \text{ where } S = \{1, 3, 8, 13\}$
    & (b) $\ds \sum_{i=1}^{80} 7$  \hspace*{1in} 
    & (c) $\ds \sum_{i=1}^{32} i$ \\
    % & (c) $\ds \sum_{k=0}^{4} (-3)^k$ \\
    & & \\
    (d) $\ds \sum_{j=0}^{9} (3^{j} - 2^{j})$
    & (e) $\ds \sum_{j=1}^{4} \sum_{k=-2}^{2} (2k + 3j)$
    & (f) $\ds \sum_{j=-1}^{1} \sum_{k=1}^{3} j*2^{k}$
\end{tabular}
    \ifprintanswers
        \vspace{-10pt}
   \fi
\begin{solution}
    \begin{enumerate}[label=(\alph*),itemsep=2pt,parsep=0pt,
    topsep=0pt,partopsep=0pt]
        \item $(2\cdot1 - 1)+(2\cdot3 - 1)+(2\cdot8 -1)+(2\cdot13-1) = 1+5+15+25 = 46$ 
        \item $\ds \sum_{i=1}^80 7 = 80\cdot7 = 560$
        \item $\ds \sum_{i=1}^{32} i = \frac{32(32+1)}{2}  = 528$
        \item $\ds \sum_{j=0}^9 3^{j} - \sum_{j=0}^9 2^{j} = 29523 -1022 = 28501$
        \item $\ds \sum_{j=1}^4 \left( \sum_{k=-2}^2 2k + \sum_{k=-2}^2 3j \right) = \sum_{j=1}^4 \left( 0 + 3j\cdot(2-(-2)+1)\cdot1 \right)  = \sum_{j=1}^4 15j = 150$
        \item $\sum_{j=-1}^1 \sum_{k=1}^3 j*2^k = 0$
    \end{enumerate}
\end{solution}




\uplevel{\ifprintanswers 
\else 
A \textit{closed form} of a summation is an equation
in which no summation symbol appears.  The classic example is
$\displaystyle \sum_{i=1}^n i = \frac{n(n+1)}{2}$.  The fraction on
the right is the \textit{closed form} of the summation. 
\fi

\ifprintanswers
\small
\begin{tabular}{llll}
    \multicolumn{2}{l}{Fact} & Example: \\
  Fact 1: & $\displaystyle \sum_{i=1}^n c  = nc$     & Fact 2: & $\displaystyle \sum_{i=j}^n c  = (n-j+1)c$ \\
    & when $c$ is a constant & & when $c$ is a constant \\ 
  Fact 3: & $\displaystyle \sum_{i=j}^n (f(i) \pm g(i)) = \sum_{i=j}^n f(i) \pm \sum_{i=j}^n g(i)$ \hspace{0.2in} & Fact 4: & $\displaystyle \sum_{i=j}^n cf(i) = c \sum_{i=j}^n f(i)$ \\
    & & & when $c$ is a constant \\
\end{tabular}
\else
\small
\begin{tabular}{lll}
    \multicolumn{2}{l}{Fact} & Example: \\
  Fact 1: & $\displaystyle \sum_{i=1}^n c  = nc$     & $\displaystyle \sum_{i=1}^n 7 = 7n$ \\
    & when $c$ is a constant \\
  Fact 2: & $\displaystyle \sum_{i=j}^n c  = (n-j+1)c$  & $\displaystyle \sum_{i=0}^n = 7(n+1) $ \\
    & when $c$ is a constant \\
  Fact 3: & $\displaystyle \sum_{i=j}^n (f(i) \pm g(i)) = \sum_{i=j}^n f(i) \pm \sum_{i=j}^n g(i)$ \hspace{0.2in} &  $\displaystyle \sum_{i=j}^n (2n - n^2) = \sum_{i=j}^n 2n - \sum_{i=j}^n n^2 $ \\
    & \\
  Fact 4: & $\displaystyle \sum_{i=j}^n cf(i) = c \sum_{i=j}^n f(i)$  & $\displaystyle \sum_{i=j}^n (3 \times 2^i) = 3 \sum_{i=j}^n 2^i $ \\
   & when $c$ is a constant \\
\end{tabular}
\fi
}

% \gquestion{8}{8}{}   For this
% problem, you will determine closed forms for summation, $\displaystyle \sum_{i=1}^n \sum_{j=1}^n (3i+ 2j)$.  Justify each step using the four facts of the arithmetic properties of summations and Table 2, p. 166 in the book.
%     \ifprintanswers
%         \vspace{-10pt}
%    \fi
% \begin{solution}
% \begin{tabular}{rll}
%   $\displaystyle \sum_{i=1}^n \sum_{j=1}^n (3i+ 2j)$ & $\displaystyle = \sum_{i=1}^n \left[ \sum_{j=1}^n (3i + 2j) \right] $ & add implied parentheses \\
%    & $\displaystyle = \sum_{i=1}^n \left[ \sum_{j=1}^n 3i + \sum_{j=1}^n 2j \right]$ & Fact 3 \\
%    & $\displaystyle = \sum_{i=1}^n \left[ 3i \sum_{j=1}^n 1 + 2 \sum_{j=1}^n j \right]$ & Fact 4 and Fact 4 \\
%    & $\displaystyle = \sum_{i=1}^n \left[ 3i\cdot n + 2 \sum_{j=1}^n j \right]$ & Fact 1  \\
%    & $\displaystyle = \sum_{i=1}^n \left[ 3i\cdot n + 2 \cdot \frac{n(n+1)}{2} \right]$ & Table 2 \\
%    & $\displaystyle = \sum_{i=1}^n 3i\cdot n + \sum_{i=1}^n n(n+1)$ & Fact 3 \\
%    & $\displaystyle = 3n \sum_{i=1}^n i + n(n+1) \sum_{i=1}^n 1$ & Fact 4 (twice) \\
%    & $\displaystyle = 3n \cdot \frac{n(n+1)}{2} + n(n+1) \sum_{i=1}^n 1$ & Table 2 \\
%    & $\displaystyle = 3n \cdot \frac{n(n+1)}{2} + n(n+1)\cdot n$ & Fact 1 \\
%    & $\displaystyle = \frac{5n^2(n+1)}{2} $ & algebra \\
% \end{tabular}
% \end{solution}

\gquestion{8}{8}{all} Find a closed form for the summation $\displaystyle \sum_{i=1}^n \sum_{j=1}^n (6i^2 - 2j)$.  Show how you find the closed form solution; justify each step using the four facts of the arithmetic properties of summations and Table 2, p. 166 in the book.
    \ifprintanswers
        \vspace{-10pt}
   \fi
\begin{solution}
\begin{align*}
        \sum_{i=1}^n \sum_{j=1}^n (6i^2 - 2j) 
        &= \sum_{i=1}^n ( \sum_{j=1}^n (6i^2 - 2j) ) \tag{implied parentheses} \\
        &= \sum_{i=1}^n (\sum_{j=1}^n 6i^2 - \sum_{j=1}^n 2j) \tag{Fact 3} \\
        &= \sum_{i=1}^n (6i^2 \sum_{j=1}^n 1 - 2 \sum_{j=1}^n j) \tag{Fact 4, twice} \\
        &= \sum_{i=1}^n (6i^2n - 2 \sum_{j=1}^n j) \tag{Fact 1} \\
        &= \sum_{i=1}^n (6i^2n - 2 \frac{n(n+1)}{2}) \tag{Table 2} \\
        &= \sum_{i=1}^n 6i^2n - \sum_{i=1}^n  n(n+1) \tag{Fact 3} \\
        &= 6n \sum_{i=1}^n i^2 - n(n+1) \sum_{i=1}^n 1 \tag{Fact 4, twice} \\
        &= 6n (\frac{n(n+1)(2n+1)}{6}) - n(n+1) \sum_{i=1}^n 1 \tag{Table 2} \\
        &= n^2(n+1)(2n+1) - n^2(n+1) \tag{algebra} \\
        &= n^2(n+1)(2n + 1  - 1) \\
        &= 2n^3(n+1) 
    \end{align*}
\end{solution}




\ugquestion{8}{8} Find a closed form for the summation $\displaystyle \sum_{i=5}^n 3^i$.  Show how you find the closed form solution; justify each step using the four facts of the arithmetic properties of summations (listed above) and Table 2, p. 166 in the book.
    \ifprintanswers
        \vspace{-10pt}
   \fi
\begin{solution}
    \begin{align*}
        \sum_{i=5}^n 3^i  & = \sum_{j=0}^{n-5} 3^{j+5} \tag{change of index} \\
        & = \sum_{j=0}^{n-5} 3^{j}3^{5} \tag{algebra} \\
        & = 3^5 \sum_{j=0}^{n-5} 3^j    \tag{Fact 4} \\
        & = 3^5 \left[ \frac{3^{n-5+1} - 1}{3 - 1} \right] \tag{Table } \\
        & = \frac{ 3^5(3^{n-4} - 1) }{ 2 } \tag{algebra}
    \end{align*}

    Alternatively, 
    \begin{align*}
        \sum_{i=5}^n 3^i &= \sum_{i=0}^n 3^i - \sum_{i=0}^4 3^i \tag{change index} \\
        &= \frac{3^{n+1} - 1}{3 - 1} - \frac{3^{4+1} - 1}{3 -1} \tag{Table 2}\\
        &= \frac{3^{n+1} - 3^{5}}{2} = \frac{3^5(3^{n-4} - 1)}{2} \tag{algebra} \\
    \end{align*}
\end{solution}




\ugquestion{4} Rosen Ch 2.4 \#40, p. 169
    \ifprintanswers
        \vspace{-10pt}
   \fi
\begin{solution}
 We can find $\ds \sum_{k=1}^{200} k^3 = \frac{200^2(201)^2}{4}$.
 Also, $\ds \sum_{k=1}^{98} k^3 = \frac{98^2(99)^2}{4}$.

 Therefore,
$$ \sum_{k=99}^{200} = \sum_{k=1}^{200} k^3 - \sum_{k=1}^{98} k^3
   = \frac{200^2(201)^2}{4} - \frac{98^2(99)^2}{4} 
   = 380,477,799 $$
\end{solution}




\section*{Bonus Questions} 


\bonusquestion[4] Give a direct proof that for all real numbers $x$ and $y$,
if $x$ and $y$ are rational, then $x-y$ is rational.
    \ifprintanswers
        \vspace{-10pt}
    \fi
\begin{solution} \textbf{Proof:} Assume $x$ and $y$ are rational. Then, there are
    integers $p, q\neq 0, r$, and $s\neq 0$ such that
    $x = \frac{p}{q}$ and $y = \frac{r}{s}$ using the definition of rational.  Then,
    \[ x - y = \frac{p}{q} - \frac{r}{s} = \frac{ps - qr}{qs} \]
    Since $ps - qr$ and $qs \neq 0$ are integers, it follows
    that $x - y$ is rational.  We have proved the difference of
    two rational numbers is rational.
\end{solution}



\bonusquestion[4] Rosen Ch. 2.4 \#22, p. 168.
\begin{solution}
\begin{enumerate}[label=(\alph*), itemsep=0pt,parsep=0pt,topsep=0pt,partopsep=0pt]
  \item $a_n = 1+ 1.05a_{n-1}$, $a_0 = 50$. 
  \item $n = 8$. $a_8 = \sim 83,400$
  \item $a_n = 1 + 1.05a_{n-1}$ 
  \begin{align*}
    a_n &= 1 + 1.05a_{n-1} \\
     &= 1 + 1.05(1 + 1.05a_{n-2}) = 1 + 1.05 + 1.05^2a_{n-2} \\
     &= 1 + 1.05(1 + 1.05(1 + 1.05a_{n-3})) = 1 + 1.05 + 1.05^2 + 1.05^3a_{n-3} \\
     & \vdots \\
     &= 1 + 1.05 + 1.05^2 + \cdots + 1.05^{n-1} + 1.05^n a_{n-n} \\
     &= 1 + 1.05 + 1.05^2 + \cdots + 1.05^{n-1} + 1.05^n a_{0} \\
     &= \sum_{i=0}^n 1.05^n + (1.05)^n a_{0} \\
     &= \frac{1.05^n - 1}{1.05 - 1} + 50\cdot (1.05)^n \\
     &= 70\cdot (1.05)^n - 20 
  \end{align*}
\end{enumerate}
\end{solution}



% \bonusquestion[4] Prove the following proposition, ``For any integer $n \geq 2$, $n^2-3$ is never divisible by 4."
%     \ifprintanswers
%         \vspace{-10pt}
%     \fi
% \begin{solution}
%     Consider two cases:
%     \begin{itemize}
%         \item[Case 1]: If $n$ is even, then by definition $n^2$is also even and $n^2-3$ will be odd.  Therefore, $n^2-3$ is not divisible by 4.
%         \item[Case 2]: If $n$ is odd, there are 4 cases to consider. $n$ can be written as $4k$, $4k+1$, $4k+2$, and $4k+3$ for an integer $k$.  The forms of $4k$ and $4k+2$ are even, not matching the condition $n$ is odd, and will not be considered further.
%             \begin{itemize}
%                 \item[Case a]:
%                 \begin{align*}
%                     n &= 4k + 1 \\
%                     n^2 - 3 &= (4k + 1)^2 - 3 \\
%                         &= 16k^2 + 8k + 1 - 3 \\
%                         &= 16k^2 +8k - 2
%                 \end{align*}
%                 The number $n^2-3$ is not divisible by 4.
%                 \item[Case b]:
%                 \begin{align*}
%                     n &= 4k+3 \\
%                     n^2 - 3 &= (4k + 3)^2 - 3 \\
%                      &= 16k^2 + 24k + 9 - 3 \\
%                      &= 16k^2 + 24k + 6
%                 \end{align*}
%                 The number $n^2-3$ is not divisible by 4.
%                 Therefore, if $n$ is odd, $n^2-3$ is not divisible by 4.
%             \end{itemize}
%         \end{itemize}
%         Therefore, we have shown in all cases that for any integer $n \geq 2$, $n^2 - 3$ is not divisible by 4.
% \end{solution}


\end{questions}
\end{document}