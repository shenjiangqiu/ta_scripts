\documentclass[11pt,addpoints]{exam}
% can include option [answers] to print out solutions, or command \printanswers
%  can turn addpoint on and off with commands, \addpoints and \noaddpoints

\usepackage{amsthm}
\usepackage{amssymb}
\usepackage{amsmath}
\usepackage{epsfig,graphicx}
\usepackage{color}
\usepackage{enumitem}
\usepackage[top=0.75in,bottom=0.75in,left=0.9in,right=0.9in]{geometry}

\setlength{\itemsep}{0pt} \setlength{\topsep}{0pt}
\newcommand{\ra}{\rightarrow}
\newcommand{\lra}{\leftrightarrow}
\newcommand{\xor}{\oplus}
\newcommand{\es}{\emptyset}
\newcommand{\s}{\subseteq}
\newcommand{\pss}{\subset}
\newcommand{\N}{\mathbf{N}}
\newcommand{\Z}{\mathbf{Z}}
\newcommand{\Zp}{\mathbb{Z}^+}
\newcommand{\Zn}{\mathbb{Z}^-}
\newcommand{\Q}{\mathbb{Q}}
\newcommand{\R}{\mathbb{R}}
\newcommand{\ds}{\displaystyle}

\begin{document}
\extrawidth{0.5in} \extrafootheight{-0in} \pagestyle{headandfoot}
\headrule \header{\textbf{cs2311 - Fall 2013}}{\textbf{HW
 7 - \numpoints$\;$ points - Solutions}}{\textbf{Due: Mon. 11/4/13}} \footrule \footer{}{Page \thepage\
of \numpages}{}

% \noindent \textbf{Instructions:} All assignments are due \underline{by \textbf{5pm} on the due date} specified.  There will be a box in the CS department office (Rekhi 221) where assignments may be turned in.  Solutions will be handed out (or posted on-line) shortly thereafter.  Every student
% must write up their own solutions in their own manner.

% \smallskip
% \noindent Please present your solutions in a clean, understandable
% manner; pages should be stapled before class, no ragged edges of
% paper.

\begin{questions}
\printanswers

% Ferland, Sec. 4.1 # 13, 16, 37, 41
\question[10] What are the first 4 terms of each sequence:

\begin{tabular}{ll}
    (a) $a_n = 4-2n\;\;\forall n \geq 0$ \hspace{0.35in} 
    	& (b) $b_1 = 6- 3\cdot 2^n \;\;\forall n \geq 0$ \\
    (c) $c_1 = 4,  c_n = 3\cdot c_{n-1} - 2 \;\; \forall n \geq 2$ \hspace{0.3in}
    	& (d) $d_1 = -1, d_n = 5\cdot d_{n-1} + 1\;\; \forall n \geq 2$ \\
    \multicolumn{2}{c}{(e) $e_0 = 1, e_1 = 1$, $e_n = ne_{n-1} + n^2e_{n-2} \;\forall n \geq 2$}
\end{tabular}
    \ifprintanswers
        \vspace{-10pt}
   \fi
\begin{solution}

	\begin{tabular}{ll}
		(a) $a_1 = 4$, $a_2 = 2$, $a_3 = 0$, $a_4 = -2$ \hspace{0.2in}
			& (b) $b_1 = 3$, $b_2 = 0$, $b_3 = -6$, $b_4 = -18$ \\
		(c) $c_1 = 4$, $c_2 = 10$, $c_3 = 28$, $c_4 = 82$
			& (d) $d_1 = -1$, $d_2 = -4$, $d_3 = -19$, $d_4 = -94$ \\
		(e) $e_0 = 1$, $e_1 = 1$, $e_2 = 6$, $e_3 = 27$
	\end{tabular}
    % \begin{enumerate}[label=(\alph*),itemsep=0pt,parsep=0pt,
    % topsep=0pt,partopsep=0pt]
    %     \item $a_1 = 4$, $a_2 = 2$, $a_3 = 0$, $a_4 = -2$
    %     \item $b_1 = 3$, $b_2 = 0$, $b_3 = -6$, $b_4 = -18$
    %     \item $c_1 = 3$, $c_2 = 10$, $c_3 = 28$, $c_4 = 82$
    %     \item $d_1 = -1$, $d_2 = -4$, $d_3 = -19$, $d_4 = -94$
    %     \item $e_0 = 1$, $e_1 = 1$, $e_2 = 6$, $e_3 = 27$
    % \end{enumerate}
\end{solution}


% Ferland, Sec. 4.1 # 41
\bonusquestion[3] Consider the sequence in problem 1(d).  With $d_1 = -1$ and $d_n = 5\cdot d_{n-1} + 1 \;\;\forall n \geq 2$, the sequence is decreasing; changing the value of the initial condition may change the properties of the sequence.  Determine the smallest value for $d_1$ so that $\{d_n\}$ is not decreasing.
\ifprintanswers
        \vspace{-10pt}
   \fi
\begin{solution}
 An initial condition $d_1 \geq -\frac{1}{4}$ results in a non-decreasing sequence.
\end{solution}


\question[10] Find a closed formula; assume the sequence starts with $n=0, 1, 2, \ldots$.

\begin{tabular}{ll}
	(a) $1, -4, 9, -16, 25, \ldots$ 
		& (b) $3, 6, 12, 24, 48, \ldots$ \\
	(c) $2, 6, 18, 54, 162, 486, 1458\ldots$  \hspace{0.3in}
		& (d) $7, 10, 15, 22, 31, 42, 55, 70, \ldots$ \\
	\multicolumn{2}{c}{(e) $9, 4, -1, -6, -11, \ldots$ }
\end{tabular}
    \ifprintanswers
        \vspace{-10pt}
   \fi
\begin{solution}
	
	\begin{tabular}{lll}
		(a) $a_n = (n+1)^2 (-1)^n$ \hspace{0.6in}
			& (b) $a_n = 3\cdot 2^n$ \hspace{0.6in}
			& (c) $a_n = 2\cdot 3^n$ \\
		(d) $a_n = (n+1)^2 + 6$
			& (e) $a_n = 9 - 5n$
	\end{tabular}
    % \begin{enumerate}[label=(\alph*),itemsep=0pt,parsep=0pt,
    % topsep=0pt,partopsep=0pt]
    %     \item $a_n = (n+1)^2 (-1)^n$
    %     \item $a_n = 3\cdot 2^n$
    %     \item $a_n = 2\cdot 3^n$
    %     \item $a_n = (n+1)^2 + 6$
    %     \item $a_n = 9 - 5n$
    % \end{enumerate}
\end{solution}


\question[6] Find a recursive formula; assume the sequnce starts with $n=0,1,2,\ldots$

\begin{tabular}{ll}
	(a) $3, 6, 12, 24, 48, \ldots$  \hspace{0.3in}
		& (b) $7, 10, 15, 22, 31, 42, 55, 70, \ldots$\\
	\multicolumn{2}{c}{(c) $9, 4, -1, -6, -11, -16, \ldots$}
\end{tabular}
    \ifprintanswers
        \vspace{-10pt}
   \fi
\begin{solution}
	(a) $a_0 = 3$, $a_n = 2*a_{n-1}$ \\
	(b)  $a_0 = 7$, $a_n = a_{n-1} + 2n + 1$  
		\hspace{0.6in} (c) $a_0 = 9$, $a_n = a_{n-1} + -5$
    % \begin{enumerate}[label=(\alph*),itemsep=0pt,parsep=0pt,
    % topsep=0pt,partopsep=0pt]
    %     \item $a_0 = 3$, $a_n = 2*a_{n-1}$
    %     \item $a_0 = 7$, $a_n = a_{n-1} + 2n + 1$ 
    %     \item $a_0 = 9$, $a_n = a_{n-1} + -5$
    % \end{enumerate}
\end{solution}


\question[4] Rosen Ch 2.4 \# 12(b,c), p. 168.
    \ifprintanswers
        \vspace{-10pt}
   \fi
\begin{solution}
 (b) $a_n = -3a_{n-1} + 4a_{n-2} = -3\cdot 1 + 4 \cdot 1 = 1$ \\
 (c) $a_n = -3a_{n-1} + 4a_{n-2} = -3(-4)^{n-1} + 4(-4^{n-2}) = (-4)^{n-2}(-3\cdot -4 + 4) = (-4)^{n-2}\cdot 16 = (-4)^{n-2}(-4)^2 = -4^{n}$
\end{solution}


\question[16] Rosen Ch 2.4 \# 16(a,c), p. 168.
    \ifprintanswers
        \vspace{-10pt}
   \fi
\begin{solution}
	% \begin{itemize}[itemsep=0pt,parsep=0pt,topsep=0pt,partopsep=0pt]
	%     \item[a)]
		\small \vspace{-0.2in}
    	\begin{align*}
	        (a) \quad \quad a_n &= -a_{n-1} = -1\cdot a_{n-1} \\
	          &= -(-a_{n-2}) = (-1)^2a_{n-2} \\
	          &= -(-(-a_{n-3})) = (-1)^3a_{n-3} \\
	          &= \cdots \\
	          &= (-1)^n\cdot a_{n-n} = (-1)^n\cdot a_{0} = 5\cdot(-1)^n
    	\end{align*}
        % \item[(c)] 
        \vspace{-0.1in} 
        \begin{align*}
            (c) \quad \quad a_n &= a_{n-1} - n \\
            &= a_{n-2} -(n-1) - n = a_{n-1} - (n + (n-1)) \\
            &= a_{n-3} -(n-2) - (n + (n-1)) = a_{n-2} - (n + (n-1) + (n-2)) \\
            &= \cdots \\
            &= a_{n-n} -(n + (n-1) + (n-2) + \ldots + (n- (n-1)) \\
            &= a_0 - (n + (n-1) + (n-2) + \ldots + 1) \\
            &= 4 - (n + (n-1) + (n-2) + \ldots + 1) = 4 -\frac{n(n+1)}{2}
        \end{align*}
    % \end{itemize}
\end{solution}


\question[8] A mortage loan is paid off in periodic (monthly) installments, while interest is also charged each period.  A mortage with an annual interest rate of $r$ has a monthly interest rate of $i = \frac{r}{12}$.  A mortage of $M$ dollars at monthly interest rate $i$ has payments of $P$ dollars.  At the end of the month, interest is added to the previous balance, and then the payment $P$ is subtracted from the result.  Let $m_n$ be the balance due after $n$ months, where $m_0 = M$, $m_1 = M(1+i) - P, \ldots$
\begin{enumerate}[label=(\alph*),itemsep=0pt,parsep=0pt,
    topsep=0pt,partopsep=0pt]
    \item Let $M = 10,000$, $i = 0.03$ and $P = 105.13$, determine $m_2$ and $m_3$.
    % \item If $m_{20} = 79,495.98$, $i=0.02$ and $R = 822.89$, determine $m_{21}$.
    \item Find a recursive formula for the mortage balance, $m_n$
\end{enumerate}
    \ifprintanswers
        \vspace{-10pt}
   \fi
\begin{solution}
	\begin{enumerate}[label=(\alph*),itemsep=0pt,parsep=0pt,
    topsep=0pt,partopsep=0pt]
    	\item $m_0 = 10,000$, $m_1 = 10194.87$, $m_2 = 10395.59$, $m_3 =  10602.32$
    	\item $m_n = m_{n-1}(1+i) - P$, $m_o = M$
    \end{enumerate}
\end{solution}


\question[10] A \textit{closed form} of a summation is an equation
in which no summation symbol appears.  The classic example is
$\displaystyle \sum_{i=1}^n i = \frac{n(n+1)}{2}$.  The fraction on
the right is the \textit{closed form} of the summation.  For this
problem you will determine closed forms for summation, $\displaystyle \sum_{i=1}^n \sum_{j=1}^n (3i+ 2j)$.  Justify each step using the four facts on the arithmetic properties of summations and Table 2, p. 166 in the book.

\small
\begin{tabular}{lll}
    \multicolumn{2}{l}{Fact} & Example: \\
  Fact 1: & $\displaystyle \sum_{i=1}^n c  = nc$     & $\displaystyle \sum_{i=1}^n 7 = 7n$ \\
    & when $c$ is a constant \\
  Fact 2: & $\displaystyle \sum_{i=j}^n c  = (n-j+1)c$  & $\displaystyle \sum_{i=0}^n = 7(n+1) $ \\
    & when $c$ is a constant \\
  Fact 3: & $\displaystyle \sum_{i=j}^n (f(i) \pm g(i)) = \sum_{i=j}^n f(i) \pm \sum_{i=j}^n g(i)$ \hspace{0.2in} &  $\displaystyle \sum_{i=j}^n (2n - n^2) = \sum_{i=j}^n 2n - \sum_{i=j}^n n^2 $ \\
    & \\
  Fact 4: & $\displaystyle \sum_{i=j}^n cf(i) = c \sum_{i=j}^n f(i)$  & $\displaystyle \sum_{i=j}^n (3 \times 2^i) = 3 \sum_{i=j}^n 2^i $ \\
   & when $c$ is a constant \\
\end{tabular}

    \ifprintanswers
        \vspace{-10pt}
   \fi
\begin{solution}
\begin{tabular}{rll}
  $\displaystyle \sum_{i=1}^n \sum_{j=1}^n (3i+ 2j)$ & $\displaystyle = \sum_{i=1}^n \left[ \sum_{j=1}^n (3i + 2j) \right] $ & add implied parentheses \\
   & $\displaystyle = \sum_{i=1}^n \left[ \sum_{j=1}^n 3i + \sum_{j=1}^n 2j \right]$ & Fact 3 \\
   & $\displaystyle = \sum_{i=1}^n \left[ 3i \sum_{j=1}^n 1 + 2 \sum_{j=1}^n j \right]$ & Fact 4 and Fact 4 \\
   & $\displaystyle = \sum_{i=1}^n \left[ 3i\cdot n + 2 \sum_{j=1}^n j \right]$ & Fact 1  \\
   & $\displaystyle = \sum_{i=1}^n \left[ 3i\cdot n + 2 \cdot \frac{n(n+1)}{2} \right]$ & Table 2 \\
   & $\displaystyle = \sum_{i=1}^n 3i\cdot n + \sum_{i=1}^n n(n+1)$ & Fact 3 \\
   & $\displaystyle = 3n \sum_{i=1}^n i + n(n+1) \sum_{i=1}^n 1$ & Fact 4 (twice) \\
   & $\displaystyle = 3n \cdot \frac{n(n+1)}{2} + n(n+1) \sum_{i=1}^n 1$ & Table 2 \\
   & $\displaystyle = 3n \cdot \frac{n(n+1)}{2} + n(n+1)\cdot n$ & Fact 1 \\
   & $\displaystyle = \frac{5n^2(n+1)}{2} $ & algebra \\
\end{tabular}
\end{solution}


\question[8] Determine the closed form of the summation $\displaystyle \sum_{i=5}^n 3^i$.
    \ifprintanswers
        \vspace{-10pt}
   \fi
\begin{solution}
	\begin{align*}
		\sum_{i=5}^n 3^i  & = \sum_{j=0}^{n-5} 3^{j+5} \tag{change of index} \\
		& = \sum_{j=0}^{n-5} 3^{j}3^{5} \tag{algebra} \\
		& = 3^5 \sum_{j=0}^{n-5} 3^j  	\tag{Fact 4} \\
		& = 3^5 \left[ \frac{3^{n-5+1} - 1}{3 - 1} \right] \tag{Table } \\
		& = \frac{ 3^5(3^{n-4} - 1) }{ 2 } \tag{algebra}
	\end{align*}

	Alternatively, 
	\begin{align*}
		\sum_{i=5}^n 3^i &= \sum_{i=0}^n 3^i - \sum_{i=0}^4 3^i \tag{change index} \\
		&= \frac{3^{n+1} - 1}{3 - 1} - \frac{3^{4+1} - 1}{3 -1} \tag{Table 2}\\
		&= \frac{3^{n+1} - 3^{5}}{2} = \frac{3^5(3^{n-4} - 1)}{2} \tag{algebra} \\
	\end{align*}
\end{solution}

\question[2] Rosen Ch 2.4 \#30(c), p. 169
    \ifprintanswers
        \vspace{-10pt}
   \fi
\begin{solution}
\begin{tabular}{ll}
 % (a) $1 + 3 + 5 + 7 = 16$ \hspace{0.5in} & %
 (c) $\frac{1}{1} + \frac{1}{3} + \frac{1}{5} + \frac{1}{7} = \frac{176}{105}$
\end{tabular} 
\end{solution}


\question[4] Rosen Ch 2.4 \#34 (a,d), p. 169
    \ifprintanswers
        \vspace{-10pt}
   \fi
\begin{solution}
\begin{tabular}{l}
(a) $(1-1) + (1-2) + (2-1) + (2-2) + (3-1) + (3-2) = 3$ \\
 (d) $(0+0+0+0)+(0+1+8+27)+(0+4+32+108) = 180$
\end{tabular}
\end{solution}


\uplevel{\textbf{Ch 5 Induction and Recursion}}

\question[10] Rosen, Ch 5.1 \# 4, p. 329
    \ifprintanswers
        \vspace{-10pt}
   \fi
\begin{solution}
Let $P(n)$ be the statement that $1^3 + 2^3 + \cdots + n^3 = \left( \frac{n(n+1)}{2} \right)^2$ for the positive integer $n$.
\begin{parts}
    \part (1 point) What is the statement P(1)? \\
        $P(1)$ is the statement $ 1^3 = [ \frac{1 \cdot (1+1) }{2} ]^2 $.
    \part (1 point) Show that P(1) is true, completing the basis step of the proof.
        \[ \left[ \frac{1 \cdot (1+1) }{2} \right]^2 = 1^2 = 1 = 1^3 \]
    \part (2 points) What is the inductive hypothesis? \\
        The inductive hypothesis is to assume $P(k)$ is true for an arbitrary, fixed integer $k \geq 1$, that is
        \[ 1^3 + 2^3 + \cdots + k^3 = \left( \frac{k(k+1)}{2} \right)^2 \]
    \part (2 point) What do you need to prove in the inductive step? \\
        For the inductive step, show for each $k \geq 1$ that $P(k)$ implies $P(k+1)$. \\
        That is, show $P(k+1)$:
        \[ 1^3 + 2^3 + \cdots + k^3 + (k+1)^3 = \left( \frac{(k+1)(k+2)}{2} \right)^2 \]
    \part (3 points) Complete the inductive step. \\
        Since, $P(k)$ holds,
        \begin{align*}
        	1^3 + 2^3 + \cdots + k^3 &= \left( \frac{k(k+1)}{2} \right)^2 \tag{Ind. Hyp.} \\
            1^3 + 2^3 + \cdots + k^3 + (k+1)^3 &=  \left( \frac{k(k+1)}{2} \right)^2 + (k+1)^3   \\
              &= \left(\frac{k^2(k+1)^2}{2^2} \right) + (k+1)^3 \\
              &= (k+1)^2 \left( \frac{k^2}{4} + (k + 1) \right) \\
              &= (k+1)^2 \left( \frac{k^2 + 4k + 4}{4} \right) \\
              &= \frac{(k+1)^2(k+2)^2}{2^2} =  \left( \frac{(k+1)(k+2)}{2} \right)^2\\
        \end{align*}
        That is, $1^3 + 2^3 + \cdots +(k+1)^3 = \left( \frac{(k+1)(k+2)}{2} \right)^2 $, the $P(k+1)$ statement.  This completes the inductive step.
    \part (1 point) Explain why these steps show that this formula is true whenever $n$ is a positive integer. \\
        The basis step and inductive step are completed.  Therefore by principle of mathematical induction, the statement is true for every positive integer $n$.
\end{parts}
\end{solution}


% Ferland, p. 189
\question[10] Use induction to prove the following statement:
	\[\forall n \geq 0, \;\; 4 \,|\, (5^{n} - 1)\]
That is, for the natural numbers, 4 divides $5^n - 1$.
    \ifprintanswers
        \vspace{-10pt}
   \fi
\begin{solution}
Let $P(n)$ be the statement ``4 divides $5^n - 1$" for the natural numbers $n$. 

\textit{Base Case:} Show $P(0)$ is true. 
	4 does divide $5^n - 1$ where $n=0$;  $4 \,|\, 5^0 - 1 = 1 - 1 = 0$ because there exists an integer $k$ such that $4k = 0$.

\textit{Inductive Step:} \\
\textit{IH} Assume $P(n)$ holds for some $k \geq 0$, that is, there exists an integer $a$ such that $4a = 5^k - 1$. 

Show that $P(k+1)$ holds, that is, there exists an integer $b$ such that $4b = 5^{k+1} - 1$. 

\begin{align*}
	5^{k+1} - 1 &= 5\cdot 5^{k} - 1 \\
	 &= (4 + 1)5^{k} - 1 \\
	 &= 4\cdot 5^{k} + 5^{k} - 1 \\
	 &= 4\cdot 5^{k} + 4a \tag{Ind. Hyp.} \\
	 &= 4(5^{k} + a)
	 = 4b
\end{align*}
This shows that assuming $P(k)$, $P(k+1)$ holds. \\
Therefore, by completing the basis and inductive step and principle of mathematical induction, the statement $P(n)$ is true for every natural number.
\end{solution}

\end{questions}
\end{document}


