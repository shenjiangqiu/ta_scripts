% Homework Specific Commands


\begin{document}
\extrawidth{0.5in} \extrafootheight{-0in} \pagestyle{headandfoot}
\headrule \header{\textbf{cs2311 - Fall 2016}}{\textbf{HW
 8 \ifprintanswers - Solutions \fi}}{\textbf{Due: Mon. 03/28/16}} \footrule \footer{}{Page \thepage\
of \numpages}{}

\ifprintanswers
% \noindent You should \underline{complete all problems}, but \underline{only a subset will be graded} (which will be graded is not known to you ahead of time). 
\else
\noindent \textbf{Instructions:} All assignments are due \underline{by \textbf{4pm} on the due date} specified.  There will be a box in the CS department office (Rekhi 221) where assignments may be turned in.  Solutions will be posted on-line at the following lecture.  Every student
must write up their own solutions in their own manner.

\medskip
\noindent Please present your solutions in a clean, understandable
manner; pages should be stapled before being turned in, no ragged edges of
paper.

% \medskip
% \noindent Be clear with you penmanship to distinguish set brackets and parentheses.
%   For example, $\{1, 2\}$ is a set where $(1,2)$ is an ordered pair.  Also,
% differentiate the empty set $\emptyset$ and the number zero 0.

\medskip
\noindent You should \underline{complete all problems}, but \underline{only a subset will be graded} (which will be graded is not known to you ahead of time). 

\medskip
\noindent \textbf{Include the following information at the top, left corner of your submission:} (not including this information will result in -3pt)\\
\textbf{Last Name, First Name (as listed in Canvas)} \\
\textbf{Section (R01, R02)}
\fi

\begin{questions}


\gquestion{9}{9}{all} Prove the following is true for all positive integers $n$, $n$ is even if and only if $5n^2 + 4$ is even.
%Prove that if $n$ is a positive integer, then $n$ is
%even if and only if $7n+4$ is even.
    \ifprintanswers
        \vspace{-10pt}
    \fi
\begin{solution}
    The statement is a biimplication therefore both sides of the
    implication must be shown.  Let $p$ be ``$n$ is even" and $q$ be
    ``$5n^2 + 4$ is even."
    
    \textbf{Prove ``if p, then q":}
    Assume $n$ is even. By definition of even, there exists an
    integer $k$ s.t. $n=2k$.  Compute $5n^2+4$:
    \[ 5n^2 + 4 = 5(2k)^2 + 4 = 5(4k) + 4 = 20k + 4 = 2(10k+2) \]
    $5n^2+4$ is also of the form of an even number.  Therefore, if $n$
    is even, then $5n^2+4$ is even.

    \smallskip
    \textbf{Prove ``if q, then p": using proof by contradiction.}
    Assume $5n^2+4$ is even and $n$ is odd.  Then by definition of odd
    there exists an integer $k$ s.t. $n=2k+1$.  Compute $5n^2+4$:
    \[ 5n^2+4 = 5(2k+1)^2+4 = 5(2k^2+4k+1)+4 = 10k^2 + 20k + 5 + 4 = 2(5k^2 + 10k + 4) + 1 \]
    $5n^2+4$ is of the form of an odd number leading to a
    contradiction, therefore it must be that if $5n^2+4$ is even, then
    $n$ is even.

    \smallskip
    Combining both parts, we have shown if $n$ is a positive integer,
    then $n$ is even if and only if $5n^2+4$ is even.
\end{solution}


\vspace{-4pt}
\ugquestion{6} Use a proof by cases to show that $min(a, min(b,c)) =
min(min(a,b),c)$ whenever $a$, $b$, and $c$ are real numbers.
% SEE Jean Mayo hw5
    \ifprintanswers
        \vspace{-10pt}
    \fi
\begin{solution}
    There are 6 cases for the values of $a$,$b$, $c$. 

    \begin{minipage}{0.5\textwidth}
    \begin{itemize}[itemsep=1pt, topsep=0pt]
      \item \textbf{case 1:} $a \leq b \leq c$, then \\
        $min(a, min(b,c)) = min(a,c) = a$ \\
        $min(min(a,b),c)) = min(a,c) = a$ 
      \item \textbf{case 2:} $a \leq c \leq b$, then \\
        $min(a, min(b,c)) = min(a,c,) = a$ \\
        $min(min(a,b),c) = min(a,c) = a$ 
      \item \textbf{case 3:} $b \leq a \leq c$, then \\
        $min(a, min(b,c)) = min(a,b) = b$\\
        $min(min(a,b),c) = min(b,c) = b$ 
    \end{itemize}
    \end{minipage}
    %
    \begin{minipage}{0.5\textwidth}
    \begin{itemize}[itemsep=1pt, topsep=0pt]
      \item \textbf{case 4:} $b \leq c \leq a$, then \\
        $min(a, min(b,c)) = min(a,b) = b$ \\
        $min(min(a,b),c) = min(b,c) = b$ 
      \item \textbf{case 5:} $c \leq a \leq b$, then \\
        $min(a, min(b,c)) = min(a,c) = c$\\
        $min(min(a,b),c) = min(a,c) = c$ 
      \item \textbf{case 6:} $c \leq b \leq a$, then \\
        $min(a, min(b,c)) = min(a,c) = c$ \\
        $min(min(a,b),c) = min(b,c) = c$ 
    \end{itemize}
    \end{minipage}

    \smallskip
    We have shown it to be true, in all cases.
\end{solution}


\vspace{-4pt}
\ugquestion{3} Prove that there is a positive integer that equals the
sum of the positive integers not exceeding it.
    \ifprintanswers
        \vspace{-12pt}
    \fi
\begin{solution}
    This is an existence proof.  3 is an example of such a positive
    integer, 3 = 1 + 2.
\end{solution}

\vspace{-4pt}
\ugquestion{3} Prove/disprove: If $a$ and $b$ are rational
numbers, then $a^b$ is also rational.
    \ifprintanswers
        \vspace{-12pt}
    \fi
\begin{solution}
    \textbf{Disprove:} Let $a=2$ and $b=\frac{1}{2}$, which are both rational numbers.  Then $a^b =
    \sqrt{2}$ which is irrational.
\end{solution}

\vspace{-4pt}
\gquestion{3}{3}{all} Prove/disprove: The sum of four consecutive integers is divisible by 4.
    \ifprintanswers
        \vspace{-10pt}
    \fi
\begin{solution}
    \textbf{Disprove:} Let the consecutive integers be 1, 2, 3, 4; the sum is 10 which is not divisible by 4.
\end{solution}


\ifprintanswers
  \newpage 
\fi 
\gquestion{8}{8}{all} Prove that if $n$ is an integer that $n^3 - n$ is even.
    \ifprintanswers
        \vspace{-10pt}
    \fi
%\begin{EnvFullwidth}
%\begin{TheSolution}
\begin{solution} \textbf{Proof:} Let $n$ be an integer.
    \begin{itemize}[itemsep=0pt,parsep=0pt,topsep=0pt,partopsep=0pt]
        \item[Case] (i): Let $n$ be even. By definition, there exists an integer $k$ s.t. $n=2k$.
            \[ n^3 - n = (2k)^3 - 2k = 8k^3 - 2k = 2(4k^3 - k) \]
        This is of the form of an even number.
        \item[Case] (ii):  Let $n$ be odd.  By definition, there exists an integer $k$ s.t. $n = 2k+1$.
            \begin{align*}
                n^3 - n &= (2k+1)^3 - (2k+1) = (4k^2 + 4k+ 1)(2k + 1) - (2k+1) \\
                &= 8k^3 + 12k^2 + 6k + 1 - 2k - 1 = 8k^3 + 12k^2 + 4k = 2(8k^3 + 6k^2 + 2k)
            \end{align*}
        This is of the form of an even number.

        Therefore, because $n^3 - n$ is even in all cases, it holds that for any integer $n$, $n^3 - n$ is even.
    \end{itemize}
\end{solution}



\bonusquestion[4] Prove the following proposition, ``For any integer $n \geq 2$, $n^2-3$ is never divisible by 4."
    \ifprintanswers
        \vspace{-10pt}
    \fi
\begin{solution}
    Consider two cases:
    \begin{itemize}
        \item[Case 1]: If $n$ is even, then by definition $n^2$is also even and $n^2-3$ will be odd.  Therefore, $n^2-3$ is not divisible by 4.
        \item[Case 2]: If $n$ is odd, there are 4 cases to consider. $n$ can be written as $4k$, $4k+1$, $4k+2$, and $4k+3$ for an integer $k$.  The forms of $4k$ and $4k+2$ are even, not matching the condition $n$ is odd, and will not be considered further.
            \begin{itemize}
                \item[Case a]:
                \begin{align*}
                    n &= 4k + 1 \\
                    n^2 - 3 &= (4k + 1)^2 - 3 \\
                        &= 16k^2 + 8k + 1 - 3 \\
                        &= 16k^2 +8k - 2
                \end{align*}
                The number $n^2-3$ is not divisible by 4.
                \item[Case b]:
                \begin{align*}
                    n &= 4k+3 \\
                    n^2 - 3 &= (4k + 3)^2 - 3 \\
                     &= 16k^2 + 24k + 9 - 3 \\
                     &= 16k^2 + 24k + 6
                \end{align*}
                The number $n^2-3$ is not divisible by 4.
                Therefore, if $n$ is odd, $n^2-3$ is not divisible by 4.
            \end{itemize}
        \end{itemize}
        Therefore, we have shown in all cases that for any integer $n \geq 2$, $n^2 - 3$ is not divisible by 4.
\end{solution}




\ifprintanswers
  \newpage 
\fi 
\uplevel{{\bf Sequences}}

\ugquestion{8} What are the first 4 terms of each sequence:

\begin{tabular}{ll}
    (a) $a_n = 4-2n\;\;\forall n \geq 0$ \hspace{0.35in} 
        & (b) $b_n = 6- 3\cdot 2^n \;\;\forall n \geq 0$ \\
    (c) $c_1 = 4,  c_n = 3\cdot c_{n-1} - 2 \;\; \forall n \geq 2$ \hspace{0.3in}
        & (d) $d_1 = -1, d_n = 5\cdot d_{n-1} + 1\;\; \forall n \geq 2$ \\
    % \multicolumn{2}{c}{(e) $e_0 = 1, e_1 = 1$, $e_n = ne_{n-1} + n^2e_{n-2} \;\forall n \geq 2$}
\end{tabular}
    \ifprintanswers
        \vspace{-10pt}
   \fi
\begin{solution}

    \begin{tabular}{ll}
        (a) $a_1 = 4$, $a_2 = 2$, $a_3 = 0$, $a_4 = -2$ \hspace{0.2in}
            & (b) $b_1 = 3$, $b_2 = 0$, $b_3 = -6$, $b_4 = -18$ \\
        (c) $c_1 = 4$, $c_2 = 10$, $c_3 = 28$, $c_4 = 82$
            & (d) $d_1 = -1$, $d_2 = -4$, $d_3 = -19$, $d_4 = -94$ \\
        % (e) $e_0 = 1$, $e_1 = 1$, $e_2 = 6$, $e_3 = 27$
    \end{tabular}
\end{solution}



\gquestion{8}{4}{d,e} Find a closed formula; assume the sequence starts with $n=0, 1, 2, \ldots$.

\begin{tabular}{ll}
    (a) $1, -4, 9, -16, 25, \ldots$ 
        & (b) $8, 4, 2, 1, \frac{1}{2}, \frac{1}{4},  \ldots$ \\
    % (c) $2, 6, 18, 54, 162, 486, 1458\ldots$  \hspace{0.3in}
        (c) $3, 4, 7, 12, 19, 28, 39 \ldots$ 
    & (d) $6, 1, -4, -9, -14, -19 \ldots$ \\
\end{tabular}
    \ifprintanswers
        \vspace{-10pt}
   \fi
\begin{solution}
    
    \begin{tabular}{ll}
        (a) $a_n = (n+1)^2 (-1)^n$ \hspace{0.6in}
            & (b) $a_n = 8\cdot (\frac{1}{2})^n$ \hspace{0.6in} \\
            % & (c) $a_n = 2\cdot 3^n$ \\
        (c) $a_n = n^2 + 3$
            & (d) $a_n = 6 - 5n$
    \end{tabular}
\end{solution}



\gquestion{6}{4}{a,b} Find a recursive formula; assume the sequnce starts with $n=0,1,2,\ldots$

\begin{tabular}{ll}
    (a) $6, 18, 54, 162, 486, \ldots$  \hspace{0.3in}
        & (b) $5, 9, 14, 20, 27, 35, \ldots$\\
    \multicolumn{2}{c}{(c) $7, 3, -1, -5, -9, -13, \ldots$}
\end{tabular}
    \ifprintanswers
        \vspace{-10pt}
   \fi
\begin{solution}
    (a) $a_0 = 6$, $a_n = 3*a_{n-1}$ \\
    (b)  $a_0 = 5$, $a_n = a_{n-1} + n + 3$  
        \hspace{0.6in} (c) $a_0 = 7$, $a_n = a_{n-1} -4$
\end{solution}




\gquestion{6}{3}{a} Determine whether each answer is a solution to the recurrence relation, 
$$ a_n = a_{n-1} + 2a_{n-2} + 2n - 9 $$
    \begin{enumerate}[label=(\alph*),topsep=0pt,itemsep=0pt,parsep=0pt]
        \item $a_n = 0$
        \item $a_n = -n +2$
    \end{enumerate}
    \ifprintanswers
        \vspace{-10pt}
    \fi
\begin{solution}
    \begin{enumerate}[label=(\alph*),topsep=0pt,itemsep=0pt,parsep=0pt]
        \item Not a solution.
        \begin{align*}
            a_n &= a_{n-1} + 2a_{n-2} + 2n - 9 \\
            &= (0) + 2\cdot0 + 2n - 9 \\
            &= 2n - 9 \\
            0 &\neq 2n - 9
        \end{align*}
        \item Yes, $a_n = -n + 2$ is a solution.
        \begin{align*}
            a_n &= a_{n-1} + 2a_{n-2} + 2n - 9 \\
            &= (-(n-1) + 2) + 2(-(n-2) + 2) + 2n -9 \\
            &= -n + 1 + 2 -2n + 4 + 4 + 2n -9 \\
            &= -n +2 \\
        \end{align*}
    \end{enumerate}
\end{solution}




\gquestion{10}{5}{c} Rosen Ch 2.4 \# 16(c,e), p. 168.
    \ifprintanswers
        \vspace{-10pt}
   \fi
\begin{solution}
    % \begin{itemize}[itemsep=0pt,parsep=0pt,topsep=0pt,partopsep=0pt]
    %     \item[a)]
        % \small \vspace{-0.2in}
        % \begin{align*}
        %     (a) \quad \quad a_n &= -a_{n-1} = -1\cdot a_{n-1} \\
        %       &= -(-a_{n-2}) = (-1)^2a_{n-2} \\
        %       &= -(-(-a_{n-3})) = (-1)^3a_{n-3} \\
        %       &= \cdots \\
        %       &= (-1)^n\cdot a_{n-n} = (-1)^n\cdot a_{0} = 5\cdot(-1)^n
        % \end{align*}
        % \item[(c)] 
        \vspace{-0.1in} 
        \begin{align*}
            (c) \quad \quad a_n &= a_{n-1} - n \\
            &= a_{n-2} -(n-1) - n = a_{n-1} - (n + (n-1)) \\
            &= a_{n-3} -(n-2) - (n + (n-1)) = a_{n-2} - (n + (n-1) + (n-2)) \\
            &= \cdots \\
            &= a_{n-n} -(n + (n-1) + (n-2) + \ldots + (n- (n-1)) \\
            &= a_0 - (n + (n-1) + (n-2) + \ldots + 1) \\
            &= 4 - (n + (n-1) + (n-2) + \ldots + 1) = 4 -\frac{n(n+1)}{2}
        \end{align*}
    % \end{itemize}
        \vspace{-0.1in}
        \begin{align*}
            (e) \quad \quad a_n &= (n+1)a_{n-1} \\
              &= (n+1)na_{n-2} \\
              &= (n+1)n(n-1)a_{n-3} \\
              &= \cdots \\
              &= (n+1)n(n-1)\cdots(n-(n-2))a_{n-n} \\
              &= (n+1)n(n-1)\cdots2a_{0} \\
              &= 2(n+1)!
        \end{align*}
\end{solution}


\ugquestion{6} A mortage loan is paid off in periodic (monthly) installments, while interest is also charged each period.  A mortage with an annual interest rate of $r$ has a monthly interest rate of $i = \frac{r}{12}$.  A mortage of $M$ dollars at monthly interest rate $i$ has payments of $P$ dollars.  At the end of the month, interest is added to the previous balance, and then the payment $P$ is subtracted from the result.  Let $m_n$ be the balance due after $n$ months, where $m_0 = M$, $m_1 = M(1+i) - P, \ldots$
\begin{enumerate}[label=(\alph*),itemsep=0pt,parsep=0pt,
    topsep=0pt,partopsep=0pt]
    \item Let $M = 10,000$, $i = 0.03$ and $P = 105.13$, determine $m_2$ and $m_3$.
    % \item If $m_{20} = 79,495.98$, $i=0.02$ and $R = 822.89$, determine $m_{21}$.
    \item Find a recursive formula for the mortage balance, $m_n$
\end{enumerate}
    \ifprintanswers
        \vspace{-10pt}
   \fi
\begin{solution}
    \begin{enumerate}[label=(\alph*),itemsep=0pt,parsep=0pt,
    topsep=0pt,partopsep=0pt]
        \item $m_0 = 10,000$, $m_1 = 10194.87$, $m_2 = 10395.59$, $m_3 =  10602.32$
        \item $m_n = m_{n-1}(1+i) - P$, $m_o = M$
    \end{enumerate}
\end{solution}


\uplevel{{\bf Summations}}

\ugquestion{3} Write out in sigma notation the sum of the first 50 terms of the series: \\
\hspace*{0.2in} $1 - \frac{1}{2} + \frac{1}{4} - \frac{1}{8} + \frac{1}{16} - \ldots$
    \ifprintanswers
        \vspace{-10pt}
   \fi
\begin{solution}
    The sum can take many form depending on index of summation and limits.  Here are three samples
    
    \begin{center}
    \begin{tabular}{ccc}
        $\ds \sum_{i=1}^{50} (-1)^{n+1}\cdot\frac{1}{2^{n-1}}$ \hspace{0.25in}
        & $\ds \sum_{i=0}^{49} (-1)^{n}\cdot\frac{1}{2^{n}}$ \hspace{0.2in}
        & $\ds \sum_{i=0}^{49} \left(\frac{-1}{2}\right)^n$
    \end{tabular}
    \end{center}
\end{solution}  



\gquestion{8}{4}{c,e} What are the values of the sums:\\
% \begin{tabular}{lll}
%     (a) $\ds \sum_{j\in S} (2j - 1), \text{ where } S = \{1, 2, 4, 6\}$
%     & (b) $\ds \sum_{i=1}^{8} 4$  \hspace*{1in} 
%     & (c) $\ds \sum_{k=0}^{4} (-3)^k$ \\
%     & & \\
%     (d) $\ds \sum_{j=0}^{7} (2^{j+1} - 2^{j-1})$
%     & (e) $\ds \sum_{i=1}^{4} \sum_{j=1}^{3} (2i + 3j)$
%     & (f) $\ds \sum_{j=0}^{3} \sum_{k=1}^{3} j*2^{k}$
% \end{tabular}
\begin{tabular}{lll}
    (a) $\ds \sum_{j\in S} (2j - 1), \text{ where } S = \{1, 2, 4, 6\}$
    & (b) $\ds \sum_{k=0}^{4} (-3)^k$ \\
    & & \\
    (c) $\ds \sum_{i=1}^{4} \sum_{j=1}^{3} (2i + 3j)$
    & (d) $\ds \sum_{j=0}^{3} \sum_{k=1}^{3} j*2^{k}$
\end{tabular}
    \ifprintanswers
        \vspace{-10pt}
   \fi
\begin{solution}
    \begin{enumerate}[label=(\alph*),itemsep=3pt,parsep=0pt,
    topsep=0pt,partopsep=0pt]
        \item $\ds (2\cdot1 - 1)+(2\cdot2 - 1)+(2\cdot4 -1)+(2\cdot6-1) = 1+3+7+11 = 22$ 
        % \item $\sum_{i=1}^8 4 = 8\cdot4 = 32$
        \item $\ds (-3^0) + (-3^1) + (-3^2) + (-3^3) + (-3^4) = 1 -3 + 9 -27 + 81 = 61 $
        % \item $\sum_{j=0}^7 2^{j+1} - \sum_{j=0}^7 2^{j-1} = 510 - 127.5 = 382.5$
        \item $\ds \sum_{i=1}^4 \left( \sum_{j=1}^3 2i + \sum_{j=1}^3 3j \right) = 132$
        \item $\ds \sum_{j=0}^3 \sum_{k=1}^3 j*2^k = 84$
    \end{enumerate}
\end{solution}



\end{questions}
\end{document}