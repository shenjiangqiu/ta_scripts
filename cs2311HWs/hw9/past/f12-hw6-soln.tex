\documentclass[12pt,addpoints]{exam}
% can include option [answers] to print out solutions, or command \printanswers
%  can turn addpoint on and off with commands, \addpoints and \noaddpoints

\usepackage{amsthm}
\usepackage{amssymb}
\usepackage{amsmath}
\usepackage{epsfig,graphicx}
\usepackage{color}
\usepackage{enumitem}
\usepackage[top=0.75in,bottom=0.75in,left=0.9in,right=0.9in]{geometry}

\setlength{\itemsep}{0pt} \setlength{\topsep}{0pt}
\newcommand{\ra}{\rightarrow}
\newcommand{\lra}{\leftrightarrow}
\newcommand{\xor}{\oplus}
\newcommand{\es}{\emptyset}
\newcommand{\s}{\subseteq}
\newcommand{\pss}{\subset}
\newcommand{\N}{\mathbf{N}}
\newcommand{\Z}{\mathbf{Z}}
\newcommand{\Zp}{\mathbb{Z}^+}
\newcommand{\Zn}{\mathbb{Z}^-}
\newcommand{\Q}{\mathbb{Q}}
\newcommand{\R}{\mathbb{R}}
\newcommand{\ds}{\displaystyle}

\begin{document}
\extrawidth{0.5in} \extrafootheight{-0in} \pagestyle{headandfoot}
\headrule \header{\textbf{cs2311 - Fall 2012}}{\textbf{HW
 6 - \numpoints$\;$ points - Solutions}}{\textbf{Due: Fri. 10/19/12}} \footrule \footer{}{Page \thepage\
of \numpages}{}

\noindent \textbf{Instructions:} All assignments are due \underline{by \textbf{5pm} on the due date} specified.  There will be a box in the CS department office (Rekhi 221) where assignments may be turned in.  Solutions will be handed out (or posted on-line) shortly thereafter.  Every student
must write up their own solutions in their own manner.

\smallskip
\noindent Please present your solutions in a clean, understandable
manner; pages should be stapled before class, no ragged edges of
paper.

\begin{questions}
\printanswers

\question[11]\label{prob1} For each list of integers, (i) provide a simple formula or rule that generates the terms of the sequence given and (ii) determine the next three terms of the sequence.  Assume the sequence starts with $n=0, 1, 2, 3, \ldots$.
	\begin{enumerate}[label=(\alph*),topsep=0pt,itemsep=0pt,parsep=0pt]
		\item (3 pt) $4, 15, 26, 37, 48, 59, \ldots $
		\item (3 pt) $2, 6, 18, 54, 162, 486, 1458, \ldots $
		\item (3 pt) $7, 10, 15, 22, 31, 42, 55, 70, \ldots $
		\item (1 pt) What is $a_{300}$ for the sequence in (a)?
		\item (1 pt) What is $a_{58}$ for the sequence in (b)?  (scientific notation is fine, e.g.  $3.14159 \times 10^{0}$)
	\end{enumerate}
    \ifprintanswers
        \vspace{-10pt}
   \fi
\begin{solution}
	\begin{enumerate}[label=(\alph*),topsep=0pt,itemsep=0pt,parsep=0pt]
		\item  $4, 15, 26, 37, 48, \ldots $, is an arithmetic sequence with a initial term of 4, and a common difference of 11.  The rule is $\mathbf{a_n = 4 + 11n}$.  The next three terms are: \textbf{70, 81, 92}.
		\item $2, 6, 18, 54, 162, 486, 1458, \ldots $, is a geometric sequence with a initial term of 2, and a common ratio of 3.  The rule is $\mathbf{a_n = 2\cdot3^n}$.  The next three terms are: \textbf{4374, 13122, 39366}.
		\item $7, 10, 15, 22, 31, 42, 55, 70, \ldots $, sequence adds 2n+1 to the previous term.  The rule is $\mathbf{a_0 = 7, \quad a_n = a_{n-1} + 2n + 1}$ or $\mathbf{a_n = (n+1)^2 + 6}$.  The next three terms are: \textbf{ 87, 106, 127 }. 
		\item $a_{300} = 4 + 11*300 = \mathbf{3304}$
		\item $a_{58} = 2\cdot3^{58} = \mathbf{9.4203 \times 10^{27}}$
	\end{enumerate}
\end{solution}

 

\question[6] Determine whether each answer is a solution to the recurrence relation, 
$$ a_n = a_{n-1} + 2a_{n-2} + 2n - 9 $$
	\begin{enumerate}[label=(\alph*),topsep=0pt,itemsep=0pt,parsep=0pt]
		\item $a_n = 0$
		\item $a_n = -n +2$
	\end{enumerate}
    \ifprintanswers
        \vspace{-10pt}
   	\fi
\begin{solution}
	\begin{enumerate}[label=(\alph*),topsep=0pt,itemsep=0pt,parsep=0pt]
		\item Not a solution.
		\begin{align*}
			a_n &= a_{n-1} + 2a_{n-2} + 2n - 9 \\
			&= (0) + 2\cdot0 + 2n - 9 \\
			&= 2n - 9 \\
			0 &\neq 2n - 9
		\end{align*}
		\item Yes, $a_n = -n + 2$ is a solution.
		\begin{align*}
			a_n &= a_{n-1} + 2a_{n-2} + 2n - 9 \\
			&= (-(n-1) + 2) + 2(-(n-2) + 2) + 2n -9 \\
			&= -n + 1 + 2 -2n + 4 + 4 + 2n -9 \\
			&= -n +2 \\
		\end{align*}
	\end{enumerate}
\end{solution}
		

\question[3] Write out in sigma notation the sum of the first 50 terms of the series: $1 - \frac{1}{2} + \frac{1}{4} - \frac{1}{8} + \frac{1}{16} - \ldots$
    \ifprintanswers
        \vspace{-10pt}
   \fi
\begin{solution}
    The sum can take many form depending on index of summation and limits.  Here are three samples
    
    \begin{center}
    \begin{tabular}{ccc}
		$\ds \sum_{i=1}^{50} (-1)^{n+1}\cdot\frac{1}{2^{n-1}}$ \hspace{0.25in}
		& $\ds \sum_{i=0}^{49} (-1)^{n}\cdot\frac{1}{2^{n}}$ \hspace{0.2in}
		& $\ds \sum_{i=0}^{49} \left(\frac{-1}{2}\right)^n$
	\end{tabular}
	\end{center}
\end{solution}



\question[10] A \textit{closed form} of a summation is an equation
in which no summation symbol appears.  The classic example is
$\displaystyle \sum_{i=1}^n i = \frac{n(n+1)}{2}$.  The fraction on
the right is the \textit{closed form} of the summation.  For this
problem you will determine closed forms for summation, $\displaystyle \sum_{i=1}^n \sum_{j=1}^n (6i^2 - 2j)$.  Justify each step using the four facts on the arithmetic properties of summations and Table 2, p. 166 in the book.

\small
\begin{tabular}{lll}
    \multicolumn{2}{l}{Fact} & Example: \\
  Fact 1: & $\displaystyle \sum_{i=1}^n c  = nc$     & $\displaystyle \sum_{i=1}^n 7 = 7n$ \\
    & when $c$ is a constant \\
  Fact 2: & $\displaystyle \sum_{i=j}^n c  = (n-j+1)c$  & $\displaystyle \sum_{i=0}^n = 7(n+1) $ \\
    & when $c$ is a constant \\
  Fact 3: & $\displaystyle \sum_{i=j}^n (f(i) \pm g(i)) = \sum_{i=j}^n f(i) \pm \sum_{i=j}^n g(i)$ \hspace{0.2in} &  $\displaystyle \sum_{i=j}^n (2n - n^2) = \sum_{i=j}^n 2n - \sum_{i=j}^n n^2 $ \\
    & \\
  Fact 4: & $\displaystyle \sum_{i=j}^n cf(i) = c \sum_{i=j}^n f(i)$  & $\displaystyle \sum_{i=j}^n (3 \times 2^i) = 3 \sum_{i=j}^n 2^i $ \\
   & when $c$ is a constant \\
\end{tabular}

    \ifprintanswers
        \vspace{-10pt}
   \fi
\begin{solution}
	\begin{align*}
		\sum_{i=1}^n \sum_{j=1}^n (6i^2 - 2j) 
		&= \sum_{i=1}^n ( \sum_{j=1}^n (6i^2 - 2j) ) \tag{implied parentheses} \\
		&= \sum_{i=1}^n (\sum_{j=1}^n 6i^2 - \sum_{j=1}^n 2j) \tag{Fact 3} \\
		&= \sum_{i=1}^n (6i^2 \sum_{j=1}^n 1 - 2 \sum_{j=1}^n j) \tag{Fact 4, twice} \\
		&= \sum_{i=1}^n (6i^2n - 2 \sum_{j=1}^n j) \tag{Fact 1} \\
		&= \sum_{i=1}^n (6i^2n - 2 \frac{n(n+1)}{2}) \tag{Table 2} \\
		&= \sum_{i=1}^n 6i^2n - \sum_{i=1}^n  n(n+1) \tag{Fact 3} \\
		&= 6n \sum_{i=1}^n i^2 - n(n+1) \sum_{i=1}^n 1 \tag{Fact 4, twice} \\
		&= 6n (\frac{n(n+1)(2n+1)}{6}) - n(n+1) \sum_{i=1}^n 1 \tag{Table 2} \\
		&= n^2(n+1)(2n+1) - n^2(n+1) \tag{algebra} \\
		&= n^2(n+1)(2n + 1  - 1) \\
		&= 2n^3(n+1) 
	\end{align*}
\end{solution}


\question[2] Rosen Ch 2.4 \#30(a), p. 169
    \ifprintanswers
        \vspace{-10pt}
   \fi
\begin{solution}
\begin{tabular}{ll}
 (a) $1 + 3 + 5 + 7 = 16$ \hspace{0.5in} & %(c) $\frac{1}{1} + \frac{1}{3} + \frac{1}{5} + \frac{1}{7} = \frac{176}{105}$
\end{tabular} 
\end{solution}


\question[2] Rosen Ch 2.4 \#34 (d), p. 169
    \ifprintanswers
        \vspace{-10pt}
   \fi
\begin{solution}
\begin{tabular}{l}
% (a) $(1-1) + (1-2) + (2-1) + (2-2) + (3-1) + (3-2) = 3$ \\
 (d) $(0+0+0+0)+(0+1+8+27)+(0+4+32+108) = 180$
\end{tabular}
\end{solution}


\question[4] Rosen Ch 2.4 \#40, p. 169
    \ifprintanswers
        \vspace{-10pt}
   \fi
\begin{solution}
 We can find $\ds \sum_{k=1}^{200} k^3 = \frac{200^2(201)^2}{4}$.
 Also, $\ds \sum_{k=1}^{98} k^3 = \frac{98^2(99)^2}{4}$.

 Therefore,
$$ \sum_{k=99}^{200} = \sum_{k=1}^{200} k^3 - \sum_{k=1}^{98} k^3
   = \frac{200^2(201)^2}{4} - \frac{98^2(99)^2}{4} 
   = 380,477,799 $$
\end{solution}


\uplevel{\textbf{Ch 5 Induction and Recursion}}

\question[9] Prove that $\ds \frac{1}{1\cdot 3} + \frac{1}{3\cdot 5} + \frac{1}{5\cdot 7} + \cdots + \frac{1}{(2n-1)(2n+1)} = \frac{n}{2n+1}$ for the positive integers.
    \ifprintanswers
        \vspace{-10pt}
   \fi
\begin{solution}
Let $P(n)$ be $\frac{1}{1\cdot 3} + \frac{1}{3\cdot 5} + \frac{1}{5\cdot 7} + \cdots + \frac{1}{(2n-1)(2n+1)} = \frac{n}{2n+1}$ for $n \geq 1$

\smallskip
\begin{tabular}{lp{4in}}
  \textit{Basis Step:} & Show $P(1)$ is true, $\frac{1}{1\cdot 3}= \frac{1}{3} = \frac{1}{2\cdot1 +1} = \frac{1}{3}$ \\
   & \\
 \textit{Inductive Step:} &  \\
\end{tabular}

Assume $P(k)$ is true for an arbitrary, fixed integer $k \geq 0$, that is,
\begin{align*}
	\frac{1}{1\cdot 3} + \frac{1}{3\cdot 5} + \frac{1}{5\cdot 7} + \cdots + \frac{1}{(2k-1)(2k+1)} = \frac{k}{2k+1} \tag{IH}
\end{align*}

Show $P(k+1)$ is true, that is, 
\[ \frac{1}{1\cdot 3} + \frac{1}{3\cdot 5} + \frac{1}{5\cdot 7} + \cdots + \frac{1}{(2k+1)(2k+3)} = \frac{k+1}{2(k+1)+1} = \frac{k+1}{2k+3} \]

Start with $P(k)$ and add the next term of the sequence $\frac{1}{(2k+1)(2k+3)}$ to both sides:
\begin{align*}
	\frac{1}{1\cdot 3} + \frac{1}{3\cdot 5} + \frac{1}{5\cdot 7} + \cdots + \frac{1}{(2k+1)(2k+3)} &\overset{IH}{=} \frac{k}{2k+1} + \frac{1}{(2k+1)(2k+3)} \\
	&= \frac{2k+3}{2k+3}\cdot\frac{k}{2k+1} + \frac{1}{(2k+1)(2k+3)} \\
	&= \frac{2k^2 + 3k + 1}{(2k+1)(2k+3)} \\
	&= \frac{(2k+1)(k+1)}{(2k+1)(2k+3)} \\
	&= \frac{k+1}{2k+3}
\end{align*}
This shows $P(k+1)$ is true, assuming $P(k)$ is true, completing the inductive step.

\smallskip
Consequently, we have shown the basis step and inductive step, by mathematical induction, $P(n)$ is true for all $n \geq 1$.
\end{solution}


%\question[9] Prove that $\ds \sum_{i=0}^n (2i-2) = (n-2)(n+1)$ for natural numbers using induction.
%    \ifprintanswers
%        \vspace{-10pt}
%   \fi
%\begin{solution}
%	Let $P(n)$ be $\ds \sum_{i=0}^n (2i-2) = (n-2)(n+1)$  for every natural number $n \geq 0$.
%	
%\smallskip
%\begin{tabular}{lp{4in}}
%  \textit{Basis Step:} & Show $P(0)$ is true, $\sum_{i=0}^0 (2i-2) = 2\cdot0 -2 = -2 = (0-2)(0+1) = -2$ \\
%   & \\
% \textit{Inductive Step:} &  \\
%\end{tabular}
%
%Assume $P(k)$ is true for an arbitrary, fixed integer $k \geq 0$, that is,
%\begin{align*}
%	\sum_{i=0}^k (2i-2) = (k-2)(k+1) \tag{IH} \\
%\end{align*}
%
%Show $P(k+1)$ is true, that is,
%\[ \sum_{i=0}^{k+1} (2i-2) = ((k+1)-2)((k+1)+1) = (k-1)(k+2) = k^2 + k - 2\]
%
%Start with equation for $P(k)$ add $2(k+1)-2$ to both sides (the next term in the summation),
%\begin{align*}
%	\sum_{i=0}^k (2i-2) + 2(k+1)-2 &\overset{IH}{=} (k-2)(k+1) + 2(k+1)-2 \\
%	\sum_{i=0}^{k+1} (2i-2) &= (k-2)(k+1) + 2k \\
%	&= k^2 -k -2 +2k \\
%	&= k^2 +k -2 \\
%\end{align*}
%
%This shows $P(k+1)$ is true, assuming $P(k)$ is true, completing the inductive step.
%
%\smallskip
%Consequently, we have shown the basis step and inductive step, by mathematical induction, $P(n)$ is true for all $n \geq 0$.
%\end{solution}


\question[9] Rosen Ch 5.1 \# 22, p. 330
    \ifprintanswers
        \vspace{-10pt}
   \fi
\begin{solution}
$n^2 \leq n!$ hold for $n=0$, $1$, and $n \geq 4$.  Prove $n^2 \leq n!$ for all $n \geq 4$.

\smallskip 
Let $P(n)$ be the proposition $n^2 \leq n!$.

\smallskip
\begin{tabular}{lp{4in}}
  \textit{Basis Step:} & Show $P(4)$ is true, $4^2 = 16 \leq 24 = 4!$ holds \\
   & \\
 \textit{Inductive Step:} &  \\
\end{tabular}

Assume $P(k)$ is true for an arbitrary, fixed integer $k \geq 4$, that is,
\begin{align*}
	k^2 \leq k! \tag{IH} \\
\end{align*}

Show $P(k+1)$ is true, that is,
\[  (k+1)^2 \leq (k+1)! \]

Start by expanding the \textit{lhs} of the expression $P(k+1)$
\begin{align*}
	(k+1)^2 = k^2 + 2k + 1 &\overset{IH}{\leq} k! + 2k +1 \\
	 &\leq k! + 2k + k = k! + 3k  \tag{$k \geq 1$} \\
	 &\leq k! + k\cdot k \tag{$k \geq 3$} \\
	 &\leq k! + k\cdot k! = (k+1)k! = (k+1)! 
\end{align*}

This shows $P(k+1)$ is true, assuming $P(k)$ is true, completing the inductive step.

\smallskip
Consequently, we have shown the basis step and inductive step, by mathematical induction, $P(n)$ is true for all $n \geq 4$.
\end{solution}


%\question[9] Rosen Ch 5.1 \# 18, p. 330
%    \ifprintanswers
%        \vspace{-10pt}
%   \fi
%\begin{solution}
%\begin{enumerate}[label=(\alph*), itemsep=0pt, topsep=0pt, parsep=0pt]
%    \item (1 point) $P(2) \;:\;  2! < 2^2$.
%    \item (1 point) $P(2) \;:\; 2 = 2! < 2^2 = 4$
%    \item (2 points) The inductive hypothesis is that $P(k)$ is true for an arbitrary, fixed integer $k \geq 2$, that is $k! < k^k.$
%    \item (2 points) The inductive step should show for any $k \geq 2$ if $P(k)$ is true then $P(k+1)$ is true. That is prove, $(k+1)! < (k+1)^{k+1}$ given the inductive hypothesis.
%    \item (4 points)
%    \begin{align*}
%        k! &< k^k \tag{mult. both sides by (k+1)} \\
%        (k+1)\cdot k! &< k^k \cdot (k+1) \\
%        (k+1)! &< (k+1)\cdot k^k \\
%        (k+1)! &< (k+1)\cdot (k+1)^k  \tag{ $k < k+1$ } \\
%        (k+1)! &< (k+1)^{k+1} \\
%    \end{align*}
%    This completes the inductive step.
%    \item (2 points) By completing the basis and inductive step, by the principle of mathematical induction, the statement $P(n) \;:\; n! < n^n$ must hold for integers $n \geq 2$.
%\end{enumerate}
%\end{solution}




\question[8] Rosen Ch 5.1 \# 32, p. 330
    \ifprintanswers
        \vspace{-10pt}
   \fi
\begin{solution}
Let $P(n)$ be the proposition is $3 \;|\; n^3 + 2n$.

\smallskip
\begin{tabular}{ll}
  \textit{Basis Step:} & Show $P(1)$ is true, $3 \;|\; n^3 + 2n$ or $3 \;|\; 1 + 2$ or $3 \;|\; 3$ which is true. \\
   & \\
 \textit{Inductive Step:} &  \\
\end{tabular}

Assume $P(k)$ is true for an arbitrary, fixed
 integer $k \geq 1$, that is,
\begin{align*}
	3 \;|\; k^3 + 2k \tag{IH} 
\end{align*}

Show $P(k+1)$ is true, that is,
\[ 3 \;|\; (k+1)^3 + 2(k+1) \]

Start with the expression used in $P(k+1)$, we want to show this is divisible by 3.
\begin{align*}
	(k+1)^3 + 2(k+1) &=  k^3 + 3k^2 + 3k + 1 + 2k + 2 \\
	 &= (k^3 + 2k) + (3k^2 + 3k + 3) \\
\end{align*}
Each of the terms in parenthesis are divisible by 3.  The first term $k^3 + 2k$ is divisible by 3 using the Inductive Hypothesis, the second term has a 3 pulled from the expression, $3\cdot(k^2 + k + 1)$ so it is also divisible by 3. 

This shows $P(k+1)$ is true when $P(k)$ is true, completing the
inductive step.

\smallskip
Hence, the basis step and inductive step are completed, by mathematical induction $P(n)$ is true for all $n$ such
that $n\geq 1$.
\end{solution}



\question[9] Rosen Ch 5.2 \# 4, p. 341-342
    \ifprintanswers
        \vspace{-10pt}
   \fi
\begin{solution}
\begin{parts}
    \part (1 points) Show $P(18)$, $P(19)$, $P(20)$, $P(21)$ are true.

        \begin{tabular}{l}
           $7 + 7 + 4 = 18$ \\
           $7 + 4 + 4 + 4 = 19$\\
           $4 + 4 + 4 + 4 + 4 = 20$\\
           $7 + 7 + 7 = 21$
        \end{tabular}
    \part (2 points) What is the inductive hypothesis? \\
        Using just 4-cent and 7-cent stamps we can form $j$ cents of postage, $P(j)$ for all $j$ with $ 18 \leq j \leq k$, assuming $k \geq 21$.
    \part (2 points) What do you need to prove in the inductive step? \\
        Show, assuming the inductive hypothesis, that we can form $k+1$ cents using only 4-cent and 7-cent stamps, $P(k+1)$.
    \part (2 points) Complete the inductive step. \\
        From the hypothesis, we can assume $P(k-3)$ is true, that is we can form $k-3$ cents of postage.  To form $k+1$ cents, add another 4-cent stamp.
    \part (2 points) Explain why this statement is true for $n \geq 18$. \\
        We have completed both the basis step and inductive step, therefore, by strong induction, the statement is true for all $n \geq 18$.
\end{parts}
\end{solution}



\question[4] Rosen Ch 5.3 \# 4c, p. 357
    \ifprintanswers
        \vspace{-10pt}
   \fi
\begin{solution}
Find $f(2)$, $f(3)$, $f(4)$, and $f(5)$ if $f$ is defined
recursively by $f(0) = f(1) = 1$ and for $n=1,2,\ldots$
\begin{itemize}
%    \item[(a)] $f(n+1) = f(n) - f(n-1)$
%    \begin{align*}
%        f(2) &= f(1) - f(0) = 1 - 1 = 0 \\
%        f(3) &= f(2) - f(1) = 0 - 1 = -1 \\
%        f(4) &= f(3) - f(2) = -1 - 0 = -1 \\
%        f(5) &= f(4) - f(3) = -1 - -1 = 0 \\
%    \end{align*}
%    \item $f(n+1) = f(n)f(n-1)$
%    \begin{quote}
%    \begin{align*}
%        f(2) &= f(1)f(0) = 1 \\
%        f(3) &= f(2)f(1) = 1 \\
%        f(4) &= f(3)f(2) = 1 \\
%        f(5) &= f(4)f(3) = 1 \\
%    \end{align*}
%    \end{quote}
    \item[(c)] $f(n+1) = f(n)^2 + f(n-1)^3$.
    \begin{align*}
        f(2) &= f(1)^2 + f(0)^3 = 1^2 + 1^3 = 2 \\
        f(3) &= f(2)^2 + f(1)^3 = 2^2 + 1^3 = 5 \\
        f(4) &= f(3)^2 + f(2)^3 = 5^2 + 2^3 = 33 \\
        f(5) &= f(4)^2 + f(3)^2 = 33^2 + 5^3 = 1214 \\
    \end{align*}
\end{itemize}
\end{solution}


\question[6] Rosen Ch 5.3 \# 26a, p. 358. (See book examples and \#
27 to help with this problem)
\begin{solution}
Let $S$ be the subset of the set of ordered pairs of integers defined recursively by \\
\textit{Basis Step}: $(0,0) \in S$. \\
\textit{Recursive Step}: If $(a,b) \in S$, then $(a+2,b+3)\in S$ and $(a+3,b+2) \in S$. \\
\begin{enumerate}
    \item List the elements of $S$ produced by the first five applications of the recursive definition.
\end{enumerate}
\begin{quote}
    From the basis step $(0,0) \in S$.  Each application of the recursive definition will be given:

    \begin{tabular}{rllllll}
    1 & (2,3), & (3,2) \\
    2 & (4,6), & (5,5), & (6,4) \\
    3 & (6,9), & (7,8), & (8,7), & (9,6) \\
    4 & (8,12), & (9,11), & (10,10), & (11,9), & (12,8) \\
    5 & (10,15), & (11,14), & (12,13), & (13,12), & (14,11), & (15,10) \\
    \end{tabular}
\end{quote}
\end{solution}


\question[5] Rosen Ch 5.3 \# 28a, p. 358
    \ifprintanswers
        \vspace{-10pt}
   \fi
\begin{solution}
	The basis step for the set are $(1,2)$ and $(2,1)$.  
	
	The parity of the sum of $(a,b), a + b$ must remain constant, therefore, either $a$ may be increased by 2, $b$ may be increased by 2, or $a$ and $b$ are increased by 1.  The recursive step is: if $(a,b) \in S$, then $(a+2,b) \in S$, $(a,b+2) \in S$, and $(a+1,b+1)\in S$.
\end{solution}


\bonusquestion[3] The sequence in \ref{prob1}(a) starts with a term 4, that is a perfect square --- an integer of the form $k^2$, where $k$ is also an integer.  Find the next three perfect squares in this sequence.
    \ifprintanswers
        \vspace{-10pt}
   \fi
\begin{solution}
	The terms of the sequence that are perfect squares are: $4, 81, 169, 400, 576$.
	Each term can be tested by asking $\sqrt{a_i} \; \text{mod}\; 1 == 0$
\end{solution}


\bonusquestion[4] Rosen Ch 2.4, \# 22, p. 168-169. \\
Let $a_n$ be the salary in thousands of dollars, $n$ years after 2009.
    \ifprintanswers
        \vspace{-10pt}
   \fi
\begin{solution}
\begin{enumerate}[label=(\alph*),topsep=0pt,itemsep=0pt,parsep=0pt]
	\item (1 pt) $a_0 = 50$, $a_n = 1 + 1.05a_{n-1}$ 
	\item (1 pt) $a_8 =  83.4$
	\item (2 pt) Use the iterative approach with the answer to (a) 
	\begin{align*} 
		a_n &= 1 + 1.05a_{n-1} \\
		&= 1 + 1.05(1 + 1.05a_{n-2})  = 1 + 1.05 + 1.05^2a_{n-2}\\
		&= 1 + 1.05 + 1.05^2(1 + 1.05a_{n-3})) = 1 + 1.05 + 1.05^2 + 1.05^3a_{n-3} \\
		&= \cdots \\
		&= 1 + 1.05 + 1.05^2 + \cdots + 1.05^{n-1} + 1.05^na_0 \\
		&= \frac{1.05^n - 1}{1.05 - 1} + 50 \cdot 1.05^n \\
		&= 70 \cdot 1.05^n - 20 
	\end{align*}
\end{enumerate} 
\end{solution}

\end{questions}
\end{document} 