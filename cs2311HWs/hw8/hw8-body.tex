\begin{document}
\extrawidth{0.5in} \extrafootheight{-0in} \pagestyle{headandfoot}
\headrule \header{\textbf{cs2311 - Fall 2017}}{\textbf{HW
 8 \ifprintanswers - Solutions \fi}}{\textbf{Due: Fri. 11/10/17}} \footrule \footer{}{Page \thepage\
of \numpages}{}
\ifprintanswers
\noindent \textbf{Instructions:} All assignments are due \underline{by
\textbf{midnight} on the due date} specified.  Assignments should be typed or
scanned and submitted as a PDF in Canvas.

\medskip
\noindent Every student or student group must write up their own solutions in
their own manner.

\medskip
\noindent You should \underline{complete all problems}, but \underline{only a
subset will be graded} (which will be graded is not known to you ahead of
time).
\else
\noindent \textbf{Instructions:} All assignments are due \underline{by \textbf{midnight} on the due date} specified.

\medskip
\noindent Every student or student group  must write up their own solutions in
their own manner.

\medskip
\noindent Please present your solutions in a clean, understandable
manner.  Use the provided files that give mathematical notation in Word, Open Office, Google Docs, and \LaTeX.

\medskip
\noindent Assignments should be typed or scanned and submitted as a PDF.

\medskip
\noindent You should \underline{complete all problems}, but \underline{only a
subset will be graded} (which will be graded is not known to you ahead of
time).
\fi
\begin{questions}

\ifprintanswers
\else
\uplevel{
For many of these proof problems you will be asked to proof expressions related to even (definition p. 83), odd (definition p. 83), and divides (definition p. 238).}


  % For the next two problems, you will use the definition of
  % $n$ being \textit{divisible} by $m$, specifically:

  % \smallskip
  % \textbf{Definition:} For integers $n$ and $m$ and $m\neq0$, $n$ is
  % \underline{divisible} by $m$ if and only if there is a integer $k$ such that $n=km$.  The notation $m | n$ denotes, ``$m$ divides $n$".}

% \smallskip
% \uplevel{
%   For example, 56 is divisible by 8; there is a natural number $k$
%   (specifically, 7) such that $56 = 8k$. Also, thinking of the
%   bi-implication in the other direction, you can state because $39 = 3
%   \cdot 13$, 39 is divisible by 3 (or 39 is divisible by 13).}

%   \uplevel{An example of a proof using this new definition is to show,
%   \begin{center}
%    "For all integers $n$, if $n$ is divisible by 6, then $n^2$
%    is divisible by 9."
%   \end{center}
%   Follow the examples in class,
%   \begin{quote}
%   Assume $n$ is divisible by 6.  Then, by definition of divisible
%   there is some integer $k$ s.t. $n=6k$. Compute $n^2$:
%   \[ n^2 = (6k)^2 = 36k^2 = 9(4k^2). \]
%   where $n^2$ is divisible by 9 by definition. Therefore, for all
%   integers, if $n$ is divisible by 6, then $n^2$ is divisible
%   by 9.
%   \end{quote}}
\fi


\gquestion{6}{6}{all} Prove: the product of two odd integers is odd. 
    \ifprintanswers
        \vspace{-10pt}
    \fi
\begin{solution} 
\textbf{Proof:} Assume there are two odd integers $a$ and $b$.  Then, by definition of odd there exists two integers $k_a$ and $k_b$ such that $a=2k_a + 1$ and $b=2k_b + 1$.  Compute $a \cdot b$:
    $$ a\cdot b = (2k_a + 1)\cdot(2k_b + 1) = 4k_ak_b + 2k_a + 2k_b + 1 = 2(2k_ak_b + k_a + k_b) + 1$$
    the product of the two odd integers is the form of an odd number.
    Consequently, the product of two odd numbers is odd.
\end{solution}



\gquestion{6}{6}{all} Prove: For all natural numbers $m$ and $n$, if $m$ is divisible by 5 and $n$ is divisible by 4, then $m\cdot n$ is divisible by 10.
    \ifprintanswers
        \vspace{-10pt}
    \fi
  \begin{solution} \textbf{Proof:} Assume $m$ is divisible by 5 and also assume $n$ is divisible by 4.  By definition of divisibility, then there exists a integer $k_m$ and a integer $k_n$ such that $m = 5k_m$ and $n=4k_n$.  Then, $m\cdot n$ is:
    \[ m\cdot n = 5k_m \cdot 4k_n = 20k_mk_n = 10(2k_mk_n). \]
  From this expression, you can see $m\cdot n$ is divisible by 10.  Therefore, if $m$ is divisible by 5 and $n$ is divisible by 4, then $m \cdot n$ is divisible by 10.
  \end{solution}



\ugquestion{6} Prove: For any three consecutive natural numbers, the sum of the consecutive numbers is divisible by 3.
    \ifprintanswers
        \vspace{-10pt}
    \fi
  \begin{solution} \textbf{Proof:} Assume you have three consecutive natural numbers, with values $n$, $n+1$ and $n+2$.  Compute the sum:
    \[ n + (n+1) + (n+2) = 3n+3 = 3(n+1) = 3k. \]
  The sum is shown to be divisible by 3.  Consequently, the sum of three consecutive natural numbers is divisible by 3.
  \end{solution}




% discretebook-fall2015
\ugquestion{6} Prove: If $a$ is an even integer and $b$ is divisible by 3, then $ab$ is divisible by 6. 
    \ifprintanswers
        \vspace{-10pt}
    \fi
\begin{solution}
\textbf{Proof:}  Assume $a$ is even and $b$ is divisible by 3.  Then by definition, $a = 2m$ and $b=2n$ for some integers $m$ and $n$.  Consider $ab$: 

\[ ab = (2m)(3n) = 6mn = 6(mn), \]

this shows that $ab$ is divisible by 6.  Therefore, if $a$ is even and $b$ is divisible by 3, then $ab$ is divisible by 6.
\end{solution}




\gquestion{12}{6}{a}  Prove that if $n$ is an integer and $n^2 - 2n + 1$ is odd,
then $n$ is even using: (a) proof by contraposition and (b) proof by
contradiction.
    \ifprintanswers
        \vspace{-10pt}
    \fi
\begin{solution} \textbf{Proof by contraposition:}
    Assume $n$ is odd.  Then by definition of odd, there exists an
    integer $k$ such that $n=2k+1$.  Compute $n^2 - 2n + 1$:
    \[ n^2 - 2n + 1 = (2k+1)^2 - 2(2k+1) + 1 = 4k^2 + 4k + 1 - 4k - 2 + 1 = 4k^2 = 2(2k^2) \]
    Here, $n^2 - 2n + 1$ is in the form of an even number.  Therefore, by
    contraposition, if $n$ is an integer and $n^2 - 2n + 1$ is odd,
    then $n$ is even.

    \medskip
    \textbf{Proof by contradiction:} Assume $n$ is odd and $n^2 - 2n + 1$ is
    odd.  By definition of odd, there there exists an
    integer $k$ such that $n=2k+1$.  Compute $n^2 - 2n + 1$:
    \[ n^2 - 2n + 1 = (2k+1)^2 - 2(2k+ 1) + 1 = 4k^2 + 4k+ 1 - 4k - 2 + 1 = 4k^2 = 2(2k^2) \]
    Here, $n^2 - 2n + 1$ is in the form of an even number, however, we assumed
    $n^2 -2n +1$ is odd giving a contradiction. Therefore, by
    contradiction, it must be that if $n$ is an integer and $n^2 - 2n + 1$ is
    odd, then $n$ is even.
\end{solution}



% Book of proof
\gquestion{12}{6}{b}  Prove that if $n$ is an integer and $n^3 -1$ is even,
then $n$ is odd using: (a) proof by contraposition and (b) proof by
contradiction.
    \ifprintanswers
        \vspace{-10pt}
    \fi
\begin{solution} \textbf{Proof by contraposition:}
  Assume $n$ is even.  Then by definition of even, there exists an integer $k$ such that $n=2k$.  Compute $n^3 - 1$: 
  \[ n^3 - 1 = (2k)^3 - 1 = 8k^3 - 1 = 8k^3 -2 +1 = 2(4k^3 - 1) +1 = 2k' + 1. \]
  Here $n^3 - 1$ is of the form of an odd integer.  Therefore, by contraposition, if $n^3 -1$ is even, then $n$ is odd. 

  \medskip
  \textbf{Proof by contractiction:}
  Assume $n^3 -1$ is even and $n$ is even.  By definition of even, there exists an integer $k$ such that $n=2k$.  Compute $n^3 - 1$: 
  \[ n^3 - 1 = (2k)^3 - 1 = 8k^3 - 1 = 8k^3 -2 +1 = 2(4k^3 - 1) +1 = 2k' + 1. \]
  Here $n^3 - 1$ is of the form of an odd integer, however, we assumed $n^3 -1$ is odd giving a contradiction. Therefore, by contradiction, if $n^3 - 1$ is even, then $n$ is odd.   
\end{solution}




\ugquestion{13} Prove for all natural numbers $n$ and $m$, $nm$ is odd if
and only if $n$ and $m$ are both odd.
    \ifprintanswers
        \vspace{-10pt}
    \fi
\begin{solution}
     This proof is for a theorem using ``if and only if" therefore,
     it must be proved in both directions.  \\
    \textbf{Prove ``if p, then q" by contradiction:} Assume $nm$ is
    odd and $n$ and $m$ are not both odd.  Then at least one of $n$
    and $m$ are even.  Suppose $n$ is the one that is even. Then
    $n=2k$ for some natural number $k$.  Then $nm = 2km$ and so $nm$
    is even.  This contradicts the assumption that $nm$ is odd.
    Similarly, if we assume instead that $m$ is the one that is even,
    we reach the same contradiction.   If both $m$ and $n$ are both
    even we reach the same result.  Thus, it must be that $n$ and
    $m$ are both odd.

    \smallskip
    \textbf{Prove ``if q then p" directly:} Assume $n$ and $m$ are
    both odd.  Then, there are natural numbers $k$ and $j$ such
    that $n = 2k+1$ and $m=2j+1$.  Then,
    \[ nm = (2k+1)(2j+1) = 4kj + 2(k+j) +1 = 2(2kj +k + j) +1 \]
    So, $nm$ is odd by definition. Therefore, if $n$ and $m$ are
    odd, $nm$ is odd.

    \smallskip
    Combining both parts, we have shown for any natural number $n$ and $m$, $nm$ is odd if and only if $n$ and $m$ are both odd.
\end{solution}



\ugquestion{2} Prove that there is a positive integer that equals the
sum of the positive integers not exceeding it.
    \ifprintanswers
        \vspace{-10pt}
    \fi
\begin{solution}
    This is an existence proof.  3 is an example of such a positive
    integer, 3 = 1 + 2.
\end{solution}


\ugquestion{3} Prove or disprove: If $a$ and $b$ are rational
numbers, then $a^b$ is also rational.
    \ifprintanswers
        \vspace{-10pt}
    \fi
\begin{solution}
    \textbf{Disprove:} Let $a=2$ and $b=\frac{1}{2}$, which are both rational numbers.  Then $a^b =
    \sqrt{2}$ which is irrational.
\end{solution}


\ugquestion{3} Prove or disprove: The sum of four consecutive integers is divisible by 4.
    \ifprintanswers
        \vspace{-10pt}
    \fi
\begin{solution}
    \textbf{Disprove:} Let the consecutive integers be 1, 2, 3, 4; the sum is 10 which is not divisible by 4.
\end{solution}



\ugquestion{6} Prove that if $n$ is an integer that $n^3 - n$ is even.
    \ifprintanswers
        \vspace{-10pt}
    \fi
%\begin{EnvFullwidth}
%\begin{TheSolution}
\begin{solution} \textbf{Proof:} Let $n$ be an integer.
    \begin{itemize}[itemsep=0pt,parsep=0pt,topsep=0pt,partopsep=0pt]
        \item[Case] (i): Let $n$ be even. By definition, there exists an integer $k$ s.t. $n=2k$.
            \[ n^3 - n = (2k)^3 - 2k = 8k^3 - 2k = 2(4k^3 - k) \]
        This is of the form of an even number.
        \item[Case] (ii):  Let $n$ be odd.  By definition, there exists an integer $k$ s.t. $n = 2k+1$.
            \begin{align*}
                n^3 - n &= (2k+1)^3 - (2k+1) = (4k^2 + 4k+ 1)(2k + 1) - (2k+1) \\
                &= 8k^3 + 12k^2 + 6k + 1 - 2k - 1 = 8k^3 + 12k^2 + 4k = 2(8k^3 + 6k^2 + 2k)
            \end{align*}
        This is of the form of an even number.

        Therefore, because $n^3 - n$ is even in all cases, it holds that for any integer $n$, $n^3 - n$ is even.
    \end{itemize}
\end{solution}


\gquestion{6}{6}{all} Prove: Suppose $a$ and $b$ are integers. If $a^2(b^2 - 2b)$ is odd, then $a$ and $b$ are odd. 
    \ifprintanswers
        \vspace{-10pt}
    \fi
%\begin{EnvFullwidth}
%\begin{TheSolution}
\begin{solution} \textbf{Proof:} This proof technique using proof by contraposition and proof by cases. 

Suppose it is not the case that $a$ and $b$ are odd. Then, at least one of $a$ and $b$ is even.  Suppose: 
\begin{itemize}
  \item Case 1.  Suppose $a$ is even.  Then, $a = 2c$ for some integer $c$.  Thus, $a^2(b^2 -2b) = (2c)^2(b^2 - 2b) = 2(2c(b^2 - 2b)) = 2k$, which is of the form of an even integer. 
  \item Case 2.  Suppose $b$ is even.  Then, $b = 2c$ for some integer $c$.  Thus, $a^2(b^2 -2b) = a^2((2c)^2 - 2(2c)) = 2(a^2(2c^2 - 2c)) = 2k'$, which is of the form of an even integer. 
\end{itemize}

For each case, we have shown that $a^2(b^2 - 2b)$ is even.  Therefore, by contraposition, if $a^2(b^2 - 2b)$ is odd, then $a$ and $b$ are odd.
\end{solution}




\section*{Sequences} 

% f14, s16
\ugquestion{5}  What are the first four terms of each sequence: 


\begin{minipage}{0.5\textwidth}
\begin{enumerate}[label=(\alph*),itemsep=0pt,parsep=0pt,topsep=0pt]
  \item $a_n = 4-2n\;\;\forall n \geq 0$ \hspace{0.25in} 
  \item $b_n = 6- 3\cdot 2^n \;\;\forall n \geq 0$
  \item $c_1 = 4,  c_n = 3\cdot c_{n-1} - 2 \;\; \forall n \geq 2$ 
  \item $d_1 = -1, d_n = 5\cdot d_{n-1} + n\;\; \forall n \geq 2$ 
  \item $e_0 = 1, e_1 = 1$, $e_n = ne_{n-1} + n^2e_{n-2} + 1\;\forall n \geq 2$
\end{enumerate}
\end{minipage}
% 
\begin{minipage}{0.5\textwidth}
\begin{solution}
\begin{enumerate}[label=(\alph*),itemsep=0pt,parsep=0pt,topsep=0pt]
  \item $a_0 = 4$, $a_1 = 2$, $a_2 = 0$, $a_3 = -2$
  \item $b_0 = 3$, $b_1 = 0$, $b_2 = -12$, $b_3 = -48$
  \item $c_1 = 4$, $c_2 = 10$, $c_3 = 28$, $c_4 = 82$
  \item $d_1 = -1$, $d_2 = -3$, $d_3 = -12$, $d_4 = -46$ 
  \item $e_0 = 1$, $e_1 = 1$, $e_2 = 6$, $e_3 = 27$
\end{enumerate}
\end{solution}
\end{minipage}



% f14, s16
\gquestion{12}{6}{a-b,d} Find a closed formula for each sequence; assume the sequence starts with $n=0, 1, 2, \ldots$. 

\begin{minipage}{0.5\textwidth}
\begin{enumerate}[label=(\alph*),itemsep=0pt,parsep=0pt,topsep=0pt]
  \item $1, -4, 9, -16, 25, \ldots$ 
  \item $8, 4, 2, 1, \frac{1}{2}, \frac{1}{4},  \ldots$
  \item $3, 4, 7, 12, 19, 28, 39 \ldots$
  \item $6, 1, -4, -9, -14, -19 \ldots$ 
  \item $3, 6, 12, 24, 48, \ldots$
  \item $1, 0, 1, -4, 9, -16, 25, -36 \ldots$
\end{enumerate}
\end{minipage}
% 
\begin{minipage}{0.5\textwidth}
\begin{solution}
\begin{enumerate}[label=(\alph*),itemsep=0pt,parsep=0pt,topsep=0pt]
  \item $a_n = (n+1)^2 (-1)^n$ 
  \item $a_n = 8\cdot (\frac{1}{2})^n$ 
  \item $a_n = n^2 + 3$
  \item $a_n = 6 - 5n$
  \item $a_n = 3\cdot 2^n$ 
  \item $a_n = (n-1)^2 (-1)^n$
\end{enumerate}
\end{solution}
\end{minipage}



\gquestion{8}{4}{b-c} Find a recursive formula for each sequence; assume the sequence with $n=0, 1, 2, \ldots$.

\begin{minipage}{0.5\textwidth}
\begin{enumerate}[label=(\alph*),itemsep=0pt,parsep=0pt,topsep=0pt]
  \item $3, 6, 12, 24, 48, \ldots$  
  \item $7, 10, 15, 22, 31, 42, 55, 70, \ldots$\
  \item $9, 4, -1, -6, -11, -16, \ldots$
  \item $2, 5, 13, 42, 171, 858, 5151, \ldots$
\end{enumerate}
\end{minipage}
% 
\begin{minipage}{0.5\textwidth}
\begin{solution}
\begin{enumerate}[label=(\alph*),itemsep=0pt,parsep=0pt,topsep=0pt]
  \item $a_0 = 3$, $a_n = 2\cdot a_{n-1}$
  \item $a_0 = 7$, $a_n = a_{n-1} + 2n + 1$ 
  \item $a_0 = 9$, $a_n = a_{n-1} + -5$
  \item $a_0 = 2$, $a_n = n\cdot a_{n-1} + 3$
\end{enumerate}
\end{solution}
\end{minipage}



\ugquestion{6} Determine whether each answer is a solution to the recurrence relation, 
$$ a_n = a_{n-1} + 2a_{n-2} + 2n - 9 $$
    \begin{enumerate}[label=(\alph*),topsep=0pt,itemsep=0pt,parsep=0pt]
        \item $a_n = 0$
        \item $a_n = -n +2$
    \end{enumerate}
    \ifprintanswers
        \vspace{-10pt}
    \fi
\begin{solution}
    \begin{enumerate}[label=(\alph*),topsep=0pt,itemsep=0pt,parsep=0pt]
        \item Not a solution.
        \begin{align*}
            a_n &= a_{n-1} + 2a_{n-2} + 2n - 9 \\
            &= (0) + 2\cdot0 + 2n - 9 \\
            &= 2n - 9 \\
            0 &\neq 2n - 9
        \end{align*}
        \item Yes, $a_n = -n + 2$ is a solution.
        \begin{align*}
            a_n &= a_{n-1} + 2a_{n-2} + 2n - 9 \\
            &= (-(n-1) + 2) + 2(-(n-2) + 2) + 2n -9 \\
            &= -n + 1 + 2 -2n + 4 + 4 + 2n -9 \\
            &= -n +2 \\
        \end{align*}
    \end{enumerate}
\end{solution}


\section*{Bonus Questions} 




\bonusquestion[6] Prove: For an integer $a$ if $7 | 4a$, then $7 | a$. \\
\textit{Hint: you may want to use the definition of even and knowledge of the products of even and odd integers.}
 \ifprintanswers
        \vspace{-10pt}
    \fi
  \begin{solution} 
  \textbf{Proof:}  Assume $a$ is an integer and $7 | 4a$.  By definition of divisibility, there is some integer $c$ such that $4a = 7c$.  We know that $4a$ is even because it can be written as $2(2a)$.  Also, because $4a = 7c$ we know that $7c$ is even.  If $7c$ is even, then $c$ must be even (odd*odd = odd and odd*even = even).  Then, we can replace $c$ with $2d$ for some integer $d$.  

  From the initial equation, replace $c=2d$, we get the following: 
  \begin{align*}
    4a &= 7c \\
      &= 7(2d) = 14d \\
    2a &= 7d \tag{divide by 2}
  \end{align*}
  Since $7d$ is equal to $2a$ if follows that $7d$ is even. Using the same reasoning as above $d$ must be even (odd*odd = odd and odd*even = even).  With $d$ as an even number, then it can be respresented as $d = 2e$ for some integer $e$. 

  Replacing $d$ in the equation above, we get the following:
  \begin{align*}
    2a &= 7d \\
      &= 7(2e) = 14e \\
    a &= 7e \tag{divide by 2}
  \end{align*}
  At this point we have shown, $a = 7e$ which means that $7 | a$ by definition of divisibility. 

  Therefore, for an integer $a$ if $7 | 4a$, then $7 | a$.

  \end{solution}


\bonusquestion{4} Prove the statement: For all integers $a$, $b$, and $c$, if $a^2 + b^2 = c^2$, then $a$ or $b$ is even.
   \ifprintanswers
        \vspace{-10pt}
    \fi
\begin{solution} 
\textbf{Proof by contradiction:} Assume there are integers $a$, $b$, and $c$ such that $a^2 + b^2 = c^2$ and $a$ and $b$ are both odd.  By definition, there exists and integer $k_a$ and integer $k_b$ such that $a = 2k_a + 1$ and $b = 2k_b + 1$.  Consider the expression: 

\begin{align*}
  a^2 + b^2 &= (2k_a +1)^2 + (2k_b + 1)^2 = 4k_a^2 + 4k_a + 1 + 4k_b^2 + 4k_b + 1 \\
   &= 4(k_a^2 + k_b^2 + k_a + k_b) + 2,
\end{align*}

where $c^2 = 4(k_a^2 + k_b^2 + k_a + k_b) + 2$, this shows $c^2$ is even, $c^2 = 2(2k_a^2 + 2k_b^2 + 2k_a + 2k_b + 1) = 2k'$.  With $c^2$ even, this means that $c$ is even.  But, then $c^2$ must be a multiple of 4. However, this is a contradiction because $4(k_a^2 + k_b^2 + k_a + k_b) + 2$ is not a multiple of 4. 

Therefore, by contradiction, if $a^2 + b^2 = c^2$, then $a$ or $b$ is even.
\end{solution}


% \bonusquestion[4] Prove the following proposition, ``For any integer $n \geq 2$, $n^2-3$ is never divisible by 4."
%     \ifprintanswers
%         \vspace{-10pt}
%     \fi
% \begin{solution}
%     Consider two cases:
%     \begin{itemize}
%         \item[Case 1]: If $n$ is even, then by definition $n^2$is also even and $n^2-3$ will be odd.  Therefore, $n^2-3$ is not divisible by 4.
%         \item[Case 2]: If $n$ is odd, there are 4 cases to consider. $n$ can be written as $4k$, $4k+1$, $4k+2$, and $4k+3$ for an integer $k$.  The forms of $4k$ and $4k+2$ are even, not matching the condition $n$ is odd, and will not be considered further.
%             \begin{itemize}
%                 \item[Case a]:
%                 \begin{align*}
%                     n &= 4k + 1 \\
%                     n^2 - 3 &= (4k + 1)^2 - 3 \\
%                         &= 16k^2 + 8k + 1 - 3 \\
%                         &= 16k^2 +8k - 2
%                 \end{align*}
%                 The number $n^2-3$ is not divisible by 4.
%                 \item[Case b]:
%                 \begin{align*}
%                     n &= 4k+3 \\
%                     n^2 - 3 &= (4k + 3)^2 - 3 \\
%                      &= 16k^2 + 24k + 9 - 3 \\
%                      &= 16k^2 + 24k + 6
%                 \end{align*}
%                 The number $n^2-3$ is not divisible by 4.
%                 Therefore, if $n$ is odd, $n^2-3$ is not divisible by 4.
%             \end{itemize}
%         \end{itemize}
%         Therefore, we have shown in all cases that for any integer $n \geq 2$, $n^2 - 3$ is not divisible by 4.
% \end{solution}


\end{questions}
\end{document}