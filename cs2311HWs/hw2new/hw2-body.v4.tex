\begin{document}
\extrawidth{0.5in} \extrafootheight{-0in} \pagestyle{headandfoot}
\headrule \header{\textbf{CS2311 - Spring 2021}}{\textbf{HW
 2  \ifprintanswers - Solutions \fi}}{\textbf{Due: ***. **/**/21}} \footrule \footer{}{Page \thepage\
of \numpages}{}


\ifprintanswers\
\noindent\ \textbf\{Instructions:\}\ All\ assignments\ are\ due\ \underline\{by\
\textbf\{midnight\}\ on\ the\ due\ date\}\ specified\.\ \ Assignments\ should\ be\ typed\ or\
scanned\ and\ submitted\ as\ a\ PDF\ in\ Canvas\.\ \ \ \
\
\medskip\
\noindent\ Every\ student\ or\ student\ group\ must\ write\ up\ their\ own\ solutions\ in\
their\ own\ manner\.\
\
\medskip\
\noindent\ You\ should\ \underline\{complete\ all\ problems\},\ but\ \underline\{only\ a\
subset\ will\ be\ graded\}\ \(which\ will\ be\ graded\ is\ not\ known\ to\ you\ ahead\ of\
time\)\.\ \
\else\
\noindent\ \textbf\{Instructions:\}\ All\ assignments\ are\ due\ \underline\{by\ \textbf\{midnight\}\ on\ the\ due\ date\}\ specified\.\ \ \
\
\medskip\
\noindent\ Every\ student\ or\ student\ group\ \ must\ write\ up\ their\ own\ solutions\ in\
their\ own\ manner\.\
\
\medskip\
\noindent\ Please\ present\ your\ solutions\ in\ a\ clean,\ understandable\
manner\.\ \ Use\ the\ provided\ files\ that\ give\ mathematical\ notation\ in\ Word,\ Open\ Office,\ Google\ Docs,\ and\ \LaTeX\.\ \
\
\medskip\
\noindent\ Assignments\ should\ be\ typed\ or\ scanned\ and\ submitted\ as\ a\ PDF\.\ \ \ \
\
\medskip\
\noindent\ You\ should\ \underline\{complete\ all\ problems\},\ but\ \underline\{only\ a\
subset\ will\ be\ graded\}\ \(which\ will\ be\ graded\ is\ not\ known\ to\ you\ ahead\ of\
time\)\.\ \
\fi
\begin{solution}
	\vspace{-5pt}
	\begin{enumerate}[label=(\alph*),itemsep=3pt,parsep=0pt,topsep=0pt,partopsep=0pt]
		\item $\lfloor 8.6 \rfloor = 8$
		\item $\lceil 4.3 \rceil = 5$
		\item $\lfloor -3.6 \rfloor = -4$
		\item $\lfloor 10.5\bar{3} \rfloor $ = 10
		\item $\lceil -2.1 \rceil = -2$
		\item $\lfloor \frac{-3}{4} \rfloor = -1$ 
		\item $\lceil \frac{13}{3} + \lfloor \frac{-5}{4} \rfloor \rceil = 3$
		\item $\lfloor 3.2 - \lceil 10.4 \rceil \rfloor = -8$
	\end{enumerate}
\end{solution}

   % \ifprintanswers
   %      \vspace{-10pt}
   % \fi
% \begin{solution}
% \renewcommand{\arraystretch}{1.3}
% 	\begin{tabular}{llll}
% 		(a) $\lfloor 8.6 \rfloor = 8$ \hspace*{0.5in} 
% 		& (b) $\lceil 4.3 \rceil = 5$ \hspace*{0.5in} &
% 		(c) $\lfloor -3.6 \rfloor = -4$ 
% 		& (d) $\lfloor 10.5\bar{3} \rfloor $ = 10\\
% 		(e) $\lceil -2.1 \rceil = -2$ 
% 		& (f) $\lfloor \frac{-3}{4} \rfloor = -1$ &
% 		(g) $\lceil \frac{13}{3} + \lfloor \frac{-5}{4} \rfloor \rceil = 3$ & 
% 		(h) $\lfloor 3.2 - \lceil 10.4 \rceil \rfloor = -7$   \\
% 	\end{tabular}

% \end{solution}



\gquestion{14}{6}{c-e} Determine whether each of the following functions are (i) one-to-one and (ii) onto. 
\begin{enumerate}[label=(\alph*),itemsep=0pt,parsep=0pt,topsep=0pt,partopsep=0pt]
    \item \csol{ one-to-one \tab onto \tab } 
        Rosen Ch 2.3.10(a), p. 162 (p. 162 for 8th ed)
    \item \csol{ not one-to-one \tab not onto \tab }
        Rosen Ch 2.3.10(b), p. 162 (p. 162 for 8th ed)
    \item \csol{ not one-to-one \tab not onto \tab }
    	$f: \Z \ra \Z$ where $f(x) = x^2 - 5x + 5$.
    \item \csol{ one-to-one \tab not onto \tab }
    	$g: \N \ra \N$ where $g(n) = n+1$
    \item \csol{ not one-to-one \tab onto \tab }
    	$h: \N \ra \N$ where $h(n) = \lfloor \frac{n}{2} \rfloor$
    \item \csol{ not one-to-one \tab not onto \tab }
    	$i: \Z \times \Z \ra \Z$ where $i(m,n) = 2n - 4m$
    \item \csol{ not one-to-one \tab not onto \tab }
    	$j: \R \ra \R$ where $j(x) = \sqrt{x}$
\end{enumerate}
%     \ifprintanswers
%         \vspace{-10pt}
%     \fi
% \begin{solution}
% \begin{enumerate}[label=(\alph*),itemsep=0pt,parsep=0pt,
% topsep=0pt,partopsep=0pt]
%     \item one-to-one, onto
%     \item not one-to-one, not onto
%     \item not one-to-one, not onto
%     \item one-to-one, not onto 
%     \item not one-to-one, onto 
%     \item not one-to-one, not onto
%     \item not one-to-one, not-onto
% \end{enumerate}
% \end{solution}




\gquestion{4}{4}{all} Let $A$ and $B$ be finite sets, and $f$ be a function is $f: A \ra B$.  Determine which of the following statements are true. 
\begin{enumerate}[label=(\alph*),itemsep=0pt,parsep=0pt,topsep=0pt,partopsep=0pt]
    \item \csol{False }
    	If $f: A \ra B$ is onto, then the domain and range are not only the same size, but the same set. 
    \item \csol{True }
    	If $f: A \ra B$ is both one-to-one and onto, then $A$ and $B$ have the same cardinality. 
    \item \csol{True }
    	If $f: A \ra B$ is one-to-one, then $|A| \leq |B|$.
    \item \csol{True } 
    	If $f: A \ra B$ is onto, then $|A| \geq |B|$.
\end{enumerate}
    \ifprintanswers
        \vspace{-10pt}
    \fi
\begin{solution}
% \begin{enumerate}[label=(\alph*),itemsep=0pt,parsep=0pt,
% topsep=0pt,partopsep=0pt]
%     \item False 
%     \item True 
%     \item True 
%     \item True 
% \end{enumerate}
    % (a) False \hfill 
    % (b) True \hfill 
    % (c) True \hfill
    % (d) True \hfill
\end{solution}



\gquestion{12}{8}{c-f} Let $A = \{a, b, c, d, e \}$ and $B = \{ a, b, d, f, g\}$.  Let $f : A \ra B$ and $g : B \ra B$ with 
\begin{align*}
    f &= \{(a, b), (b, d), (c, g), (d, a), (e, b) \} \text{  and} \\
    g &= \{ (a, f), (b, d), (d, a), (f, g), (g, b) \}
\end{align*}
For each of the following compositions, define the function or explain why it is not defined.

\begin{tabular}{p{0.5in}llll}
  & (a) $f \circ g$ \hspace{0.3in} & (b) $g \circ f$ \hspace{0.3in} & (c) $f
\circ f$ \hspace{0.3in} & (d) $g \circ g$
\end{tabular}
\begin{itemize}[itemsep=0pt,parsep=0pt,topsep=0pt,partopsep=0pt]
    \item[(e)] Find $f^{-1}$ if it exists. If it doesn't, explain why not.
    \item[(f)] Find $g^{-1}$ if it exists. If it doesn't, explain why not.
\end{itemize}
   \ifprintanswers
        \vspace{-5pt}
   \fi
	\begin{solution}
	\begin{enumerate}[label=(\alph*),itemsep=3pt,parsep=0pt,topsep=0pt,partopsep=0pt]
		\item $f \circ g$ is not defined, the range of $g$ is not a subset of the domain of $f$
		\item $g \circ f = \{ (a,d), (b,a), (c,b), (d,f), (e,d) \}$
		\item $f \circ f$ is not defined, the range of $f$ is not a subset of the domain of $f$
		\item $g \circ g = \{ (a,g), (b,a), (d,f), (f,b), (g,d) \}$
		\item $f^{-1}$ does not exists, $f$ is not one-to-one and not onto
		\item $g^{-1}$ does exist, $= \{ (f,a), (d,b), (a,d), (g,f), (b,g) \}$
	\end{enumerate}
	\end{solution}



\ugquestion{8} Let $f$, $g$, and $h$ all be functions mapping from $A$ to $\mathbb{R}$ where $A = \{x \in \mathbb{R} \;|\; x >0 \}$,
$$ f(x) = \frac{1}{x+1}, \qquad g(x) = \frac{x+1}{x}, \text{ and }\qquad h(x) =
x-1. $$
 Compute (a) $(f \circ g)(x)$, (b) $(g \circ f)(x)$, (c) $(h \circ g \circ f)(x)$, (d) $(f \circ g \circ h)(x)$
    \ifprintanswers
        \vspace{-5pt}
   	\fi
	\begin{solution}
	\begin{enumerate}[label=(\alph*),itemsep=3pt,parsep=3pt,topsep=0pt,partopsep=0pt]
	    \item $\ds (f \circ g)(x) = f(g(x)) 
	      = f(\frac{x+1}{x}) 
	      = \frac{1}{\left(\frac{x+1}{x}\right) + 1}$
	    \item $\ds (g \circ f)(x) = g(f(x)) 
	      = g(\frac{1}{x+1}) 
	      = \frac{\left(\frac{1}{x+1}\right)+1}{\left(\frac{1}{x+1}\right)}$
	    \item $\ds (h \circ g \circ f)(x) = h(g(f(x)))
	      = h\left( g\left(\frac{1}{x+1} \right) \right)
	      = h\left( \frac{\left(\frac{1}{x+1}\right)+1}{\left(\frac{1}{x+1}\right)} \right)
	      = \frac{\left(\frac{1}{x+1}\right)+1}{\left(\frac{1}{x+1}\right)} - 1$
	    \item $\ds (f \circ g \circ h)(x) = f(g(h(x))) 
	      = f(g(x-1)) 
	      = f\left(\frac{(x-1) +1}{(x-1)}\right) 
	      = \frac{1}{\left(\frac{(x-1) +1}{(x-1)}\right) + 1}$
	\end{enumerate}
	\end{solution}



\gquestion{6}{3}{a} 
	\ifprintanswers
		Let $P$, be a set of \textit{Patients} who have ever been admitted to the hospital at some time, $B$, be a set \textit{Beds} available for patients. 

		The function \textit{currentBed} maps \textit{Patients} to \textit{Beds}. The function relates a patient to the bed that he/she is currently occupying in the hospital. 

		The function \textit{date1stAdmitted} that maps \textit{Patients} to \textit{Dates}, relating a patient to the date he/she was first admitted to the hospital.
        \vspace{-5pt}
   	\else
\begin{enumerate}[label=(\alph*),itemsep=0pt,parsep=3pt,topsep=0pt,partopsep=0pt]
	\item A hospital maintains a set of \textit{Patients}, $P$, who have ever been admitted to the hospital at some time. (No patient is ever deleted from this set, even after they leave the hospital.) 

	There is also a set \textit{Beds}, $B$, describing the beds available for patients. 

	Consider the function \textit{currentBed} that maps \textit{Patients} to \textit{Beds}. The function relates a patient to the bed that he/she is currently occupying in the hospital. 
	\begin{enumerate}
		\item Is this function total?
		\item Is the function one-to-one?
		\item Is the function onto?
	\end{enumerate}
	Explain your answers.

	\item Let \textit{Dates} be a set of dates. 

	Consider the function \textit{date1stAdmitted} that maps \textit{Patients} to \textit{Dates}, relating a patient to the date he/she was first admitted to the hospital.
	\begin{enumerate}
		\item Is this function total?
		\item Is the function one-to-one?
		\item Is the function onto?
	\end{enumerate}
	Explain your answers.
\end{enumerate}
	\fi	
\begin{solution}
	\begin{enumerate}[label=(\alph*),itemsep=4pt,parsep=0pt,topsep=0pt,partopsep=0pt]
		\item function, \textit{currentBed} :  \textit{Patients} $\ra$ \textit{Beds} 
		\begin{enumerate}
			\item No, the function is not total; not every patient is currently occupying a bed.
			\item Yes, no two patients may occupy the same bed. 
			\item No, there may be beds that are not occupied.
		\end{enumerate}

		\item function \textit{date1stAdmitted} : \textit{Patients} $\ra$ \textit{Dates}
		\begin{enumerate}
			\item Yes, the function is total; every patient admitted to the hospital has a date of first admittance. 
			\item No, the function is not one-to-one, multiple patients may be admitted on the same day. 
			\item No, the function is not onto, there may be a day with no patients admitted. 
		\end{enumerate}

	\end{enumerate}
	\end{solution}



\section*{Bonus}

\bonusquestion[1] For the following sets, state what the  corresponding shaded region of the Venn diagram represents. 


%http://randyridenour.net/2016/04/11/venn-diagrams-with-latex-and-tikz/

\def\sub{(0,0) circle (1.5cm)}
\def\mid{(-60:2cm) circle (1.5cm)}
\def\pred{(0:2cm) circle (1.5cm)}

\begin{tikzpicture}[thick]

\begin{scope}
    \draw (-2,-0) node {$A$};
    \draw (1,-4) node {$B$};
    \draw (4,0) node {$C$};

\begin{scope}[even odd rule]% Shade S without M
        \clip \mid (-1.5,-1.5) rectangle (1.5,1.5);
        \fill[gray] \sub;
        \end{scope}

\begin{scope} %Shade intersection of M and P
  \clip \pred;
  \fill[gray] \mid;
\end{scope}

\draw \sub;
\draw \pred;
\draw \mid;
\end{scope}

\end{tikzpicture}


\csol{$(A - B) \cup (B \cap C)$ or  $(A \cap \overline{B}) \cup (B \cap C)$ }





\end{questions}
\end{document}
