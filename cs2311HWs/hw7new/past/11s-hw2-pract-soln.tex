\documentclass[12pt]{exam}
% can include option [answers] to print out solutions, or command \printanswers
%  can turn addpoint on and off with commands, \addpoints and \noaddpoints

\usepackage{amsthm}
\usepackage{amsmath, amssymb}

\usepackage[normalmargins,normalsections,normalindent,normalleading]{savetrees}

% Tight lists
%\usepackage{mdwlist} - then use "starred" versions of itemize, enumerate, description
%  this still has spacing above and below.

\newenvironment{my_parts}{
\begin{parts}
    \setlength{\itemsep}{1pt}
    \setlength{\parskip}{0pt}
    \setlength{\parsep}{0pt}
}{\end{parts}}

\newenvironment{my_item}{
\begin{itemize}
    \setlength{\itemsep}{1pt}
    \setlength{\parskip}{0pt}
    \setlength{\parsep}{0pt}
}{\end{itemize}}

% or can include lines
%   \begin{parts}
%     \itemsep 1pt
%     \parskip 0pt
%     \item One
%     ...
%   \end{parts}

\newcommand{\ra}{$\rightarrow$\xspace}

\begin{document}
\extrawidth{0.5in} \extrafootheight{-0.75in} \pagestyle{headandfoot}
\headrule \header{\textbf{cs2311 - Spring 2011}}{\textbf{HW 2 -
Practice Problems Solutions}}{} \footrule \footer{}{Page \thepage\ of
\numpages}{}

\addpoints

\noindent \textbf{Instructions:} All assignments are due at the
beginning of class on the due date specified.  Solutions will be
handed out (or posted on-line) shortly thereafter.  Every student
must write up their own solutions in their own manner.

%\noindent Please follow the format in the book when asked to produce
%truth tables (this will help in grading in order to avoid any
%errors).

\noindent The assignment has \numpoints\ points.

\begin{questions}
\printanswers

\question Rosen Ch 1.3 \# 24(d,e), p. 48. \\
Use the following predicates:
\begin{itemize}
    \item $C(x)$ be ``$x$ is in your class"
    \item $Q(x)$ be ``$x$ can solve quadratic equation"
    \item $R(x)$ be ``$x$ wants to be rich"
\end{itemize}
    \begin{solution}
    \begin{my_item}
        \item[(d)] All students in your class can solve quadratic equations.
        $$ \forall x\; Q(x) \hspace{2in} \forall x\; (C(x) \rightarrow Q(x))$$
        \item[(e)] Some student in your class does not want to be rich.
        $$ \exists x\; \neg R(x) \hspace{2in} \exists x\; (C(x) \wedge \neg R(x))$$
    \end{my_item}
    \end{solution}

\question Rosen Ch 1.3 \#28(a,e), p. 48 and the statement below:\\
No tool is in the correct place. \\
Use the following predicates:
\begin{itemize}
    \item $C(x)$ be ``$x$ is in the correct place"
    \item $T(x)$ be ``$x$ is a tool"
    \item $E(x)$ be ``$x$ is in excellent condition"
\end{itemize}
    \begin{solution}
    \begin{my_item}
        \item[(a)] Something is not in the correct place
        $$ \exists x\; \neg C(x) $$
        \item[(e)] One of your tools is not in the correct place, but it is in excellent condition
        $$ \exists x\; (T(x) \wedge \neg C(x) \wedge E(x)) $$
        \item No tool is in the correct place
        \begin{align*}
            \forall &x\; (T(x) \rightarrow \neg R(x)) \\
            &\equiv \forall x\; (\neg T(x) \vee \neg R(x)) \\
            &\equiv \forall x\; \neg(T(x) \wedge R(x)) \\
            &\equiv \neg \exists x\; (T(x) \wedge R(x))
        \end{align*}
    \end{my_item}
    \end{solution}


\question Rosen Ch 1.4 \#6(b,c,f), p. 58
    \begin{solution}
    \begin{my_item}
        \item[(b)] $\exists x\; C(x, Math 695)$: There is a student enrolled in Math 695.
        \item[(c)] $\exists y\; C(Carol Sitea, y)$. Carol Sitea is enrolled in at least one course.
        \item[(f)] $\exists x\; \exists y\; \forall z\; ((x \neq y) \wedge (C(x,z) \leftrightarrow C(y,z)))$: There are two distinct students, $x$ and $y$,  where $x$ and $y$ are enrolled in the same courses.
    \end{my_item}
    \end{solution}

\question Rosen Ch 1.4, \#10(b,c,e), p. 59
    \begin{solution}
    \begin{my_item}
        \item[(b)] $\forall x\; F(Evelyn, x)$
        \item[(c)] $\forall x\; \exists y\; F(x,y)$
        \item[(e)] $\forall y\; \exists x\; F(x,y)$
%        \item[(d)] $\neg \exists x\; \forall y\; F(x,y)$
%        \item[(i)] $\neg \exists x\; F(x,x)$ \\
    \end{my_item}
    \end{solution}

\question Rosen Ch 1.4, \#25(a,b), p. 61
    \begin{solution}
    \begin{my_parts}
        \part $\exists x\; \forall y \; (x \cdot y = y)$ - ``There is a
        multiplicative identity for real numbers.  That is, there exists a
        number $x$ that can be multiplied with all numbers $y$ such that $x
        \cdot y = y$."
        \part $\forall x\; \forall y\; (((x < 0) \wedge (y < 0))
        \rightarrow (x \cdot y > 0))$ - ``The product of two negative real
        numbers is always a positive real number."
    \end{my_parts}
    \end{solution}


\question Rosen Ch 1.4, \#26(b,d,f), p. 61
    \begin{solution}
    \begin{my_item}
        \item[(b)] True
        \item[(d)] False, when $y=2$ rewrite the statement as $x + 2 = x - 2 \equiv x+4 = x$ where for no value of $x$ this statement holds.
        \item[(f)] True, choose $y=0$.
    \end{my_item}
    \end{solution}


\question Rosen Ch 1.4, \#28(a,c,e), p. 61
    \begin{solution}
    \begin{my_item}
        \item[(a)] True
        \item[(c)] True
        \item[(e)] True
    \end{my_item}
    \end{solution}


\question Rosen Ch 1.4, \#30(a,c,e), p. 61
    \begin{solution}

    \begin{tabular}{ll}
        \multicolumn{2}{l}{(a) $\neg \exists y \exists x\; P(x,y)$} \\
        \hspace{0.5in} & $\equiv \forall y\; \neg \exists x\; P(x,y)$ \\
          & $\equiv \forall y\; \forall x\; \neg P(x,y)$ \\
    \hline
        \multicolumn{2}{l}{(c) $\neg \exists y (Q(y) \wedge \forall x \neg R(x,y))$} \\
            & $\equiv \forall y \neg (Q(y) \wedge \forall x \neg R(x,y))$ \\
            & $\equiv \forall y \neg Q(y) \vee \neg \forall x \neg R(x,y)$ \\
            & $\equiv \forall y \neg Q(y) \vee \exists x \neg \neg R(x,y)$ \\
            & $\equiv \forall y \neg Q(y) \vee \exists x R(x,y)$ \\
    \hline
        \multicolumn{2}{l}{(e) $\neg \exists y (\forall x \exists z T(x,y,z) \vee \exists x \forall z U(x,y,z))$} \\
            & $\equiv \forall y \neg (\forall x \exists z T(x,y,z) \vee \exists x \forall z U(x,y,z))$ \\
            & $\equiv \forall y (\neg \forall x \exists z T(x,y,z) \wedge \neg \exists x \forall z U(x,y,z))$ \\
            & $\equiv \forall y (\exists x \neg \exists z T(x,y,z) \wedge \forall x \neg \forall z U(x,y,z))$ \\
            & $\equiv \forall y (\exists x \forall z \neg T(x,y,z) \wedge \forall x \exists z \neg U(x,y,z))$ \\
    \end{tabular}
    \end{solution}


\question Rosen Ch 1.5, \#10(e,f), p.73
    \begin{solution}
    \textbf{(e):} Let $h(x)$ be ``$x$ is healthy to eat", $t(x)$ be ``$x$ tastes good", $e(x)$ be ``You eat $x$".  Then the premises are:
    \begin{itemize}
        \item[1.] $\forall x\; (h(x) \rightarrow \neg t(x))$
        \item[2.] $h(tofu)$
        \item[3.] $\forall x\; (e(x) \rightarrow t(x))$
        \item[4.] $\neg e(tofu)$
        \item[5.] $\neg h(cheeseburger)$
    \end{itemize}

    \smallskip
    We can apply universal generalization and modus ponens on statement 1 and statement 2.  The conclusion is $\neg t(tofu)$ which means ``tofu does not taste good."

    From statement 3 and $\neg t(tofu)$, applying universal generalization and modus tollens, we draw the same conclusion $\neg e(tofu)$ as statement 4.

    No conclusion can be drawn about cheeseburgers.

    \medskip
    \textbf{(f):} Let $d$ be ``I am dreaming", $h(x)$ be ``I am hallucinating", $e$ be ``I see elephants running down
    the road."
    \begin{itemize}
        \item[1.] $d \vee h$
        \item[2.] $\neg d$
        \item[3.] $h \rightarrow e$
    \end{itemize}

    \smallskip
    From statement 1 and 2, using disjunctive syllogism we can
    conclude $h$. From this and statement 3, with modus ponens we
    can conclude $e$, ``I see elephants running down the road."
    \end{solution}


\question Rosen Ch 1.5, \#14(b,d), p.73
    \begin{solution}
    \textbf{(b):} Let $r(x)$ be ``$x$ is one of the five roommates", $d(x)$ be ``$x$ has taken a course in discrete math", and $a(x)$ be ``$x$ can take a course in algorithms." The premises are:
    \begin{itemize}
        \item[1.] $\forall x\;(r(x) \rightarrow d(x))$
        \item[2.] $\forall x\;(d(x) \rightarrow a(x))$
    \end{itemize}
    The conclusion is: $\forall x\;(r(x) \rightarrow a(x))$.  Let $y$ be an arbitrary person.

    \smallskip
    \begin{tabular}{lll}
        Step        & \hspace{0.2in} & Reason \\
        1. $\forall x\;(r(x) \rightarrow d(x))$     &   & Hypothesis \\
        2. $r(y) \rightarrow d(y)$                  & & Univ. Inst. with (1) \\
        3. $\forall x\; (d(x) \rightarrow a(x))$    &   & Hypothesis \\
        4. $d(u) \rightarrow a(y)$                  &   & Univ. Inst. with (3) \\
        5. $r(y) \rightarrow a(y)$                  &   & Hyp. syl. with (2) and (4) \\
        6. $\forall x\;(r(x) \rightarrow a(x))$     &   & Univ. Generalization with (5) \\
    \end{tabular}

    \medskip
    \textbf{(d):} Let $c(x)$ be ``$x$ is in this class", $f(x)$ be ``$x$ has been to France", $l(x)$ be ``$x$ has visited the Louvre." The premises are:
    \begin{itemize}
        \item[1.] $\exists x\;(c(x) \wedge f(x))$
        \item[2.] $\forall x\;(f(x) \rightarrow l(x))$
    \end{itemize}
    The conclusion is: $\exists x\; (c(x) \wedge l(x))$
    Let $y$ be an arbitrary person.

    \smallskip
    \begin{tabular}{lll}
        Step        & \hspace{0.2in} & Reason \\
        1. $\exists x\;(c(x) \wedge f(x))$      & & Hypothesis \\
        2. $c(y) \wedge f(y)$                   & & Exis. Inst. \\
        3. $f(y)$                               & & Simplification with (2) \\
        4. $c(y)$                               & & Simplification with (2) \\
        5. $\forall x\;(f(x) \rightarrow l(x))$ & & Hypothesis \\
        6. $f(y) \rightarrow l(y)$              & & Univ. Inst. \\
        7. $l(y)$                               & & Modus ponens using (3) and (6) \\
        8. $c(y) \wedge l(y)$                   & & Conjuntion with (7) and (4) \\
        9. $\exists x\; (c(x) \wedge l(x))$     & & Exis. Gen. \\
    \end{tabular}
    \end{solution}


\question Rosen Ch 1.5, \#16(b,d), p.73-4
    \begin{solution}
    \begin{my_item}
        \item[(b)] Incorrect, after applying universal instantiation it uses the fallacy of denying the hypothesis.
        \item[(d)] Correct, using universal instantiation and modus ponens.
    \end{my_item}
    \end{solution}

\end{questions}
\end{document}
