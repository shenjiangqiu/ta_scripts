\documentclass[12pt,addpoints]{exam}
% can include option [answers] to print out solutions, or command \printanswers
%  can turn addpoint on and off with commands, \addpoints and \noaddpoints

\usepackage{amsthm}
\usepackage{amsmath, amssymb}

\usepackage[normalmargins,normalsections,normalindent,normalleading]{savetrees}

\newenvironment{my_parts}{
\begin{parts}
    \setlength{\itemsep}{1pt}
    \setlength{\parskip}{0pt}
    \setlength{\parsep}{0pt}
}{\end{parts}}

\setlength{\itemsep}{0pt} \setlength{\topsep}{0pt}

\begin{document}
\extrawidth{0.5in} \extrafootheight{-0.75in} \pagestyle{headandfoot}
\headrule \header{\textbf{cs2311 - Spring 2011}}{\textbf{HW
1 - Solutions}}{\textbf{Due: Thu. 1/20/11}} \footrule \footer{}{Page \thepage\
of \numpages}{}

\noindent \textbf{Instructions:} All assignments are due at the
beginning of class on the due date specified.  Solutions will be
handed out (or posted on-line) shortly thereafter.  Every student
must write up their own solutions in their own manner.

\noindent Please follow the format in the book when asked to produce
truth tables (this will help in grading in order to avoid any
errors).

\noindent Please present your solutions in a clean, understandable
manner; pages should be stapled before class, no ragged edges of
paper.

%\noindent The assignment has \totalpoints\ points.

\begin{questions}
\printanswers

\question[8] Determine which of the statements are propositions? What are the truth values of those that are propositions?
    \begin{parts}
    \part Open the door.
    \part 6 + 6 = 14.
    \part 4 - x = 8.
    \part Who is the president?
    \end{parts}
    \begin{solution}
        \begin{parts}
        \part not a proposition, is a command
        \part proposition, $F$
        \part not a proposition, depends on value of $x$
        \part not a proposition, is a question
        \end{parts}
    \end{solution}

\question[10] Let $p$ and $q$ be the propositions
 \begin{itemize}
    \item[$p$:] I twisted my ankle.
    \item[$q$:] I went skiing this week.
 \end{itemize}
 Express each of these propositions as an English sentence.
    \begin{parts}
    \part $\neg q$
    \part $p \vee q$
    \part $p \rightarrow q$
    \part $\neg p \wedge \neg q$
    \part $p \leftrightarrow q$
    \end{parts}
    \begin{solution}
        \begin{parts}
        \part It is not the case that I went skiing this week.
        \part I twisted my ankle or I went skiing this week.
        \part If I twisted my ankle then I went skiing this week.
        \part I did not twist my ankle and I did not go skiing this
        week.
        \part I twisted my ankle if and only if I went skiing this
        week.
        \end{parts}
    \end{solution}

\question[6] Rosen Ch 1.1, \#10 (a,c,e), p. 17 
%Let $p$, $q$, and $r$ be the propositions
% \begin{itemize}
%    \item[$p$:] You get an A on the final exam.
%    \item[$q$:] You do every exercise in this book.
%    \item[$r$:] You get an A in this class.
% \end{itemize}
% Write these propositions using $p$, $q$, and $r$ and logical connectives.
%    \begin{parts}
%    \part You get an A in this class, but you do not do every exercise in this book.
%    \part To get an A in this class, it is necessary for you to get an A on the final.
%    \part Getting an A on the final and doing every exercise in this book is sufficient for getting an A in this class.
%    \end{parts}
    \begin{solution}
        \begin{parts}
        \part $r \wedge \neg q$
        \part $r \rightarrow p$
        \part $(p \wedge q) \rightarrow r$
        \end{parts}
    \end{solution}

\question[6] Rosen Ch 1.1, \#20 (a,c,f), p. 18 
%Write each statement in the form ``if $p$, then $q$" in English.
%    \begin{parts}
%    \part I will remember to send you the address only if you send me an e-mail message.
%    \part If you keep your textbook, it will be a useful reference in your future courses.
%    \part The beach erodes whenever there is a storm.
%    \end{parts}
    \begin{solution}
        \begin{parts}
        \part If I remember to send you the address, then you sent me an e-mail message.
        \part If you keep your textbook, then it will be a useful reference in your future courses.
        \part If the there is a storm, then the beach erodes.
        \end{parts}
    \end{solution}

\question[6] State the converse, contrapositive, and inverse of each of the conditional statements.
    \begin{parts}
    \part If it is snowing, Paul will shovel snow.  %If it is snowing, then Paul will shovel snow.
    \part When Ann is up late studying, it is necessary that she drinks coffee. % drinks coffee, p study late
    % q is neccessary for p
    \end{parts} % p->q  Converse: q->p, Contrapositive \neg q \ra \neg p,  Inverse \neg p \ra \neg q
    \begin{solution}
        \begin{parts}
        \part \begin{tabular}{lp{4in}}
            Converse &  If Paul will shovel snow, then it is snowing.\\
            Contrapositive & If Paul will not shovel snow, then it is not snowing.\\
            Inverse & If it is not snowing, then Paul will not shovel snow.\\
        \end{tabular}
        \part \begin{tabular}{lp{4in}}
            Converse & If Ann drinks coffee, then she is up late studying.\\
            Contrapositive & If Ann does not drink coffee, then she is not up late studying.\\
            Inverse & If Ann is not up late studying, then she does not drink coffee.\\
        \end{tabular}
        \end{parts}
    \end{solution}


\question[20] Give the truth tables for the following propositions. \\
 \begin{tabular}{lll}
    (a) $\neg p \leftrightarrow q$ & (b) $q \rightarrow \neg p$ & (c) $(p \rightarrow \neg q) \wedge \neg (q \leftrightarrow p)$ \\
    (d) $(p \oplus \neg q) \vee \neg (q \oplus p)$ & (e) $(p \wedge \neg q) \vee r$ & (f) $(p \rightarrow q) \rightarrow \neg r$ \\
    \multicolumn{3}{l}{(g) Which pairs of propositions (if any) are logically equivalent?} \\
    \multicolumn{3}{l}{(h) Which propositions (if any) are tautologies? contradictions?} \\
 \end{tabular}
 \begin{solution}
 
	\begin{tabular}{|cc|cc|c|c|ccc|c|}
	\hline
		& & & & (a) & (b) & & & & (c) $(p \rightarrow \neg q)$ \\
		$p$ & $q$ & $\neg p$ & $\neg q$ & $\neg p \leftrightarrow q$  & $q \rightarrow \neg p$
		    & $p \rightarrow \neg q$ & $q \leftrightarrow p$ & $\neg (q \leftrightarrow p)$ 
		    & $\wedge \neg (q \leftrightarrow p)$ \\
	\hline
		T & T & F & F & F & F & F & T & F & F \\
		T & F & F & T & T & T & T & F & T & T \\
		F & T & T & F & T & T & T & F & T & T \\   
		F & F & T & T & F & T & T & T & F & F \\
	\hline
	\end{tabular}
    
	\begin{tabular}{|cc|cccc|c|}
	\hline	
		& & & & & & (d) $p \oplus \neg q$  \\
		$p$ & $q$ & $\neg q$ & $p \oplus \neg q$ & $q \oplus p$ & $\neg(q \oplus p)$ & $\vee \neg (q \oplus p)$ \\
    \hline
    	T & T & F & T & F & T & T \\
    	T & F & T & F & T & F & F \\
    	F & T & F & F & T & F & F \\
    	F & F & T & T & F & T & T \\
    \hline
    \end{tabular}
    
    \begin{tabular}{|ccc|cc|cc|cc|}
    \hline
    	& & & & & & (e) & & (f) \\
    	$p$ & $q$ & $r$ & $\neg q$ & $\neg r$ & $p \wedge \neg q$ & $(p \wedge \neg q) \vee r$ 
    	  & $p \rightarrow q$ & $(p \rightarrow q) \rightarrow \neg r$ \\
    \hline
    	T & T & T & F & F & F & T & T & F \\
    	T & T & F & F & T & F & F & T & T \\
    	T & F & T & T & F & T & T & F & T \\
    	T & F & F & T & T & T & T & F & T \\
    	F & T & T & F & F & F & T & T & F \\
    	F & T & F & F & T & F & F & T & T \\	   
    	F & F & T & T & F & F & T & T & F \\
    	F & F & F & T & T & F & F & T & T \\
    	\multicolumn{9}{|c|}{} \\
     	\multicolumn{9}{|l|}{(g) a and c are logically equivalent.} \\
     	\multicolumn{9}{|l|}{(h) There are no tautologies or contradictions.} \\
    \hline
    \end{tabular}
   \end{solution}


\question[8] There are 16 possible truth tables for propositions of two variables $p$ and $q$.
 All sixteen possibilities are given in the table below (numbered 15, 14, \ldots, 0).
 For example, the proposition $p \vee q$ is 14. What are the numbers of the truth
 tables for propositions:\\
 \begin{tabular}{l}
    (a) $p \leftrightarrow q$ \\
    (b) $p \wedge (\neg p \vee q)$ \\
    (c) $\neg p \oplus q $ \\
    (d) $\neg q \rightarrow p$ \\
 \end{tabular}

 \begin{tabular}{cc|cccc|cccc|cccc|cccc}
    $p$ & $q$ & 15 & 14 & 13 & 12 & 11 & 10 & 9 & 8 & 7 & 6 & 5 & 4 & 3 & 2 & 1 & 0 \\
    \hline
    T & T & T & T & T & T & T & T & T & T & F & F & F & F & F & F & F & F \\
    T & F & T & T & T & T & F & F & F & F & T & T & T & T & F & F & F & F \\
    F & T & T & T & F & F & T & T & F & F & T & T & F & F & T & T & F & F \\
    F & F & T & F & T & F & T & F & T & F & T & F & T & F & T & F & T & F \\
 \end{tabular}
 \begin{solution}
    \begin{tabular}{llll}
        (a) 9 \hspace{0.5in} & (b) 8 \hspace{0.5in} & (c) 9 \hspace{0.5in} & (d) 14 \\
    \end{tabular}
 \end{solution}

\question[8] Show the proposition is a tautology using a truth table
and logical equivalences.
 $$ (p \vee q) \wedge (\neg p \vee r) \rightarrow (q \vee r) $$

	\begin{solution}
   \begin{tabular}{ccc|cccccc}
         &     &     &          &            &                 & $(p
         \vee q) \wedge$ & & $(p \vee q) \wedge (\neg p \vee r)$ \\
     $p$ & $q$ & $r$ & $\neg p$ & $p \vee q$ & $\neg p \vee r$ &
     $(\neg p \vee r)$ & $q \vee r$ & $\rightarrow (q \vee r)$ \\
     \hline
     T & T & T & F & T & T & T & T & T\\
     T & T & F & F & T & F & F & T & T\\
     T & F & T & F & T & T & T & T & T\\
     T & F & F & F & T & F & F & F & T\\
     F & T & T & T & T & T & T & T & T\\
     F & T & F & T & T & T & T & T & T\\
     F & F & T & T & F & T & F & T & T\\
     F & F & F & T & F & T & F & F & T\\
   \end{tabular}

   \begin{align*}
        ((p \vee q) & \wedge (\neg p \vee r)) \rightarrow (q \vee r)
        \notag \\
         & \equiv \neg ((p \vee q) \wedge (\neg p \vee r)) \vee (q
         \vee r) \tag{Table 7, rule 1} \\
         & \equiv \neg (p \vee q) \vee \neg (\neg p \vee r) \vee (q
         \vee r) \tag{De Morgan's} \\
         & \equiv (\neg p \wedge \neg q) \vee (\neg \neg p \wedge
         \neg r) \vee (q \vee r) \tag{De Morgan's x2} \\
         & \equiv (\neg p \wedge \neg q) \vee (p \wedge
         \neg r) \vee (q \vee r) \tag{Double negation} \\
         & \equiv q \vee (\neg p \wedge \neg q) \vee r \vee (p \wedge
         \neg r) \tag{Commutative} \\
         & \equiv (q \vee \neg p) \wedge (q \vee \neg q) \vee (r
         \vee p) \wedge (r \vee \neg r) \tag{Distributive, x2}\\
         & \equiv (q \vee \neg p) \wedge {\bf T} \vee (r \vee p)
         \wedge {\bf T} \tag{Negation, x2} \\
         & \equiv q \vee \neg p \vee r \vee p \tag{Identity, x2} \\
         & \equiv q \vee r \vee \neg p \vee p \tag{Commutative} \\
         & \equiv q \vee r \vee {\bf T} \tag{Negation} \\
         & \equiv q \vee {\bf T} \tag{Domination} \\
         & \equiv {\bf T} \tag{Domination}
   \end{align*}
   \end{solution}



\question[12] Prove the following logical equivalences using the
equivalences from Table 6-8 (model the solutions in the style of
book Examples 6-8, pp. 26-27).  Justify each step with laws.
    \begin{parts}
    \part $(q \wedge \neg p) \vee (\neg p \wedge \neg (p \vee q))
    \equiv \neg p$
    \part $\neg(\neg p \vee (p \wedge q)) \vee q \equiv q \vee p $
    \end{parts}
\newpage
	\begin{solution}
    \begin{my_parts}
    \part
    \begin{tabular}{|l|l|}
    \hline
    $(q \wedge \neg p) \vee (\neg p \wedge \neg (p \vee q))$ & given
    \\
    $\equiv (q \wedge \neg p) \vee (\neg p \wedge \neg p \wedge \neg
    q)$ & De Morgan's laws \\
    $\equiv (q \wedge \neg p) \vee (\neg p \wedge \neg q)$ &
    Idempotent laws \\
    $\equiv (\neg p \wedge q) \vee (\neg p \wedge \neg q)$ &
    Commutative laws \\
    $\equiv \neg p \wedge (q \vee \neg q)$ & Distributive laws \\
    $\equiv \neg p \wedge \mathbf{T}$ & Negation laws \\
    $\equiv \neg p$ & Identity laws \\
    \hline
    \end{tabular}
    \part
    \begin{tabular}{|l|l|}
    \hline
    $\neg(\neg p \vee (p \wedge q)) \vee q$ & given \\
    $\equiv (\neg\neg p \wedge \neg(p \wedge q)) \vee q$ & De
    Morgan's laws \\
    $\equiv (p \wedge \neg(p \wedge q)) \vee q$ & Double negative
    laws \\
    $\equiv (p \wedge (\neg p \vee \neg q)) \vee q$ & De Morgan's
    laws \\
    $\equiv (p \wedge \neg p) \vee (p \wedge \neg q) \vee q$ &
    Distributive laws \\
    $\equiv \mathbf{F} \vee (p \wedge \neg q) \vee q$ & Negation
    laws \\
    $\equiv q \vee \mathbf{F} \vee (p \wedge \neg q)$ & Commutative
    laws \\
    $\equiv q \vee (p \wedge \neg q)$ & Identity laws \\
    $\equiv (q \vee p) \wedge (q \vee \neg q)$ & Distributive laws
    \\
    $\equiv (q \vee p) \wedge \mathbf{T}$ & Negation laws \\
    $\equiv q \vee p $ & Identity laws \\
     \hline
     \end{tabular}
    \end{my_parts}
    \end{solution}

\question[4] Rosen Ch 1.2 \# 60(b), p. 30
%A compound proposition is \textbf{satisfiable}
% if there is an assignment to the variables that makes the statement
% true.  Which of the propositions are satisfiable?
%\begin{itemize}
%    \item[b] $(\neg p \vee \neg q \vee r) \wedge
%    (\neg p \vee q \vee \neg s) \wedge
%    (p \vee \neg q \vee \neg s) \wedge
%    (\neg p \vee \neg r \vee \neg s) \wedge
%    (p \vee q \vee \neg r) \wedge
%    (p \vee \neg r \vee \neg s)$
%\end{itemize}
	\begin{solution}
    (b) Satisfiable, let $p$ and $s$ be false, $q$ be true, and $r$ be
  any truth value.
    \end{solution}

\question[4] Rosen Ch 1.3 \# 8(a,b), p. 47 
	\begin{solution}
     \begin{tabular}{l}
        (a) Every rabbit hops. \\
        (b) Every animal is a rabbit and hops. \\
    \end{tabular}
    \end{solution}

\question[4] Rosen Ch 1.3 \# 10(a,b), p. 47 
	\begin{solution}
    \begin{tabular}{l}
    (a) $\exists x\; (C(x) \wedge D(x) \wedge F(x))$ \\
    (b) $\forall x\; (C(x) \vee D(x) \vee F(x))$ \\
    \end{tabular}
    \end{solution}

\question[4] Rosen Ch 1.3 \# 18(a,b), p. 47 
	\begin{solution}
    \begin{tabular}{l}
    (a) $P(-2) \vee P(-1) \vee P(0) \vee P(1) \vee P(0)$ \\
    (b) $P(-2) \wedge P(-1) \wedge P(0) \wedge P(1) \wedge P(0)$ \\
    \end{tabular}
    \end{solution}


\end{questions}
\end{document}
