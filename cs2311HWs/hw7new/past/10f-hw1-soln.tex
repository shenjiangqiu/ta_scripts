\documentclass[12pt]{exam}
% can include option [answers] to print out solutions, or command \printanswers
%  can turn addpoint on and off with commands, \addpoints and \noaddpoints

\usepackage{amsthm}
\usepackage{amsmath, amssymb}

\usepackage[normalmargins,normalsections,normalindent,normalleading]{savetrees}

% Tight lists
%\usepackage{mdwlist} - then use "starred" versions of itemize, enumerate, description
%  this still has spacing above and below.

\newenvironment{my_parts}{
\begin{parts}
    \setlength{\itemsep}{1pt}
    \setlength{\parskip}{0pt}
    \setlength{\parsep}{0pt}
}{\end{parts}}
% or can include lines
%   \begin{parts}
%     \itemsep 1pt
%     \parskip 0pt
%     \item One
%     ...
%   \end{parts}

\begin{document}
\extrawidth{0.5in}
\extrafootheight{-0.75in}
\pagestyle{headandfoot}
\headrule
\header{\textbf{cs2311 - Fall 2010}}{\textbf{HW 1}}{\textbf{Due: Fri. 9/10/10}}
\footrule
\footer{}{Page \thepage\ of \numpages}{}

\addpoints

\noindent \textbf{Instructions:} All assignments are due at the
beginning of class on the due date specified.  Solutions will be
handed out (or posted on-line) shortly thereafter.  Every student
must write up their own solutions in their own manner.

\noindent Please follow the format in the book when asked to produce
truth tables (this will help in grading in order to avoid any
errors).

%\noindent The assignment has \numpoints\ points.

\begin{questions}
\printanswers

\question[8] Determine which of the statements are propositions? What are the truth values of those that are propositions?
    \begin{my_parts}
    \part Close the door.
    \part Who is the president?
    \part 6 + 8 = 14.
    \part 4 - y = 9.
    \end{my_parts}
    \begin{solution}
        \begin{my_parts}
        \part not a proposition, is a command
        \part not a proposition, is a question
        \part proposition, $T$
        \part not a proposition, depends on value of $y$
        \end{my_parts}
    \end{solution}

\question[8] Let $p$ and $q$ be the propositions
 \begin{itemize}
    \item[$p$:] I played poker this week.
    \item[$q$:] I lost my money on Tuesday.
 \end{itemize}
 Express each of these propositions as an English sentence.
    \begin{my_parts}
    \part $\neg q$
    \part $p \vee q$
    \part $p \rightarrow q$
    \part $\neg p \wedge \neg q$
    \end{my_parts}
    \begin{solution}
        \begin{my_parts}
        \part It is not the case that I lost my money on Tuesday.
        \part I played poker this week or I lost my money on Tuesday.
        \part If I played poker this week then I lost my money on Tuesday.
        \part I did not play poker this week and I did not lose my money on Tuesday.
        \end{my_parts}
    \end{solution}

\question[6] Rosen Ch 1.1, \#10 (a,c,e), p. 17
%Let $p$, $q$, and $r$ be the propositions
% \begin{itemize}
%    \item[$p$:] You get an A on the final exam.
%    \item[$q$:] You do every exercise in this book.
%    \item[$r$:] You get an A in this class.
% \end{itemize}
% Write these propositions using $p$, $q$, and $r$ and logical connectives.
%    \begin{parts}
%    \part You get an A in this class, but you do not do every exercise in this book.
%    \part To get an A in this class, it is necessary for you to get an A on the final.
%    \part Getting an A on the final and doing every exercise in this book is sufficient for getting an A in this class.
%    \end{parts}
    \begin{solution}
        \begin{itemize}
        \item[(a)] $r \wedge \neg q$
        \item[(c)] $r \rightarrow p$
        \item[(e)] $(p \wedge q) \rightarrow r$
        \end{itemize}
    \end{solution}

\question[6] Rosen Ch 1.1, \#18 (a,c,f), p. 18
%Write each statement in the form ``if $p$, then $q$" in English.
%    \begin{parts}
%    \part It is necessary to wash the boss's car to get promoted.
%    \part A sufficient condition for the warranty to be good is that you bought the computer less than a year ago.
%    \part Getting elected follows from knowing the right people.
%    \end{parts}
    \begin{solution}
        \begin{itemize}
        \item[(a)] If you get promoted, then you washed the boss's car.
        \item[(c)] If you bought the computer less than a year ago, then the warranty is good.
        \item[(f)] If you know the right people, then you get elected.
        \end{itemize}
    \end{solution}

\question[6] State the converse, contrapositive, and inverse of each of the conditional statements.
    \begin{my_parts}
    \part If it is raining, Paul will stay inside.
    \part When Kim is up late working on a programming assignment, it is necessary that she drinks coffee.
    \end{my_parts}
    \begin{solution}
        \begin{my_parts}
        \part
            \begin{tabular}{ll}
             Converse & Paul will stay inside only if it is
             raining.\\
             Contrapositive & If Paul will not stay inside then it isn't
             raining.\\
             Inverse & If it is not raining, then Paul will not stay inside.
             \end{tabular}
        \part
            \begin{tabular}{lp{4in}}
            Converse & If Kim drinks coffee, then she is up late working on a programming
            assignment.\\
            Contrapositive & If Kim does not drink coffee, then she is not up late working on a programming
            assignment.\\
            Inverse & If Kim is not up late working on a programming assignment, then she does not drink
            coffee.\\
            \end{tabular}
        \end{my_parts}
    \end{solution}


\question[14] Give the truth tables for the following propositions. \\
 \begin{tabular}{lll}
    (a) $\neg p \leftrightarrow q$ & (b) $p \rightarrow \neg q$ & (c) $(q \rightarrow \neg p) \vee \neg (q \leftrightarrow p)$ \\
    (d) $(p \oplus \neg q) \vee \neg (q \oplus p)$ & (e) $(p \vee \neg q) \wedge r$ & (f) $(\neg p \rightarrow q) \rightarrow r$ \\
    \multicolumn{3}{l}{(g) Which pairs of propositions (if any) are logically equivalent?} \\
    \multicolumn{3}{l}{(h) Which propositions (if any) are tautologies? contradictions?} \\
 \end{tabular}
 \begin{solution}
        \begin{tabular}{|cc|c|c|c|c|}
         & & (a) & (b) & (c) & (d)  \\
        $p$ & $q$ & $\neg p \leftrightarrow q$  & $p \rightarrow \neg q$ & $(q \rightarrow \neg p) \vee \neg (q \leftrightarrow p)$ & $(p \oplus \neg q) \vee \neg (q \oplus p)$ \\
    \hline
        T & T & F & F & F & T \\
        T & F & T & T & T & F \\
        F & T & T & T & T & F \\
        F & F & F & T & T & T \\
    \end{tabular}

    \begin{tabular}{|ccc|c|c|}
      & & & (e) & (f) \\
     $p$ & $q$ & $r$ & $(p \vee \neg q) \wedge r$ & $(\neg p \rightarrow q) \rightarrow r$ \\
     \hline
     T & T & T & T & T \\
     T & T & F & F & F \\
     T & F & T & T & T \\
     T & F & F & F & F \\
     F & T & T & F & T \\
     F & T & F & F & F \\
     F & F & T & T & T \\
     F & F & F & F & T \\
     \multicolumn{5}{|c|}{} \\
     \multicolumn{5}{|l|}{(g) Proposition (b) and (c) are logically equivalent.} \\
     \multicolumn{5}{|l|}{(h) There are no tautologies or contradictions.} \\
    \end{tabular}
   \end{solution}


\question[8] There are 16 possible truth tables for propositions of two variables $p$ and $q$.  All sixteen possiblities are given in the table below (numbered 15, 14, \ldots, 0).  For example, the proposition $p \vee q$ is 14. What are the numbers of the truth tables for propositions:\\
 \begin{tabular}{l}
    (a) $p \leftrightarrow q$ \\
    (b) $p \vee \neg p \wedge q$ \\
    (c) $\neg p \vee q $ \\
    (d) $q \rightarrow \neg p$ \\
 \end{tabular}

 \begin{tabular}{cc|cccc|cccc|cccc|cccc}
    $p$ & $q$ & 15 & 14 & 13 & 12 & 11 & 10 & 9 & 8 & 7 & 6 & 5 & 4 & 3 & 2 & 1 & 0 \\
    \hline
    T & T & T & T & T & T & T & T & T & T & F & F & F & F & F & F & F & F \\
    T & F & T & T & T & T & F & F & F & F & T & T & T & T & F & F & F & F \\
    F & T & T & T & F & F & T & T & F & F & T & T & F & F & T & T & F & F \\
    F & F & T & F & T & F & T & F & T & F & T & F & T & F & T & F & T & F \\
 \end{tabular}
 \begin{solution}
    \begin{tabular}{ll}
        (a) & 9 \\
        (b) & 14 \\
        (c) & 11 \\
        (d) & 7 \\
    \end{tabular}
 \end{solution}


\question[10] Show the proposition is a tautology using a truth
table and logical equivalences.
 $$ (p \vee q) \wedge (\neg p \vee r) \rightarrow (q \vee r) $$
 \begin{solution}
   \begin{tabular}{ccc|cccccc}
         &     &     &          &            &                 & $(p
         \vee q) \wedge$ & & $(p \vee q) \wedge (\neg p \vee r)$ \\
     $p$ & $q$ & $r$ & $\neg p$ & $p \vee q$ & $\neg p \vee r$ &
     $(\neg p \vee r)$ & $q \vee r$ & $\rightarrow (q \vee r)$ \\
     \hline
     T & T & T & F & T & T & T & T & T\\
     T & T & F & F & T & F & F & T & T\\
     T & F & T & F & T & T & T & T & T\\
     T & F & F & F & T & F & F & F & T\\
     F & T & T & T & T & T & T & T & T\\
     F & T & F & T & T & T & T & T & T\\
     F & F & T & T & F & T & F & T & T\\
     F & F & F & T & F & T & F & F & T\\
   \end{tabular}

   \begin{align*}
        ((p \vee q) & \wedge (\neg p \vee r)) \rightarrow (q \vee r)
        \notag \\
         & \equiv \neg ((p \vee q) \wedge (\neg p \vee r)) \vee (q
         \vee r) \tag{Table 7, rule 1} \\
         & \equiv \neg (p \vee q) \vee \neg (\neg p \vee r) \vee (q
         \vee r) \tag{De Morgan's} \\
         & \equiv (\neg p \wedge \neg q) \vee (\neg \neg p \wedge
         \neg r) \vee (q \vee r) \tag{De Morgan's x2} \\
         & \equiv (\neg p \wedge \neg q) \vee (p \wedge
         \neg r) \vee (q \vee r) \tag{Double negation} \\
         & \equiv q \vee (\neg p \wedge \neg q) \vee r \vee (p \wedge
         \neg r) \tag{Commutative} \\
         & \equiv (q \vee \neg p) \wedge (q \vee \neg q) \vee (r
         \vee p) \wedge (r \vee \neg r) \tag{Distributive, x2}\\
         & \equiv (q \vee \neg p) \wedge {\bf T} \vee (r \vee p)
         \wedge {\bf T} \tag{Negation, x2} \\
         & \equiv q \vee \neg p \vee r \vee p \tag{Identity, x2} \\
         & \equiv q \vee r \vee \neg p \vee p \tag{Commutative} \\
         & \equiv q \vee r \vee {\bf T} \tag{Negation} \\
         & \equiv q \vee {\bf T} \tag{Domination} \\
         & \equiv {\bf T} \tag{Domination}
   \end{align*}


    %\begin{align}
%        (p \vee & q) \wedge (\neg p \vee r) \rightarrow (q \vee r) \nonumber\\
%         & \equiv ((p \vee q) \wedge \neg p) \vee ((p \vee q) \wedge
%         r) \rightarrow (q \vee r) \text{Distributive} \\
%         & \equiv (\neg p \wedge (p \vee q)) \vee (r \wedge (p \vee
%         q)) \rightarrow (q \vee r) \text{Commutative x2} \\
%         & \equiv ((\neg p \wedge p) \vee (\neg p \wedge q)) \vee ((r
%         \wedge p) \vee (r \wedge q)) \rightarrow (q \vee r)
%         \text{Distributive x2} \\
%         & \equiv ({\bf F} \vee (\neg p \wedge q)) \vee ((r \wedge
%         p) \vee (r \wedge q)) \rightarrow (q \vee r)
%         \text{Negation} \\
%         & \equiv (\neg p \wedge q) \vee ((r \wedge p) \vee (r
%         \wedge q)) \rightarrow (q \vee r) \text{Identity} \\
%         & \equiv \neg\left[ (\neg p \wedge q) \vee ((r \wedge p) \vee (r
%         \wedge q)) \right] \vee (q \vee r) \text{Table 7, rule 1}
%         \\
%         & \equiv \left[ \neg (\neg p \wedge q) \wedge \neg
%         ((r\wedge p) \vee (r \wedge q)) \right] \vee (q \vee r)
%         \text{De Morgans} \\
%         & \equiv \left[ (\neg \neg p \vee \neg q) \wedge (\neg(r
%         \wedge p) \wedge \neg (r \wedge q)) \right] \vee (q \vee r)
%         \text{De Morgans x2} \\
%         & \equiv \left[ (p \vee \neg q) \right] \\
%    \end{align}

 \end{solution}

\question[14] Prove the following logical equivalences using the
equivalences from Table 6-8 (model the solutions in the style of
book Examples 6-8, pp. 26-27).  Justify each step with laws.
    \begin{my_parts}
    \part $(q \wedge \neg p) \vee (\neg p \wedge \neg (p \vee q))
    \equiv \neg p$
    \part $\neg(\neg p \vee (p \wedge q)) \vee q \equiv q \vee p $
    \end{my_parts}
    \begin{solution}
    \begin{my_parts}
    \part
    \begin{tabular}{|l|l|}
    \hline
    $(q \wedge \neg p) \vee (\neg p \wedge \neg (p \vee q))$ & given
    \\
    $\equiv (q \wedge \neg p) \vee (\neg p \wedge \neg p \wedge \neg
    q)$ & De Morgan's laws \\
    $\equiv (q \wedge \neg p) \vee (\neg p \wedge \neg q)$ &
    Idempotent laws \\
    $\equiv (\neg p \wedge q) \vee (\neg p \wedge \neg q)$ &
    Commutative laws \\
    $\equiv \neg p \wedge (q \vee \neg q)$ & Distributive laws \\
    $\equiv \neg p \wedge \mathbf{T}$ & Negation laws \\
    $\equiv \neg p$ & Identity laws \\
    \hline
    \end{tabular}
    \part
    \begin{tabular}{|l|l|}
    \hline
    $\neg(\neg p \vee (p \wedge q)) \vee q$ & given \\
    $\equiv (\neg\neg p \wedge \neg(p \wedge q)) \vee q$ & De
    Morgan's laws \\
    $\equiv (p \wedge \neg(p \wedge q)) \vee q$ & Double negative
    laws \\
    $\equiv (p \wedge (\neg p \vee \neg q)) \vee q$ & De Morgan's
    laws \\
    $\equiv (p \wedge \neg p) \vee (p \wedge \neg q) \vee q$ &
    Distributive laws \\
    $\equiv \mathbf{F} \vee (p \wedge \neg q) \vee q$ & Negation
    laws \\
    $\equiv q \vee \mathbf{F} \vee (p \wedge \neg q)$ & Commutative
    laws \\
    $\equiv q \vee (p \wedge \neg q)$ & Identity laws \\
    $\equiv (q \vee p) \wedge (q \vee \neg q)$ & Distributive laws
    \\
    $\equiv (q \vee p) \wedge \mathbf{T}$ & Negation laws \\
    $\equiv q \vee p $ & Identity laws \\
     \hline
     \end{tabular}
    \end{my_parts}
    \end{solution}


\question[4] Rosen Ch 1.2 \# 60(b), p. 30,
%A compound proposition is \textbf{satisfiable}
% if there is an assignment to the variables that makes the statement
% true.  Which of the propositions are satisfiable?
%\begin{itemize}
%    \item[b] $(\neg p \vee \neg q \vee r) \wedge
%    (\neg p \vee q \vee \neg s) \wedge
%    (p \vee \neg q \vee \neg s) \wedge
%    (\neg p \vee \neg r \vee \neg s) \wedge
%    (p \vee q \vee \neg r) \wedge
%    (p \vee \neg r \vee \neg s)$
%\end{itemize}
    \begin{solution}
    (b) Satisfiable, let $p$ and $s$ be false, $q$ be true, and $r$ be
  any truth value.
    \end{solution}

\question[4] Rosen Ch 1.3 \# 8(a,c), p. 47
    \begin{solution}
     \begin{tabular}{l}
        (a) Every rabbit hops. \\
        (c) There exists an animal such that if it is a rabbit, then it hops. \\
    \end{tabular}
    \end{solution}

\question[6] Rosen Ch 1.3 \# 10(a,c,e), p. 47
    \begin{solution}
    \begin{tabular}{l}
    (a) $\exists x\; (C(x) \wedge D(x) \wedge F(x))$ \\
    (c) $\exists x\; (C(x) \wedge F(x) \wedge \neg D(x))$ \\
    (e) $(\exists x\; C(x)) \wedge (\exists x\; D(x)) \wedge (\exists x\; F(x))$ or it can be written as \\
    \hspace{0.5in} $(\exists x\; C(x)) \wedge (\exists y\; D(y)) \wedge (\exists z\; F(z))$ \\
    \end{tabular}
    \end{solution}

\question[6] Rosen Ch 1.3 \# 18(a,c,e), p. 47
    \begin{solution}
    \begin{tabular}{l}
    (a) $P(-2) \vee P(-1) \vee P(0) \vee P(1) \vee P(0)$ \\
    (c) $\neg P(-2) \vee \neg P(-1) \vee \neg P(0) \vee \neg P(1) \vee \neg P(0)$ \\
    (e) $\neg ( P(-2) \vee P(-1) \vee P(0) \vee P(1) \vee P(0) )$ \\
    \end{tabular}
    \end{solution}

\end{questions}
\end{document}
