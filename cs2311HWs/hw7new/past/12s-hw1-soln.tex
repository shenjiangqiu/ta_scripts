\documentclass[12pt,addpoints]{exam}
% can include option [answers] to print out solutions, or command \printanswers
%  can turn addpoint on and off with commands, \addpoints and \noaddpoints

\usepackage{amsthm}
\usepackage{amssymb}
\usepackage{amsmath}
\usepackage{color}
\usepackage{enumitem}
\usepackage[top=0.75in,bottom=0.75in,left=0.9in,right=0.9in]{geometry}

\setlength{\itemsep}{0pt} \setlength{\topsep}{0pt} 
\newcommand{\ra}{\rightarrow}
\newcommand{\lra}{\leftrightarrow}
\newcommand{\xor}{\oplus}

\begin{document}
\extrawidth{0.5in} \extrafootheight{-0in} \pagestyle{headandfoot}
\headrule \header{\textbf{cs2311 - Spring 2012}}{\textbf{HW
1 - \numpoints$\;$ points}}{\textbf{Due: Wed. 1/18/12}} \footrule \footer{}{Page \thepage\
of \numpages}{}

\noindent \textbf{Instructions:} All assignments are due \underline{by 5pm on the due date} specified.  There will be a box in the CS department office where assignments may be turned in.  Solutions will be handed out (or posted on-line) shortly thereafter.  Every student
must write up their own solutions in their own manner.

\smallskip
\noindent Please follow the format in the book when asked to produce
truth tables (this will help in grading in order to avoid any
errors).

\smallskip
\noindent Please present your solutions in a clean, understandable
manner; pages should be stapled before class, no ragged edges of
paper.

%\noindent The assignment has \totalpoints\ points.

\begin{questions}
\printanswers

\question[4] Determine which of the statements are propositions? What are the truth values of those that are propositions?
    \begin{enumerate}[label=(\alph*),itemsep=0pt,parsep=0pt,topsep=0pt,partopsep=0pt]
        \item 10 is a prime number.
        \item Who won the World Series?
        \item 4 + 4 = 8.
        \item $x - 5 > 2$.
    \end{enumerate}
    \ifprintanswers
        \vspace{-15pt}
    \fi
    \begin{solution}
        \begin{enumerate}[label=(\alph*),itemsep=0pt,parsep=0pt,topsep=0pt,partopsep=0pt]
        \item proposition, $F$
        \item not a proposition, is a question
        \item proposition, $T$
        \item proposition, depends on $x$
        \end{enumerate}
    \end{solution}

\question[4] For each of the sentences, determine whether an inclusive or, or an exclusive or, is intended.
    \begin{enumerate}[label=(\alph*),itemsep=0pt,parsep=0pt,topsep=0pt,partopsep=0pt]
        \item If you fail to make a payment on time or fail to pay the full amount due, you will incur a penalty.
        \item I will pass or fail the course.
    \end{enumerate}
    \ifprintanswers
        \vspace{-15pt}
    \fi
    \begin{solution}
        \begin{enumerate}[label=(\alph*),itemsep=0pt,parsep=0pt,topsep=0pt,partopsep=0pt]
        \item This is an inclusive or, you can fail to make a payment on time and the late payment may not be for the full amount.
        \item This is an exclusive or, both can not be true together.
        \end{enumerate}
    \end{solution}

\question[10] Let $p$, $q$ and $r$ be the propositions:
    \[ p : \text{ I take a vacation}, \quad q : \text{ It is summer}, \quad r : \text{ I work} \]
%    \begin{itemize}[itemsep=0pt,parsep=0pt,topsep=0pt,partopsep=0pt]
%        \item[$p$ :] I take a vacation
%        \item[$q$ :] It is summer
%        \item[$r$ :] I work
%    \end{itemize}
    Express each of these propositions as an English sentence.
    \begin{enumerate}[label=(\alph*),itemsep=0pt,parsep=0pt,topsep=0pt,partopsep=0pt]
        \item $\neg p$
        \item $p \vee q$
        \item $q \ra p$
        \item $q \ra \neg r$
        \item $p \lra q$
    \end{enumerate}
    \ifprintanswers
        \vspace{-15pt}
    \fi
    \begin{solution}
    \begin{enumerate}[label=(\alph*),itemsep=0pt,parsep=0pt,topsep=0pt,partopsep=0pt]
        \item $\neg p$ - It is not the case I take a vacation. \\
        OR:  I do not take a vacation.
        \item $p \vee q$ - I take a vacation or (inclusive) it is summer.
        \item $q \ra p$  - If it is summer, then I take a vacation. \\
        OR: I take a vacation in the summer.
        \item $q \ra \neg r$ - If it is summer, then I do not work. \\
        OR: I do not work in the summer.
        \item $p \lra q$ - I take a vacation if and only if it is summer.
    \end{enumerate}
    \end{solution}


\question[6] Rosen Ch 1.1 \#14 (b,e,f), p. 13-14
    \ifprintanswers
        \vspace{-10pt}
    \fi
    \begin{solution}
        \begin{itemize}[itemsep=0pt,parsep=0pt,topsep=0pt,partopsep=0pt]
            \item[(b)] $p \wedge q \wedge r$
            \item[(e)] $(p \wedge q) \ra r$
            \item[(f)] $r \lra (q \vee p)$
        \end{itemize}
    \end{solution}


\question[8] Rosen Ch 1.1, \#24 (b,d,g,h), p. 15
    \ifprintanswers
        \vspace{-10pt}
    \fi
    \begin{solution}
        \begin{itemize}[itemsep=0pt,parsep=0pt,topsep=0pt,partopsep=0pt]
            \item[(b)] If you were born in the United States, then you are a citizen of this country.
            \item[(d)] If their goalie plays well, then the Red Wings will win the Stanley Cup.
            \item[(g)] If there is a storm, then the beach erodes.
            \item[(h)] If you do not begin your climb too late, then you will reach the summit.
        \end{itemize}
    \end{solution}

\question[4] Rosen Ch 1.1 \#26 (a,c), p. 15
    \ifprintanswers
        \vspace{-10pt}
    \fi
    \begin{solution}
        \begin{itemize}[itemsep=0pt,parsep=0pt,topsep=0pt,partopsep=0pt]
            \item[(a)] You will get an A in this course if and only if you learn how to solve discrete mathematics problems.
            \item[(c)] It rains if and only if it is a weekend day.
        \end{itemize}
    \end{solution}

\question[6] State the converse, contrapositive, and inverse of each of the conditional statements.
    \begin{enumerate}[label=(\alph*),itemsep=0pt,parsep=0pt,topsep=0pt,partopsep=0pt]
        \item If it is snowing, then I will go skiing.
        \item You sleep late if it is Saturday.
    %  q is necessary for p
    % Converse: q \ra p, Contrapositive: \neg q \ra \neg p, Inverse: \neg p \ra \neg q
    \end{enumerate}
    \ifprintanswers
        \vspace{-10pt}
    \fi
    \begin{solution}
    \begin{enumerate}[label=(\alph*),itemsep=0pt,parsep=0pt,topsep=0pt,partopsep=0pt]
        \item
        \begin{tabular}{ll}
            Converse &  If I go skiing, then it is snowing. \\
            Contrapositive & If I do not go skiing, then it is not snowing. \\
            Inverse &  If it is not snowing, then I will not go snowing.
        \end{tabular}
        \item
        \begin{tabular}{ll}
            Converse &  If you sleep late, then it is a Saturday. \\
            Contrapositive & If you do not sleep late, then it is not Saturday. \\
            Inverse &  If it is not Saturday, then you do not sleep late.
        \end{tabular}
    \end{enumerate}
    \end{solution}

\question[2] Rosen Ch 1.1 \#30 (b,c), p. 15
% How many rows appear in a truth table for each of these compound propositions?
%\begin{parts}
%    \part $ (q \ra \neg q) \vee (\neg p \ra \neg q)$
%    \part $ (p \vee \neg t) \wedge (p \vee \neg s) $
%    \part $ (p \ra r) \vee (\neg s \ra \neg t) \vee (\neg u \ra v) $
%    \part $ (p \wedge r \wedge s) \vee (q \wedge t) \vee (r \wedge \neg t) $
%\end{parts}
    \ifprintanswers
        \vspace{-10pt}
    \fi
    \begin{solution} (b) 8;  (c) 64
%    \begin{itemize}[itemsep=0pt,parsep=0pt,topsep=0pt,partopsep=0pt]
%%        \item[(a)] 4
%        \item[(b)] 8
%        \item[(c)] 64
%%        \item[(d)] 32
%    \end{itemize}
    \end{solution}


% 1.1 \# 32 (a,c,d,e)
% 1.1 \# 36 (c,e)
\question[24]\label{tt} Construct a truth table for each compound proposition.
    \begin{center}
    \begin{tabular}{ll}
       (a) $p \lra \neg p$  & (b) $p \xor \neg p$ \\
       (c) $(p \vee q) \ra (p \wedge q)$ & (d) $ (p \lra q) \xor (p \vee \neg q)$ \\
       (e) $(p \vee q) \wedge r$  & (f) $(p \ra q) \vee (\neg q \lra r)$
    \end{tabular}
    \end{center}
    \ifprintanswers
        \vspace{-10pt}
    \fi
    \begin{solution}

        \begin{tabular}{c|c||c}
            \multicolumn{3}{l}{ (a) } \\
            $p$ & $\neg p$ & $p \lra \neg p$ \\
         \hline
            T & F & F \\
            F & T & F \\
         \end{tabular} \hspace{1in}
         \begin{tabular}{c|c||c}
            \multicolumn{3}{l}{ (b) } \\
            $p$ & $\neg p$ & $p \xor \neg p$ \\
         \hline
            T & F & T \\
            F & T & T \\
         \end{tabular}

        \smallskip
        \begin{tabular}{c|c||c|c|c}
            \multicolumn{4}{l}{ (c) } & $(p \vee q) \ra $ \\
            $p$ & $q$ & $p \vee q$ & $p \wedge q$ & $(p \wedge q)$ \\
         \hline
            T & T & T & T & T \\
            T & F & T & F & F \\
            F & T & T & F & F \\
            F & F & F & F & T \\
        \end{tabular} \hspace{0.5in}
        \begin{tabular}{c|c||c|c|c|c}
            \multicolumn{5}{l}{(d)} & $(p \lra q) \xor$ \\
            $p$ & $q$ & $p \lra q$ & $\neg q$ & $p \vee \neg q$ & $ (p \vee \neg q)$ \\
         \hline
            T & T & T & F & T & F \\
            T & F & F & T & T & T \\
            F & T & F & F & F & F \\
            F & F & T & T & T & F \\
        \end{tabular}

        \smallskip
        \begin{tabular}{c|c|c||c|c}
            \multicolumn{4}{l}{(e)} & \\
            $p$ & $q$ & $r$ & $p \vee q$ & $(p \vee q) \wedge r$ \\
         \hline
            T & T & T & T & T \\
            T & T & F & T & F \\
            T & F & T & T & T \\
            T & F & F & T & F \\
          \hline
            F & T & T & T & T \\
            F & T & F & T & F \\
            F & F & T & F & F \\
            F & F & F & F & F
         \end{tabular} \hspace{0.4in}
         \begin{tabular}{c|c|c||c|c|c|c}
            \multicolumn{6}{l}{(f)} & $(p \ra q) \vee$ \\
            $p$ & $q$ & $r$ & $\neg q$ & $\neg q \lra r$ & $p \ra q$ & $ (\neg q \lra r)$ \\
         \hline
            T & T & T & F & F & T & T\\
            T & T & F & F & T & T & T\\
            T & F & T & T & T & F & T\\
            T & F & F & T & F & F & F\\
          \hline
            F & T & T & F & F & T & T\\
            F & T & F & F & T & T & T\\
            F & F & T & T & T & T & T\\
            F & F & F & T & F & T & T
         \end{tabular}
    \end{solution}

\question[3] List those expressions from problem \ref{tt} that are tautologies, contradictions, and contingencies?
    \ifprintanswers
        \vspace{-10pt}
    \fi
    \begin{solution}
       Tautology: (b);
       Contradiction: (a);
       Contingency: (c), (d), (e), (f)
    \end{solution}

\question[4] There are 16 possible truth tables for propositions of two variables $p$ and $q$.
 All sixteen possibilities are given in the table below (numbered 15, 14, \ldots, 0).
 For example, the proposition $p \vee q$ is 14. What are the numbers of the truth
 tables for propositions:
\begin{center}
 \begin{tabular}{ll}
    (a) $(p \vee \neg q) \wedge (\neg p \ra q)$ \quad &  (b) $\neg p \vee (\neg q \wedge p)$ \\
    (c) $\neg p \oplus q $ \quad \quad& (d) $(\neg p \ra q) \wedge (\neg q \wedge p)$ \\
 \end{tabular}

 \begin{tabular}{cc|cccc|cccc|cccc|cccc}
    $p$ & $q$ & 15 & 14 & 13 & 12 & 11 & 10 & 9 & 8 & 7 & 6 & 5 & 4 & 3 & 2 & 1 & 0 \\
    \hline
    T & T & T & T & T & T & T & T & T & T & F & F & F & F & F & F & F & F \\
    T & F & T & T & T & T & F & F & F & F & T & T & T & T & F & F & F & F \\
    F & T & T & T & F & F & T & T & F & F & T & T & F & F & T & T & F & F \\
    F & F & T & F & T & F & T & F & T & F & T & F & T & F & T & F & T & F \\
 \end{tabular}
 \end{center}
     \ifprintanswers
        \vspace{-10pt}
    \fi
    \begin{solution} (a) 12, (b) 7, (c) 9, (d) 4
%    \begin{enumerate}[label=(\alph*),itemsep=0pt,parsep=0pt,topsep=0pt,partopsep=0pt]
%        \item  9
%        \item 12
%        \item 9
%        \item 7
%    \end{enumerate}
    \end{solution}


\question[9] Prove the following statement is a tautology without using truth tables (use the logical equivalences from Table 6-8 of the book).  Justify each step with the law used.  Model the solutions in the style of Examples 6-8, pp. 29-30 of the book.
\[ [ \neg p \wedge (p \vee q) ] \ra q \]
    \ifprintanswers
        \vspace{-30pt}
    \fi
\begin{solution}  There are many possible solutions; two are given.
\small
    \begin{align*}
        [ \neg p & \wedge (p \vee q) ] \ra q \\
        & \equiv \neg [ \neg p \wedge (p \vee q) ] \vee q \tag{Table 7, rule 1} \\
        & \equiv \neg \neg p \vee \neg (p \vee q) \vee q \tag{DeMorgan's law} \\
        & \equiv p \vee \neg (p \vee q) \vee q \tag{Double Negation} \\
        & \equiv p \vee q \vee \neg (p \vee q) \tag{Commutative} \\
        & \equiv (p \vee q) \vee \neg (p \vee q) \\
        & \equiv \mathbf{T} \tag{Negation} 
    \end{align*}

    \begin{align*}
        [ \neg p & \wedge (p \vee q) ] \ra q \\
        & \equiv \neg [ \neg p \wedge (p \vee q) ] \vee q \tag{Table 7, rule 1} \\
        & \equiv \neg \neg p \vee \neg (p \vee q) \vee q \tag{DeMorgan's law} \\
        & \equiv p \vee \neg (p \vee q) \vee q \tag{Double Negation} \\
        & \equiv p \vee q \vee \neg (p \vee q) \tag{Commutative} \\
        & \equiv p \vee q \vee (\neg p \wedge \neg q) \tag{DeMorgan's} \\
        & \equiv p \vee (q \vee \neg p) \wedge (q \vee \neg q) \tag{Distributive} \\
        & \equiv p \vee (q \vee \neg p) \wedge \mathbf{T} \tag{Negation} \\
        & \equiv p \vee (q \vee \neg p) \tag{Identity} \\
        & \equiv p \vee (\neg p \vee q) \tag{Commutative} \\
        & \equiv (p \vee \neg p) \vee q \tag{Associative} \\
        & \equiv \mathbf{T} \vee q \tag{Negation} \\
        & \equiv \mathbf{T} \tag{Domination}
    \end{align*}
\end{solution}

\question[16] Prove the following logical equivalences using the
equivalences from Table 6-8 (model the solutions in the style of
book Examples 6-8, pp. 29-30).  Justify each step with laws.
\begin{enumerate}[label=(\alph*),itemsep=0pt,parsep=0pt,topsep=0pt,partopsep=0pt]
    \item $p \ra (q \vee r) \equiv (p \wedge \neg q) \ra r$
    \item $\neg [ r \vee (q \wedge (\neg r \ra \neg p))] \equiv \neg r \wedge (p \vee \neg q) $
\end{enumerate}
    \begin{solution}
    
    (a)
    \begin{align*}
        p \ra (q \vee r) &\equiv (p \wedge \neg q) \ra r \\
            & \equiv \neg p \vee (q \vee r) \tag{Table 7, rule 1} \\
            & \equiv (\neg p \vee q) \vee r) \tag{Associative} \\
            & \equiv \neg (\neg p \vee q) \ra r \tag{Table 7, rule 1} \\
            & \equiv (\neg \neg p \vee \neg q) \ra r \tag{DeMorgan's} \\
            & \equiv (p \vee \neg q) \ra r \tag{Double Negation} 
    \end{align*}
    
    (b)
    \begin{align*} 
      \neg [ r \vee (q & \wedge (\neg r \ra \neg p))] \equiv \neg r \wedge (p \vee \neg q) \\
        & \equiv \neg [r \vee (q \wedge (p \ra r))] \tag{Table 7, rule 2} \\
        & \equiv \neg [r \vee (q \wedge (\neg p \vee r))] \tag{Table 7, rule 1} \\
        & \equiv \neg [r \vee (q \wedge \neg p) \vee (q \wedge r)] \tag{Distributive} \\
        & \equiv \neg [r \vee (q \wedge r) \vee (q \wedge \neg p)] \tag{Commutative} \\
        & \equiv \neg [r \vee (q \wedge \neg p)] \tag{Absorption} \\
        & \equiv \neg [r \vee (\neg p \wedge q)] \tag{Commutative} \\
        & \equiv \neg r \wedge \neg (\neg p \wedge q) \tag{DeMorgan's} \\
        & \equiv \neg r \wedge (\neg \neg p \vee \neg q) \tag{DeMorgan's} \\
        & \equiv \neg r \wedge (p \vee \neg q) \tag{Double Negation} 
    \end{align*}
    \end{solution}

\bonusquestion[2] Give a compound proposition with three variables $p$, $q$, and $r$ that is 
true when at most one of the three variables is true, and false otherwise.
    \begin{solution}
    \[ (p \wedge \neg q \wedge \neg r) \vee (\neg p \wedge q \neg r) 
        \vee (\neg p \wedge \neg q \wedge r) \vee (\neg p \wedge \neg q \neg r)\]
    \end{solution}
\end{questions}
\end{document}
