\documentclass[12pt,addpoints]{exam}
% can include option [answers] to print out solutions, or command \printanswers
%  can turn addpoint on and off with commands, \addpoints and \noaddpoints

\usepackage{amsthm}
\usepackage{amssymb}
\usepackage{amsmath}
\usepackage{color}
\usepackage{enumitem}
\usepackage[top=0.75in,bottom=0.75in,left=0.9in,right=0.9in]{geometry}

\setlength{\itemsep}{0pt} \setlength{\topsep}{0pt}
\newcommand{\ra}{\rightarrow}
\newcommand{\lra}{\leftrightarrow}
\newcommand{\xor}{\oplus}

\renewcommand{\solutiontitle}{\noindent\textbf{Soln:}\enspace}

\begin{document}
\extrawidth{0.5in} \extrafootheight{-0in} \pagestyle{headandfoot}
\headrule \header{\textbf{cs2311 - Fall 2012}}{\textbf{HW
3 - Practice Problems}}{\textbf{Solutions}} \footrule \footer{}{Page \thepage\
of \numpages}{}

\noindent \textbf{Instructions:} The following practice problems are
not due and are not graded.  The solutions will be provided to allow
for extra practice.

\begin{questions}
\printanswers


\question Rosen Ch 1.6 \#14 (a,b), p. 79
    \ifprintanswers
        \vspace{-15pt}
    \fi
\begin{solution}
    \begin{itemize}[itemsep=0pt,parsep=0pt,topsep=0pt,partopsep=0pt]
    \item[(a):] Let $c(x)$ be ``$x$ is in this class", $r(x)$ be ``$x$ owns a red convertible", $t(x)$ be ``$x$ has gotten a speeding ticket."  The premises are:
    \begin{itemize}[itemsep=0pt,parsep=0pt,topsep=0pt,partopsep=0pt]
        \item[1.] $c(Linda)$
        \item[2.] $r(Linda)$
        \item[3.] $\forall x\; (r(x) \rightarrow t(x))$
    \end{itemize}
    To conclude: $\exists x\; (c(x) \wedge t(x))$

    \smallskip
    \begin{tabular}{lll}
        Step        & \hspace{0.2in} & Reason \\
        1. $\forall x\; (r(x) \rightarrow t(x))$    &   & Hypothesis \\
        2. $r(Linda) \rightarrow t(Linda)$          &   & Universal Instantiation \\
        3. $r(Linda)$                               &   & Hypothesis \\
        4. $t(Linda)$                               &   & Modus ponens using (2) and (3) \\
        5. $c(Linda)$                               &   & Hypothesis \\
        6. $c(Linda) \wedge t(Linda)$               &   & Conjunction with (4) and (5) \\
        7. $\exists x\; (c(x) \wedge t(x))$         &   & Existential generalizaion \\
    \end{tabular}

    \item[(b):] Let $r(x)$ be ``$x$ is one of the five roommates", $d(x)$ be ``$x$ has taken a course in discrete math", and $a(x)$ be ``$x$ can take a course in algorithms." The premises are:
    \begin{itemize}[itemsep=0pt,parsep=0pt,topsep=0pt,partopsep=0pt]
        \item[1.] $\forall x\;(r(x) \rightarrow d(x))$
        \item[2.] $\forall x\;(d(x) \rightarrow a(x))$
    \end{itemize}
    The conclusion is: $\forall x\;(r(x) \rightarrow a(x))$.  Let $y$ be an arbitrary person.

    \smallskip
    \begin{tabular}{lll}
        Step        & \hspace{0.2in} & Reason \\
        1. $\forall x\;(r(x) \rightarrow d(x))$     &   & Hypothesis \\
        2. $r(y) \rightarrow d(y)$                  & & Univ. Inst. with (1) \\
        3. $\forall x\; (d(x) \rightarrow a(x))$    &   & Hypothesis \\
        4. $d(u) \rightarrow a(y)$                  &   & Univ. Inst. with (3) \\
        5. $r(y) \rightarrow a(y)$                  &   & Hyp. syl. with (2) and (4) \\
        6. $\forall x\;(r(x) \rightarrow a(x))$     &   & Univ. Generalization with (5) \\
    \end{tabular}
    \end{itemize}
\end{solution}


\question Rosen Ch 1.6 \# 16(a,b,c), p. 79
    \ifprintanswers
        \vspace{-15pt}
    \fi
\begin{solution}
    \begin{itemize}[itemsep=0pt,parsep=0pt,topsep=0pt,partopsep=0pt]
        \item[(a)] Correct, using universal instantiation and modus tollens.
        \item[(b)] Incorrect, first apply universal instantiation then fallacy of denying the hypothesis.
        \item[(c)] Incorrect, first apply universal instantiation then fallacy of denying the hypothesis.
%        \item[(d)] Correct, using universal instantiation and modus ponens.
    \end{itemize}
\end{solution}


\question Rosen Ch 1.6 \# 18, p. 79
    \ifprintanswers
        \vspace{-15pt}
    \fi
\begin{solution}
    It is true that there is some $s$ in the domain s.t. $S(s,Max)$; however, there is no guarantee that Max is the $s$.  This first step of the proof is invalid.
\end{solution}


\question Rosen Ch 1.6 \# 24, p. 79
    \ifprintanswers
        \vspace{-15pt}
    \fi
\begin{solution}
    The incorrect steps are 3 and 5.  The simplification rule can not be applied to disjunctions.  Also, step 7 the conjunction rule should ``AND" terms together.
\end{solution}


\question Rosen Ch 1.6 \# 26, p. 80
    \ifprintanswers
        \vspace{-15pt}
    \fi
\begin{solution}
    This is to show that the conditional statement is true for all elements in the domain, that is $\forall x\; P(x) \ra R(x)$.  Show if $P(x)$ is true for a particular $x$ then $R(x)$ is also true.  For an item in the domain $x$, universal modus ponens with the first hypothesis gives $Q(x)$.  Then, apply universal modus ponens with the second hypothesis to get $R(x)$.
\end{solution}


\question Consider the statements of Rosen Example 27, Ch 1.4, p. 51.  Show from the premises a valid argument that leads to the conclusion.
    \ifprintanswers
        \vspace{-10pt}
    \fi
\begin{solution}
    For the problem, the domain description and translation into logical expression has already been performed.

    \begin{tabular}{lll}
            Step                    & \hspace{0.15in} & Reason \\
            1. $\forall x\; (P(x) \ra S(x))$            & & Given \\
            2. $\neg \exists x\; (Q(x) \wedge R(x))$    & & Given \\
            3. $\forall x\; (\neg R(x) \ra \neg S(x))$  & & Given \\
            4. $\forall x\; \neg (Q(x) \wedge R(x))$    & & DeMorgan's with Quantifiers with (2) \\
            5. $P(a) \ra S(a)$                          & & Univ. instantiation with (1) \\
            6. $\neg R(a) \ra \neg S(a)$                & & Univ. instantiation with (3) \\
            7. $\neg (Q(a) \wedge R(a))$                & & Univ. instantiation with (4) \\
            8. $\neg Q(a) \vee \neg R(a)$               & & DeMorgan's with (7) \\
            9. $\neg R(a) \vee \neg Q(a)$               & & Commutative with (8) \\
            10. $\neg \neg R(a) \ra \neg Q(a)$          & & Table 7, rule 3 with (9) \\
            11. $R(a) \ra \neg Q(a)$                    & & Double negation with (10) \\
            12. $S(a) \ra R(a)$                         & & Table 7, rule 2 with (6) \\
            13. $P(a) \ra R(a)$                         & & Hyp. syllogism with (5) and (12) \\
            14. $P(a) \ra \neg Q(a)$                    & & Hyp. syllogism with (13) and (11) \\
            15. $\forall x\; (P(x) \ra \neg Q(x))$      & & Univ. generalization with (14)
    \end{tabular}
\end{solution}


\question  Match the term in the left column with the definition
from the right column that best matches the word.  Put the letter
associated with a word's definition in the space next to the word.

\footnotesize
\begin{tabular}{p{1.4in}p{4.2in}}
\rule{0.4in}{.01in} Argument    & (a) A declarative statement that
is either true of false, but not both.\\
\ifprintanswers
\else
	&\\[2pt]
\fi
\rule{0.4in}{.01in} Axiom       & (b) A form of incorrect reasoning\\
\ifprintanswers
\else
	&\\[2pt]
\fi
\rule{0.4in}{.01in} Conjecture  & (c) A proposition that can be
established directly from a theorem that has been proved\\
\ifprintanswers
\else
	&\\[2pt]
\fi
\rule{0.4in}{.01in} Contingency & (d) A proposition that is always false\\
\ifprintanswers
\else
	&\\[2pt]
\fi
\rule{0.4in}{.01in} Contradiction & (e) A proposition that is always true\\
\ifprintanswers
\else
	&\\[2pt]
\fi
\rule{0.4in}{.01in} Corollary   & (f) A proposition that is neither
always true nor always false\\
\ifprintanswers
\else
	&\\[2pt]
\fi
\rule{0.4in}{.01in} Fallacy     & (g) A sequence of propositions.\\
\ifprintanswers
\else
	&\\[2pt]
\fi
\rule{0.4in}{.01in} Lemma       & (h) A simple theorem used in proof of another theorem\\
\ifprintanswers
\else
	&\\[2pt]
\fi
\rule{0.4in}{.01in} Proposition & (i) A statement accepted as true
as the basis for argument or inference\\
\ifprintanswers
\else
	&\\[2pt]
\fi
\rule{0.4in}{.01in} Tautology   & (j) A statement proposed to be
true.\\
\ifprintanswers
\else
	&\\[2pt]
\fi
\rule{0.4in}{.01in} Theorem     & (k) A statement that can be shown to be true\\
\end{tabular}
\normalsize
    \ifprintanswers
        \vspace{-15pt}
    \fi
\begin{solution}
    The following matches are correct:

    \begin{tabular}{ll}
        (g) - Argument      & (b) - Fallacy \\
        (i) - Axiom         & (h) - Lemma  \\
        (j) - Conjecture    & (a) - Proposition \\
        (f) - Contingency   & (e) - Tautology \\
        (d) - Contradiction & (k) - Theorem \\
        (c) - Corollary     & \\
    \end{tabular}
\end{solution}


\question Rosen Ch 1.7 \# 2, p. 91
%Use a direct proof to show that the sum of two even integers is even.
    \ifprintanswers
        \vspace{-10pt}
    \fi
\begin{solution} \textbf{Proof:} Assume you have two even integers $a$ and $b$.  By definition of even numbers, then there exists integers $k_a$ and $k_b$ such that $a=2k_a$ and $b=2k_b$.  Compute $a+b$:
    $$a+b = 2k_a + 2k_b = 2(k_a + k_b) = 2k_c.$$
    The sum of the two even numbers is also in the form of a even number.
    Therefore, the sum of two even integers is even.
\end{solution}


\question Give a direct proof that for all real numbers $x$ and $y$,
if $x$ and $y$ are rational, then $x-y$ is rational.
    \ifprintanswers
        \vspace{-10pt}
    \fi
\begin{solution} \textbf{Proof:} If $x$ and $y$ are rational, then there are
    integers $p, q\neq 0, r$, and $s\neq 0$ such that
    $x = \frac{p}{q}$ and $y = \frac{r}{s}$ using the definition of rational.  Then,
    \[ x - y = \frac{p}{q} - \frac{r}{s} = \frac{ps - qr}{qs} \]
    Since $ps - qr$ and $qs \neq 0$ are integers, it follows
    that $x - y$ is rational.  We have proved the difference of
    two rational numbers is rational.
\end{solution}


\question For all integers $n$, if $n$ is divisible by 5,
then $5n+5$ is divisible by 5.
    \ifprintanswers
        \vspace{-10pt}
    \fi
\begin{solution} \textbf{Proof:} Assume $n$ is divisible by 5.
    Then by definition of divisible there exists a integers
    $k$ s.t. $n=5k$. Compute $5n+5$:
    \[ 5n + 5 = 5(5k) + 5 = 5(5k + 1) \]
    It follows that $5n+5$ is divisible by 5. Therefore, for all
    integers $n$, if $n$ is divisible by 5, then $5n+5$ is
    divisible by 5.
\end{solution}


\question Give a proof by contradiction that if $n$ is a natural
number and $3n+3$ is odd, then $n$ is even.
    \ifprintanswers
        \vspace{-10pt}
    \fi
\begin{solution}
    \textbf{Proof by contradiction:} Assume $3n+3$ is odd and $n$ is
    odd.  Then $n=2k + 1$ for some integer $k$.  Compute
    $3n+3$:
    \[ 3n + 3 = 3(2k+1) + 3 = 6k + 6 = 2(3k + 3)\]
    showing $3n+3$ is even. This contradicts the assumption that $3n
    + 3$ is odd. Consequently, if $n$ is a natural number and $3n + 3$ is odd, then $n$ is even.
\end{solution}


\question Prove for all natural numbers $n$ and $m$, $nm$ is odd if
and only if $n$ and $m$ are both odd.
    \ifprintanswers
        \vspace{-10pt}
    \fi
\begin{solution}
     This proof is for a theorem using ``if and only if" therefore,
     it must be proved in both directions.  \\
    \textbf{Prove ``if p, then q" by contradiction:} Assume $nm$ is
    odd and $n$ and $m$ are not both odd.  Then at least one of $n$
    and $m$ are even.  Suppose $n$ is the one that is even. Then
    $n=2k$ for some natural number $k$.  Then $nm = 2km$ and so $nm$
    is even.  This contradicts the assumption that $nm$ is odd.
    Similarly, if we assume instead that $m$ is the one that is even,
    we reach the same contradiction.   If both $m$ and $n$ are both
    even we reach the same result.  Thus, it must be that $n$ and
    $m$ are both odd.

    \smallskip
    \textbf{Prove ``if q then p" directly:} Assume $n$ and $m$ are
    both odd.  Then, there are natural numbers $k$ and $j$ such
    that $n = 2k+1$ and $m=2j+1$.  Then,
    \[ nm = (2k+1)(2j+1) = 4kj + 2(k+j) +1 = 2(2kj +k + j) +1 \]
    So, $nm$ is odd by definition. Therefore, if $n$ and $m$ are
    odd, $nm$ is odd.
\end{solution}


\question Use a proof by cases to show that $min(a, min(b,c)) =
min(min(a,b),c)$ whenever $a$, $b$, and $c$ are real numbers.
% SEE Jean Mayo hw5
    \ifprintanswers
        \vspace{-10pt}
    \fi
\begin{solution}
    There are 6 cases for the values of $a$,$b$, $c$. \\
    \textbf{case 1:} $a \leq b \leq c$, then \\
        $min(a, min(b,c)) = min(a,c) = a$ \\
        $min(min(a,b),c)) = min(a,c) = a$ \\
    \textbf{case 2:} $a \leq c \leq b$, then \\
        $min(a, min(b,c)) = min(a,c,) = a$ \\
        $min(min(a,b),c) = min(a,c) = a$ \\
    \textbf{case 3:} $b \leq a \leq c$, then \\
        $min(a, min(b,c)) = min(a,b) = b$\\
        $min(min(a,b),c) = min(b,c) = b$ \\
    \textbf{case 4:} $b \leq c \leq a$, then \\
        $min(a, min(b,c)) = min(a,b) = b$ \\
        $min(min(a,b),c) = min(b,c) = b$ \\
    \textbf{case 5:} $c \leq a \leq b$, then \\
        $min(a, min(b,c)) = min(a,c) = c$\\
        $min(min(a,b),c) = min(a,c) = c$ \\
    \textbf{case 6:} $c \leq b \leq a$, then \\
        $min(a, min(b,c)) = min(a,c) = c$ \\
        $min(min(a,b),c) = min(b,c) = c$ \\
    We have shown it to be true, in all cases.
\end{solution}


\question Prove the following proposition, ``For any integer $n \geq 2$, $n^2-3$ is never divisible by 4."
    \ifprintanswers
        \vspace{-10pt}
    \fi
\begin{solution}
    Consider two cases:
    \begin{itemize}
        \item[Case 1]: If $n$ is even, then by definition $n^2$is also even and $n^2-3$ will be odd.  Therefore, $n^2-3$ is not divisible by 4.
        \item[Case 2]: If $n$ is odd, there are 4 cases to consider. $n$ can be written as $4k$, $4k+1$, $4k+2$, and $4k+3$ for an integer $k$.  The forms of $4k$ and $4k+2$ are even, not matching the condition $n$ is odd, and will not be considered further.
            \begin{itemize}
                \item[Case a]:
                \begin{align*}
                    n &= 4k + 1 \\
                    n^2 - 3 &= (4k + 1)^2 - 3 \\
                        &= 16k^2 + 8k + 1 - 3 \\
                        &= 16k^2 +8k - 2
                \end{align*}
                The number $n^2-3$ is not divisible by 4.
                \item[Case b]:
                \begin{align*}
                    n &= 4k+3 \\
                    n^2 - 3 &= (4k + 3)^2 - 3 \\
                     &= 16k^2 + 24k + 9 - 3 \\
                     &= 16k^2 + 24k + 6
                \end{align*}
                The number $n^2-3$ is not divisible by 4.
                Therefore, if $n$ is odd, $n^2-3$ is not divisible by 4.
            \end{itemize}
        \end{itemize}
        Therefore, we have shown in all cases that for any integer $n \geq 2$, $n^2 - 3$ is not divisible by 4.
\end{solution}


%\question Prove that if $n$ is a perfect square, then $n+2$ is not a perfect square.
%    \ifprintanswers
%        \vspace{-10pt}
%    \fi
%\begin{solution}
%    Assume $n$ is a perfect square then by definition there exists an integer $a$ s.t. $n=a^2$.  Also, assume $n+2$ is a perfect square, then there exists an integer $b$ s.t. $n+2 = b^2$.  Plug $n=a^2$ into the equation $n + 2 = b^2$.  Thus,
%    \[ a^2 + 2 = b^2 \;\text{or}\; 2 = b^2 - a^2 \]
%    This can be written as $2 = (b-a)(b+a)$. In this expression, we know that both $b-a$ and $b+a$ are integers because $a$ and $b$ are integers.  The only two integers that when multiplied together equal 2 are 2,1 or -2,-1.  Consider the cases:
%    \begin{itemize}
%        \item[Case 1]:
%        \[ \text{Solve:}\;\;\;
%            \begin{matrix}
%                b & -a \\
%                b & +a
%            \end{matrix}
%            =
%            \begin{matrix}
%                2 \\
%                1
%            \end{matrix}
%            \hspace*{0.5in} Solution:  b=1.5, a=-0.5 \]
%        \item[Case 2]:
%        \[ \text{Solve:}\;\;\;
%            \begin{matrix}
%                b & -a \\
%                b & +a
%            \end{matrix}
%            =
%            \begin{matrix}
%                1 \\
%                2
%            \end{matrix}
%            \hspace*{0.5in} Solution: b=1.5, a=0.5 \]
%        \item[Case 3]:
%        \[ \text{Solve:}\;\;\;
%            \begin{matrix}
%                b & -a \\
%                b & +a
%            \end{matrix}
%            =
%            \begin{matrix}
%                -2 \\
%                -1
%            \end{matrix}
%            \hspace*{0.5in} Solution: b=-1.5, a=0.5 \]
%        \item[Case 4]:
%        \[ \text{Solve:}\;\;\;
%            \begin{matrix}
%                b & -a \\
%                b & +a
%            \end{matrix}
%            =
%            \begin{matrix}
%                -1 \\
%                -2
%            \end{matrix}
%            \hspace*{0.5in} Solution: b=-1.5, a=-0.5 \]
%    \end{itemize}
%    In all cases, the only solutions for $b$ and $a$ in each case are not integers which breaks the assumption.  Thus a contradiction.  Therefore using a proof by contradiction, if $n$ is a perfect square, then $n+2$ is not a perfect square.
%\end{solution}

\end{questions}
\end{document}
