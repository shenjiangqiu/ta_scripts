\documentclass[12pt,addpoints]{exam}
% can include option [answers] to print out solutions, or command \printanswers
%  can turn addpoint on and off with commands, \addpoints and \noaddpoints

\usepackage{amsthm}
\usepackage{amssymb}
\usepackage{amsmath}
\usepackage{color}
\usepackage{enumitem}
\usepackage[top=0.75in,bottom=0.75in,left=0.9in,right=0.9in]{geometry}

\setlength{\itemsep}{0pt} \setlength{\topsep}{0pt}
\newcommand{\ra}{\rightarrow}
\newcommand{\lra}{\leftrightarrow}
\newcommand{\xor}{\oplus}

\begin{document}
\extrawidth{0.5in} \extrafootheight{-0in} \pagestyle{headandfoot}
\headrule \header{\textbf{cs2311 - Fall 2012}}{\textbf{HW
1 - \numpoints$\;$ points - Solutions}}{\textbf{Due: Wed. 9/12/12}} \footrule \footer{}{Page \thepage\
of \numpages}{}

%\noindent \textbf{Instructions:} All assignments are due \underline{by 5pm on %the due date} specified.  There will be a box in the CS department office where %assignments may be turned in.  Solutions will be handed out (or posted on-line) %shortly thereafter.  Every student
% must write up their own solutions in their own manner.

%\smallskip
%\noindent Please follow the format in the book when asked to produce
%truth tables (this will help in grading in order to avoid any
%errors).

%\smallskip
%\noindent Please present your solutions in a clean, understandable
%manner; pages should be stapled before class, no ragged edges of
%paper.


\begin{questions}
\printanswers

\question[4] Determine which of the statements are propositions? What are the truth values of those that are propositions?
    \ifprintanswers
        \vspace{-15pt}
    \fi
    \begin{solution}
        \begin{enumerate}[label=(\alph*),itemsep=0pt,parsep=0pt,
        topsep=0pt,partopsep=0pt]
        	\item 8 is even. \hspace{1.6in} proposition, $T$
        	\item $z - 5 > 2$. \hspace{1.55in} not a proposition, depends on $z$
        	\item Who won the World Series? \hspace{0.27in} not a proposition, question
        	\item 4 + 3 = 8. \hspace{1.5in} proposition, $F$
        \end{enumerate}
    \end{solution}


\question[9] Rosen Ch 1.1, \#10 (a,b,c), p. 13 
	\ifprintanswers
        \vspace{-15pt}
    \fi
    \begin{solution}
    	\begin{enumerate}[label=(\alph*),itemsep=0pt,parsep=0pt,
    	topsep=0pt,partopsep=0pt]
    		\item It is not the case that the election is decided. \textbf{OR} \\The election is not decided.
    		\item The election is decided, or the votes have been counted.
    		\item It is not the case that the election is decided and the votes have been counted. \textbf{OR} The election is not decided, and the votes have been counted.
    	\end{enumerate}
    \end{solution}


\question[6] Rosen Ch 1.1 \#14 (b,e,f), p. 13-14
    \ifprintanswers
        \vspace{-15pt}
    \fi
    \begin{solution}
    	(b) $p \wedge q \wedge r$ \hfill (e) $(p \wedge q) \ra r$ \hfill (f) $r \lra (q \vee p)$
    \end{solution}
%        \begin{itemize}[itemsep=0pt,parsep=0pt,topsep=0pt,partopsep=0pt]
%            \item[(b)] $p \wedge q \wedge r$
%            \item[(e)] $(p \wedge q) \ra r$
%            \item[(f)] $r \lra (q \vee p)$
%        \end{itemize}
    

\question[9] Rosen Ch 1.1 \#22 (d,e,f), p. 14
    \ifprintanswers
        \vspace{-15pt}
    \fi
    \begin{solution}
    	\begin{itemize}[itemsep=0pt,parsep=0pt,topsep=0pt,partopsep=0pt]
    		\item[(d)] If Willy cheats, then he gets caught.
    		\item[(e)] If you access the website, then you must pay a subscription fee.
    		\item[(f)] If you know the right people, then you will get elected.
		\end{itemize}
	\end{solution}
	

\question[4] Rosen Ch 1.1 \#26 (b,d), p. 15
%            \item[(a)] You will get an A in this course if and only if you learn how to solve discrete mathematics problems.
%            \item[(c)] It rains if and only if it is a weekend day.
%        \end{itemize}
%    \end{solution}
    \ifprintanswers
        \vspace{-15pt}
    \fi
    \begin{solution}
        \begin{itemize}[itemsep=0pt,parsep=0pt,topsep=0pt,partopsep=0pt]
        	\item[(b)] You will be informed if and only if you read the newspaper every day. \textbf{OR} \\ You read the newspaper every day if and only if you will be informed.
        	\item[(d)] You can see the wizard if and only if he is not in.
        \end{itemize}
    \end{solution}


\question[6] Rosen Ch 1.1 \# 28 (a,c), p. 15
    \ifprintanswers
        \vspace{-15pt}
    \fi
    \begin{solution}
        \begin{itemize}[itemsep=0pt,parsep=0pt,topsep=0pt,partopsep=0pt]
        	\item[(a)] Converse: If I stay home, then it will snow tonight. \\
        		Contrapositive: If I do not stay at home, then it will not snow tonight. \\
        		Inverse: If it does not snow tonight, then I will not stay home.
        	\item[(c)] Converse: If I sleep until noon, then I stayed up late. \\
        		Contrapositive: If I do not sleep until noon, then I did not stay up late. \\
        		Inverse: If I don't stay up late, then I don't sleep until noon.
        \end{itemize}
    \end{solution}


\question[2] Rosen Ch 1.1 \#30 (b,d), p. 15
    \ifprintanswers
        \vspace{-15pt}
    \fi
    \begin{solution} (b) 8;  (d) 32
    \end{solution}



\question[20]\label{tt} Construct a truth table for each compound proposition.
    \ifprintanswers
        \vspace{-10pt}
    \fi
    \footnotesize
    \begin{center}
    \begin{tabular}{ll}
       (a) $p \lra \neg p$  & (b) $p \ra \neg p$ \\
       (c) $(p \vee \neg q) \ra (p \wedge q)$ & (d) $ (p \lra \neg q) \xor (p \wedge \neg q)$ \\
       (e) %$(p \vee q) \wedge r$  & (f) 
       $(p \vee q) \wedge (\neg p \vee r) \wedge (q \vee \neg r)$
    \end{tabular}
    \end{center}
    \normalsize
    \ifprintanswers
        \vspace{-15pt}
    \fi
    \begin{solution}
    \footnotesize
        \begin{tabular}{c|c||c}
            \multicolumn{3}{l}{ (a) } \\
            $p$ & $\neg p$ & $p \lra \neg p$ \\
         \hline
            T & F & F \\
            F & T & F \\
         \end{tabular} \hspace{1in}
         \begin{tabular}{c|c||c}
            \multicolumn{3}{l}{ (b) } \\
            $p$ & $\neg p$ & $p \ra \neg p$ \\
         \hline
            T & F & F \\
            F & T & T \\
         \end{tabular}

        \smallskip
        \begin{tabular}{c|c|c||c|c|c}
            \multicolumn{5}{l}{ (c) } & $(p \vee \neg q) \ra $ \\
            $p$ & $q$ & $\neg q$ & $p \vee \neg q$ & $p \wedge q$ & $(p \wedge q)$ \\
         \hline
            T & T & F & T & T & T \\
            T & F & T & T & F & F \\
            F & T & F & F & F & T \\
            F & F & T & T & F & F \\
        \end{tabular} \hspace{0.5in}
        \begin{tabular}{c|c|c||c|c|c|c} %$ (p \lra \neg q)\xor (p\wedge \neg q)$
            \multicolumn{5}{l}{(d)} & $(p \lra \neg q) \xor$ \\
            $p$ & $q$ & $\neg q$ & $p \lra \neg q$ & $p \wedge \neg q$ & $ (p \wedge \neg q)$ \\
         \hline
            T & T & F & F & F  & F \\
            T & F & T & T & T  & F \\
            F & T & F & T & F  & T \\
            F & F & T & F & F  & F \\
        \end{tabular}

        \smallskip
        % $(p \vee q) \wedge (\neg p \vee r) \wedge (q \vee \neg r$
        \begin{tabular}{c|c|c|c|c||c|c|c|c}
            \multicolumn{4}{l}{(e)} & \\
            $p$ & $q$ & $r$ & $\neg p$ & $\neg r$ & $p \vee q$ & $\neg p \vee r$ & $q \vee \neg r$ & (e) \\
         \hline
            T & T & T & F & F & T & T & T & T \\
            T & T & F & F & T & T & F & T & F \\
            T & F & T & F & F & T & T & F & F \\
            T & F & F & F & T & T & F & T & F \\
          \hline
            F & T & T & T & F & T & T & T & T \\
            F & T & F & T & T & T & T & T & T \\
            F & F & T & T & F & F & T & F & F \\
            F & F & F & T & T & F & T & T & F
         \end{tabular} \hspace{0.4in}
    \end{solution}


\question[3] List those expressions from problem \ref{tt} that are tautologies, contradictions, and contingencies?
    \ifprintanswers
        \vspace{-15pt}
    \fi
    \begin{solution}
       Tautology: none;
       Contradiction: (a);
       Contingency: (b), (c), (d), (e)
    \end{solution}


\question[8] There are 16 possible truth tables for propositions of two variables $p$ and $q$.
 All sixteen possibilities are given in the table below (numbered 15, 14, \ldots, 0).
 For example, the proposition $p \vee q$ is 14. What are the numbers of the truth
 tables for propositions:
     \ifprintanswers
        \vspace{-10pt}
    \fi
\begin{center}
 \begin{tabular}{ll}
    (a) $(p \wedge \neg q) \ra (\neg p \vee q)$ \quad &  (b) $p \vee (\neg q \wedge p)$ \\
    (c) $p \oplus \neg q $ \quad \quad& (d) $(\neg p \ra q) \wedge (\neg q \vee p)$ \\
 \end{tabular}

 \footnotesize
 \begin{tabular}{cc|cccc|cccc|cccc|cccc}
    $p$ & $q$ & 15 & 14 & 13 & 12 & 11 & 10 & 9 & 8 & 7 & 6 & 5 & 4 & 3 & 2 & 1 & 0 \\
    \hline
    T & T & T & T & T & T & T & T & T & T & F & F & F & F & F & F & F & F \\
    T & F & T & T & T & T & F & F & F & F & T & T & T & T & F & F & F & F \\
    F & T & T & T & F & F & T & T & F & F & T & T & F & F & T & T & F & F \\
    F & F & T & F & T & F & T & F & T & F & T & F & T & F & T & F & T & F \\
 \end{tabular}
 \end{center}
 \normalsize
     \ifprintanswers
        \vspace{-15pt}
    \fi
    \begin{solution} (a) 11, (b) 12, (c) 9, (d) 12
    \end{solution}



\question[10] Prove the following statement is a tautology without using truth tables (use the logical equivalences from Table 6-8 of the book).  Justify each step with the law used.  Model the solutions in the style of Examples 6-8, pp. 29-30 of the book.
\[ [ p \wedge (p \ra q) ] \ra q \]
    \ifprintanswers
        \vspace{-40pt}
    \fi
\begin{solution}  There are many possible solutions; two are given.
\small
	\begin{align*}
	[p \wedge (p \ra q)] \ra q \\
	 & \equiv \neg [p \wedge (\neg p \vee q)] \vee q \tag{Table7.1, x2} \\
	 & \equiv [\neg p \vee \neg (\neg p \vee q)] \vee q \tag{DeMorgan's} \\
	 & \equiv [\neg p \vee (\neg \neg p \wedge \neg q)] \vee q \tag{DeMorgan's} \\
	 & \equiv [\neg p \vee (p \wedge \neg q)] \vee q \tag{Double Negation} \\
	 & \equiv [(\neg p \vee p) \wedge (\neg p \vee \neg q)] \vee q \tag{Distributive} \\
	 & \equiv [\mathbf{T} \wedge (\neg p \vee \neg q)] \vee q \tag{Negation} \\
	 & \equiv [(\neg p \vee \neg q) \wedge \mathbf{T}] \vee q \tag{Commutative} \\
	 & \equiv (\neg p \vee \neg q) \vee q \tag{Identity} \\
	 & \equiv \neg p \vee (\neg q \vee q) \tag{Associative} \\
	 & \equiv \neg p \vee \mathbf{T} \tag{Negation} \\
	 & \equiv \mathbf{T} \tag{Domination}
	 \end{align*}
	\textbf{OR}
	\begin{align*}
	& [p \wedge (p \ra q)] \ra q  \\
	& \equiv [p \wedge (\neg p \vee q)] \ra q \tag{Table 7, rule1} \\
	& \equiv [(p \wedge \neg p) \vee (p \wedge q)] \ra q \tag{Distributive} \\
	& \equiv [\mathbf{F} \vee (p \wedge q)] \ra q \tag{Negation} \\
	& \equiv [(p \wedge q) \vee \mathbf{F}] \ra q \tag{Commutative} \\
	& \equiv (p \wedge q) \ra q \tag{Identity} \\
	& \equiv \neg (p \wedge q) \vee q \tag{Table 7, rule 1} \\
	& \equiv (\neg p \vee \neg q) \vee q \tag{DeMorgan's} \\
	& \equiv \neg p \vee (\neg q \vee q) \tag{Associative} \\
	& \equiv \neg p \vee \mathbf{T} \tag{Negation} \\
	& \equiv \mathbf{T} \tag{Domination}
	\end{align*}
\end{solution}


\question[8] Prove the following logical equivalence using the
equivalence laws from Table 6-8 (model the solutions in the style of
book Examples 6-8, pp. 29-30).  Justify each step with laws.
\[\neg [ r \vee (q \wedge (\neg r \ra \neg p))] \equiv \neg r \wedge (p \vee \neg q) \]
    \ifprintanswers
        \vspace{-30pt}
    \fi
    \begin{solution}
    \begin{align*}
      \neg [ r \vee (q & \wedge (\neg r \ra \neg p))] \equiv \neg r \wedge (p \vee \neg q) \\
        & \equiv \neg [r \vee (q \wedge (p \ra r))] \tag{Table 7, rule 2} \\
        & \equiv \neg [r \vee (q \wedge (\neg p \vee r))] \tag{Table 7, rule 1} \\
        & \equiv \neg [r \vee (q \wedge \neg p) \vee (q \wedge r)] \tag{Distributive} \\
        & \equiv \neg [r \vee (q \wedge r) \vee (q \wedge \neg p)] \tag{Commutative} \\
        & \equiv \neg [r \vee (q \wedge \neg p)] \tag{Absorption} \\
        & \equiv \neg [r \vee (\neg p \wedge q)] \tag{Commutative} \\
        & \equiv \neg r \wedge \neg (\neg p \wedge q) \tag{DeMorgan's} \\
        & \equiv \neg r \wedge (\neg \neg p \vee \neg q) \tag{DeMorgan's} \\
        & \equiv \neg r \wedge (p \vee \neg q) \tag{Double Negation}
    \end{align*}
    \end{solution}
    
    
\question[3] Express the following statement using only the connectives $\neg$ and $\wedge$:
\[ p \ra (\neg q \wedge r) \]
    \ifprintanswers
        \vspace{-30pt}
    \fi
	\begin{solution}
		\[p \ra (\neg q \wedge r) \equiv \neg p \vee (\neg q \wedge r) \equiv \neg (p \wedge \neg (\neg q \wedge r))\]
	\end{solution}
	


\bonusquestion[2] Give a compound proposition with three variables $p$, $q$, and $r$ that is
true when at most one of the three variables is true, and false otherwise.
    \begin{solution}
    \begin{align*}
    (p \wedge \neg q \wedge \neg r) \vee (\neg p \wedge q \wedge \neg r)
      \vee (\neg p \wedge \neg q \wedge r) \vee (\neg p \wedge \neg q \wedge \neg r) \\
    \text{OR} \\
    (p \wedge \neg q \wedge \neg r) \vee (\neg p \wedge q \wedge \neg r)
      \vee (\neg p \wedge \neg q \wedge r) 
    \end{align*}
    \end{solution}
\end{questions}
\end{document}
