
\begin{document}
\extrawidth{0.5in} \extrafootheight{-0in} \pagestyle{headandfoot}
\headrule \header{\textbf{cs2311 - Fall 2016}}{\textbf{HW
 6 \ifprintanswers - Solutions \fi}}{\textbf{Due: Fri. 10/14/16}} \footrule \footer{}{Page \thepage\
of \numpages}{}

\ifprintanswers
\noindent \textbf{Instructions:} All assignments are due \underline{by \textbf{midnight} on the due date} specified.  Assignments should be typed and submitted as a PDF.  Every student must write up their own solutions in their own manner.

\medskip
\noindent You should \underline{complete all problems}, but \underline{only a subset will be graded} (which will be graded is not known to you ahead of time). 
\else
\noindent \textbf{Instructions:} All assignments are due \underline{by \textbf{midnight} on the due date} specified.  Every student must write up their own solutions in their own manner.

\noindent Please present your solutions in a clean, understandable
manner.  Use the provided files that give mathematical notation in Word, Open Office, Google Docs, and \LaTeX. 

\noindent Assignments should be typed and submitted as a PDF.   

\noindent You should \underline{complete all problems}, but \underline{only a subset will be graded} (which will be graded is not known to you ahead of time). 
\fi

\begin{questions}


\uplevel{\color{blue} \large For each question, there are many possible solutions.  I have typically provided one or two common solutions to these problems. }

\gquestion{8}{8}{all} Prove the following statement is a tautology without using truth tables (use the logical equivalences from Table 6-8 of the book).  Justify each step with the law used.  Model the solutions in the style of Examples 6-8, pp. 29-30 of the book.
\[ [ \neg p \wedge (p \vee q) ] \ra q \]
    \ifprintanswers
        \vspace{-30pt}
    \fi
\begin{solution}  There are many possible solutions; two are given.
\small
    \begin{align*}
        [ \neg p & \wedge (p \vee q) ] \ra q \\
        & \equiv \neg [ \neg p \wedge (p \vee q) ] \vee q \tag{Table 7, rule 1} \\
        & \equiv \neg \neg p \vee \neg (p \vee q) \vee q \tag{DeMorgan's law} \\
        & \equiv p \vee \neg (p \vee q) \vee q \tag{Double Negation} \\
        & \equiv p \vee q \vee \neg (p \vee q) \tag{Commutative} \\
        & \equiv (p \vee q) \vee \neg (p \vee q) \tag{algebra, add ()} \\
        & \equiv \mathbf{T} \tag{Negation} 
    \end{align*}

    \begin{center}
    OR
    \end{center}

    \vspace{-20pt}
    \begin{align*}
        [ \neg p & \wedge (p \vee q) ] \ra q \\
        & \equiv \neg [ \neg p \wedge (p \vee q) ] \vee q \tag{Table 7, rule 1} \\
        & \equiv \neg \neg p \vee \neg (p \vee q) \vee q \tag{DeMorgan's law} \\
        & \equiv p \vee \neg (p \vee q) \vee q \tag{Double Negation} \\
        & \equiv p \vee q \vee \neg (p \vee q) \tag{Commutative} \\
        & \equiv p \vee q \vee (\neg p \wedge \neg q) \tag{DeMorgan's} \\
        & \equiv p \vee (q \vee \neg p) \wedge (q \vee \neg q) \tag{Distributive} \\
        & \equiv p \vee (q \vee \neg p) \wedge \mathbf{T} \tag{Negation} \\
        & \equiv p \vee (q \vee \neg p) \tag{Identity} \\
        & \equiv p \vee (\neg p \vee q) \tag{Commutative} \\
        & \equiv (p \vee \neg p) \vee q \tag{Associative} \\
        & \equiv \mathbf{T} \vee q \tag{Negation} \\
        & \equiv \mathbf{T} \tag{Domination}
    \end{align*}
\end{solution}


% \ugquestion{8} 
% Show the following statement is a tautology without using truth tables. Use the equivalence laws from Table 6-8 (model the solutions in the style of book Examples 6-8, pp. 29-30).  Justify each step with laws.

% \vspace{-0.15in}
% \begin{center}
%     $[(p \vee q) \wedge (p \ra r) \wedge (q \ra r)] \ra r$
% \end{center}
% \vspace{-10pt}

%     \begin{solution}
%     \begin{align*}
%       [(p \vee q) &\wedge ((p \ra r) \wedge (q \ra r))] \ra r \\
%       &\equiv \neg[(p \vee q) \wedge \neg(p \vee r) \wedge \neg (q \vee r)] \vee r \tag{Table 7, rule 1, x3} \\
%       &\equiv [\neg(p \vee q) \vee \neg(\neg(p \vee r) \wedge \neg(q \vee r))] \vee r \tag{DeMorgan's} \\
%       & \equiv (\neg p \wedge \neg q) \vee (\neg \neg(p \vee r) \vee \neg \neg(q \vee r)) \vee r \tag{DeMorgan's, x2} \\
%       & \equiv (\neg p \wedge \neg q) \vee (p \vee r) \vee (q \vee r) \vee r \tag{Double Negation, x2} \\
%       & \equiv (\neg p \wedge \neg q) \vee p \vee q \vee r \vee r \vee r \tag{Commutative} \\
%       & \equiv (\neg p \wedge \neg q) \vee p \vee q \vee r \vee r \tag{Idempotent} \\
%       & \equiv (\neg p \wedge \neg q) \vee p \vee q \vee r \tag{Idempotent} \\
%       & \equiv p \vee (\neg p \wedge \neg q) \vee q \vee r \tag{Commutative} \\
%       & \equiv (p \vee \neg p) \wedge (p \vee \neg q) \vee q \vee r \tag{Distributive} \\
%       & \equiv (\mathbf{T} \wedge (p \vee \neg q) \vee q \vee r \tag{Negation} \\
%       & \equiv (p \vee \neg q) \vee q \vee r \tag{Identity} \\
%       & \equiv p \vee (\neg q \vee q) \vee r \tag{Associative} \\
%       & \equiv p \vee \mathbf{T} \vee r \tag{Negation} \\
%       & \equiv \mathbf{T} \vee r \tag{Domination} \\
%       & \equiv \mathbf{T} \tag{Domination}
%     \end{align*}

%     An alternative solution is:
%     \begin{align*}
%       [(p \vee q) &\wedge ((p \ra r) \wedge (q \ra r))] \ra r \\
%        &\equiv [(p \vee q) \wedge ((p \vee q) \ra r)] \ra r \tag{Table 7, rule 7} \\
%        &\equiv [(p \vee q) \wedge (\neg (p \vee q) \vee r)] \ra r \tag{Table 7, rule 1} \\
%        &\equiv [((p \vee q) \wedge \neg (p \vee q)) \vee ((p \vee q) \wedge r)] \ra r \tag{Distributive} \\
%        &\equiv [\mathbf{F} \vee ((p \vee q) \wedge r)] \ra r \tag{Negation} \\
%        & \equiv [((p \vee q) \wedge r) \vee \mathbf{F}] \ra r \tag{Commutative} \\
%        & \equiv [((p \vee q) \wedge r)] \ra r \tag{Identity} \\
%        & \equiv [(p \vee q) \wedge r] \ra r \tag{drop extra ()} \\
%        & \equiv \neg [(p \vee q) \wedge r] \vee r \tag{Table 7, rule 1} \\
%        & \equiv  [\neg (p \vee q) \vee \neg r] \vee r \tag{DeMorgans} \\
%        & \equiv  \neg (p \vee q) \vee [\neg r \vee r ] \tag{Associative} \\
%        & \equiv \neg (p \vee q) \vee [r \vee \neg r] \tag{Commutative} \\
%        & \equiv \neg (p \vee q) \vee \mathbf{T} \tag{Negation} \\
%        & \equiv \mathbf{T} \tag{Domination} 
%     \end{align*}
%     \end{solution}

\newpage
\ugquestion{8} Prove the following statement is a tautology without using truth tables (use the logical equivalences from Table 6-8 of the book).  Justify each step with the law used.  Model the solutions in the style of Examples 6-8, pp. 29-30 of the book.
\[ [(p \rightarrow q) \wedge (q \rightarrow r)] \rightarrow (p \rightarrow r) \]
    \ifprintanswers
        \vspace{-30pt}
    \fi
	\begin{solution}
   \begin{align*}
    [(p \rightarrow q) &\wedge (q \rightarrow r)] \rightarrow (p \rightarrow r) \\
     & \equiv \neg [ (\neg p \vee q) \wedge (\neg q \vee r)] \vee (\neg p \vee r) \tag{Table 7, rule 1, x4}\\
     & \equiv [\neg (\neg p \vee q) \vee \neg (\neg q \vee r)] \vee (\neg p \vee r) \tag{DeMorgan's} \\
     & \equiv [(\neg \neg p \wedge \neg q) \vee (\neg \neg q \wedge \neg r)] \vee (\neg p \vee r) \tag{DeMorgan's, x2} \\
     & \equiv (p \wedge \neg q) \vee (q \wedge \neg r) \vee \neg p \vee r \tag{Double Negation, x2} \\
     & \equiv (p \wedge \neg q) \vee \neg p \vee (q \wedge \neg r) \vee r \tag{Commutative} \\
     & \equiv \neg p \vee (p \wedge \neg q) \vee r \vee (q \wedge \neg r) \tag{Commutative, x2} \\
     & \equiv (\neg p \vee p) \wedge (\neg p \vee \neg q) \vee (r \vee q) \wedge (r \vee \neg r) \tag{Distributive, x2} \\
     & \equiv \mathbf{T} \wedge (\neg p \vee \neg q) \vee (r \vee q) \wedge \mathbf{T} \tag{Negation, x2} \\
     & \equiv (\neg p \vee \neg q) \vee (r \vee q) \tag{Identity, x2} \\
     & \equiv \neg p \vee \neg q \vee r \vee q \tag{algebra} \\
     & \equiv \neg p \vee r \vee \neg q \vee q \tag{Commutative} \\
     & \equiv \neg p \vee r \vee \mathbf{T} \tag{Negation} \\
     & \equiv \neg p \vee \mathbf{T} \tag{Domination} \\
     & \equiv \mathbf{T} \tag{Domination}
  \end{align*}


  \begin{center}
  OR
  \end{center}
  \vspace{-20pt}
  \begin{align*}
    [(p \rightarrow q) \wedge (q \rightarrow r)] &\rightarrow (p \rightarrow r) \\
    & \equiv [(p \vee q) \ra r] \ra (p \ra r) \tag{Table 7, r7} \\
    & \equiv \neg[\neg(p \vee q) \vee r] \vee (\neg p \vee r) \tag{Table 7, r1, x3} \\
    & \equiv [\neg \neg (p \vee q) \vee r] \vee (\neg p \vee r) \tag{DeMorgans} \\
    & \equiv [(p \vee q) \vee r] \vee (\neg p \vee r) \tag{Double Negation} \\
    & \equiv p \vee q \vee r \vee \neg p \vee r  \tag{algebra} \\
    & \equiv p \vee q \vee \neg p \vee r \vee r \tag{Commutative} \\
    & \equiv p \vee \neg p \vee q \vee r \vee r \tag{Commutative} \\
    & \equiv \mathbf{T} \vee q \vee r \vee r \tag{Negation} \\
    & \equiv \mathbf{T} \vee q \vee r \tag{Idempotent} \\
    & \equiv q \vee \mathbf{T} \vee r \tag{Commutative} \\
    & \equiv \mathbf{T} \vee r \tag{Domination} \\
    & \equiv r \vee \mathbf{T} \tag{Commutative} \\
    & \equiv \mathbf{T} \tag{Domination}
  \end{align*}
  \end{solution}


\newpage
\gquestion{18}{10}{a-b} Prove the following logical equivalences using the
equivalences from Table 6-8 (model the solutions in the style of
book Examples 6-8, pp. 29-30).  Justify each step with laws.
\begin{enumerate}[label=(\alph*),itemsep=0pt,parsep=0pt,topsep=0pt,partopsep=0pt]
    \item $(p \wedge \neg(q \vee \neg p)) \equiv p \wedge \neg q$
    \item $p \rightarrow (\neg q \rightarrow r) \equiv \neg(q \vee r)
    \rightarrow \neg p$
    \item $\neg(\neg p \vee (p \wedge q)) \vee q \equiv q \vee p $
\end{enumerate}

    \begin{solution}
    \begin{parts}
    \part  $(p \wedge \neg(q \vee \neg p)) \equiv 
      p \wedge \neg q$
    \begin{align*}
      (p &\wedge \neg(q \vee \neg p)) \\
      & \equiv p \wedge (\neg q \wedge \neg \neg p) \tag{DeMorgan's} \\
      & \equiv p \wedge \neg q \wedge p \tag{Double Negation} \\
      & \equiv p \wedge p \wedge \neg q \tag{Commutative} \\
      & \equiv p \wedge \neg q \tag{Idempotent}
    \end{align*}
    
    \part $p \rightarrow (\neg q \rightarrow r) \equiv 
      \neg(q \vee r) \rightarrow \neg p$
    \begin{align*}
      p & \rightarrow (\neg q \rightarrow r) \\
      & \equiv p \ra (\neg \neg q \vee r) \tag{Table 7, rule 1} \\
      & \equiv p \ra (q \vee r) \tag{Double Negation} \\
      & \equiv \neg (q \vee r) \ra \neg p \tag{Table 7, rule 2} \\
    \end{align*}

    \vspace{-30pt}
    \begin{center}
    OR
    \end{center}

    \vspace{-30pt}
    \begin{align*}
      p & \rightarrow (\neg q \rightarrow r) \\
       & \equiv p \ra (q \vee r) \tag{Table 7, rule 3} \\
       & \equiv \neg (q \vee r) \ra \neg p  \tag{Table 7, rule 2}
    \end{align*}
    
    \part $\neg(\neg p \vee (p \wedge q)) \vee q \equiv q \vee p $
    \begin{align*}
    \neg(\neg p & \vee (p \wedge q)) \vee q \\
     & \equiv (\neg \neg p \wedge \neg (p \wedge q)) \vee q \tag{DeMorgan's} \\
     & \equiv (p \wedge \neg (p \wedge q)) \vee q \tag{Double Negation}\\
     & \equiv (p \wedge (\neg p \vee \neg q)) \vee q \tag{DeMorgan's}\\
     & \equiv (p \wedge \neg p) \vee (p \wedge \neg q) \vee q \tag{Distributive}\\
     & \equiv \mathbf{F} \vee (p \wedge \neg q) \vee q \tag{Negation}\\
     & \equiv (p \wedge \neg q) \vee q \tag{Identity}\\
     & \equiv q \vee (p \wedge \neg q) \tag{Commutative}\\
     & \equiv (q \vee p) \wedge (q \vee \neg q) \tag{Distributive}\\
     & \equiv (q \vee p) \wedge \mathbf{T} \tag{Negation}\\
     & \equiv  q \vee p \tag{Identity} \\
     & \equiv p \vee q \tag{Commutative}
     \end{align*}

    \vspace{-30pt}
    \begin{center}
    OR
    \end{center}

    \vspace{-30pt}
    \begin{align*}
    \neg(\neg p & \vee (p \wedge q)) \vee q \\
      & \equiv \neg[ (\neg p \vee p) \wedge (\neg p \vee q) ] \vee q \tag{Distributive} \\
      & \equiv \neg[ \mathbf{T} \wedge (\neg p \vee q) ] \vee q \tag{Negation} \\
      & \equiv \neg[ (\neg p \vee q)  ] \vee q \tag{Identity} \\
      & \equiv \neg( \neg p \vee q) \vee q \tag{algebra} \\
      & \equiv (\neg \neg p \wedge \neg q) \vee q \tag{DeMorgans} \\
      & \equiv (p \wedge \neg q) \vee q \tag{Double Negation} \\
      & \equiv q \vee (p \wedge \neg q) \tag{Commutative} \\
      & \equiv (q \vee p) \wedge (q \vee \neg q) \tag{Distributive} \\
      & \equiv (q \vee p) \wedge \mathbf{T} \tag{Negation} \\
      & \equiv (q \vee p) \tag{Identity} \\
      & \equiv q \vee p \tag{algebra}
    \end{align*}
    \end{parts}
    \end{solution}



% \newpage
\ugquestion{8}  Rosen Ch 1.6 \#6, p. 78.\\
Let $r$ be the proposition ``It rains", let $f$ be ``It is foggy",
    let $s$ be ``The sailing race will be held", let $l$ be ``The life
    saving demonstration will go on", and let $t$ be ``The trophy will
    be awarded".
    \ifprintanswers
        \vspace{-10pt}
    \fi
\begin{solution}
    The premises are: $(\neg r \vee \neg f) \rightarrow (s \wedge l)$, $s \rightarrow t$, $\neg t$. \\
    The conclusion we want is: $r$.

    \begin{tabular}{lll}
        Step    & \hspace{0.2in} & Reason \\
        \hline
        1. $\neg t$                 &       & Hypothesis (Premise, Given) \\
        2. $s \rightarrow t$        &       & Hypothesis (Premise, Given) \\
        3. $\neg s$                 &       & Modus tollens with (1) and (2) \\
        4. $(\neg r \vee \neg f) \rightarrow (s \wedge l)$  &   & Hypothesis (Premise, Given)  \\
        5. $(\neg(s \wedge l)) \rightarrow \neg(\neg r \vee \neg f)$    & & Contrapositive with (4) \\
        6. $(\neg s \vee \neg l) \rightarrow (\neg \neg r \wedge \neg \neg f)$ & & De Morgans law with (5) (x2) \\
        7. $(\neg s \vee \neg l) \rightarrow (r \wedge f)$  & & Double negation with (6) (x2) \\
        8. $\neg s \vee \neg l$     &       & Addition with (3) \\
        9. $r \wedge f$             &       & Modus ponens with (7) and (8) \\
        10. $r$                     &       & Simplification with (9)
    \end{tabular}

    \medskip
    Another valid argument is:

    \begin{tabular}{lll}
        Step   & \hspace{0.2in}     & Reason \\
        \hline
        1. $\neg t$                 &   & Hypothesis \\
        2. $s \ra t$                &   & Hypothesis \\
        3. $(\neg r \vee \neg f) \ra (s \wedge l)$  &   & Hypothesis \\
        4. $\neg s$                 &   & Modus tollens, (1), (2) \\
        5. $\neg (\neg r \vee \neg f) \vee (s \wedge l) $  & & Table 7, rule 1, (3) \\
        6. $(\neg \neg r \wedge \neg \neg f) \vee (s \wedge l)$  & & DeMorgans,  (5) \\
        7. $(r \wedge f) \vee (s \wedge l)$     & & Double Negation (x2), (6) \\
        8. $\neg s \vee \neg l$     &   & Addition with (4) \\
        9. $\neg (s \wedge l)$      &   & DeMorgans, (8) \\
        10. $r \wedge f$            &   & Disjunctive Syllogism, (7), (9) \\
        11. $r$                     &   & Simplification, (10)
    \end{tabular}


    \newpage
    Or, another valid argument is:

    \begin{tabular}{lll}
        Step   & \hspace{0.2in}     & Reason \\
        \hline
        1. $(\neg r \vee \neg f) \ra (s \wedge l)$                &   & Hypothesis \\
        2. $s \ra t$                &   & Hypothesis \\
        3. $\neg t$  &   & Hypothesis \\
        4. $\neg s$                 &   & Modus tollens, (2), (3) \\
        5. $\neg s \vee \neg l$     &   & Addition with (4) \\
        6. $\neg(s \wedge l)$       &   & DeMorgans with (5) \\
        7. $\neg (\neg r \vee \neg f)$  & & Modus tollens with (1) and (6)\\
        8. $\neg \neg r \wedge \neg \neg f$ & & DeMorgans with (7) \\
        9. $r \wedge f$             &   & Double Negation with (8) \\
        10. $r$                     &   & Simplification, (9)
    \end{tabular}
\end{solution}



\gquestion{10}{10}{all} Use the rules of inference to show that the hypotheses imply the conclusion:
\begin{itemize}[itemsep=0pt,parsep=0pt,topsep=0pt,partopsep=0pt]
    \item ``If I graduate in four years, then I will have completed the CS courses", and
    \item ``If I do not work on CS for 10 hours a week, then I will not complete the CS courses", and
    \item ``If I work on CS for 10 hours a week, then I can not procrastinate."
\end{itemize}
Conclusion: ``If I procrastinate, then I will not graduate in four years."
Let
\begin{itemize}[itemsep=0pt,parsep=0pt,topsep=0pt,partopsep=0pt]
    \item[$w = $] ``I work on CS for 10 hours a week",
    \item[$g = $] ``I graduate in four years",
    \item[$c = $] ``I will complete the CS courses", and
    \item[$p = $] ``I procrastinate."
\end{itemize}

\begin{parts}
  \part (4 pts) Translate the hypotheses and conclusion to logical statements. 
  \part (6 pts) Construct a valid argument (justify each step). 
\end{parts}

\textit{Hint: Remember you can also use the logical equivalences as a step in the argument.}
    \ifprintanswers
        \vspace{-12pt}
    \fi
\begin{solution}
  (a) The hypotheses are: $g \ra c$, $\neg w \ra \neg c$ and $w \ra \neg p$. \\
  The conclusion is: $p \ra \neg g$.
  
  (b)

    \begin{tabular}{lll}
        Step    & \hspace{0.2in} & Reason \\
        \hline
        1. $g \ra c$                & & hypothesis \\
        2. $\neg w \ra \neg c$          & & hypothesis \\
        3. $w \ra \neg p$             & & hypothesis \\
        4. $\neg \neg p \ra \neg w$   & & Table 7, rule 2, with (3) \\
        5. $p \ra \neg w$       & & Double Negation with (4) \\
        6. $p \ra \neg c$               & & Hyp. syl., with (2) \& (5) \\
        7. $\neg c \ra \neg g$          & & Table 7, rule 2 with (1) \\
        8. $p \ra \neg g$       & & Hyp. syl. with (6) \& (7)
    \end{tabular}
    
    \emph{Note, this is one possible valid argument; many others exists.}

    \begin{tabular}{lll}
        Step    & \hspace{0.2in} & Reason \\
        \hline
        1. $g \ra c$                    & & premise \\
        2. $\neg w \ra \neg c$          & & premise \\
        3. $w \ra \neg p$               & & premise \\
        4. $\neg g \vee c$              & & Table 7.1 with (1) \\
        5. $\neg \neg w \vee \neg c$    & & Table 7.1 with (2) \\
        6. $\neg w \vee \neg p$         & & Table 7.1 with (3) \\
        7. $w \vee \neg c$              & & Double Negation with (5) \\
        8. $\neg p \vee \neg c$         & & Resolution with (6) and (7) \\
        9. $\neg c \vee \neg p$         & & Commutative with (8) \\
        10. $c \vee \neg g$             & & Commutative with (9) \\
        11. $\neg p \vee \neg g$        & & Resolution with (9) and (10) \\
        12. $p \ra \neg g$              & & Table 7.1 with (11)
    \end{tabular}
\end{solution}


\gquestion{12}{12}{all} Use the rules of inference to show that the hypotheses imply the conclusion:

\begin{quote}
If you eat carefully then you will have a healthy digestive system. If you exercise regularly you will be very fit. If you have a healthy digestive system or you are very fit, you will live to a ripe old age. You do not live to a ripe old age. Therefore, you did not eat carefully and you did not exercise regularly.
\end{quote}

Let
\begin{itemize}[itemsep=0pt,parsep=0pt,topsep=0pt,partopsep=0pt]
    \item[$c = $] ``you eat carefully",
    \item[$h = $] ``you have a healthy digestive system",
    \item[$e = $] ``you excercise regularly",
    \item[$f = $] ``you will be very fit", 
    \item[$l = $] ``you will live to a ripe old age"
\end{itemize}

\begin{parts}
  \part (4 pts) Translate the hypotheses and conclusion to logical statements. 
  \part (8 pts) Construct a valid argument (justify each step). 
\end{parts}

\textit{Hint: Remember you can also use the logical equivalences as a step in the argument.}
    \ifprintanswers
        \vspace{-2pt}
    \fi
\begin{solution}
  % (a) The hypotheses are: 
  % 	\begin{itemize}[itemsep=0pt,parsep=0pt,topsep=0pt,partopsep=0pt]
  % 		\item $c \ra h$ 
  % 		\item $e \ra f$ 
  % 		\item $(h \vee f) \ra l$ 
  % 		\item $\neg l$ 
  % 	\end{itemize}
  % The conclusion is: $\neg c \wedge \neg e$.
  (a) The hypotheses are:  $\;\;c \ra h$ \qquad $e \ra f$ \qquad $(h \vee f) \ra l$ 
  \qquad $\neg l$ 

  The conclusion is: $\neg c \wedge \neg e$
  
  (b) 

  \begin{tabular}{lll}
        Step    & \hspace{0.2in} & Reason \\
        \hline
        1. $c \ra h$                & & hypothesis \\
        2. $e \ra f$           & & hypothesis \\
        3. $(h \vee f) \ra l$            & & hypothesis \\
        4. $\neg l$ 				& & hypothesis \\
        5. $\neg (h \vee f)$   & & modus tollens (3) \& (4) \\
        6. $\neg h \wedge \neg f$       & & DeMorgans (5) \\
        7. $\neg h$               & & Simplification (6) \\
        8. $\neg c$          & & modus tollens (7) \& (1) \\
        9. $\neg f$       & & Simplification (6) \\
        10. $\neg e$  		& & modus tollens (2) \& (9) \\
        11. $\neg c \wedge \neg e$ 		 & & Conjunction (8) \& (10) \\
    \end{tabular}
    
    \emph{Note, this is one possible valid argument; many others exists.}

    % \begin{tabular}{lll}
    %     Step    & \hspace{0.2in} & Reason \\
    %     \hline
    %     1. $c \ra h$                & & premise \\
    %     2. $e \ra f$                & & premise \\
    %     3. $(h \vee f) \ra l$       & & premise \\
    %     4. $\neg l$                 & & premise \\
    %     5. $\neg (h \vee f)$        & & modus tollens (3) \& (4) \\
    %     6. $
\end{solution}



\bonusquestion[2] Give a compound proposition with three variables $p$, $q$, and $r$ that is 
true when at most one of the three variables is true, and false otherwise.
    \begin{solution}
    \[ (p \wedge \neg q \wedge \neg r) \vee (\neg p \wedge q \neg r) 
        \vee (\neg p \wedge \neg q \wedge r) \vee (\neg p \wedge \neg q \neg r)\]
    \end{solution}





\end{questions}

\end{document}